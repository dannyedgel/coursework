%%% Econ714: Macroeconomics II
%%% Spring 2021
%%% Danny Edgel
%%%
% Due on Canvas Friday, April 23rd, 11:59pm Central Time
%%%

%%%
%							PREAMBLE
%%%

\documentclass{article}

%%% declare packages
\usepackage{amsmath}
\usepackage{amssymb}
\usepackage{array}
\usepackage{bm}
\usepackage{changepage}
\usepackage{centernot}
\usepackage{graphicx}
\usepackage{xcolor}
\usepackage[shortlabels]{enumitem}
\usepackage{fancyhdr}
	\fancyhf{} % sets both header and footer to nothing
	\renewcommand{\headrulewidth}{0pt}
    \rfoot{Edgel, \thepage}
    \pagestyle{fancy}
	
%%% define shortcuts for set notation
\newcommand{\Z}{\mathbb{Z}}
\newcommand{\R}{\mathbb{R}}
\newcommand{\Q}{\mathbb{Q}}
\newcommand{\lmt}{\underset{x\rightarrow\infty}{\text{lim }}}
\newcommand{\neglmt}{\underset{x\rightarrow-\infty}{\text{lim }}}
\newcommand{\zerolmt}{\underset{x\rightarrow 0}{\text{lim }}}
\newcommand{\loge}[1]{\text{log}\left(#1\right)}
\newcommand{\usmax}[1]{\underset{#1}{\text{max }}}
\newcommand{\usmin}[1]{\underset{#1}{\text{min }}}
\newcommand{\Mt}{M_{t+1}^t}
\newcommand{\vhat}{\hat{v}}
\newcommand{\olp}{\overline{p}}
\renewcommand{\L}{\mathcal{L}}
\newcommand{\olq}{\overline{q}}
\newcommand{\zinf}{_{t=0}^\infty}
\newcommand{\aneg}{A^{-1}}
\newcommand{\sneg}{s^{-1}}
\newcommand{\olk}{\overline{k}}
\newcommand{\olc}{\overline{c}}
\newcommand{\olr}{\overline{r}}
\newcommand{\olpi}{\overline{\pi}}
\newcommand{\Aneg}{A^{-1}}
\renewcommand{\sneg}{s^{-1}}
\newcommand{\dc}[1]{\Delta c_{#1}}
\newcommand{\N}{\mathcal{N}}
\newcommand{\suminf}{\sum_{t=0}^\infty}
\newcommand{\sumn}{\sum_{i=1}^{n}}
\newcommand{\sumnk}{\sum_{i=1}^{N_k}}
\newcommand{\red}[1]{{\color{red}#1}}
\newcommand{\Tau}{\mathrm{T}}
\newcommand{\phat}{\hat{p}}
\newcommand{\qs}{q^*}
\newcommand{\pl}{\partial}

\newcommand{\E}[1]{\mathbb{E}\left[#1\right]} % expected value
\newcommand{\Et}[1]{\mathbb{E}_t\left[#1\right]}

%%% define column vector command (from Michael Nattinger)
\newcount\colveccount
\newcommand*\colvec[1]{
        \global\colveccount#1
        \begin{pmatrix}
        \colvecnext
}
\def\colvecnext#1{
        #1
        \global\advance\colveccount-1
        \ifnum\colveccount>0
                \\
                \expandafter\colvecnext
        \else
                \end{pmatrix}
        \fi
}

%%% define function for drawing matrix augmentation lines
\newcommand\aug{\fboxsep=-\fboxrule\!\!\!\fbox{\strut}\!\!\!}

\makeatletter
\let\amsmath@bigm\bigm

\renewcommand{\bigm}[1]{%
  \ifcsname fenced@\string#1\endcsname
    \expandafter\@firstoftwo
  \else
    \expandafter\@secondoftwo
  \fi
  {\expandafter\amsmath@bigm\csname fenced@\string#1\endcsname}%
  {\amsmath@bigm#1}%
}


%________________________________________________________________%

\begin{document}

\title{	Problem Set \#2 }
\author{ 	Danny Edgel 					        	      \\ 
			Econ 761: Industrial Organization Theory	\\
			Fall 2021						                      \\
		}
\maketitle\thispagestyle{empty}


%%%________________________________________________________________%%%

\section*{Question 1}
\begin{itemize}
    \item[(a)] Using the demand function, (1), we can solve: \[
    \frac{dQ}{dP}\frac{P}{Q} = -\frac{1}{a_1}\frac{a_0 - a_1Q + \nu}{Q} = 1 - \frac{a_0 + \nu}{a_1Q}
    \] 
    \item[(b)] A Cournot equilibrium with homogenous firms is characterized by:\[
      \frac{dC}{dq} - \frac{Q}{N}\frac{dP}{dQ} = P
    \]
    Solving for $Q^*$ and $P^*$ with a fixed $N$ and $F$ yields:\[
      Q^* = \frac{b_0-a_0 + \eta+\nu}{2b_1-\left(\frac{1}{N}+1\right)a_1},\quad\quad 
      P^* = a_0 - \frac{b_0 - a_0 + \eta + \nu}{2\frac{b_1}{a_1} - 1/N - 1} + \nu
    \]
    Letting $b_1=0$,\[
      Q^* = \left[\frac{b_0-a_0 + \eta+\nu}{(N+1)a_1}\right]N,\quad\quad 
      P^* = a_0 + \left[\frac{b_0 - a_0 + \eta + \nu}{1+N}\right]N + \nu
    \]
    \item[(c)] N/A %If firms enter until it is no longer profitable, then we can determine the equilibrium number of firms, $N^*$, by setting profit, given $P^*$ and $Q^*$, equal to zero: \begin{align*} 
    %  P^*Q^* &= F  + (b_0 + b_1Q^* + \eta)Q^*
    %\end{align*}
    \item[(d)] Using the values calculated above, we can calculate the Lerner index, $L_I$, and Herfindahl index, $H$, as follows: \begin{align*}
      L_I &=  -1/\varepsilon = -\left(1 - \frac{a_0 + \nu}{a_1Q^*}\right)^{-1} = \frac{a_1Q^*}{a_0 + \nu - a_1Q^*} \\ &= \left(\frac{b_0-a_0 + \eta+\nu}{a_0 + \nu -(b_0-a_0 + \eta+\nu)\frac{N}{N+1}}\right)\frac{N}{N+1} \\ &= \left(\frac{b_0-a_0 + \eta+\nu}{(2a_0 - b_0 - \eta)N + a_0 + \nu}\right)N \\
      H &= \sum_{i=1}^N\left(\frac{q^*}{Q^*}\right)^2 = \sum_{i=1}^N\frac{1}{N^2} = \frac{1}{N}
    \end{align*} 
    \item[(e)] Equilibrium elasticity is: \[
      \varepsilon^* = \frac{(b_0 + \eta - 2a_0)N - a_0 - \nu}{(b_0 - a_0 + \eta + \nu)N}
    \]
    Thus, we can calculate:\begin{align*}
      \frac{\pl\varepsilon^*}{\pl F}    &= 0  \\
      \frac{\pl\varepsilon^*}{\pl \nu}  &= \frac{-(b_0 - a_0 + \eta + \nu)N - [(b_0 + \eta - 2a_0)N - a_0 - \nu]N}{(b_0 - a_0 + \eta + \nu)^2N^2}   \\
      \frac{\pl\varepsilon^*}{\pl \eta} &= \frac{(b_0 - a_0 + \eta + \nu)N^2- [(b_0 + \eta - 2a_0)N - a_0 - \nu]N}{(b_0 - a_0 + \eta + \nu)^2N^2}
    \end{align*}
    \item[(f)]
    \item[(g)]
\end{itemize}

\section*{Question 2}
\begin{itemize}
    \item[(a)]
    \item[(b)]
    \item[(c)]
    \item[(d)]
    \item[(e)]
    \item[(f)]
\end{itemize}


\section*{Question 3}
\begin{itemize}
    \item[(a)]
    \item[(b)]
    \item[(c)]
\end{itemize}

%%%________________________________________________________________%%%




\end{document}






