%%% Econ714: Macroeconomics II
%%% Spring 2021
%%% Danny Edgel
%%%
% Due on Canvas Friday, April 23rd, 11:59pm Central Time
%%%

%%%
%							PREAMBLE
%%%

\documentclass{article}

%%% declare packages
\usepackage{amsmath}
\usepackage{amssymb}
\usepackage{array}
\usepackage{bm}
\usepackage{changepage}
\usepackage{centernot}
\usepackage{graphicx}
\usepackage{xcolor}
\usepackage[shortlabels]{enumitem}
\usepackage{fancyhdr}
	\fancyhf{} % sets both header and footer to nothing
	\renewcommand{\headrulewidth}{0pt}
    \rfoot{Edgel, \thepage}
    \pagestyle{fancy}
	
%%% define shortcuts for set notation
\newcommand{\Z}{\mathbb{Z}}
\newcommand{\R}{\mathbb{R}}
\newcommand{\Q}{\mathbb{Q}}
\newcommand{\lmt}{\underset{x\rightarrow\infty}{\text{lim }}}
\newcommand{\neglmt}{\underset{x\rightarrow-\infty}{\text{lim }}}
\newcommand{\zerolmt}{\underset{x\rightarrow 0}{\text{lim }}}
\newcommand{\loge}[1]{\text{log}\left(#1\right)}
\newcommand{\usmax}[1]{\underset{#1}{\text{max }}}
\newcommand{\usmin}[1]{\underset{#1}{\text{min }}}
\newcommand{\Mt}{M_{t+1}^t}
\newcommand{\vhat}{\hat{v}}
\newcommand{\olp}{\overline{p}}
\renewcommand{\L}{\mathcal{L}}
\newcommand{\olq}{\overline{q}}
\newcommand{\zinf}{_{t=0}^\infty}
\newcommand{\aneg}{A^{-1}}
\newcommand{\sneg}{s^{-1}}
\newcommand{\olk}{\overline{k}}
\newcommand{\olc}{\overline{c}}
\newcommand{\olr}{\overline{r}}
\newcommand{\olpi}{\overline{\pi}}
\newcommand{\Aneg}{A^{-1}}
\renewcommand{\sneg}{s^{-1}}
\newcommand{\dc}[1]{\Delta c_{#1}}
\newcommand{\N}{\mathcal{N}}
\newcommand{\suminf}{\sum_{t=0}^\infty}
\newcommand{\sumn}{\sum_{i=1}^{n}}
\newcommand{\sumnk}{\sum_{i=1}^{N_k}}
\newcommand{\red}[1]{{\color{red}#1}}
\newcommand{\Tau}{\mathrm{T}}
\newcommand{\phat}{\hat{p}}
\newcommand{\qs}{q^*}
\newcommand{\pl}{\partial}

\newcommand{\E}[1]{\mathbb{E}\left[#1\right]} % expected value
\newcommand{\Et}[1]{\mathbb{E}_t\left[#1\right]}

%%% define column vector command (from Michael Nattinger)
\newcount\colveccount
\newcommand*\colvec[1]{
        \global\colveccount#1
        \begin{pmatrix}
        \colvecnext
}
\def\colvecnext#1{
        #1
        \global\advance\colveccount-1
        \ifnum\colveccount>0
                \\
                \expandafter\colvecnext
        \else
                \end{pmatrix}
        \fi
}

%%% define function for drawing matrix augmentation lines
\newcommand\aug{\fboxsep=-\fboxrule\!\!\!\fbox{\strut}\!\!\!}

\makeatletter
\let\amsmath@bigm\bigm

\renewcommand{\bigm}[1]{%
  \ifcsname fenced@\string#1\endcsname
    \expandafter\@firstoftwo
  \else
    \expandafter\@secondoftwo
  \fi
  {\expandafter\amsmath@bigm\csname fenced@\string#1\endcsname}%
  {\amsmath@bigm#1}%
}


%________________________________________________________________%

\begin{document}

\title{	Problem Set \#2 }
\author{ 	Danny Edgel 					        	      \\ 
			Econ 761: Industrial Organization Theory	\\
			Fall 2021						                      \\
		}
\maketitle\thispagestyle{empty}


%%%________________________________________________________________%%%

\section*{Question 1}
\begin{itemize}
    \item[(a)] Using the demand function, (1), we can solve: \[
    \frac{dQ}{dP}\frac{P}{Q} = -\frac{1}{a_1}\frac{a_0 - a_1Q + \nu}{Q} = 1 - \frac{a_0 + \nu}{a_1Q}
    \] 
    Thus, elasticity is increasing in $Q$ and decreasing in $\nu$.
    \item[(b)] A Cournot equilibrium with homogenous firms is characterized by:\[
      \frac{dC}{dq} - \frac{Q}{N}\frac{dP}{dQ} = P
    \]
    Solving for $Q^*$ and $P^*$ with a fixed $N$ and $F$ yields:\[
      Q^* = \left(\frac{a_0-b_0+\nu-\eta}{2(a_1-b_1)}\right)N,\quad\quad 
      P^* = a_0 - \left(\frac{a_0-b_0+\nu-\eta}{2(a_1-b_1)}\right)a_1N + \nu
    \]
    \item[(c)] If firms enter until it is no longer profitable, then we can determine the equilibrium number of firms, $N^*$, by setting profit, given $P^*$ and $Q^*$, equal to zero: \begin{align*} 
      P^*(Q^*/N) &= F  + (b_0 + b_1Q^* + \eta)(Q^*/N) \\
      N^* &= \frac{2(a_1-b_1)}{a_1 + b_1} - \frac{4F(a_1-b_1)^2}{(a_0-b_0+\nu-\eta)^2(a_1-b_1)}
    \end{align*}
    Letting $b_1=0$, the equilibrium value for N reduces to:\[
      N^* = 2-\frac{4Fa_1}{a_0-b_0+\nu-\eta}
    \]
    \item[(d)] Using the values calculated above, we can calculate the Lerner index, $L_I$, and Herfindahl index, $H$, as follows (letting ${b_1=0}$ in the final step): \begin{align*}
      L_I &=  -1/\varepsilon = -\left(1 - \frac{a_0 + \nu}{a_1Q^*}\right)^{-1} = \frac{a_1Q^*}{a_0 + \nu - a_1Q^*} \\ &= \frac{\left(\frac{a_0 -b_0 + \nu - \eta}{2(a_1-b_1)}\right)a_1N}{a_0 + \nu - \left(\frac{a_0 -b_0 + \nu - \eta}{2(a_1-b_1)}\right)a_1N} \\ &= \frac{(a_0-b_0+\nu-\eta)N}{2(a_0+\nu)-(a_0-b_0+\nu-\eta)N} \\
      H &= \sum_{i=1}^N\left(\frac{q^*}{Q^*}\right)^2 = \sum_{i=1}^N\frac{1}{N^2} = \frac{1}{N}
    \end{align*} 
    \item[(e)] Equilibrium elasticity (letting ${b_1=0}$ in the final step) is: \[
      \varepsilon^* = 1-\frac{2(a_0+\nu)(a_1-b_1)}{(a_0-b_0+\nu-\eta)a_1N}
      = 1 - \frac{2(a_0+\nu)}{(a_0-b_0+\nu-\eta)N}
    \]
    Thus, we can calculate:\begin{align*}
      \frac{\pl\varepsilon^*}{\pl F}    &= 0                                        \\
      \frac{\pl\varepsilon^*}{\pl \nu}  &= \frac{2(b_0+\eta)}{(a_0-b_0+\nu-\eta)^2N} \\
      \frac{\pl\varepsilon^*}{\pl \eta} &= \frac{2(a_0+\nu)}{(a_0-b_0+\nu-\eta)^2N} 
    \end{align*}
    Using the equations from (d), we can calculate $\loge{L_1}$ and $\loge{H}$:\begin{align*}
      \loge{L_1}  &= \loge{a_0-b_0+\nu-\eta} + \loge{N} - \loge{2(a_0+\nu)-(a_0-b_0+\nu-\eta)N} \\
      \loge{H}    &= -\loge{N}
    \end{align*}
    Thus, neither index changes with $F$, and $\loge{H}$ does not change with any variable other than $N$.
    \item[(f)] If firms collude and split the profits, the new equilibrium will be determined by: \[
      \usmax{Q}(a_0-a_1Q+\nu)Q - F - (b_0-b_1Q + \nu)Q
    \]
    Which results in the following equilibrium price and quantity:\[
      Q^* = \frac{b_0-a_0}{2(b_1-a_1)},\quad P^* = a_0 - \left(\frac{b_0-a_0}{b_1-a_1}\right)\frac{a_1}{2} + \nu
    \]
    Assuming that the colluding firms split profit equally, we can determine the endogenous number of firms in equilibrium as follows:\begin{align*}
      0 &= \pi(Q^*/N,P^*)                                    \\
      FN^2 &= (a_0 - b_0 + \nu - \eta)QN - a_1Q^2N + b_1Q^2 \\
      N^* &= \frac{a_1Q^2-(a_0-b_0+\nu-\eta)Q\pm\sqrt{[-a_1Q^2+(a_0-b_0+\nu-\eta)Q]^2+4Fb_1Q^2}}{-2F}
    \end{align*}
    Letting ${b_1=0}$, this problem simplifies nicely, with $N^*=0$ as one solution and, for the other:\[
      N^* = \frac{a_0 - b_0 + \nu - \eta - a_1Q^2}{F} 
          = \frac{a_0 - b_0 + \nu - \eta}{F} - \frac{(b_0-a_0)^2}{4Fa_1}
    \]
    \item[(g)] The elasticity of (3) is solved as follows: \begin{align*}
      P &= e^{c_0 + \xi}Q^{-c_1}  \\
      \frac{dQ}{dP} &= -\frac{1}{c_1}\left[e^{c_0 + \xi}Q^{-c_1}\right]^{\frac{-1}{c_1}-1}e^{\frac{c_0+\xi}{c_1}} = -\frac{1}{c_1}Q^{1+c_1} \\
      \frac{P}{Q} &= e^{c_0 + \xi}Q^{c_1-1}     \\
      \varepsilon &= \frac{dQ}{dP}\frac{P}{Q} = -\frac{1}{c_1}e^{c_0 + \xi}
    \end{align*}
    This does not change with $Q$ or $\xi$. Using the same Cournot equilibrium formula from (a), we can solve for the equilbrium under (3):\begin{align*}
      \frac{dc}{dq} - \frac{Q}{N}\frac{dQ}{dP} &= P \\
      \frac{c_1}{N} + \frac{Q}{e^{c_0 + \xi}}\left(b_0+\eta-2b_1\frac{Q}{N}\right) &= 1
    \end{align*}
    Again letting ${b_1=0}$, we can solve:\[
      Q^* = e^{\frac{c_0-\xi}{c_1}}\left(\frac{N-c_1}{N(b_0+\eta)}\right)^\frac{1}{c_1}, \quad\quad 
      P^* = \frac{N(b_0+\eta)}{N-c_1}
    \]
    The Lerner index, $L_I$, and Herfindahl index, $H$, for this system are:\begin{align*}
      &L_I = -1/\varepsilon = c_1e^{-c_0-\xi}, &H = 1/N
    \end{align*}
    Equilibrium elasticity does not depend on $F$ or $\eta$, but is is decreasing in $\xi$:\[
      \frac{\pl\varepsilon^*}{\pl\xi} = -\frac{1}{c_1}e^{c_0 + \xi}
    \]
    The Herfindahl index (and its log) are not changing in $F$, $\eta$, or $\xi$, as it is only changing in $N$. However, the log of the Lerner index is decreasing in $\xi$:\[
      \loge{L_I} = c_1 - c_0 - \xi,\quad \frac{\pl\loge{L_I}}{\pl\xi} = -1
    \]
\end{itemize}

\section*{Question 2}
\begin{itemize}
    \item[(a)]
    \item[(b)]
    \item[(c)]
    \item[(d)]
    \item[(e)]
    \item[(f)]
\end{itemize}


\section*{Question 3}
\begin{itemize}
    \item[(a)]
    \item[(b)]
    \item[(c)]
\end{itemize}

%%%________________________________________________________________%%%




\end{document}






