%%% Econ714: Macroeconomics II
%%% Spring 2021
%%% Danny Edgel
%%%
% Due on Canvas Friday, April 23rd, 11:59pm Central Time
%%%

%%%
%							PREAMBLE
%%%

\documentclass{article}

%%% declare packages
\usepackage{amsmath}
\usepackage{amssymb}
\usepackage{array}
\usepackage{bm}
\usepackage{changepage}
\usepackage{centernot}
\usepackage{graphicx}
\usepackage{xcolor}
\usepackage[shortlabels]{enumitem}
\usepackage{fancyhdr}
	\fancyhf{} % sets both header and footer to nothing
	\renewcommand{\headrulewidth}{0pt}
    \rfoot{Edgel, \thepage}
    \pagestyle{fancy}
	
%%% define shortcuts for set notation
\newcommand{\Z}{\mathbb{Z}}
\newcommand{\R}{\mathbb{R}}
\newcommand{\Q}{\mathbb{Q}}
\newcommand{\lmt}{\underset{x\rightarrow\infty}{\text{lim }}}
\newcommand{\neglmt}{\underset{x\rightarrow-\infty}{\text{lim }}}
\newcommand{\zerolmt}{\underset{x\rightarrow 0}{\text{lim }}}
\newcommand{\loge}[1]{\text{log}\left(#1\right)}
\newcommand{\usmax}[1]{\underset{#1}{\text{max }}}
\newcommand{\usmin}[1]{\underset{#1}{\text{min }}}
\newcommand{\Mt}{M_{t+1}^t}
\newcommand{\vhat}{\hat{v}}
\newcommand{\olp}{\overline{p}}
\renewcommand{\L}{\mathcal{L}}
\newcommand{\olq}{\overline{q}}
\newcommand{\zinf}{_{t=0}^\infty}
\newcommand{\aneg}{A^{-1}}
\newcommand{\sneg}{s^{-1}}
\newcommand{\olk}{\overline{k}}
\newcommand{\olc}{\overline{c}}
\newcommand{\olr}{\overline{r}}
\newcommand{\olpi}{\overline{\pi}}
\newcommand{\Aneg}{A^{-1}}
\renewcommand{\sneg}{s^{-1}}
\newcommand{\dc}[1]{\Delta c_{#1}}
\newcommand{\N}{\mathcal{N}}
\newcommand{\suminf}{\sum_{t=0}^\infty}
\newcommand{\sumn}{\sum_{i=1}^{n}}
\newcommand{\sumnk}{\sum_{i=1}^{N_k}}
\newcommand{\red}[1]{{\color{red}#1}}
\newcommand{\Tau}{\mathrm{T}}
\newcommand{\phat}{\hat{p}}
\newcommand{\qs}{q^*}
\newcommand{\pl}{\partial}

\newcommand{\E}[1]{\mathbb{E}\left[#1\right]} % expected value
\newcommand{\Et}[1]{\mathbb{E}_t\left[#1\right]}

%%% define column vector command (from Michael Nattinger)
\newcount\colveccount
\newcommand*\colvec[1]{
        \global\colveccount#1
        \begin{pmatrix}
        \colvecnext
}
\def\colvecnext#1{
        #1
        \global\advance\colveccount-1
        \ifnum\colveccount>0
                \\
                \expandafter\colvecnext
        \else
                \end{pmatrix}
        \fi
}

%%% define function for drawing matrix augmentation lines
\newcommand\aug{\fboxsep=-\fboxrule\!\!\!\fbox{\strut}\!\!\!}

\makeatletter
\let\amsmath@bigm\bigm

\renewcommand{\bigm}[1]{%
  \ifcsname fenced@\string#1\endcsname
    \expandafter\@firstoftwo
  \else
    \expandafter\@secondoftwo
  \fi
  {\expandafter\amsmath@bigm\csname fenced@\string#1\endcsname}%
  {\amsmath@bigm#1}%
}


%________________________________________________________________%

\begin{document}

\title{	Problem Set \#2 }
\author{ 	Danny Edgel 					        	      \\ 
			Econ 761: Industrial Organization Theory	\\
			Fall 2021						                      \\
		}
\maketitle\thispagestyle{empty}


%%%________________________________________________________________%%%

\section*{Question 1}
\begin{itemize}
    \item[(a)] Using the demand function, (1), we can solve: \[
    \frac{dQ}{dP}\frac{P}{Q} = -\frac{1}{a_1}\frac{a_0 - a_1Q + \nu}{Q} = 1 - \frac{a_0 + \nu}{a_1Q}
    \] 
    Thus, elasticity is increasing in $Q$ and decreasing in $\nu$.
    \item[(b)] In a Cournot equilibrium, each firm solves:\[
      \usmax{q_i}\left(a_0 - a_1\sum{i=1}^Nq_i + \nu\right)q_i - F - (b_0+\eta)q_i
    \]
    The resulting best response function is:\[
      q_i = \frac{a_0-b_0+\nu-\eta-a_1\sum_{j\neq i}q_j}{2a_1}
    \]
    Since firms are homogenous, $q_i=q_j$, yielding the following per-firm equilibrium quantity:\[
      q^* = \frac{a_0-b_0+\nu-\eta}{a_1(N+1)}
    \]
    Solving for $Q^*$ and $P^*$ with a fixed $N$ yields:\begin{align*}
      Q^* &= \frac{1}{a_1}\left(a_0-b_0+\nu-\eta\right)N(N+1) \\ 
      P^* &= a_0 - \left(a_0-b_0+\nu-\eta\right)N(N+1) + \nu
    \end{align*}
    \item[(c)] If firms enter until it is no longer profitable, then we can determine the equilibrium number of firms, $N^*$, by setting profit, given $P^*$ and $Q^*$, equal to zero:\footnote{$b_1$ is only present in the initial equality; in all steps that follow, ${b_1=0}$.} \begin{align*} 
      P^*q^*&= F  + (b_0 + b_1q^* + \eta)q^* \\
      a_0-(a_0-b_0+\nu-\eta)\frac{N}{N+1} &= \frac{Fa_1N}{a_0-b_0 + \nu - \eta}+b_0+\nu
    \end{align*}
    Letting ${\Gamma=a_0-b_0 + \nu - \eta}$, we can solve:\begin{align*}
      \frac{1}{N+1}\Gamma^2 - Fa_1N &= 0  \\
      Fa_1N^2 + Fa_1N - \Gamma^2 &= 0     \\
      N &= \frac{-Fa_1 \pm \sqrt{F^2a_1^2 + 4Fa_1\Gamma^2}}{2Fa_1}  \\
      N &= -1 \pm \sqrt{\frac{1}{4} + \frac{\Gamma^2}{Fa_1}}
    \end{align*}
    Since $N$ must be positive, this equation yields:\[
      N^* = \sqrt{\frac{1}{4} + \frac{(a_0 - b_0 + \nu - \eta)^2}{Fa_1}}-1
    \]
    \item[(d)] Using the values calculated above, we can calculate the Lerner index, $L_I$, and Herfindahl index, $H$, as follows (letting ${b_1=0}$ in the final step): \begin{align*} 
      H &= \sum_{i=1}^N\left(\frac{q^*}{Q^*}\right)^2 = \sum_{i=1}^N\frac{1}{N^2} = \frac{1}{N} \\
      L_I &=  -HHI/\varepsilon = -\frac{1}{N}\left(1 - \frac{a_0 + \nu}{a_1Q^*}\right)^{-1} =  -\frac{1}{N}\left(\frac{a_1Q^*}{a_0 + \nu - a_1Q^*}\right) \\ &= -\frac{\left(a_0-b_0+\nu-\eta\right)N(N+1)}{a_0 + \nu - \left(a_0-b_0+\nu-\eta\right)N^2(N+1)}
    \end{align*} 
    \item[(e)] Equilibrium elasticity is: \[
      \varepsilon^* = 1- \frac{a_0 + \nu}{(a_0-b_0+\nu-\eta)N^2(N+1)}
    \]
    Thus, we can calculate:\begin{align*}
      \frac{\pl\varepsilon^*}{\pl F}    &= 0                                        \\
      \frac{\pl\varepsilon^*}{\pl \nu}  &= \frac{(b_0 + \eta)N^2(N+1)}{(a_0-b_0+\nu-\eta)^2N^4(N+1)^2} \\
      \frac{\pl\varepsilon^*}{\pl \eta} &= -\frac{(b_0 + \eta)N^2(N+1)}{(a_0-b_0+\nu-\eta)^2N^4(N+1)^2}
    \end{align*}
    Using the equations from (d), we can calculate $\loge{L_1}$ and $\loge{H}$:\begin{align*}
      \loge{L_1}  &= \loge{a_0-b_0 + \nu - \eta} + \loge{N} + \loge{N+1} \\
                  &- \loge{a_0 + \nu - \left(a_0-b_0+\nu-\eta\right)N^2(N+1)} \\
      \loge{H}    &= -\loge{N}
    \end{align*}
    Thus, neither index changes with $F$, and $\loge{H}$ does not change with any variable other than $N$.
    \item[(f)] If firms collude and split the profits, the new equilibrium will be determined by: \[
      \usmax{Q}(a_0-a_1Q+\nu)Q - F - (b_0-b_1Q + \nu)Q
    \]
    Which results in the following equilibrium price and quantity:\[
      Q^* = \frac{b_0-a_0}{2(b_1-a_1)},\quad P^* = a_0 - \left(\frac{b_0-a_0}{b_1-a_1}\right)\frac{a_1}{2} + \nu
    \]
    Assuming that the colluding firms split profit equally, we can determine the endogenous number of firms in equilibrium as follows:\begin{align*}
      0 &= \pi(Q^*/N,P^*)                                    \\
      FN^2 &= (a_0 - b_0 + \nu - \eta)QN - a_1Q^2N + b_1Q^2 \\
      N^* &= \frac{a_1Q^2-(a_0-b_0+\nu-\eta)Q\pm\sqrt{[-a_1Q^2+(a_0-b_0+\nu-\eta)Q]^2+4Fb_1Q^2}}{-2F}
    \end{align*}
    Letting ${b_1=0}$, this problem simplifies nicely, with $N^*=0$ as one solution and, for the other:\[
      N^* = \frac{a_0 - b_0 + \nu - \eta - a_1Q^2}{F} 
          = \frac{a_0 - b_0 + \nu - \eta}{F} - \frac{(b_0-a_0)^2}{4Fa_1}
    \]
    The new Herfindahl index, $H$, is simply the reciprocal of $N^*$:\[
      H = \frac{F}{a_0-b_0+\nu-\eta - \frac{(b_0-a_0)^2}{4a_1}}
    \]
    While the new Lerner index, $L_I$, under collusion is (letting ${b1=0}$ in the final step):\[
      L_I = \frac{a_1Q^*}{(a_0 + \nu - a_1Q^*)N} = \frac{a_1\frac{b_0-a_0}{2(b_1-a_1)}}{\left(a_0 + \nu - a_1\frac{b_0-a_0}{2(b_1-a_1)}\right)N} = \frac{a_0- b_0}{(a_0 + b_0 + 2\nu)N}
    \]
    \item[(g)] The elasticity of (3) is solved as follows: \begin{align*}
      P &= e^{c_0 + \xi}Q^{-c_1}  \\
      \frac{dQ}{dP} &= -\frac{1}{c_1}\left[e^{c_0 + \xi}Q^{-c_1}\right]^{\frac{-1}{c_1}-1}e^{\frac{c_0+\xi}{c_1}} = -\frac{1}{c_1}Q^{1+c_1} \\
      \frac{P}{Q} &= e^{c_0 + \xi}Q^{c_1-1}     \\
      \varepsilon &= \frac{dQ}{dP}\frac{P}{Q} = -\frac{1}{c_1}e^{c_0 + \xi}
    \end{align*}
    This does not change with $Q$ and is decreasing in $\xi$. Using the same Cournot equilibrium formula from (a), we can solve for the equilbrium under (3):\begin{align*}
      \frac{dc}{dq} - \frac{Q}{N}\frac{dQ}{dP} &= P \\
      \frac{c_1}{N} + \frac{Q}{e^{c_0 + \xi}}\left(b_0+\eta-2b_1\frac{Q}{N}\right) &= 1
    \end{align*}
    Again letting ${b_1=0}$, we can solve:\[
      Q^* = e^{\frac{c_0-\xi}{c_1}}\left(\frac{N-c_1}{N(b_0+\eta)}\right)^\frac{1}{c_1}, \quad\quad 
      P^* = \frac{N(b_0+\eta)}{N-c_1}
    \]
    The Lerner index, $L_I$, and Herfindahl index, $H$, for this system are:\begin{align*}
      &H = 1/N, &L_I = -H/\varepsilon = \frac{c_1}{N}e^{-c_0-\xi}
    \end{align*}
    Equilibrium elasticity does not depend on $F$ or $\eta$, but is is decreasing in $\xi$:\[
      \frac{\pl\varepsilon^*}{\pl\xi} = -\frac{1}{c_1}e^{c_0 + \xi}
    \]
    The Herfindahl index (and its log) are not changing in $F$, $\eta$, or $\xi$, as it is only changing in $N$. However, the log of the Lerner index is decreasing in $\xi$:\[
      \loge{L_I} = c_1 - c_0 - \xi,\quad \frac{\pl\loge{L_I}}{\pl\xi} = -1
    \]
\end{itemize}
\pagebreak 
\section*{Question 2}
The table below displays the results from the requested analyses.
\begin{center}
  \begin{tabular}{r|cccc}
 & (1) & (2) & (3) & (4) \\ 
& OLS & OLS & IV & IV \\\hline 
$\alpha$ & -30.036 & -27.988  & -30.071  & -39.603 \\ 
& (0.216) & (0.918) & (0.216) & (0.743) \\ 
 &&&& \\ 
FE? & & X & & X \\ 
 &&&& \\ 
$R^2$ & -0.32 & 0.46 & -0.32 & 0.39 \\
N & 2256 & 2256 & 2256 & 2256 \\\hline 
\end{tabular}
\end{center}
Before interpreting the coefficients, I will first display the significance of ${\beta=1}$ in this specification:\begin{align*} 
  \loge{L_I} &= \alpha + \beta\loge{H} + e \\
  \loge{-H/\varepsilon} &= \alpha + \beta\loge{H} + e \\
  \loge{-1/\varepsilon} &= \alpha + (\beta-1)\loge{H} + e 
\end{align*}
Thus, ${\beta=1}$ is interpratively equivalent to a null result for a regression of markups on the Herfindahl index.

Using equation (3), we reject the hypothesis that ${\beta=1}$ in the collusion subsample, but we cannot reject this hypothesis in the Cournot or pooled samples. Thus, the results imply that markups are invariant in competition for Cournot and the pooled sample, but vary positively with competition in the collusion subsample. This is intuitive, since adding a firm to the market in the collusion scenario does not change aggregate profits, but each firm produces less, increasing firm-level profit margins. The results from the Cournot subsample are intuitive, since (3) is a constant elasticity demand function, so markups do not vary with $N$ (or, as a result, with $H$).

Using equation (1), we reject the hypothesis that ${\beta=1}$ in all samples, but for different reasons. In the collusion subsample, ${\beta<1}$ (as with equation(3)), while ${\beta>1}$ in the other two samples. These results suggest that markups decrease as competition increases under Cournot competition, but that markups increase as competition increases under collusion.

The results from each demand function differ because, as I mentioned in interpreting the results for equation (3), markups are constant for that demand function. Using the linear demand from (1), markups depend on aggregate quantity, so increased competition decreases firm-level markups. The results for collusion do not differ because, regardless of the demand function, colluding firms set the monopoly price and quantity and split profits equally across colluding firms. Thus, per-firm quantity decreases as the number of firms increases, but the price stays constant, resulting in a positive realtionship between markups and the number of firms.

Yes, we can learn something positive from this analysis. If we suspected collusion in markets 1-250, for example, we could run this regression on the subsample of 1-250 and test the hypothesis that ${\beta\geq1}$. A rejection of this hypothesis would be evidence of collusion.


\section*{Question 3}
The results from each regression are displayed in the table below.
\begin{center}
  \begin{tabular}{r|cccc}
 & (1) & (2) & (3) & (4) \\\hline &&&& \\ 
\E{\mu_{jt}} & -0.098 & 0.005  & -0.025  & 0.007 \\ 
Var(\mu_{jt})& 0.033 & 0.000 & 0.002 & 0.000 \\
 &&&&\\ 
\E{c_{jt}} & 0.224 & 0.121  & 0.150  & 0.119 \\ 
Var(c_{jt})& 0.035 & 0.001 & 0.003 & 0.001 \\
 &&&&\\ 
\E{m_{jt}} & -0.299 & 0.044  & -0.127  & 0.075 \\ 
Var(m_{jt})& 0.054 & 0.009 & 0.020 & 0.033 \\
&&&&\\\hline 
\end{tabular}
\end{center}
Again appealing to the simplified specification I derived in question 2, ${\beta>2}$ suggests that markups decrease as the market becomes more competitive. The results from each regression suggests that this is true. However, Markups appear to be higher and depend more strongly on the competitiveness of the market when $\eta$ is variable than when $\nu$ is. Since $\nu$ shifts maximum willingness to pay but $\eta$ shifts marginal costs, this difference is intuitive: in 1(e), we showed that equilibrium elasticity is increasing in $\nu$ and decreasing (symmetrically) in $\eta$, and higher elasticity equates to lower markups.

%%%________________________________________________________________%%%




\end{document}






