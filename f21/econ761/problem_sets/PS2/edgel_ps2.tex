%%% Econ714: Macroeconomics II
%%% Spring 2021
%%% Danny Edgel
%%%
% Due on Canvas Friday, April 23rd, 11:59pm Central Time
%%%

%%%
%							PREAMBLE
%%%

\documentclass{article}

%%% declare packages
\usepackage{amsmath}
\usepackage{amssymb}
\usepackage{array}
\usepackage{bm}
\usepackage{changepage}
\usepackage{centernot}
\usepackage{graphicx}
\usepackage{xcolor}
\usepackage[shortlabels]{enumitem}
\usepackage{fancyhdr}
	\fancyhf{} % sets both header and footer to nothing
	\renewcommand{\headrulewidth}{0pt}
    \rfoot{Edgel, \thepage}
    \pagestyle{fancy}
	
%%% define shortcuts for set notation
\newcommand{\Z}{\mathbb{Z}}
\newcommand{\R}{\mathbb{R}}
\newcommand{\Q}{\mathbb{Q}}
\newcommand{\lmt}{\underset{x\rightarrow\infty}{\text{lim }}}
\newcommand{\neglmt}{\underset{x\rightarrow-\infty}{\text{lim }}}
\newcommand{\zerolmt}{\underset{x\rightarrow 0}{\text{lim }}}
\newcommand{\loge}[1]{\text{log}\left(#1\right)}
\newcommand{\usmax}[1]{\underset{#1}{\text{max }}}
\newcommand{\usmin}[1]{\underset{#1}{\text{min }}}
\newcommand{\Mt}{M_{t+1}^t}
\newcommand{\vhat}{\hat{v}}
\newcommand{\olp}{\overline{p}}
\renewcommand{\L}{\mathcal{L}}
\newcommand{\olq}{\overline{q}}
\newcommand{\zinf}{_{t=0}^\infty}
\newcommand{\aneg}{A^{-1}}
\newcommand{\sneg}{s^{-1}}
\newcommand{\olk}{\overline{k}}
\newcommand{\olc}{\overline{c}}
\newcommand{\olr}{\overline{r}}
\newcommand{\olpi}{\overline{\pi}}
\newcommand{\Aneg}{A^{-1}}
\renewcommand{\sneg}{s^{-1}}
\newcommand{\dc}[1]{\Delta c_{#1}}
\newcommand{\N}{\mathcal{N}}
\newcommand{\suminf}{\sum_{t=0}^\infty}
\newcommand{\sumn}{\sum_{i=1}^{n}}
\newcommand{\sumnk}{\sum_{i=1}^{N_k}}
\newcommand{\red}[1]{{\color{red}#1}}
\newcommand{\Tau}{\mathrm{T}}
\newcommand{\phat}{\hat{p}}
\newcommand{\qs}{q^*}
\newcommand{\pl}{\partial}

\newcommand{\E}[1]{\mathbb{E}\left[#1\right]} % expected value
\newcommand{\Et}[1]{\mathbb{E}_t\left[#1\right]}

%%% define column vector command (from Michael Nattinger)
\newcount\colveccount
\newcommand*\colvec[1]{
        \global\colveccount#1
        \begin{pmatrix}
        \colvecnext
}
\def\colvecnext#1{
        #1
        \global\advance\colveccount-1
        \ifnum\colveccount>0
                \\
                \expandafter\colvecnext
        \else
                \end{pmatrix}
        \fi
}

%%% define function for drawing matrix augmentation lines
\newcommand\aug{\fboxsep=-\fboxrule\!\!\!\fbox{\strut}\!\!\!}

\makeatletter
\let\amsmath@bigm\bigm

\renewcommand{\bigm}[1]{%
  \ifcsname fenced@\string#1\endcsname
    \expandafter\@firstoftwo
  \else
    \expandafter\@secondoftwo
  \fi
  {\expandafter\amsmath@bigm\csname fenced@\string#1\endcsname}%
  {\amsmath@bigm#1}%
}


%________________________________________________________________%

\begin{document}

\title{	Problem Set \#2 }
\author{ 	Danny Edgel 					        	      \\ 
			Econ 761: Industrial Organization Theory	\\
			Fall 2021						                      \\
		}
\maketitle\thispagestyle{empty}


%%%________________________________________________________________%%%

\section*{Question 1}
\begin{itemize}
    \item[(a)] Using the demand function, (1), we can solve: \[
    \frac{dQ}{dP}\frac{P}{Q} = -\frac{1}{a_1}\frac{a_0 - a_1Q + \nu}{Q} = 1 - \frac{a_0 + \nu}{a_1Q}
    \] 
    Thus, elasticity is increasing in $Q$ and decreasing in $\nu$.

    \item[(b)] In a Cournot equilibrium, each firm solves:\[
      \usmax{q_i}\left(a_0 - a_1\sum_{i=1}^Nq_i + \nu\right)q_i - F - (b_0+\eta)q_i
    \]
    The resulting best response function is:\[
      q_i = \frac{a_0-b_0+\nu-\eta-a_1\sum_{j\neq i}q_j}{2a_1}
    \]
    Since firms are homogenous, $q_i=q_j$, yielding the following per-firm equilibrium quantity:\[
      q^* = \frac{a_0-b_0+\nu-\eta}{a_1(N+1)}
    \]
    Solving for $Q^*$ and $P^*$ with a fixed $N$ yields:\begin{align*}
      Q^* &= \frac{1}{a_1}\left(a_0-b_0+\nu-\eta\right)\frac{N}{N+1} \\ 
      P^* &= a_0 - \left(a_0-b_0+\nu-\eta\right)\frac{N}{N+1} + \nu
    \end{align*}
    \item[(c)] If firms enter until it is no longer profitable, then we can determine the equilibrium number of firms, $N^*$, by setting profit, given $P^*$ and $Q^*$, equal to zero:\footnote{$b_1$ is only present in the initial equality; in all steps that follow, ${b_1=0}$.} \begin{align*} 
      P^*q^*&= F  + (b_0 + b_1q^* + \eta)q^* \\
      a_0-(a_0-b_0+\nu-\eta)\frac{N}{N+1} + \nu &= \frac{Fa_1(N+1)}{a_0-b_0 + \nu - \eta}+b_0+\eta
    \end{align*}
    Letting ${\Gamma=a_0-b_0 + \nu - \eta}$ and ${n=N+1}$, we can solve:\begin{align*}
      -\Gamma\frac{n-1}{n} &= \frac{Fa_1n}{\Gamma} - \Gamma   \\
      a_1n^2 &= \Gamma^2                                      \\
      n &= \Gamma (Fa_1)^{-1/2}
    \end{align*}
    Since $N$ must be positive, this equation yields: \[
      N^* = \frac{a_0 - b_0 + \nu - \eta}{\sqrt{Fa_1}} - 1
    \]

    \item[(d)] Using the values calculated above, we can calculate the Lerner index, $L_I$, and Herfindahl index, $H$, as follows (letting ${b_1=0}$ in the final step): \begin{align*} 
      H &= \sum_{i=1}^N\left(\frac{q^*}{Q^*}\right)^2 = \sum_{i=1}^N\frac{1}{N^2} = \frac{1}{N} \\
      L_I &=  -HHI/\varepsilon = -\frac{1}{N}\left(1 - \frac{a_0 + \nu}{a_1Q^*}\right)^{-1} =  -\frac{1}{N}\left(\frac{a_1Q^*}{a_0 + \nu - a_1Q^*}\right) \\ &= \frac{a_0-b_0+\nu-\eta}{(a_0 + \nu)(N+1) - \left(a_0-b_0+\nu-\eta\right)N} \\
      &= \frac{a_0-b_0+\nu-\eta}{a_0 + \nu + (b_0 + \eta)N}
    \end{align*} 

    \item[(e)] Equilibrium elasticity is:\[
      \varepsilon^* = -\frac{a_0 + \nu + (b_0 + \eta)N}{(a_0-b_0+\nu-\eta)N}
    \]
    Thus, we can calculate: \begin{align*}
      \frac{\pl\varepsilon^*}{\pl F}    &= 0                                                \\
      \frac{\pl\varepsilon^*}{\pl \nu}  &= \frac{(b_0+\eta)(N+1)}{(a_0-b_0+\nu-\eta)^2N}   \\
      \frac{\pl\varepsilon^*}{\pl \eta} &= -\frac{(a_0+\nu)(N+1)}{(a_0-b_0+\nu-\eta)^2N}  
    \end{align*}
    Using the equations from (d), we can calculate $\loge{L_1}$ and $\loge{H}$:\begin{align*}
      \loge{H}    &= -\loge{N}  \\
      \loge{L_1}  &= \loge{a_0 + \nu + (b_0 + \eta)N} - \loge{a_0-b_0+\nu-\eta} - \loge{N}
    \end{align*}
    Thus, neither index changes with $F$, and $\loge{H}$ does not change with any variable other than $N$.

    \item[(f)] If firms collude and split the profits, the new equilibrium will be determined by: \[
      \usmax{Q}(a_0-a_1Q+\nu)Q - F - (b_0-b_1Q + \eta)Q
    \]
    Which results in the following equilibrium price and quantity:\[
      Q^* = \frac{a_0 - b_0 + \nu - \eta}{2a_1},\quad P^* = \frac{a_0 +
       b_0 + \nu + \eta}{2}
    \]
    Assuming that the colluding firms split profit equally, we can determine the endogenous number of firms in equilibrium as follows, where ${\Gamma=a_0-b_0+\nu-eta}$:\begin{align*}
      0 &= \pi(Q^*/N,P^*)                                    \\
      \frac{1}{2}\left(\Gamma+2b_0 + 2\eta\right)\left(\frac{1}{2a_1}\Gamma\right)\frac{1}{N} &= (b_0 + \eta)\frac{\Gamma}{2a_1N} + F                \\
      N &= \frac{\Gamma^2}{4a_1F}
    \end{align*}
    Re-substituting for $\Gamma$ and simplifying, this yields:\[
      N^* = \frac{(a_0-b_0+\nu-\eta)^2}{4a_1F}
    \]
    The new Herfindahl index, $H$, is simply the reciprocal of $N^*$:\[
      H = \frac{4a_1F}{(a_0-b_0+\nu-\eta)^2}
    \]
    While the new Lerner index, $L_I$, under collusion is:\[
      L_I = \frac{a_1Q^*}{(a_0 + \nu - a_1Q^*)N} = \left(\frac{1}{N}\right)\frac{\frac{1}{2}\Gamma}{a_0+\nu-\frac{1}{2}\Gamma} = \frac{a_0-b_0+\nu-\eta}{(a_0 + b_0 + \nu + \eta)N} 
    \]

    \item[(g)] The elasticity of (3) is solved as follows: \begin{align*}
      \varepsilon &= \frac{d\loge{Q}}{d\loge{P}}= -\frac{1}{c_1}
    \end{align*}
    This does not change with $Q$ or $\xi$. Using the same Cournot equilibrium formula from (a), we can solve for the equilbrium under (3):\begin{align*}
      \frac{dc}{dq} - \frac{Q}{N}\frac{dQ}{dP} &= P \\
      \frac{c_1}{N} + \frac{Q}{e^{c_0 + \xi}}\left(b_0+\eta-2b_1\frac{Q}{N}\right) &= 1
    \end{align*}
    Again letting ${b_1=0}$, we can solve:\[
      Q^* = e^{\frac{c_0-\xi}{c_1}}\left(\frac{N-c_1}{N(b_0+\eta)}\right)^\frac{1}{c_1}, \quad\quad 
      P^* = \frac{N(b_0+\eta)}{N-c_1}
    \]
    The Lerner index, $L_I$, and Herfindahl index, $H$, for this system are:\begin{align*}
      &H = 1/N, &L_I = -H/\varepsilon = \frac{c_1}{N}
    \end{align*}
    Equilibrium elasticity does not depend on $F$, $\eta$, or $\xi$. The Herfindahl index (and its log) are not changing in $F$, $\eta$, or $\xi$, as it is only changing in $N$. 
\end{itemize}

\pagebreak 
\section*{Question 2}
The table below displays the results from the requested analyses.
\begin{center}
  \begin{tabular}{r|cccc}
 & (1) & (2) & (3) & (4) \\ 
& OLS & OLS & IV & IV \\\hline 
$\alpha$ & -30.036 & -27.988  & -30.071  & -39.603 \\ 
& (0.216) & (0.918) & (0.216) & (0.743) \\ 
 &&&& \\ 
FE? & & X & & X \\ 
 &&&& \\ 
$R^2$ & -0.32 & 0.46 & -0.32 & 0.39 \\
N & 2256 & 2256 & 2256 & 2256 \\\hline 
\end{tabular}
\end{center}
Before interpreting the coefficients, I will first display the significance of ${\beta=1}$ in this specification:\begin{align*} 
  \loge{L_I} &= \alpha + \beta\loge{H} + e \\
  \loge{-H/\varepsilon} &= \alpha + \beta\loge{H} + e \\
  \loge{-1/\varepsilon} &= \alpha + (\beta-1)\loge{H} + e 
\end{align*}
Thus, ${\beta=1}$ is interpratively equivalent to a null result for a regression of markups on the Herfindahl index.

Using equation (3), we cannot reject this hypothesis in any sample. Thus, the results imply that markups are invariant in competition. This is intuitive, since (3) is a constant elasticity demand function, so markups do not vary with $N$ (or, for that matter, anything else).

Using equation (1), we reject the hypothesis that ${\beta=1}$ in all samples, but much more strongly in the sample of regulated cities. These results suggest that markups increase as competition increases in all samples, but the relationship is stronger in the regulated cities. This is counter-intuitive, as one would expect that linear demand results in markups decreasing as competition increases and the price decreases (while marginal costs remain constant). However, as I showed in 1(a), equation (1) gets less elastic as aggregate quantity increases. Since we only calculate the Lerner index and firms are homogenous, the ``markup'' we observe in this analysis does not account for the decrease in firm-level demand that occurs when the number of firms increases. Instead, the collusion in the unregulated sample appears as a weakened relationship between markups and market concentration, which is due to aggregate prices and quantities being constant in the cities where firms collude.

The results from each demand function differ because, as I mentioned in interpreting the results for equation (3), elasticity (and thus markups) is constant for that demand function. Using the linear demand from (1), markups depend on aggregate quantity, so increased competition (and its resulting impact on aggregate quantity) influences markups.

Yes, we can learn something positive from this analysis, but only if we believe (1) is the true demand function. If we suspected collusion in markets 1-250, for example, we could run this regression on the subsample of 1-250, then again for the remaining cities and test the hypothesis that $\beta$ from the 1-250 sample is less than or equal to $\beta$ from the 251-1000 sample. A rejection of this hypothesis would be evidence of collusion, as it would show a weaker relationship between markups and market structure in cities 1-250.


\section*{Question 3}
The results from each regression are displayed in the table below.
\begin{center}
  \begin{tabular}{r|cccc}
 & (1) & (2) & (3) & (4) \\\hline &&&& \\ 
\E{\mu_{jt}} & -0.098 & 0.005  & -0.025  & 0.007 \\ 
Var(\mu_{jt})& 0.033 & 0.000 & 0.002 & 0.000 \\
 &&&&\\ 
\E{c_{jt}} & 0.224 & 0.121  & 0.150  & 0.119 \\ 
Var(c_{jt})& 0.035 & 0.001 & 0.003 & 0.001 \\
 &&&&\\ 
\E{m_{jt}} & -0.299 & 0.044  & -0.127  & 0.075 \\ 
Var(m_{jt})& 0.054 & 0.009 & 0.020 & 0.033 \\
&&&&\\\hline 
\end{tabular}
\end{center}
As you can see above, letting ${\eta=0}$ and $\nu$ vary results in an SCP regression with zero predictive power, while letting $\eta$ vary results in a highly predictive regression. This is predictable by the equilibrium value of $N^*$ and the derived formulas for $H$ and $L_I$ that were found in (c), (d), (e), and (f) of question 1. Consider the simplified specification from question 2, using the parameterized solutions from question 1:{\small \begin{align*}
  \loge{-1/\varepsilon} &= \alpha + (\beta-1)\loge{H} + e \\
  \loge{a_0 - b_0 + \nu - \eta} + \loge{N} - \loge{a_0 + \nu + (b_0 + \eta)N}  
    &= \alpha - (\beta-1)\loge{N} + e 
\end{align*} }
All variance comes from only $\nu$ or $\eta$ and the only explanatory variable on the righthand side is $N$. Thus, in order for the model to expain the variance in the dependent variable, there must be variance that depends on an interaction between $N$ and $\eta$ or $\nu$. This is true for $\eta$, but not for $\nu$.


%%%________________________________________________________________%%%




\end{document}






