%%% Econ714: Macroeconomics II
%%% Spring 2021
%%% Danny Edgel
%%%
% Due on Canvas Friday, April 23rd, 11:59pm Central Time
%%%

%%%
%							PREAMBLE
%%%

\documentclass{article}

%%% declare packages
\usepackage{amsmath}
\usepackage{amssymb}
\usepackage{array}
\usepackage{bm}
\usepackage{changepage}
\usepackage{centernot}
\usepackage{graphicx}
\usepackage{xcolor}
\usepackage[shortlabels]{enumitem}
\usepackage{fancyhdr}
	\fancyhf{} % sets both header and footer to nothing
	\renewcommand{\headrulewidth}{0pt}
    \rfoot{Edgel, \thepage}
    \pagestyle{fancy}
	
%%% define shortcuts for set notation
\newcommand{\Z}{\mathbb{Z}}
\newcommand{\R}{\mathbb{R}}
\newcommand{\Q}{\mathbb{Q}}
\newcommand{\lmt}{\underset{x\rightarrow\infty}{\text{lim }}}
\newcommand{\neglmt}{\underset{x\rightarrow-\infty}{\text{lim }}}
\newcommand{\zerolmt}{\underset{x\rightarrow 0}{\text{lim }}}
\newcommand{\loge}[1]{\text{log}\left(#1\right)}
\newcommand{\usmax}[1]{\underset{#1}{\text{max }}}
\newcommand{\usmin}[1]{\underset{#1}{\text{min }}}
\newcommand{\Mt}{M_{t+1}^t}
\newcommand{\vhat}{\hat{v}}
\newcommand{\olp}{\overline{p}}
\renewcommand{\L}{\mathcal{L}}
\newcommand{\olq}{\overline{q}}
\newcommand{\zinf}{_{t=0}^\infty}
\newcommand{\aneg}{A^{-1}}
\newcommand{\sneg}{s^{-1}}
\newcommand{\olk}{\overline{k}}
\newcommand{\olc}{\overline{c}}
\newcommand{\olr}{\overline{r}}
\newcommand{\olpi}{\overline{\pi}}
\newcommand{\Aneg}{A^{-1}}
\renewcommand{\sneg}{s^{-1}}
\newcommand{\dc}[1]{\Delta c_{#1}}
\newcommand{\N}{\mathcal{N}}
\newcommand{\suminf}{\sum_{t=0}^\infty}
\newcommand{\sumn}{\sum_{i=1}^{n}}
\newcommand{\sumnk}{\sum_{i=1}^{N_k}}
\newcommand{\red}[1]{{\color{red}#1}}
\newcommand{\Tau}{\mathrm{T}}
\newcommand{\phat}{\hat{p}}
\newcommand{\qs}{q^*}
\newcommand{\pl}{\partial}

\newcommand{\E}[1]{\mathbb{E}\left[#1\right]} % expected value
\newcommand{\Et}[1]{\mathbb{E}_t\left[#1\right]}

%%% define column vector command (from Michael Nattinger)
\newcount\colveccount
\newcommand*\colvec[1]{
        \global\colveccount#1
        \begin{pmatrix}
        \colvecnext
}
\def\colvecnext#1{
        #1
        \global\advance\colveccount-1
        \ifnum\colveccount>0
                \\
                \expandafter\colvecnext
        \else
                \end{pmatrix}
        \fi
}

%%% define function for drawing matrix augmentation lines
\newcommand\aug{\fboxsep=-\fboxrule\!\!\!\fbox{\strut}\!\!\!}

\makeatletter
\let\amsmath@bigm\bigm

\renewcommand{\bigm}[1]{%
  \ifcsname fenced@\string#1\endcsname
    \expandafter\@firstoftwo
  \else
    \expandafter\@secondoftwo
  \fi
  {\expandafter\amsmath@bigm\csname fenced@\string#1\endcsname}%
  {\amsmath@bigm#1}%
}


%________________________________________________________________%

\begin{document}

\title{	Problem Set \#1 }
\author{ 		Danny Edgel 						\\ 
			Econ 761: Industrial Organization Theory	\\
			Fall 2021						\\
		}
\maketitle\thispagestyle{empty}


%%%________________________________________________________________%%%

\begin{enumerate}
	\item Consider a market in which the goods are homogenous.
		\begin{enumerate}[(a)]
			\item The elasticity of demand, $\varepsilon<0$, can be written as: 
                \[ \varepsilon = \left(P'(Q)\right)^{-1}\frac{P(Q)}{Q} \]
				Thus, letting $\varepsilon$ remain constant, we can derive:
				\begin{align*}
					P(Q)	&= QP'(Q)\varepsilon				\\
					P'(Q)	&= \left(QP''(Q) + P'(Q)\right)\varepsilon
				\end{align*}
				Thus, with $P'(Q)<0$ and $\varepsilon<0$, ${QP''(Q) + P'(Q)>0}$ for all $Q$.

			\item Under Cournot competition, each firm, $i$, solves the following problem:
                \[
                    \usmax{q_i}\Pi_i = P(Q)q_i - c(q_i),\quad\quad Q = \sum_{j=1}^N q_j
                    \]
                Which yields the following FOC, which is identical for all firms:
                \[
                    P(Q) + P'(Q)q_i = c'(q_i)\Rightarrow q_i = \left(P'(Q)\right)^{-1}\left(c'(q_i) - P(Q)\right)
                    \]
                Since cost functions are identical by assumption, $q_i=q_j=q$ $\forall i,j$ in equilibrium, so we use the implicit function theorem to solve:\footnote{Due to algebraic errors, I had to redo this several times, spending a long time on it. As a result, many intermediate steps are omitted below.}
                \begin{align*}
                    qP'(Nq) + c'(q) &= P(Nq)    \\
                    \frac{\pl q}{\pl N}\left[P'(Nq) + c''(q)\right] &= \left(q + N\frac{\pl q}{\pl N}\right)\left[P'(Nq)-qP''(Nq)\right]    \\
                    \frac{\pl q}{\pl N} &= \frac{q\left[P'(Nq)-qP''(Nq)\right]}{P'(Nq) + c''(q) + N\left[qP''(Nq) - P'(Nq)\right]}
                \end{align*}
                By assumption (A1), we know $c''(q)\geq P'(Nq)$, and by assumption (A2), we know ${-qP''(Nq)\geq P'(Nq)}$. Thus, we can reduce the equation as follows:
                \begin{align*} 
                    \frac{\pl q}{\pl N} &\leq \frac{q^2P''(Nq)}{c''(q) - NP'(Nq)}  \\
                    \frac{\pl q}{\pl N} &\leq \frac{q^2c''(q)}{c''(q) - Nc''(q)}  \\
                    \frac{\pl q}{\pl N} &\leq \frac{q^2}{1-N} < 0\quad\forall N>1  \\
                \end{align*}
                Since demand slopes downward and $Q=Nq$, an increase in $q$ necessarily increases $Q$, decreasing price. Thus, equilibrium price and price per firm quantity are decreasing in $N$.
		\end{enumerate}

        \item  
          \begin{enumerate}
            \item Each player, $i\in\{1,2\}$ chooses $b_i\in\R_+$ to maximize: \[
              \pi_i(b_i,b_j) = \begin{cases}
                V-b_i,&b_i>b_j  \\
                \frac{1}{2}\left(V-b_i\right),&b_i=b_j \\
                0,&b_i<b_j
              \end{cases}
            \]
            Since payoffs and valuations are symmetric, $b_i=b_j$ in equilibrium. For all $b_i=b_j<V$, each player has an incentive to raise their bid. Thus, the unique equilibrium is:
            \begin{align*}
              &b_1^*=b_2^*=v & \pi_1^*=\pi_2^*=0
            \end{align*}

            \item In all all-pay auction, player 1's payoff function is: 
            \[
              \pi_1(b_1,b_2) = \begin{cases}
                V-b_1,&b_1>b_2  \\
                \frac{1}{2}V-b_1,&b_1=b_2 \\
                -b_1,&b_1<b_2
              \end{cases}
            \]

            \item suppose $\exists$ a pure-strategy equilibrium with bids $(b_1^*,b_2^*)$. Since Payoffs and valuations are identical, any pure strategy equilibrium has ${b_1^*=b_2^*=b^*}$. Then, ${\pi^*=\frac{1}{2}V-b}$. Thus, either player could improve their payoff by deviating to ${b_i = b^*+\varepsilon}$ for $\varepsilon>0$. Thus, $b_1=b_2$ cannot be a pure-strategy Nash equilibrium.\footnote{The nonexistence of a $b_1\neq b_2$ equilibrium is trivial.}

            \item A mixed-strategy Nash equilibrium is a pair of distribution functions, $(F_1(b), F_2(b))$, from which each player draws their bid. Since bids must be weakly positive, $F_i(0)=0$. Since payoffs are negative for all ${b>V}$ but zero for a bid of zero, $F_i(V)=1$. Each player $i$ chooses $F_i$ to maximizes expected payoff:\footnote{Using the same logic as in (c), we can rule out any mass points, since such mass points will exist in both players' distributions, and either player could improve their payoffs by shifting the mass to a slightly higher bid.} \[
              \E{\pi_i(b_i,b_j)} = F_j(b_i)V-b_i
            \]
            From the first-order condition of this problem, we can obtain the symmetric equilibrium distribution function:
            \begin{align*}
              Vf_j(b_i)  - 1  &= 0                      \\
              f_j(b_i)        &= \frac{1}{V}            \\
              F_j(b)          &= \int_0^b\frac{1}{V}dx =\frac{b}{V} 
            \end{align*}
            Since payoffs and valuations are constant, ${F_i^*(b)=F_j^*(b)=F^*(b)}$. Thus, the mixed-strategy equilibrium is for each player to submit a uniformly random bid between $0$ and $V$. The seller's expected revenue is: \[
              R = 2\E{b^*} = 2\int_0^V\left(\frac{1}{V}\right)bdb 
                = \frac{1}{V}[b^2]_0^V = V
            \]

            \item If the seller sets some reserve price $R\in(0,V)$, then the lower bound of the equilibrium distribution will be truncated such that $F^*(b)$ is instead be a uniform distribution from $R$ to $V$. Intuitively, this would increase the seller's revenue by increasing the mean of the equilibrium bid distribution.
          \end{enumerate}
\end{enumerate}

%%%________________________________________________________________%%%




\end{document}






