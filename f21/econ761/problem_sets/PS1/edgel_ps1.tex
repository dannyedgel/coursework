%%% Econ714: Macroeconomics II
%%% Spring 2021
%%% Danny Edgel
%%%
% Due on Canvas Friday, April 23rd, 11:59pm Central Time
%%%

%%%
%							PREAMBLE
%%%

\documentclass{article}

%%% declare packages
\usepackage{amsmath}
\usepackage{amssymb}
\usepackage{array}
\usepackage{bm}
\usepackage{changepage}
\usepackage{centernot}
\usepackage{graphicx}
\usepackage{xcolor}
\usepackage[shortlabels]{enumitem}
\usepackage{fancyhdr}
	\fancyhf{} % sets both header and footer to nothing
	\renewcommand{\headrulewidth}{0pt}
    \rfoot{Edgel, \thepage}
    \pagestyle{fancy}
	
%%% define shortcuts for set notation
\newcommand{\Z}{\mathbb{Z}}
\newcommand{\R}{\mathbb{R}}
\newcommand{\Q}{\mathbb{Q}}
\newcommand{\lmt}{\underset{x\rightarrow\infty}{\text{lim }}}
\newcommand{\neglmt}{\underset{x\rightarrow-\infty}{\text{lim }}}
\newcommand{\zerolmt}{\underset{x\rightarrow 0}{\text{lim }}}
\newcommand{\loge}[1]{\text{log}\left(#1\right)}
\newcommand{\usmax}[1]{\underset{#1}{\text{max }}}
\newcommand{\usmin}[1]{\underset{#1}{\text{min }}}
\newcommand{\Mt}{M_{t+1}^t}
\newcommand{\vhat}{\hat{v}}
\newcommand{\olp}{\overline{p}}
\renewcommand{\L}{\mathcal{L}}
\newcommand{\olq}{\overline{q}}
\newcommand{\zinf}{_{t=0}^\infty}
\newcommand{\aneg}{A^{-1}}
\newcommand{\sneg}{s^{-1}}
\newcommand{\olk}{\overline{k}}
\newcommand{\olc}{\overline{c}}
\newcommand{\olr}{\overline{r}}
\newcommand{\olpi}{\overline{\pi}}
\newcommand{\Aneg}{A^{-1}}
\renewcommand{\sneg}{s^{-1}}
\newcommand{\dc}[1]{\Delta c_{#1}}
\newcommand{\N}{\mathcal{N}}
\newcommand{\suminf}{\sum_{t=0}^\infty}
\newcommand{\sumn}{\sum_{i=1}^{n}}
\newcommand{\sumnk}{\sum_{i=1}^{N_k}}
\newcommand{\red}[1]{{\color{red}#1}}
\newcommand{\Tau}{\mathrm{T}}
\newcommand{\phat}{\hat{p}}
\newcommand{\qs}{q^*}
\newcommand{\pl}{\partial}

\newcommand{\E}[1]{\mathbb{E}\left[#1\right]} % expected value
\newcommand{\Et}[1]{\mathbb{E}_t\left[#1\right]}

%%% define column vector command (from Michael Nattinger)
\newcount\colveccount
\newcommand*\colvec[1]{
        \global\colveccount#1
        \begin{pmatrix}
        \colvecnext
}
\def\colvecnext#1{
        #1
        \global\advance\colveccount-1
        \ifnum\colveccount>0
                \\
                \expandafter\colvecnext
        \else
                \end{pmatrix}
        \fi
}

%%% define function for drawing matrix augmentation lines
\newcommand\aug{\fboxsep=-\fboxrule\!\!\!\fbox{\strut}\!\!\!}

\makeatletter
\let\amsmath@bigm\bigm

\renewcommand{\bigm}[1]{%
  \ifcsname fenced@\string#1\endcsname
    \expandafter\@firstoftwo
  \else
    \expandafter\@secondoftwo
  \fi
  {\expandafter\amsmath@bigm\csname fenced@\string#1\endcsname}%
  {\amsmath@bigm#1}%
}


%________________________________________________________________%

\begin{document}

\title{	Problem Set \#1 }
\author{ 		Danny Edgel 						\\ 
			Econ 761: Industrial Organization Theory	\\
			Fall 2021						\\
		}
\maketitle\thispagestyle{empty}


%%%________________________________________________________________%%%

\begin{enumerate}
	\item Consider a market in which the goods are homogenous.
  \begin{enumerate}[(a)]
    \item The elasticity of demand, $\varepsilon<0$, can be written as: 
              \[ \varepsilon = \left(P'(Q)\right)^{-1}\frac{P(Q)}{Q} \]
      Thus, letting $\varepsilon$ remain constant, we can derive:
      \begin{align*}
        P(Q)	&= QP'(Q)\varepsilon				\\
        P'(Q)	&= \left(QP''(Q) + P'(Q)\right)\varepsilon
      \end{align*}
      Thus, with $P'(Q)<0$ and $\varepsilon<0$, ${QP''(Q) + P'(Q)>0}$ for all $Q$.

    \item Under Cournot competition, each firm, $i$, solves the following problem:
    \[
      \usmax{q_i}\Pi_i = P(Q)q_i - c(q_i),\quad\quad Q = \sum_{j=1}^N q_j
    \]
    Which yields the following FOC, which is identical for all firms:
    \[
      P(Q) + P'(Q)q_i = c'(q_i)\Rightarrow q_i = \left(P'(Q)\right)^{-1}\left(c'(q_i) - P(Q)\right)
    \]
    Since cost functions are identical by assumption, $q_i=q_j=q$ $\forall i,j$ in equilibrium, so we use the implicit function theorem to solve:\footnote{Due to algebraic errors, I had to redo this several times, spending a long time on it. As a result, many intermediate steps are omitted below.}
    \begin{align*}
        c'(q) - qP'(Nq) &= P(Nq)    \\
        \frac{\pl q}{\pl N}\left[c''(q)-P'(Nq)\right] &= \left[P'(Nq) + qP''(Nq)\right]\left(q + N\frac{\pl q}{\pl N}\right)   \\
        \frac{\pl q}{\pl N}\left[1-N\frac{P'(Nq) + qP''(Nq)}{c''(q)-P'(Nq)}\right] &= q\left[\frac{P'(Nq) + qP''(Nq)}{c''(q)-P'(Nq)}\right]
    \end{align*}
    By assumption (A1), we know $c''(q)-P'(Nq)\geq 0$, and by assumption (A2), we know ${P'(Nq)+qP''(Nq)\leq 0}$. Thus, by the equation above, $\frac{\pl q}{\pl N}\leq 0$. \medskip \\
    To see that the market price is also decreasing in $N$, we can take a straight-forward derivative without appealing to the implicit function theorem. Consider each function in terms of $Q$, such that ${q=Q/N}$:\begin{align*}
      P(Q)                    &= c'(Q/N) - \frac{Q}{N}P'(Q)                                             \\
      \frac{\pl P(Q)}{\pl N}  &= c''(Q/N)\left(-\frac{Q}{N^2}\right) + \left(\frac{Q}{N^2}\right)P'(Q)  \\
                              &= \frac{Q}{N^2}\left[P'(Q) - c''(q)\right]                               \\
                              &\leq 0\text{ by assumption (A1)}
    \end{align*}
  \end{enumerate}

  \item  
  \begin{enumerate}
    \item Each player, $i\in\{1,2\}$ chooses $b_i\in\R_+$ to maximize: \[
      \pi_i(b_i,b_j) = \begin{cases}
        V-b_i,&b_i>b_j  \\
        \frac{1}{2}\left(V-b_i\right),&b_i=b_j \\
        0,&b_i<b_j
      \end{cases}
    \]
    Since payoffs and valuations are symmetric, $b_i=b_j$ in equilibrium. For all $b_i=b_j<V$, each player has an incentive to raise their bid. Thus, the unique equilibrium is:
    \begin{align*}
      &b_1^*=b_2^*=v & \pi_1^*=\pi_2^*=0
    \end{align*}

    \item In all all-pay auction, player 1's payoff function is: 
    \[
      \pi_1(b_1,b_2) = \begin{cases}
        V-b_1,&b_1>b_2  \\
        \frac{1}{2}V-b_1,&b_1=b_2 \\
        -b_1,&b_1<b_2
      \end{cases}
    \]

    \item suppose $\exists$ a pure-strategy equilibrium with bids $(b_1^*,b_2^*)$. Since Payoffs and valuations are identical, any pure strategy equilibrium has ${b_1^*=b_2^*=b^*}$. Then, ${\pi^*=\frac{1}{2}V-b}$. Thus, either player could improve their payoff by deviating to ${b_i = b^*+\varepsilon}$ for $\varepsilon>0$. Thus, $b_1=b_2$ cannot be a pure-strategy Nash equilibrium.\footnote{The nonexistence of a $b_1\neq b_2$ equilibrium is trivial.}

    \item A mixed-strategy Nash equilibrium is a pair of distribution functions, $(F_1(b), F_2(b))$, from which each player draws their bid. Since bids must be weakly positive, $F_i(0)=0$. Since payoffs are negative for all ${b>V}$ but zero for a bid of zero, $F_i(V)=1$. Each player $i$ chooses $F_i$ to maximizes expected payoff:\footnote{Using the same logic as in (c), we can rule out any mass points, since such mass points will exist in both players' distributions, and either player could improve their payoffs by shifting the mass to a slightly higher bid.} \[
      \E{\pi_i(b_i,b_j)} = F_j(b_i)V-b_i
    \]
    From the first-order condition of this problem, we can obtain the symmetric equilibrium distribution function:
    \begin{align*}
      Vf_j(b_i)  - 1  &= 0                      \\
      f_j(b_i)        &= \frac{1}{V}            \\
      F_j(b)          &= \int_0^b\frac{1}{V}dx =\frac{b}{V} 
    \end{align*}
    Since payoffs and valuations are constant, ${F_i^*(b)=F_j^*(b)=F^*(b)}$. Thus, the mixed-strategy equilibrium is for each player to submit a uniformly random bid between $0$ and $V$. The seller's expected revenue is: \[
      R = 2\E{b^*} = 2\int_0^V\left(\frac{1}{V}\right)bdb 
        = \frac{1}{V}[b^2]_0^V = V
    \]

    \item If the seller sets some reserve price $R\in(0,V)$, then the lower bound of the equilibrium distribution will be truncated such that $F^*(b)$ is instead be a uniform distribution from $R$ to $V$. Intuitively, this would increase the seller's revenue by increasing the mean of the equilibrium bid distribution.
  \end{enumerate}
  \pagebreak
  \item 
  \begin{enumerate}
    \item The marginal consumers on either side of Esquires are indifferent to purchasing a cup of coffee from Starbucks and Esquires. Letting $p_i$ represent the price from the nearest Starbucks for $i\in\{0,1\}$ and $x_i\in[0,1]$ represent the location of the consumer on Main Street, where $i=1$ is the consumer closer to the Starbucks on the end of main street: \begin{align*}
      v - x_0^2 - p_0     &= v - (.5 - x_0)^2 - q \\
      v - (1-x_1)^2 - p_1 &= v - (x_1 - .5)^2 - q
    \end{align*}
    Solving for $x_i$ yields:
    \begin{align*}
      &x_0 = q-p_0+\frac{1}{4}    &x_1 = p_1 - q + \frac{3}{4}
    \end{align*}

    \item Assuming Starbucks can set different prices at each location, the firms' optimization problems are:
    \begin{align*}
      q(p)  &= \text{arg}\usmax{q}q\left[p_1 - q + \frac{3}{4}-\left(q-p_0+\frac{1}{4}\right)\right]  \\
            &= \frac{1}{4}\left(p_0 + p_1\right) + \frac{1}{8}  \\
      p(q)  &= \text{arg}\usmax{p_0,p_1}p_0\left[q-p_0+\frac{1}{4}\right] + p_1\left[1-\left(p_1 - q + \frac{3}{4}\right)\right]  \\
            &= \colvec{2}{\frac{1}{2}q + \frac{1}{8}}{\frac{1}{2}q + \frac{1}{8}}
    \end{align*}

    \item Since $p_0=p_1$ in equilibrium, Esquires's best response function can be simplified as ${\frac{1}{2}p + \frac{1}{8}}$. Then, we can solve for the equilibrium as follows: 
    \begin{align*}
      q   &= q(p(q)) = \frac{1}{2}\left(\frac{1}{2}q + \frac{1}{8}\right) + \frac{1}{8}  \\
      \Rightarrow q^* &= \frac{1}{4}  \\
      p^* &= p(q^*) = \frac{1}{2}\left(\frac{1}{8}\right) + \frac{1}{8} = \frac{1}{4}
    \end{align*}
    Given these equilibrium prices, we can solve for market shares using the equations derived in (a) for the marginal consumer on either side of Esquires:
    \begin{align*}
      &x_0 = \frac{1}{4}  &x_1 = \frac{3}{4}
    \end{align*}
    Thus, the middle half of the distribution buys from Esquires, while the ends buy from Starbucks. Starbucks and Esquires, then, each take half of the market.
    
    \item Assume that the Starbucks at the end of the street swaps with Esquires. Then, the best response functions are now:
    \begin{align*}
      q(p)  &= \text{arg}\usmax{q} q\left(q-p_1 + \frac{3}{4}\right) = \frac{1}{2}p_1 + \frac{3}{8}  \\
      p(q)  &= \text{arg}\usmax{p_0,p_1}p_0\left[p_1-p_0+\frac{1}{4}\right] + p_1\left[q-p_1 + \frac{3}{4}-\left(p_1 - p_0 + \frac{1}{4}\right)\right] \\
      \begin{pmatrix} 1 & -1 \\ -1 & 2 \end{pmatrix}p  &= \colvec{2}{1/8}{\frac{1}{2}q + 1/8} 
    \end{align*}
    Plugging the best response function for $q$ into the best response for $p$, we can solve for prices in the new equilibrium: 
    \begin{align*}
      p^* &= \begin{pmatrix} 1 & -1 \\ -1 & 7/4 \end{pmatrix}^{-1}\colvec{2}{1/8}{5/16}
          = \colvec{2}{17/24}{7/12} \approx \colvec{2}{.71}{.58}  \\
      q^* &= \frac{1}{2}\left(\frac{7}{12}\right) + \frac{3}{8} = \frac{2}{3}
    \end{align*}
    Again using the equations for marginal consumers from (a), we can compute market shares:
    \begin{align*} 
      &x_0 = 3/8 \approx 0.37  &x_1 = 5/6 \approx 0.83
    \end{align*}
    Since marginal consumer $i=0$ is indifferent between two Starbucks locations, only $x_1$ is informative for market share purposes.\footnote{Further note that the consumer at $x_0$ still derives positive utility under this price regime, so the market is covered.} In this equilibrium, Starbucks takes five sixths, or about 83\% of the market. \medskip \\
    This equilibrium differs from (c) because in (c), there were no marginal consumers choosing between two Starbucks locations, so Starbucks faced a tradeoff between increasing its price and increasing its market coverage at both of its locations. However, in this equilibrium, the marginal conumer on the left side of the street was indifferent between two Starbucks locations, so Starbucks only faced a tradeoff between losing consumers to its lower-price location (which competes with Esquire's) and losing consumers to Esquire's.

    \item Now suppose that Esquires is still located at the center of the road, but Starbucks sells the location at the end of the road to Seattle's Best, which charges price $z$. The resulting equilibrium is identical to the one in (c) because, as I mentioned in (d), both Starbucks locations in (c) had to price competitively to avoid losing business to Esquire's. The same will be true with Starbucks and Seattle's Best in this equilibrium, resulting in ${p^*=q^*=z^*=1/4}$, with Esquire's taking half of the market and the remaining half split equally between Starbucks and Seattle's Best.\footnote{An interesting addition to this question would be to ask what price Starbucks would charge for this location, given some prevailing interest rate and the assumption of certainty over an infinite time horizon.}
      
  \end{enumerate}

  \item  
  \begin{enumerate}
    \item This game consists of two players, $i\in\{b,d\}$, each of which choose $x_i\in\left[0,1/2\right]$\footnote{In theory, each player's action space runs from 0 to 1, but truncating the space at 1/2 heavily simplifies the notation without loss of generality.} to maximize their payoffs. Each player covers the market between their location and the town closest to it, but they compete over the consumers between their selected locations. The marginal consumer, located at $x$, is indifferent between buying from either player: \[
      (x-a)^2 - p = (1-b-x)^2 - p \Rightarrow x = \frac{(1-b)^2-a^2}{2(1-a-b)} = \frac{1-b+a}{2}
    \]
    Note that this equation is undefined for ${a=b=1/2}$. Thus, payoffs are given by:
      \begin{align*}
        &\pi_b(a,b) = p\left[a + \frac{1-b+a}{2}\right],
        &\pi_d(a,b) = p\left[b + \frac{1-a+b}{2}\right]
      \end{align*}

    \item Each player's payoff function is strictly increasing in their choice variable. Thus, each player will optimize with ${a=b=1/2}$, resulting in Jim Beam and Jack Daniels each locating at the center of the road. This makes intuitive sense, as a location-only game results in each firm simply trying to minimize distance between itself and the highest number of consumers. Locating further from the center of the road would move consumers from being certain buyers to potential marginal buyers.

    \item The Nash equilibrium does \textit{not} minimize total travel costs. We can derive socially optimal locations by solving the social planner's problem:
      \begin{align*} 
          &\usmin{a,b} \int_0^{\frac{1}{2}(1-b+a)}(x-a)^2dx + \int^1_{\frac{1}{2}(1-b+a)}(1-b-x)^2dx  \\
        = &\usmin{a,b} \frac{1}{3}a^3 + \frac{1}{3}b^3 + \frac{1}{12}(1-b-a)^3                        \\
        a: & a^2 - \frac{1}{4}(1-b-a)^2 = 0                                                           \\
        b: & b^2 - \frac{1}{4}(1-b-a)^2 = 0                                                           \\
        &\Rightarrow a^*=b^*=1/4
      \end{align*}
      Thus, the socially-optimal location choice is for each firm to locate equidistant between either town and the center of the road.
  \end{enumerate}
  \pagebreak
  \item 
  \begin{enumerate}
    \item Before solving for the equilibrium, we can reduce the space of unknown variables by recognizing that, in any equilibrium, ${p_2=p_{1R}=p_R}$. This is due to firm 2's payoff function. Let ${X=X(p_R)}$ be the demand for products at the right endpoint at ${p_R=\text{min}\{p_2,p_{1R}\}}$. Then, firm 2's payoff function (taking $X$ as given) is: \[
      \pi_2(p_2,p_{1R};X) = \begin{cases}
        p_2X(p_R) &, p_2\leq p_{1R} \\
        0         &, p_2>p_{1R}
      \end{cases}
    \]
    Thus, firm 2 always has an incentive to set $p_2$ at least as low as $p_{1R}$. Now, we can solve for the equilibrium by calculating each firm's best response function. In order to do so, we must first solve for the marginal consumer at each set of prices, $(p_{1L}, p_R)$. Let $x\in[0,1]$ be the marginal consumer's location on the product space:\[
      1-x - p_{1L} = x - p_R\quad\Rightarrow\quad x= \frac{1}{2}\left(1-p_{1L}+p_R\right)
    \]
    Next, we consider each firm's payoff function. Let us consider firm 1's payoff function without considering firm 2's strategic incentives:\[
      \pi_1(p_{1L}, p_{1R}, p_2) = \begin{cases}
        \frac{p_1}{2}\left(1-p_{1L}+p_R\right) + \frac{p_2}{2}\left(1+p_{1L}-p_R\right) &, p_{1R}<p_2     \\
        \frac{p_1}{2}\left(1-p_{1L}+p_R\right)                                          &, p_{1R}\geq p_2
      \end{cases}
    \]
    We can see that firm 1 only has an incentive to decrease $p_{1R}$ when ${p_{1R}<p_2}$. Thus, taking firm 2's strategic incentives into account, firm 1 optimizes only on $p_{1L}$, which is henceforth denoted as $p_1$. Understanding this, we can use our simplified payoff functions to derive best response functions: 
    \begin{align*}
      p_1(p_2) &= \text{arg}\usmax{p_1}\frac{p_1}{2}\left(1-p_{1L}+p_R\right) = \frac{1}{2}(1 + p_2) \\
      p_2(p_1) &= \text{arg}\usmax{p_2}\frac{p_2}{2}\left(1+p_{1L}-p_R\right) = \frac{1}{2}(1 + p_1) 
    \end{align*}
    Thus, prices and profit in equilibrium are:\begin{align*}
      &p_1^*=p_2^*=1  &\pi_1^*=\pi_2^*=\frac{1}{2}
    \end{align*}

    \item Firm 1 is neither better-off nor worse-off with product $R$. The only way that firm 1 could influence the market with product $R$ is by decreasing the profits of \textit{both} firms by undercutting firm 2's price, leading firm 2 to offer a lower price and taking demand from product $L$ (and/or requiring firm 1 to offer a lower price for $L$). Thus, it makes no difference whether firm 1 keeps or drops product $R$ (assuming neither entry/exit nor capacity costs).

  \end{enumerate}
  \pagebreak
  \item 
  \begin{enumerate}
    \item In a model with two qualities, ${s=1}$ and ${s=2}$, we have three marginal consumers: one consumer who is indifferent between consuming and not consuming, one that is indifferent between $s=1$ and $s=2$. Denote these consumers with $\theta_1$ and $\theta_2$:\begin{align*}
      \theta_1 - p_1 &= 0                  &\Rightarrow \theta_1 &= p_1                               \\
      2(\theta_2 - p_2) &= \theta_2 - p_1  &\Rightarrow \theta_2 &= 2p_2-p_1
    \end{align*}
      
    Consider a monopolist that offers both goods. The monopolist solves:\begin{align*}
      &\usmax{p_1, p_2} (p_2-p_1)(p_1-c) + (1-2p_2+p_1)(p_2-2c)  \\
      &p_1: -2p_1 + 2p_2 - c = 0  \\
      &p_2: 2p_1 - 4p_2 + 3c +1 = 0
    \end{align*}
    Rearranging the first order conditions, we can solve for optimal $p_1$ and $p_2$:\begin{align*}
      \begin{pmatrix}
        -2 & 2 \\ 2 & -4
      \end{pmatrix}\colvec{2}{p_1}{p_2} &= \colvec{2}{c}{-3c-1} \\
      \colvec{2}{p_1}{p_2}              &= \colvec{2}{\frac{1}{2}c + \frac{1}{2}}{c + \frac{1}{2}}
    \end{align*}

    \item If the monopolist instead sold one good with quality ${s=1}$, they would charge:\[
      p=\text{arg}\usmax{p}(1-p)(p-c) = \frac{1}{2}c + \frac{1}{2}
    \]
    Which is the profit-maximizing price for good one when the monopolist sells both product types.

    \item In the monopoly case, the market is covered for ${[\frac{1}{2}c + \frac{1}{2}, 1]}$.To determine the optimal market coverage, we can solve the social planner's problem, letting $t_i$ represent the minimum $\theta$ for a consumer to be allocated a product with ${s=i}$: \begin{align*}
      &\usmax{t_1, t_2} \int_{t_1}^{t_2}\theta-cd\theta + \int_{t_2}^12\theta-2cd\theta \\
      \equiv &\usmax{t_1, t_2} (t_1 + t_2 - 2)c - \frac{1}{2}(t_1^2 + t_2^2) + 1          \\
      \Rightarrow & t_1^*=t_2^*=c
    \end{align*}
    Thus, in the optimal allocation, the mass of consumers from ${\theta=c}$ to 1 consumes the product with ${s=2}$. In the case where only ${s=1}$ is available, the same mass of consumers consumes the product. In the two-product case, the monopolist's allocation is efficient if ${p_1=p_2=c}$: \[
      \frac{1}{2}(c + 1)  = c + \frac{1}{2}  \Rightarrow c = 0
    \]
    In this case, $p_1=p_2=\frac{1}{2}$. Thus, in the two-product case, ${\nexists c\in[0,1/2]}$ such that the monopolist's allocation is optimal. In the one-product case, we need ${p=c}$:\[
      \frac{1}{2}(c+1) = c  \Rightarrow c=1
    \]
    Since $c\leq1/2$, there is also no value of $c$ for which the monopolist's allocation is optimal.

    \item Since demand is exclusively determined at the extensive margin, demand for good 1, given $p_2$, is simply the mass of consumers for whom the utility of consuming good 1 is positive and exceeds that of consuming good 2:\[
      \theta - p_1 \geq 2\theta - 2p_2  \Rightarrow p_1 \leq \theta \leq 2p_2 - p_1
    \]
    Thus, the demand curve for good 1 is ${Q_1(p_1; p_2) = 2(p_2 - p_1)}$.

    \item The profit function for firm 1 is given by:\[
      \pi_1(p_1, p_2) =  2(p_2-p_1)(p_1 - c)
    \]
    This function is maximized at ${p_1(p_2; c)=\frac{1}{2}(p_2 + c)}$. Note that this value yields positive demand for all values of $p_2$'s domain for $c>0$.

  \end{enumerate}
\end{enumerate}

%%%________________________________________________________________%%%




\end{document}






