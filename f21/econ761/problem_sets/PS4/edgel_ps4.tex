%%% Econ761: IO Theory
%%% Fall 2021
%%% Danny Edgel
%%%
% Due on Canvas Thursday, November 18nd, 11:59pm Central Time
%%%

%%%
%							PREAMBLE
%%%

\documentclass{article}

%%% declare packages
\usepackage{amsmath}
\usepackage{amssymb}
\usepackage{array}
\usepackage{bm}
\usepackage{bbm}
\usepackage{changepage}
\usepackage{centernot}
\usepackage{graphicx}
\usepackage{xcolor}
\usepackage[shortlabels]{enumitem}
\usepackage{fancyhdr}
	\fancyhf{} % sets both header and footer to nothing
	\renewcommand{\headrulewidth}{0pt}
    \rfoot{Edgel, \thepage}
    \pagestyle{fancy}
	
%%% define shortcuts for set notation
\newcommand{\Z}{\mathbb{Z}}
\newcommand{\R}{\mathbb{R}}
\newcommand{\Q}{\mathbb{Q}}
\newcommand{\lmt}{\underset{x\rightarrow\infty}{\text{lim }}}
\newcommand{\neglmt}{\underset{x\rightarrow-\infty}{\text{lim }}}
\newcommand{\zerolmt}{\underset{x\rightarrow 0}{\text{lim }}}
\newcommand{\loge}[1]{\text{log}\left(#1\right)}
\newcommand{\usmax}[1]{\underset{#1}{\text{max }}}
\newcommand{\usmin}[1]{\underset{#1}{\text{min }}}
\newcommand{\Mt}{M_{t+1}^t}
\newcommand{\vhat}{\hat{v}}
\newcommand{\olp}{\overline{p}}
\renewcommand{\L}{\mathcal{L}}
\newcommand{\olq}{\overline{q}}
\newcommand{\zinf}{_{t=0}^\infty}
\newcommand{\aneg}{A^{-1}}
\newcommand{\sneg}{s^{-1}}
\newcommand{\olk}{\overline{k}}
\newcommand{\one}[1]{\mathbbm{1}\left\{#1\right\}}
\newcommand{\olc}{\overline{c}}
\newcommand{\olr}{\overline{r}}
\newcommand{\olpi}{\overline{\pi}}
\newcommand{\Aneg}{A^{-1}}
\renewcommand{\sneg}{s^{-1}}
\newcommand{\dc}[1]{\Delta c_{#1}}
\newcommand{\N}{\mathcal{N}}
\newcommand{\suminf}{\sum_{t=0}^\infty}
\newcommand{\sumn}{\sum_{i=1}^{n}}
\newcommand{\sumnk}{\sum_{i=1}^{N_k}}
\newcommand{\red}[1]{{\color{red}#1}}
\newcommand{\Tau}{\mathrm{T}}
\newcommand{\phat}{\hat{p}}
\newcommand{\bhat}{\hat{\beta}}
\newcommand{\ahat}{\hat{\alpha}}
\newcommand{\qs}{q^*}
\newcommand{\pl}{\partial}

\newcommand{\E}[1]{\mathbb{E}\left[#1\right]} % expected value
\newcommand{\Et}[1]{\mathbb{E}_t\left[#1\right]}

%%% define column vector command (from Michael Nattinger)
\newcount\colveccount
\newcommand*\colvec[1]{
        \global\colveccount#1
        \begin{pmatrix}
        \colvecnext
}
\def\colvecnext#1{
        #1
        \global\advance\colveccount-1
        \ifnum\colveccount>0
                \\
                \expandafter\colvecnext
        \else
                \end{pmatrix}
        \fi
}

%%% define function for drawing matrix augmentation lines
\newcommand\aug{\fboxsep=-\fboxrule\!\!\!\fbox{\strut}\!\!\!}

\makeatletter
\let\amsmath@bigm\bigm

\renewcommand{\bigm}[1]{%
  \ifcsname fenced@\string#1\endcsname
    \expandafter\@firstoftwo
  \else
    \expandafter\@secondoftwo
  \fi
  {\expandafter\amsmath@bigm\csname fenced@\string#1\endcsname}%
  {\amsmath@bigm#1}%
}


%________________________________________________________________%

\begin{document}

\title{	Problem Set \#4 }
\author{ 	Danny Edgel 					        	      \\ 
			Econ 761: Industrial Organization Theory	\\
			Fall 2021						                      \\
		}
\maketitle\thispagestyle{empty}


%%%________________________________________________________________%%%

\begin{enumerate}
    \item Table 1 below displays the results of the probit estimations, which are conducted in edgel\_ps4.do.
        \begin{center}
            \textbf{Table 1} \\ \medskip
            \begin{tabular}{r|cccc}
 & (1) & (2) & (3) & (4) \\ 
& OLS & OLS & IV & IV \\\hline 
$\alpha$ & -30.036 & -27.988  & -30.071  & -39.603 \\ 
& (0.216) & (0.918) & (0.216) & (0.743) \\ 
 &&&& \\ 
FE? & & X & & X \\ 
 &&&& \\ 
$R^2$ & -0.32 & 0.46 & -0.32 & 0.39 \\
N & 2256 & 2256 & 2256 & 2256 \\\hline 
\end{tabular} \\ \smallskip
            \footnotesize{Standard errors in parentheses; both specifications include an unreported intercept.}
        \end{center}
    The biggest weakness of this model is that it does not account for the timing of entry between each firm and the fact that each firm is choosing to enter strategically, based on whether and when the other firm enters. Specifically, these decisions are not modeled as simultaneous equations, even though they fundamentally are. Furthermore, this omits key profit determinants, such as household income and the number of small firms prior to entry.\footnote{The author's data do include such variables, but without a common identifier betwwen data78.out and XMat.out, we cannot merge the two files.}
\end{enumerate}

\end{document}






