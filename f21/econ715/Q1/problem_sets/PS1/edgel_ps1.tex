%%% Econ715: Econometric Methods
%%% Spring 2021
%%% Danny Edgel
%%%
% Due on Canvas Tuesday, October 19th, 11:59pm Central Time
%%%

%%%
%							PREAMBLE
%%%

\documentclass{article}

%%% declare packages
\usepackage{amsmath}
\usepackage{amssymb}
\usepackage{array}
\usepackage{bm}
\usepackage{bbm}
\usepackage{changepage}
\usepackage{centernot}
\usepackage{color}
\usepackage{courier}
\usepackage{graphicx}
\usepackage{listings}
\usepackage[shortlabels]{enumitem}
\usepackage{boondox-cal}
\usepackage{fancyhdr}
	\fancyhf{} % sets both header and footer to nothing
	\renewcommand{\headrulewidth}{0pt}
    \rfoot{Edgel, \thepage}
    \pagestyle{fancy}
	
%%% define shortcuts for set notation
\newcommand{\N}{\mathcal{N}}
\newcommand{\Z}{\mathbb{Z}}
\newcommand{\R}{\mathbb{R}}
\newcommand{\Q}{\mathbb{Q}}
\newcommand{\union}{\bigcup}
\newcommand{\intersect}{\bigcap}
\newcommand{\lmt}{\underset{x\rightarrow\infty}{\text{lim }}}
\newcommand{\neglmt}{\underset{n\rightarrow-\infty}{\text{lim }}}
\newcommand{\zerolmt}{\underset{x\rightarrow 0}{\text{lim }}}
\newcommand{\usmax}{\underset{1\leq k \leq n}{\text{max }}}
\newcommand{\usmin}[1]{\underset{#1}{\text{min }}}
\newcommand{\intinf}{\int_{-\infty}^{\infty}}
\newcommand{\olx}[1]{\overline{X}_{#1}}
\newcommand{\oly}[1]{\overline{Y}_{#1}}
\newcommand{\olz}[1]{\overline{Z}_{#1}}
%\newcommand{\est}[1]{\frac{1}{#1}\sum_{i=1}^{#1}}
\newcommand{\est}[1]{\frac{1}{\lowercase{#1}}\sum_{i=1}^{\lowercase{#1}}}
\newcommand{\sumn}{\sum_{i=1}^{n}}
\newcommand{\loge}[1]{\text{log}\left(#1\right)}
\renewcommand{\tilde}[1]{\widetilde{#1}}
\newcommand{\tb}{\tilde{\beta}}
\renewcommand{\Pr}[1]{\text{Pr}\left(#1\right)}
\newcommand{\bols}{\hat{\beta}^{OLS}}
\newcommand{\bhat}{\hat{\beta}}
\newcommand{\ahat}{\hat{\alpha}}
\newcommand{\ehat}{\hat{\varepsilon}}
\newcommand{\vols}{\hat{\varepsilon}_{OLS}}
\newcommand{\one}[1]{\mathbbm{1}\left\{#1\right\}}
\newcommand{\tr}[1]{\text{tr}\left(#1\right)}
\newcommand{\pfrac}[2]{\left(\frac{#1}{#2}\right)}
\newcommand{\bcls}{\tilde{\beta}_{CLS}}
\renewcommand{\L}{\mathcal{L}}
\newcommand{\vt}{\tilde{\varepsilon}}
\renewcommand{\Pr}[1]{Pr\left(#1\right)}
\newcommand{\biv}{\bhat^{IV}}
\newcommand{\xbar}{\overline{X}}
\newcommand{\ybar}{\overline{Y}}
\newcommand{\zbar}{\overline{Z}}
\newcommand{\eps}{\varepsilon}
\newcommand{\esti}{\frac{1}{T_i-1}\sum_{t=1}^{T_i}}
\newcommand{\oinv}{\Omega^{-1}}
\newcommand{\olg}{\overline{g}_n}
\newcommand{\e}[1]{\text{exp}\left(#1\right)}
\DeclareRobustCommand{\bbone}{\text{\usefont{U}{bbold}{m}{n}1}}
\newcommand{\that}{\hat{\theta}_n}
\newcommand{\ttilde}{\tilde{\theta}_n}
\newcommand{\ghat}{\hat{\gamma}}

\newcommand{\E}[1]{\mathbb{E}\left[#1\right]}% expected value
\renewcommand{\exp}[1]{\E\left[#1\right]}

\definecolor{mygreen}{RGB}{28,172,0} % color values Red, Green, Blue
\definecolor{mylilas}{RGB}{170,55,241}


%%% define column vector command (from Michael Nattinger)
\newcount\colveccount
\newcommand*\colvec[1]{
        \global\colveccount#1
        \begin{pmatrix}
        \colvecnext
}
\def\colvecnext#1{
        #1
        \global\advance\colveccount-1
        \ifnum\colveccount>0
                \\
                \expandafter\colvecnext
        \else
                \end{pmatrix}
        \fi
}

\makeatletter
\let\amsmath@bigm\bigm

\renewcommand{\bigm}[1]{%
  \ifcsname fenced@\string#1\endcsname
    \expandafter\@firstoftwo
  \else
    \expandafter\@secondoftwo
  \fi
  {\expandafter\amsmath@bigm\csname fenced@\string#1\endcsname}%
  {\amsmath@bigm#1}%
}


%________________________________________________________________%

\begin{document}


\title{	Problem Set \#1 }
\author{ 	Danny Edgel 					\\ 
			Econ 715: Econometrics Methods	\\
			Fall 2021						
		}
\maketitle\thispagestyle{empty}

%%%________________________________________________________________%%%
\section*{Question 1}

\begin{enumerate}[(a)]
    \item 
\end{enumerate}

%%%________________________________________________________________%%%
\section*{Question 2}

\begin{enumerate}[(a)]
    \item Letting $\tilde{\theta}$ be some value between $\theta_0$ and $\hat{\theta}$, the mean-value expansion of the first-order condition of the problem, at $\hat{\theta}$, is:\[
        \frac{\partial\hat{Q}(\that)}{\partial\theta} = \frac{\partial \hat{Q}(\theta_0)}{\partial\theta} + \est{n}\frac{\partial^2g(W_i,\ttilde,\ghat)}{\partial\theta\partial\theta'}(\that - \theta_0)
    \]
    Denote ${B_n = \est{n}\frac{\partial^2g(W_i,\ttilde,\ghat)}{\partial\theta\partial\theta'}}$, where: \[
        B_n\rightarrow_p B_0 = \frac{\partial^2g(W_i,\theta_0,\gamma_0)}{\partial\theta\partial\theta'}
    \]
    Then: \begin{align*}
        &\sqrt{n}\frac{\partial \hat{Q}(\theta_0)}{\partial\theta}  = \frac{1}{\sqrt{n}}\frac{\partial g(W_i, \theta_0, \ghat)}{\partial\theta}\rightarrow_d \N\left(0, \Omega_0\right) \\ &\text{Where }\Omega_0 = \E{\frac{\partial g(W_i,\theta_0,\gamma_0)}{\partial\theta}\frac{\partial g(W_i,\theta_0,\gamma_0)}{\partial\theta'}}
    \end{align*}
    Thus, since the conditions for ULLN are satisfied, {\small \[
        \sqrt{n}\frac{\partial\hat{Q}(\that)}{\partial\theta} = \sqrt{n}\frac{\partial \hat{Q}(\theta_0)}{\partial\theta} + \est{n}\frac{\partial^2g(W_i,\ttilde,\ghat)}{\partial\theta\partial\theta'}\sqrt{n}(\that - \theta_0)
        \rightarrow_d \N\left(0,\Omega_0\right)
    \] }
    Simplifying, this yields: \[
        \sqrt{n}(\that - \theta_0) \rightarrow_d \N\left(0, B_0^{-1}\Omega_0B_0^{-1}\right)
    \]
\end{enumerate}

%%%________________________________________________________________%%%





\end{document}








