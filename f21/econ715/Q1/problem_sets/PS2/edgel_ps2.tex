%%% Econ715: Econometric Methods
%%% Spring 2021
%%% Danny Edgel
%%%
% Due on Canvas Wednesday, November 10th, 11:59pm Central Time
%%%

%%%
%							PREAMBLE
%%%

\documentclass{article}

%%% declare packages
\usepackage{amsmath}
\usepackage{amssymb}
\usepackage{array}
\usepackage{bm}
\usepackage{bbm}
\usepackage{changepage}
\usepackage{centernot}
\usepackage{color}
\usepackage{courier}
\usepackage{graphicx}
\usepackage{listings}
\usepackage[shortlabels]{enumitem}
\usepackage{boondox-cal}
\usepackage{fancyhdr}
	\fancyhf{} % sets both header and footer to nothing
	\renewcommand{\headrulewidth}{0pt}
    \rfoot{Edgel, \thepage}
    \pagestyle{fancy}
	
%%% define shortcuts for set notation
\newcommand{\N}{\mathcal{N}}
\newcommand{\Z}{\mathbb{Z}}
\newcommand{\R}{\mathbb{R}}
\newcommand{\Q}{\mathbb{Q}}
\newcommand{\union}{\bigcup}
\newcommand{\intersect}{\bigcap}
\newcommand{\lmt}{\underset{x\rightarrow\infty}{\text{lim }}}
\newcommand{\neglmt}{\underset{n\rightarrow-\infty}{\text{lim }}}
\newcommand{\zerolmt}{\underset{x\rightarrow 0}{\text{lim }}}
\newcommand{\usmax}{\underset{1\leq k \leq n}{\text{max }}}
\newcommand{\usmin}[1]{\underset{#1}{\text{min }}}
\newcommand{\intinf}{\int_{-\infty}^{\infty}}
\newcommand{\olx}[1]{\overline{X}_{#1}}
\newcommand{\oly}[1]{\overline{Y}_{#1}}
\newcommand{\olz}[1]{\overline{Z}_{#1}}
%\newcommand{\est}[1]{\frac{1}{#1}\sum_{i=1}^{#1}}
\newcommand{\est}[1]{\frac{1}{\lowercase{#1}}\sum_{i=1}^{\lowercase{#1}}}
\newcommand{\sumn}{\sum_{i=1}^{n}}
\newcommand{\loge}[1]{\text{log}\left(#1\right)}
\renewcommand{\tilde}[1]{\widetilde{#1}}
\newcommand{\tb}{\tilde{\beta}}
\renewcommand{\Pr}[1]{\text{Pr}\left(#1\right)}
\newcommand{\bols}{\hat{\beta}^{OLS}}
\newcommand{\bhat}{\hat{\beta}}
\newcommand{\ahat}{\hat{\alpha}}
\newcommand{\ehat}{\hat{\varepsilon}}
\newcommand{\vols}{\hat{\varepsilon}_{OLS}}
\newcommand{\one}[1]{\mathbbm{1}\left\{#1\right\}}
\newcommand{\tr}[1]{\text{tr}\left(#1\right)}
\newcommand{\pfrac}[2]{\left(\frac{#1}{#2}\right)}
\newcommand{\bcls}{\tilde{\beta}_{CLS}}
\renewcommand{\L}{\mathcal{L}}
\newcommand{\vt}{\tilde{\varepsilon}}
\renewcommand{\Pr}[1]{Pr\left(#1\right)}
\newcommand{\biv}{\bhat^{IV}}
\newcommand{\xbar}{\overline{X}}
\newcommand{\ybar}{\overline{Y}}
\newcommand{\zbar}{\overline{Z}}
\newcommand{\eps}{\varepsilon}
\newcommand{\esti}{\frac{1}{T_i-1}\sum_{t=1}^{T_i}}
\newcommand{\oinv}{\Omega^{-1}}
\newcommand{\olg}{\overline{g}_n}
\newcommand{\e}[1]{\text{exp}\left(#1\right)}
\DeclareRobustCommand{\bbone}{\text{\usefont{U}{bbold}{m}{n}1}}
\newcommand{\that}{\hat{\theta}_n}
\newcommand{\tshat}{\hat{\theta}^*_n}
\newcommand{\ttilde}{\tilde{\theta}_n}
\newcommand{\ghat}{\hat{\gamma}_n}
\newcommand{\gtilde}{\tilde{\gamma}_n}
\newcommand{\chat}{\hat{c}}
\newcommand{\Qhat}{\hat{Q}_n(\beta)}
\renewcommand{\lim}[1]{\underset{#1}{\text{lim }}}
\newcommand{\xs}{X^*}
\newcommand{\olxs}{\overline{X}^*}
\newcommand{\pinv}{\Phi^{-1}}
\newcommand{\tchat}{\that^\dagger}

\newcommand{\E}[1]{\mathbb{E}\left[#1\right]}% expected value
\newcommand{\Es}[1]{\mathbb{E}^*\left[#1\right]}% expected value
\renewcommand{\exp}[1]{\E\left[#1\right]}

\definecolor{mygreen}{RGB}{28,172,0} % color values Red, Green, Blue
\definecolor{mylilas}{RGB}{170,55,241}


%%% define column vector command (from Michael Nattinger)
\newcount\colveccount
\newcommand*\colvec[1]{
        \global\colveccount#1
        \begin{pmatrix}
        \colvecnext
}
\def\colvecnext#1{
        #1
        \global\advance\colveccount-1
        \ifnum\colveccount>0
                \\
                \expandafter\colvecnext
        \else
                \end{pmatrix}
        \fi
}
\newcount\rowveccount
\newcommand*\rowvec[1]{
        \global\rowveccount#1
        \begin{pmatrix}
        \rowvecnext
}
\def\rowvecnext#1{
        #1
        \global\advance\rowveccount-1
        \ifnum\rowveccount>0
                &
                \expandafter\rowvecnext
        \else
                \end{pmatrix}
        \fi
}

\makeatletter
\let\amsmath@bigm\bigm

\renewcommand{\bigm}[1]{%
  \ifcsname fenced@\string#1\endcsname
    \expandafter\@firstoftwo
  \else
    \expandafter\@secondoftwo
  \fi
  {\expandafter\amsmath@bigm\csname fenced@\string#1\endcsname}%
  {\amsmath@bigm#1}%
}


%________________________________________________________________%

\begin{document}


\title{	Problem Set \#2 }
\author{ 	Danny Edgel 					\\ 
			Econ 715: Econometric Methods	\\
			Fall 2021						
		}
\maketitle\thispagestyle{empty}

%%%________________________________________________________________%%%
\section*{Question 1}

\begin{enumerate}[(a)]
    \item To determine ${\Pr{\sumn(\xs_i-\olxs_i)^2=0|\{W_i\}}}$, we need only find \linebreak $\Pr{\sumn\xs_i-\olxs_i=0|\{W_i\}}$. Since ${X_i\in\{0, 1\}}$, ${X_i^*\in\{0, 1\}}$, so:\[
        \Pr{\xs_i-\olxs_i=0|\{W_i\}} = \Pr{\xs_i=\xs_j \forall i,j|\{W_i\}}
    \] Where ${Pr(X^*_i=X_j^*)\geq\frac{1}{2}}$ ${\forall i,j}$. Then, \[
        \Pr{\xs_i=\xs_j \forall i,j|\{W_i\}} = \left(\frac{1}{2}\right)^n
    \]
    Where, for finite $n$, $2^{-n}>0$.

    \item If the conditions for the Lindeberg CLT are satisfied, then $\tshat$ converges to the same distribution as $\that$. Thus, we must show that\[
        \text{sup}\est{n}|X_i(s)|^{2+\delta}<\infty
    \]
    For some $\delta>0$. By the domain of $X$, this is necessarily satisfied.

    \item Since $\pinv(\cdot)$ is a constant,\begin{align*}
        \frac{F^{-1}_{\sqrt{n}(\tshat-\that)|\left\{W_i\right\}}\left(0.75|\{W_i\}\right)-F^{-1}_{\sqrt{n}(\tshat-\that)|\left\{W_i\right\}}\left(0.25|\{W_i\}\right)}{\pinv(0.75) - \pinv(0.25)} \rightarrow_p \\
        \frac{F^{-1}_{\N(0,\sigma_u^2/\sigma_X^2)}\left(0.75\right)-F^{-1}_{\N(0,\sigma_u^2/\sigma_X^2)}\left(0.25\right)}{\pinv(0.75) - \pinv(0.25)}
    \end{align*}
    Thus, \begin{align*}
        v_n^{iqr} \rightarrow_{a.s.} \frac{\frac{\sigma_u}{\sigma_X}\left(\pinv(0.75)-\pinv(0.25)\right)}{\pinv(0.75)-\pinv(0.25)} = \sigma_u/\sigma_X
    \end{align*}

    \item First, we can solve for ${Var(\tchat|\{W_i\})}$:\begin{align*}
        \E{\tchat|\{W_i\}} &= \that + \frac{\sumn(X_i - \olx{n})\hat{u}_i}{\sumn(X_i - \olx{n})^2}\E{\eps_i|\{W_i\}} \\
                    &= \that + \frac{\sumn(X_i - \olx{n})\hat{u}_i}{\sumn(X_i - \olx{n})^2}  \\
        \E{\tchat|\{W_i\}}^2 &=  \that^2 + 2\that\frac{\sumn(X_i - \olx{n})\hat{u}_i}{\sumn(X_i - \olx{n})^2} + \left(\frac{\sumn(X_i - \olx{n})\hat{u}_i}{\sumn(X_i - \olx{n})^2}\right)^2    \\
        (\tchat)^2 &= \that^2 + 2\that\frac{\sumn(X_i - \olx{n})\hat{u}_i\eps_i}{\sumn(X_i - \olx{n})^2} + \left(\frac{\sumn(X_i - \olx{n})\hat{u}_i\eps_i}{\sumn(X_i - \olx{n})^2}\right)^2   \\
        Var(\tchat|\{W_i\}) &= \E{(\tchat)^2|\{W_i\}} - \E{\tchat|\{W_i\}}^2 \\
            &= \left(\frac{\sumn(X_i - \olx{n})\hat{u}_i}{\sumn(X_i - \olx{n})^2}\right)^2\E{\eps_i^2|\{W_i\}}    \\
            &= \left(\frac{\sumn(X_i - \olx{n})\hat{u}_i}{\sumn(X_i - \olx{n})^2}\right)^2
    \end{align*}
    By the law of large numbers, this converges to its unconditional expectation. By assumption,\[
        \left(\frac{\sumn(X_i - \olx{n})\hat{u}_i}{\sumn(X_i - \olx{n})^2}\right)^2 \rightarrow_{a.s.} \frac{\E{(X-\E{X})^2}}{\E{(X-\E{X})^2}^2}\E{u^2} = \sigma_u^2/\sigma_X^2
    \]

    \item The results of the simulation are below. See edgel\_ps2.jl and functions.jl for the code that generates them.
        \begin{center}
            \begin{tabular}{r|cccc}
 & (1) & (2) & (3) & (4) \\ 
& OLS & OLS & IV & IV \\\hline 
$\alpha$ & -30.036 & -27.988  & -30.071  & -39.603 \\ 
& (0.216) & (0.918) & (0.216) & (0.743) \\ 
 &&&& \\ 
FE? & & X & & X \\ 
 &&&& \\ 
$R^2$ & -0.32 & 0.46 & -0.32 & 0.39 \\
N & 2256 & 2256 & 2256 & 2256 \\\hline 
\end{tabular}
        \end{center}
\end{enumerate}
\pagebreak
%%%________________________________________________________________%%%
\section*{Question 2}

\begin{enumerate}[(a)]
    \item The inequalities for this data generating process are:\begin{align*}
        [g(0, 0) - g(1, 0)](-1)         &\geq 0 \\
        [g(0, 0) - g(1, 1)](-1 -\theta) &\geq 0 \\
        [g(0, 0) - g(0, 1)](- \theta)   &\geq 0 \\
        [g(1, 0) - g(0, 0)]             &\geq 0 \\
        [g(1, 0) - g(0, 1)](1 - \theta) &\geq 0 \\
        [g(1, 0) - g(1, 1)](- \theta)   &\geq 0 \\
        [g(1, 1) - g(1, 0)](\theta)     &\geq 0 \\
        [g(1, 1) - g(0, 0)](1 + \theta) &\geq 0 \\
        [g(1, 1) - g(0, 1)]             &\geq 0 \\
        [g(0, 1) - g(0, 0)](\theta)     &\geq 0 \\
        [g(0, 1) - g(1, 1)](-1)         &\geq 0 
    \end{align*}
    Note that the trivial inequalities ($0\geq 0$) are excluded. The unique set of inequalities, letting $g(x_1, x_2) = F(x_1 + \theta x_2)$, is:\begin{align}
        [F(\theta) - F(0)]\theta            &\geq 0 \\
        [F(1) - F(0)]                       &\geq 0 \\
        [F(1+\theta) - F(0)](1 + \theta)    &\geq 0 \\
        [F(1) - F(\theta)](1 - \theta)      &\geq 0 \\
        [F(1+\theta) - F(1)]\theta          &\geq 0 \\
        [F(1+\theta) - F(\theta)]           &\geq 0 
    \end{align}
    Note that this set of inequalities holds for all values of $\theta$. 

    \item If $\theta=1$, then\begin{align*}
        g(1, 0) &= g(0, 1) = F(1)  \\
        g(0, 0) &= F(0)            \\
        g(1, 1) &= F(2)
    \end{align*}
    Thus, the only useful moment inequalities for identification yield:\[
        \hat{\Theta}_n = \left\{\theta: \theta\geq0\right\}
    \]

    \item In this case, any variation in $Y$ is due entirely to variation in $\varepsilon$. Then, for each $(x_1, x_2)$, $Var(Y) = Var(\varepsilon)$ and $\E{Y|(x_1, x_2)} = F(x_1 + \theta x_2)$. Thus, the standard error of $\that$ is $\frac{1}{n}\widehat{Var(Y)}$, where $\widehat{Var(Y)}$ is asymptotically normal. Then, letting $c^*_\alpha$ be the critical value of the normal distribution for $\alpha$, the confidence interval for $\alpha$ is:\[
        \left[F_\varepsilon^{-1}\left(\est{n}Y(0,1)\right) - c^*_\alpha\widehat{Var(Y)}, F_\varepsilon^{-1}\left(\est{n}Y(0,1)\right) + c^*_\alpha\widehat{Var(Y)}\right]
    \]
\end{enumerate}

%%%________________________________________________________________%%%





\end{document}








