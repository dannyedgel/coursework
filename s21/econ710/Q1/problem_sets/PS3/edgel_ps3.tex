%%% Econ710: Econometrics
%%% Spring 2020
%%% Danny Edgel
%%%
% Due on Canvas Tuesday, February 16th, 11:59pm Central Time
%%%

%%%
%							PREAMBLE
%%%

\documentclass{article}

%%% declare packages
\usepackage{amsmath}
\usepackage{amssymb}
\usepackage{array}
\usepackage{bm}
\usepackage{bbm}
\usepackage{changepage}
\usepackage{centernot}
\usepackage{color}
\usepackage{courier}
\usepackage{graphicx}
\usepackage{listings}
\usepackage[shortlabels]{enumitem}
\usepackage{boondox-cal}
\usepackage{fancyhdr}
	\fancyhf{} % sets both header and footer to nothing
	\renewcommand{\headrulewidth}{0pt}
    \rfoot{Edgel, \thepage}
    \pagestyle{fancy}
	
%%% define shortcuts for set notation
\newcommand{\N}{\mathcal{N}}
\newcommand{\Z}{\mathbb{Z}}
\newcommand{\R}{\mathbb{R}}
\newcommand{\Q}{\mathbb{Q}}
\newcommand{\union}{\bigcup}
\newcommand{\intersect}{\bigcap}
\newcommand{\lmt}{\underset{x\rightarrow\infty}{\text{lim }}}
\newcommand{\neglmt}{\underset{n\rightarrow-\infty}{\text{lim }}}
\newcommand{\zerolmt}{\underset{x\rightarrow 0}{\text{lim }}}
\newcommand{\usmax}{\underset{1\leq k \leq n}{\text{max }}}
\newcommand{\usmin}[1]{\underset{#1}{\text{min }}}
\newcommand{\intinf}{\int_{-\infty}^{\infty}}
\newcommand{\olx}[1]{\overline{X}_{#1}}
\newcommand{\oly}[1]{\overline{Y}_{#1}}
\newcommand{\olz}[1]{\overline{Z}_{#1}}
\newcommand{\est}[1]{\frac{1}{#1}\sum_{i=1}^{#1}}
\newcommand{\sumn}{\sum_{i=1}^{n}}
\newcommand{\loge}[1]{\text{log}\left(#1\right)}
\renewcommand{\tilde}[1]{\widetilde{#1}}
\newcommand{\tb}{\tilde{\beta}}
\renewcommand{\Pr}[1]{\text{Pr}\left(#1\right)}
\newcommand{\bols}{\hat{\beta}^{OLS}}
\newcommand{\bhat}{\hat{\beta}}
\newcommand{\ahat}{\hat{\alpha}}
\newcommand{\vhat}{\hat{\varepsilon}}
\newcommand{\vols}{\hat{\varepsilon}_{OLS}}
\newcommand{\one}[1]{\mathbbm{1}\left\{#1\right\}}
\newcommand{\tr}[1]{\text{tr}\left(#1\right)}
\newcommand{\pfrac}[2]{\left(\frac{#1}{#2}\right)}
\newcommand{\bcls}{\tilde{\beta}_{CLS}}
\renewcommand{\L}{\mathcal{L}}
\newcommand{\vt}{\tilde{\varepsilon}}
\renewcommand{\Pr}[1]{Pr\left(#1\right)}
\newcommand{\biv}{\bhat^{IV}}
\newcommand{\xbar}{\overline{X}}
\newcommand{\ybar}{\overline{Y}}
\newcommand{\zbar}{\overline{Z}}
\DeclareRobustCommand{\bbone}{\text{\usefont{U}{bbold}{m}{n}1}}

\newcommand{\E}[1]{\mathbb{E}\left[#1\right]}% expected value
\renewcommand{\exp}[1]{\E\left[#1\right]}

\definecolor{mygreen}{RGB}{28,172,0} % color values Red, Green, Blue
\definecolor{mylilas}{RGB}{170,55,241}


%%% define column vector command (from Michael Nattinger)
\newcount\colveccount
\newcommand*\colvec[1]{
        \global\colveccount#1
        \begin{pmatrix}
        \colvecnext
}
\def\colvecnext#1{
        #1
        \global\advance\colveccount-1
        \ifnum\colveccount>0
                \\
                \expandafter\colvecnext
        \else
                \end{pmatrix}
        \fi
}

\makeatletter
\let\amsmath@bigm\bigm

\renewcommand{\bigm}[1]{%
  \ifcsname fenced@\string#1\endcsname
    \expandafter\@firstoftwo
  \else
    \expandafter\@secondoftwo
  \fi
  {\expandafter\amsmath@bigm\csname fenced@\string#1\endcsname}%
  {\amsmath@bigm#1}%
}


%________________________________________________________________%

\begin{document}

\lstset{language=Matlab,%
    %basicstyle=\color{red},
    breaklines=true,%
    morekeywords={matlab2tikz},
    keywordstyle=\color{blue},%
    morekeywords=[2]{1}, keywordstyle=[2]{\color{black}},
    identifierstyle=\color{black},%
    stringstyle=\color{mylilas},
    commentstyle=\color{mygreen},%
    showstringspaces=false,%without this there will be a symbol in the places where there is a space
    numbers=left,%
    numberstyle={\tiny \color{black}},% size of the numbers
    numbersep=9pt, % this defines how far the numbers are from the text
    emph=[1]{for,end,break},emphstyle=[1]\color{red}, %some words to emphasise
    %emph=[2]{word1,word2}, emphstyle=[2]{style},    
}

\title{	Problem Set \#3 }
\author{ 	Danny Edgel 										\\ 
			Econ 710: Economic Statistics and Econometrics II	\\
			Spring 2021											\\
		}
\maketitle\thispagestyle{empty}

%%%________________________________________________________________%%%

\noindent\textit{Collaborated with Sarah Bass, Emily Case, Katherine Kwok, Michael Nattinger, and Alex Von Hafften}

%%%________________________________________________________________%%%

\section*{Question 1}

\begin{enumerate}[(i)]
	\item Before showing this equivalence, let us first observe $\E{ZX'}$ and $\E{ZZ'}$:
	\[
		\E{ZX'} = \begin{pmatrix} \E{Z_1X_1} & \E{X_2'Z_1} \\ \E{X_2X_1} & \E{X_2X_2'} \end{pmatrix}\text{, }
		\E{ZZ'} = \begin{pmatrix} \E{Z_1^2}  & \E{X_2'Z_1} \\ \E{X_2Z_1} & \E{X_2X_2'}  \end{pmatrix}
	\]
	Note that the only difference between these two matrices is the first column. Further, since $X_2$ is assumed non-zero, $\E{X_2X_2'}$ is invertible. Thus, we know that the rank of the last ${k-1}$ columns of $\E{ZX'}$ and $\E{ZZ'}$ is $k-1$ so long as ${\E{X_2'Z_1}\neq0}$. We will show each side of the ``if and only if" condition separately.
		\begin{itemize}
			\item[(a)] If $\E{ZZ'}$ is invertible and ${\pi\neq 0}$, then we know that ${\E{ZZ'}^{-1}y=0}$ iff $y=0$. Since ${\pi_1\neq 0}$, ${\E{ZX_1}\neq0}$. Thus, the first column of ${\E{ZX'}\neq 0}$. Therefore, $\E{ZX'}$ is invertible.
			
			\item[(b)] If $\E{ZX'}$ is invertible, then rank($\E{ZX'}$)=$k$, so ${\pi_1\neq0}$. This implies that $\E{Z_1^2}\neq 0$, so $\E{ZZ'}$ is invertible.
		\end{itemize}
	$\therefore$  $\E{ZX'}$ is invertible $\iff$ $\E{ZZ'}$ is invertible and ${\pi\neq 0}$ $\blacksquare$
	
	\item Given our asymptotic distribution of $\biv$, if we assume homoskedasticity, we have:
		\[
			\Omega = \sigma_U^2\E{ZX'}^{-1}\E{ZZ'}\E{XZ'}^{-1}
		\]
		Using the block inversion formula, we can solve:
		\begin{align*}
			\E{Z'X}^{-1} &= \frac{1}{k}\begin{pmatrix} 1 & -\E{Z_1X_2'}\E{X_2X_2'}^{-1} \\ . & . \end{pmatrix}	\\
			\E{XZ'}^{-1} &= \frac{1}{k}\begin{pmatrix} 1 & . \\ -\E{Z_1X_2}\E{X_2X_2'}^{-1} & . \end{pmatrix} 	\\
			\text{Where } k &= \E{Z_1X_1} - \E{X_2'X_1}\E{X_2X_2'}^{-1}\E{X_2Z_1} = \E{X_1\tilde{Z}_1}
		\end{align*}
		And where neither the second row of $\E{Z'X}^{-1}$ nor the second column of $\E{XZ'}^{-1}$ enter $\Omega_{11}$. Then,
		\begin{align*}
			\Omega_{11} &= \frac{\sigma_U^2}{\E{X_1\tilde{Z}_1}^2}\left(\E{Z_1^2 }- \E{Z_1X_2'}\E{X_2X_2'}^{-1}\E{X_2Z_1}\right)	\\
						&= \frac{\sigma_U^2}{\E{X_1\tilde{Z}_1}^2}\E{\tilde{Z}_1}^2 = \frac{\sigma_U^2\tilde{Z}_1}{\E{\tilde{Z}_1^2}\pi_1^2}
		\end{align*}
	
	\item $(\pi_1,\pi_2)'$ is the vector of coefficients from a regression of $X_1$ on $Z$, and $\tilde{Z}_1$ is the residualized value of $Z_1$ from said regression.
	
	\item First, recognize that $\Omega_{11}$ can be rewritten as:
		\[
			\frac{\sigma^2_U\E{\tilde{Z}_1^2}}{\E{\tilde{Z}Z_*}^2}
		\]
		Then, by deploying Cauchy-Schwarz, we can derive:
		\begin{align*}
			\Omega_{11} = 
			\frac{\sigma^2_U\E{\tilde{Z}_1^2}}{\E{\tilde{Z}Z_*}^2} 	&\geq 	\frac{\sigma^2_U\E{\tilde{Z}_1^2}}{\E{\tilde{Z}}^2\E{Z_*}^2}	\\
																	&= 		 \frac{\sigma^2_U}{\E{\E{Z_*}^2}}
		\end{align*}
		We could acheive ${\Omega_{11} = \frac{\sigma^2_U}{\E{\E{Z_*}^2}}}$ if ${\tilde{Z}=\E{Z_*}}$.
		
	\item Let $X_2=1$. Then $\pi_1$ is simply the coefficient from a regression of $X_1$ on $Z_1$ that includes a constant term, so:
		\begin{align*}
													\tilde{Z}_1 = Z_1 - \E{Z_1}		&\Rightarrow \E{\tilde{Z}_1} = Var(Z_1)	\\
																\E{X_1\tilde{Z}_1} 	&= Cov(X_1,\tilde{Z}_1)					\\
			\Omega_{11} = \frac{\sigma_U^2}{\E{X_1\tilde{Z}_1}^2}\E{\tilde{Z}_1}^2 	&\Rightarrow 
				\Omega_{11}											\\
				\therefore \frac{\Omega_{11}}{\sigma_U^2} 							&= \frac{Var(Z_1)}{Cov(X_1,\tilde{Z}_1)^2}	
		\end{align*}
	
\end{enumerate}


%%%________________________________________________________________%%%

\section*{Question 2}

\begin{enumerate}[(i)]
	\item Let us begin by simplifying $\E{h(Z)(Y-X\beta)}$:
		\begin{align*}
			\E{h(Z)(Y-X\beta)} 	&= \E{h(Z)\left(X\beta_1 + U - X\beta\right)}	\\
								&= \E{h(Z)\E{U|Z}} + \E{h(Z)X(\beta_1-\beta)}	\\
								&= (\beta_1 - \beta)\E{h(Z)X}
		\end{align*}
		If $\E{h(Z)X}\neq0$, then ${(\beta_1 - \beta)\E{h(Z)X}=0}$ if and only if ${\beta=\beta_1}$.
	
	
	\item Suppose ${\E{h(Z)X}\neq0}$. Then, we can derive $\bhat^h_1$ using the equality from (i):
		\begin{align*}
				\E{h(Z)(Y-X\beta)} = 0				&\Rightarrow \beta\E{h(Z)X} = \E{h(Z)Y}									\\
				\beta = \frac{\E{h(Z)Y}}{\E{h(Z)X}} &\Rightarrow \bhat^h_1 = \frac{\sum_{i=1}^n h(Z_i)Y_i}{\sumn h(Z_i)X_i}
		\end{align*}
	
	
	\item By the central limit theorem,
		\begin{align*}
										\bhat^h_1	&= 				\frac{\sum_{i=1}^n h(Z_i)Y_i}{\sumn h(Z_i)X_i}								
													= 				\frac{\sum_{i=1}^n h(Z_i)(X_i\beta_1 + U)}{\sumn h(Z_i)X_i}					
													= 				\beta_1 + \frac{\sum_{i=1}^n h(Z_i)U}{\sumn h(Z_i)X_i}						\\
			\sqrt{n}\left(\bhat^h_1-\beta_1\right)	&= 				\frac{\frac{1}{\sqrt{n}}\sum_{i=1}^n h(Z_i)U}{\frac{1}{n}\sumn h(Z_i)X_i}	
													\rightarrow_d 	\frac{\E{h(Z)U}}{\E{h(Z)X}}\N(0,1)																		
													=				\N\left(0,\Omega^h\right)													\\
										\Omega^h 	&= 				\frac{\E{h(Z)^2U^2}}{\E{h(Z)X}^2}
		\end{align*}
	
	
	\item Using the Cauchy-Schwarz inequality, we can show:
		\begin{align*}
			\Omega^h 	&= 		\frac{\E{h(Z)^2U^2}}{\E{h(Z)X}^2} 			= \frac{\E{h(Z)^2\E{U^2|Z}}}{\E{h(Z)\E{X|Z}}^2}	\\
						&\geq 	\E{\frac{h(Z)^2\E{U^2|Z}}{h(Z)^2\E{X|Z}^2}} = \E{\frac{\E{U^2|Z}}{\E{X|Z}^2}}				\\
	\Rightarrow\Omega^h	&\geq 	\E{\frac{\E{X|Z}^2}{\E{U^2|Z}}}^{-1}
		\end{align*}
		This lower bound is achieved by ${h(Z) = \frac{\E{X|Z}}{\E{U^2|Z}}}$:
		\begin{align*}
			\Omega^h 	&= \frac{\E{ \left(\frac{\E{X|Z}}{\E{U^2|Z}}\right)^2U^2}}{\E{ \frac{\E{X|Z}}{\E{U^2|Z}}X}^2} 
							= \frac{\E{\frac{\E{X|Z}^2}{\E{U^2|Z}^2}\E{U^2|Z}}}{\E{\frac{\E{X|Z}}{\E{U^2|Z}}\E{X|Z}}^2} 	\\
						&= \frac{\E{\E{X|Z}^2}}{\E{\frac{\E{X|Z}^2}{\E{U^2|Z}}}^2} 
							= \frac{1}{\E{\frac{\E{X|Z}^2}{\E{U^2|Z}}}} 													\\
						&= \E{\frac{\E{X|Z}^2}{\E{U^2|Z}}}^{-1}
		\end{align*}
	
\end{enumerate}

%%%________________________________________________________________%%%

\section*{Question 3}
The attached Matlab file, \texttt{edgel\_ps3.m}, calculates $\bhat^{2SLS}_1$ and $\hat{V}_\beta$ using formula (12.40) of Professor Hansen's textbook. The results for $\beta_1$ are displayed below, rounded to the nearest thousandth.
	\input{q3.tex}
The code used is provided at the end of this file.

\pagebreak
\lstinputlisting{edgel_ps3.m}


%%%________________________________________________________________%%%





\end{document}








