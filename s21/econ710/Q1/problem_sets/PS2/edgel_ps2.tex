%%% Econ710: Econometrics
%%% Spring 2020
%%% Danny Edgel
%%%
% Due on Canvas Tuesday, February 9nd, 11:59pm Central Time
%%%

%%%
%							PREAMBLE
%%%

\documentclass{article}

%%% declare packages
\usepackage{amsmath}
\usepackage{amssymb}
\usepackage{array}
\usepackage{bm}
\usepackage{bbm}
\usepackage{changepage}
\usepackage{centernot}
\usepackage{graphicx}
\usepackage[shortlabels]{enumitem}
\usepackage{boondox-cal}
\usepackage{fancyhdr}
	\fancyhf{} % sets both header and footer to nothing
	\renewcommand{\headrulewidth}{0pt}
    \rfoot{Edgel, \thepage}
    \pagestyle{fancy}
	
%%% define shortcuts for set notation
\newcommand{\N}{\mathcal{N}}
\newcommand{\Z}{\mathbb{Z}}
\newcommand{\R}{\mathbb{R}}
\newcommand{\Q}{\mathbb{Q}}
\newcommand{\union}{\bigcup}
\newcommand{\intersect}{\bigcap}
\newcommand{\lmt}{\underset{x\rightarrow\infty}{\text{lim }}}
\newcommand{\neglmt}{\underset{n\rightarrow-\infty}{\text{lim }}}
\newcommand{\zerolmt}{\underset{x\rightarrow 0}{\text{lim }}}
\newcommand{\usmax}{\underset{1\leq k \leq n}{\text{max }}}
\newcommand{\usmin}[1]{\underset{#1}{\text{min }}}
\newcommand{\intinf}{\int_{-\infty}^{\infty}}
\newcommand{\olx}[1]{\overline{X}_{#1}}
\newcommand{\oly}[1]{\overline{Y}_{#1}}
\newcommand{\est}[1]{\frac{1}{#1}\sum_{i=1}^{#1}}
\newcommand{\sumn}{\sum_{i=1}^{n}}
\newcommand{\loge}[1]{\text{log}\left(#1\right)}
\renewcommand{\tilde}[1]{\widetilde{#1}}
\newcommand{\tb}{\tilde{\beta}}
\renewcommand{\Pr}[1]{\text{Pr}\left(#1\right)}
\newcommand{\bols}{\hat{\beta}_{OLS}}
\newcommand{\bhat}{\hat{\beta}}
\newcommand{\ahat}{\hat{\alpha}}
\newcommand{\vhat}{\hat{\varepsilon}}
\newcommand{\vols}{\hat{\varepsilon}_{OLS}}
\newcommand{\one}[1]{\mathbbm{1}\left\{#1\right\}}
\newcommand{\tr}[1]{\text{tr}\left(#1\right)}
\newcommand{\pfrac}[2]{\left(\frac{#1}{#2}\right)}
\newcommand{\bcls}{\tilde{\beta}_{CLS}}
\renewcommand{\L}{\mathcal{L}}
\newcommand{\vt}{\tilde{\varepsilon}}
\renewcommand{\Pr}[1]{Pr\left(#1\right)}
\newcommand{\biv}{\bhat^{IV}}
\newcommand{\xbar}{\overline{X}}
\newcommand{\ybar}{\overline{Y}}
\newcommand{\zbar}{\overline{Z}}
\DeclareRobustCommand{\bbone}{\text{\usefont{U}{bbold}{m}{n}1}}

\newcommand{\E}[1]{\mathbb{E}\left[#1\right]}% expected value
\renewcommand{\exp}[1]{\E\left[#1\right]}


%%% define column vector command (from Michael Nattinger)
\newcount\colveccount
\newcommand*\colvec[1]{
        \global\colveccount#1
        \begin{pmatrix}
        \colvecnext
}
\def\colvecnext#1{
        #1
        \global\advance\colveccount-1
        \ifnum\colveccount>0
                \\
                \expandafter\colvecnext
        \else
                \end{pmatrix}
        \fi
}

\makeatletter
\let\amsmath@bigm\bigm

\renewcommand{\bigm}[1]{%
  \ifcsname fenced@\string#1\endcsname
    \expandafter\@firstoftwo
  \else
    \expandafter\@secondoftwo
  \fi
  {\expandafter\amsmath@bigm\csname fenced@\string#1\endcsname}%
  {\amsmath@bigm#1}%
}


%________________________________________________________________%

\begin{document}

\title{	Problem Set \#2 }
\author{ 	Danny Edgel 										\\ 
			Econ 710: Economic Statistics and Econometrics II	\\
			Spring 2021											\\
		}
\maketitle\thispagestyle{empty}

%%%________________________________________________________________%%%

\noindent\textit{Collaborated with Sarah Bass, Emily Case, Michael Nattinger, and Alex Von Hafften}

%%%________________________________________________________________%%%

\section*{Question 1}

\begin{enumerate}[(i)]
	\item Yes, ${\biv_1\rightarrow_p\beta_1}$. By the Weak Law of Large Numbers (WLLN) and the recognition that the law of iterated expectation (LIE) implies ${\E{U}=\E{\E{U|Z}}=2}$,
		\begin{align*}
			\biv_1 \rightarrow_p	  &\frac{\E{(Z-\E{Z})(Y-\E{Y})}}{\E{(Z-\E{Z})(X-\E{X})}}													\\
									= &\frac{\E{(Z-\E{Z})(\beta_0 + X\beta_1 + U-\E{\beta_0 + X\beta_1 + U})}}{\E{(Z-\E{Z})(X-\E{X})}}			\\
									= &\frac{\E{(Z-\E{Z})(\beta_0 + X\beta_1 + U-\beta_0 - \beta_1\E{X} - \E{U})}}{\E{(Z-\E{Z})(X-\E{X})}}		\\
									= &\frac{\E{\beta_1(Z-\E{Z})(X - \E{X}) + (Z-\E{Z})(U - \E{U})}}{\E{(Z-\E{Z})(X-\E{X})}}					\\
									= &\frac{\beta_1\E{(Z-\E{Z})(X - \E{X})} + \E{ZU - U\E{Z} - Z\E{U} + \E{Z}\E{U}}}{\E{(Z-\E{Z})(X-\E{X})}}	\\
									= &\beta_1 + \frac{2\E{Z}-2\E{Z}+2\E{Z}-2\E{Z}}{\E{(Z-\E{Z})(X-\E{X})}}										\\
									= &\beta_1
		\end{align*}
	
	\item Yes, ${\biv_0\rightarrow_p\beta_0}$. Given (i), we can calculate:
		\[
			\biv_0 = \ybar - \xbar\bhat_1 \rightarrow_p \E{Y} - \E{X}\beta_1 = \beta_0
		\]
	
\end{enumerate}


%%%________________________________________________________________%%%
\pagebreak
\section*{Question 2}

\begin{enumerate}[(i)]
	\item $Z$ is a valid instrument if ${Cov(Z,X)\neq0}$, i.e., if ${\pi_1\neq 0}$.
	
	\item We can derive $\gamma_0$, $\gamma_1$ and $\varepsilon$ as functions of the structural parameters by first deriving the reduced form of the model:
		\begin{align*}
			Y &= \beta_0 + \left(\pi_0 + Z\pi_1 + V\right)\beta_1 + U 	\\
			Y &= \beta_0 + \pi_0\beta_1 + Z\pi_1\beta_1 + V\beta_1 + U 	\\
			Y &= \gamma_0 + Z\gamma_1 + \varepsilon
		\end{align*}
		Where:
		\[
			\gamma_0 = \beta_0 + \pi_0\beta_1\text{, }\gamma_1 = \pi_1\beta_1\text{, }\varepsilon = V\beta_1 + U
		\]
	
	\item The IV estimator of $\beta_1$ is $$ \biv_1 = \frac{\widehat{Cov(Z,Y)}}{\widehat{Cov(Z,X)}} $$ And the two OLS estimators, of $\gamma_1$ and $\pi_1$ respectively, are
		\[
			\hat{\pi}_1 = \frac{\sumn(Z_i-\zbar)(X_i-\xbar)}{\sumn(Z_i-\zbar)^2}\text{, }\hat{\gamma}_1 = \frac{\sumn(Z_i-\zbar)(Y_i-\ybar)}{\sumn(Z_i-\zbar)^2}
		\]
		Then the indirect least squares estimator of $\beta_1$ is 
		\[
			\frac{\hat{\gamma}_1}{\hat{\pi}_1} 	= \frac{\sumn(Z_i-\zbar)(Y_i-\ybar)\sumn(Z_i-\zbar)^2}{\sumn(Z_i-\zbar)^2\sumn(Z_i-\zbar)(X_i-\xbar)}
												= \frac{\sumn(Z_i-\zbar)(Y_i-\ybar)}{\sumn(Z_i-\zbar)(X_i-\xbar)} = \biv_1
		\]
	
	\item 
	
	\item 
	
\end{enumerate}

%%%________________________________________________________________%%%

\section*{Question 3}

\begin{enumerate}[(i)]
	\item 
	
	\item 
	
	\item 
	
	\item 
	
	\item 
	
	\item 
\end{enumerate}

%%%________________________________________________________________%%%





\end{document}








