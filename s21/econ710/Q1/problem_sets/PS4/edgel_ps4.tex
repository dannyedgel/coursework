%%% Econ710: Econometrics
%%% Spring 2020
%%% Danny Edgel
%%%
% Due on Canvas Tuesday, February 23rd, 11:59pm Central Time
%%%

%%%
%							PREAMBLE
%%%

\documentclass{article}

%%% declare packages
\usepackage{amsmath}
\usepackage{amssymb}
\usepackage{array}
\usepackage{bm}
\usepackage{bbm}
\usepackage{changepage}
\usepackage{centernot}
\usepackage{color}
\usepackage{courier}
\usepackage{graphicx}
\usepackage{listings}
\usepackage[shortlabels]{enumitem}
\usepackage{boondox-cal}
\usepackage{fancyhdr}
	\fancyhf{} % sets both header and footer to nothing
	\renewcommand{\headrulewidth}{0pt}
    \rfoot{Edgel, \thepage}
    \pagestyle{fancy}
	
%%% define shortcuts for set notation
\newcommand{\N}{\mathcal{N}}
\newcommand{\Z}{\mathbb{Z}}
\newcommand{\R}{\mathbb{R}}
\newcommand{\Q}{\mathbb{Q}}
\newcommand{\union}{\bigcup}
\newcommand{\intersect}{\bigcap}
\newcommand{\lmt}{\underset{x\rightarrow\infty}{\text{lim }}}
\newcommand{\neglmt}{\underset{n\rightarrow-\infty}{\text{lim }}}
\newcommand{\zerolmt}{\underset{x\rightarrow 0}{\text{lim }}}
\newcommand{\usmax}{\underset{1\leq k \leq n}{\text{max }}}
\newcommand{\usmin}[1]{\underset{#1}{\text{min }}}
\newcommand{\intinf}{\int_{-\infty}^{\infty}}
\newcommand{\olx}[1]{\overline{X}_{#1}}
\newcommand{\oly}[1]{\overline{Y}_{#1}}
\newcommand{\olz}[1]{\overline{Z}_{#1}}
\newcommand{\est}[1]{\frac{1}{#1}\sum_{i=1}^{#1}}
\newcommand{\sumn}{\sum_{i=1}^{n}}
\newcommand{\loge}[1]{\text{log}\left(#1\right)}
\renewcommand{\tilde}[1]{\widetilde{#1}}
\newcommand{\tb}{\tilde{\beta}}
\renewcommand{\Pr}[1]{\text{Pr}\left(#1\right)}
\newcommand{\bols}{\hat{\beta}^{OLS}}
\newcommand{\bhat}{\hat{\beta}}
\newcommand{\ahat}{\hat{\alpha}}
\newcommand{\vhat}{\hat{\varepsilon}}
\newcommand{\vols}{\hat{\varepsilon}_{OLS}}
\newcommand{\one}[1]{\mathbbm{1}\left\{#1\right\}}
\newcommand{\tr}[1]{\text{tr}\left(#1\right)}
\newcommand{\pfrac}[2]{\left(\frac{#1}{#2}\right)}
\newcommand{\bcls}{\tilde{\beta}_{CLS}}
\renewcommand{\L}{\mathcal{L}}
\newcommand{\vt}{\tilde{\varepsilon}}
\renewcommand{\Pr}[1]{Pr\left(#1\right)}
\newcommand{\biv}{\bhat^{IV}}
\newcommand{\xbar}{\overline{X}}
\newcommand{\ybar}{\overline{Y}}
\newcommand{\zbar}{\overline{Z}}
\DeclareRobustCommand{\bbone}{\text{\usefont{U}{bbold}{m}{n}1}}

\newcommand{\E}[1]{\mathbb{E}\left[#1\right]}% expected value
\renewcommand{\exp}[1]{\E\left[#1\right]}

\definecolor{mygreen}{RGB}{28,172,0} % color values Red, Green, Blue
\definecolor{mylilas}{RGB}{170,55,241}


%%% define column vector command (from Michael Nattinger)
\newcount\colveccount
\newcommand*\colvec[1]{
        \global\colveccount#1
        \begin{pmatrix}
        \colvecnext
}
\def\colvecnext#1{
        #1
        \global\advance\colveccount-1
        \ifnum\colveccount>0
                \\
                \expandafter\colvecnext
        \else
                \end{pmatrix}
        \fi
}

\makeatletter
\let\amsmath@bigm\bigm

\renewcommand{\bigm}[1]{%
  \ifcsname fenced@\string#1\endcsname
    \expandafter\@firstoftwo
  \else
    \expandafter\@secondoftwo
  \fi
  {\expandafter\amsmath@bigm\csname fenced@\string#1\endcsname}%
  {\amsmath@bigm#1}%
}


%________________________________________________________________%

\begin{document}

\lstset{language=Matlab,%
    %basicstyle=\color{red},
    breaklines=true,%
    morekeywords={matlab2tikz},
    keywordstyle=\color{blue},%
    morekeywords=[2]{1}, keywordstyle=[2]{\color{black}},
    identifierstyle=\color{black},%
    stringstyle=\color{mylilas},
    commentstyle=\color{mygreen},%
    showstringspaces=false,%without this there will be a symbol in the places where there is a space
    numbers=left,%
    numberstyle={\tiny \color{black}},% size of the numbers
    numbersep=9pt, % this defines how far the numbers are from the text
    emph=[1]{for,end,break},emphstyle=[1]\color{red}, %some words to emphasise
    %emph=[2]{word1,word2}, emphstyle=[2]{style},    
}

\title{	Problem Set \#4 }
\author{ 	Danny Edgel 										\\ 
			Econ 710: Economic Statistics and Econometrics II	\\
			Spring 2021											\\
		}
\maketitle\thispagestyle{empty}

%%%________________________________________________________________%%%

\noindent\textit{Collaborated with Sarah Bass, Emily Case, Katherine Kwok, Michael Nattinger, and Alex Von Hafften}

%%%________________________________________________________________%%%

\section*{Question 1}
In order for ${\text{Pr}(\text{Defying})=0}$, $Z$ must be monotonic in $X$. In order for ${\text{Pr}(\text{Complying})>0}$, it must be the case that, in a nonzero number of cases, ${X(Z=1)=1}$ where ${X(Z=0)=0}$. Thus, ${U_1 > 0}$ and ${\text{Pr}(U_1>U_0)>0}$.


%%%________________________________________________________________%%%

\section*{Question 2}

\begin{enumerate}[(i)]
	\item The autocovariance function is defined as:
		\[
			\gamma(k) = Cov(Y_t,Y_{t-k}) = \E{\left(Y_t - \E{Y_t}\right)\left(Y_{t-k} - \E{Y_{t-k}}\right)} = \E{Y_tY_{t-k}} - \E{Y_{t}}\E{Y_{t-k}}
		\]
		Where:
		\begin{align*}
			\E{Y_t} 	&= \mu + \E{\varepsilon_t} + \theta_1\E{\varepsilon_{t-1}} + ... + \theta_q\E{\varepsilon_{t-q}} = \mu = \E{Y_{t-k}}\text{, }\forall k	\\
			Y_tY_{t-k}	&= \mu^2 + \mu\left(\varepsilon_{t-k}+ \theta_1\varepsilon_{t-k-1}+ ... + \theta_q\varepsilon_{t-k-q}\right) 
							+ \varepsilon_t\left(\varepsilon_{t-k}+ \theta_1\varepsilon_{t-k-1}+ ... + \theta_q\varepsilon_{t-k-q}\right) + ...							\\
		\E{Y_tY_{t-k}}	&= \mu^2 + \varepsilon_t^2 + ... + \varepsilon_{t-k}^2			
		\end{align*}
		Thus, letting ${\varepsilon_t^2 = \sigma^2}$ for all $t$ and recognizing that ${\theta_k=0}$ for all ${k<t-q}$,
		\[
			\gamma(k) = \begin{cases} 
							\left(\theta_k + ... + \theta_{q-k}\theta_q\right)\sigma^2, 	& k\leq q \\ 
							0, 																& k>q 
						\end{cases}
		\]
	
	\item If ${q=1}$, then:
		\[
			\gamma(k) =  \begin{cases} 
							\left(1 + \theta_1^2\right)\sigma^2	& k = 0	\\
							\theta_1\sigma^2, 					& k = 1 \\ 
							0, 									& k>1
						\end{cases} \Rightarrow 
			\rho(k) =  \begin{cases} 
							1									& k = 0	\\
							\frac{\theta_1}{1 + \theta_1^2}, 	& k = 1 \\ 
							0, 									& k>1
						\end{cases}
		\]
	
	\item $\theta_1$ is \textit{not} identified from the autocorrelation function, since it only shows up in the $k=1$ case, in which the solution is nonunique in most cases, since any value other than $-1$ or $1$ has the same autocorrelation for its reciprocal.
	
	
	\item In the case where $\theta_1\in[-1,1]$, 
	
\end{enumerate}

%%%________________________________________________________________%%%

\section*{Question 3}

\begin{enumerate}[(i)]
	\item 
	
	
	\item 

	
\end{enumerate}



%%%________________________________________________________________%%%





\end{document}








