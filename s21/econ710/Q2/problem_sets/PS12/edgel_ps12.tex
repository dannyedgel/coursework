%%% Econ710: Econometrics
%%% Spring 2020
%%% Danny Edgel
%%%
% Due on Canvas Tuesday, April 27th, 11:59pm Central Time
%%%

%%%
%							PREAMBLE
%%%

\documentclass{article}

%%% declare packages
\usepackage{amsmath}
\usepackage{amssymb}
\usepackage{array}
\usepackage{bm}
\usepackage{bbm}
\usepackage{changepage}
\usepackage{centernot}
\usepackage{color}
\usepackage{courier}
\usepackage{graphicx}
\usepackage{listings}
\usepackage[shortlabels]{enumitem}
\usepackage{boondox-cal}
\usepackage{fancyhdr}
	\fancyhf{} % sets both header and footer to nothing
	\renewcommand{\headrulewidth}{0pt}
    \rfoot{Edgel, \thepage}
    \pagestyle{fancy}
	
%%% define shortcuts for set notation
\newcommand{\N}{\mathcal{N}}
\newcommand{\Z}{\mathbb{Z}}
\newcommand{\R}{\mathbb{R}}
\newcommand{\Q}{\mathbb{Q}}
\newcommand{\union}{\bigcup}
\newcommand{\intersect}{\bigcap}
\newcommand{\lmt}{\underset{x\rightarrow\infty}{\text{lim }}}
\newcommand{\neglmt}{\underset{n\rightarrow-\infty}{\text{lim }}}
\newcommand{\zerolmt}{\underset{x\rightarrow 0}{\text{lim }}}
\newcommand{\usmax}{\underset{1\leq k \leq n}{\text{max }}}
\newcommand{\usmin}[1]{\underset{#1}{\text{min }}}
\newcommand{\intinf}{\int_{-\infty}^{\infty}}
\newcommand{\olx}[1]{\overline{X}_{#1}}
\newcommand{\oly}[1]{\overline{Y}_{#1}}
\newcommand{\olz}[1]{\overline{Z}_{#1}}
%\newcommand{\est}[1]{\frac{1}{#1}\sum_{i=1}^{#1}}
\newcommand{\est}[1]{\frac{1}{\lowercase{#1}}\sum_{i=1}^{\lowercase{#1}}}
\newcommand{\sumn}{\sum_{i=1}^{n}}
\newcommand{\loge}[1]{\text{log}\left(#1\right)}
\renewcommand{\tilde}[1]{\widetilde{#1}}
\newcommand{\tb}{\tilde{\beta}}
\renewcommand{\Pr}[1]{\text{Pr}\left(#1\right)}
\newcommand{\bols}{\hat{\beta}^{OLS}}
\newcommand{\bhat}{\hat{\beta}}
\newcommand{\ahat}{\hat{\alpha}}
\newcommand{\ehat}{\hat{\varepsilon}}
\newcommand{\vols}{\hat{\varepsilon}_{OLS}}
\newcommand{\one}[1]{\mathbbm{1}\left\{#1\right\}}
\newcommand{\tr}[1]{\text{tr}\left(#1\right)}
\newcommand{\pfrac}[2]{\left(\frac{#1}{#2}\right)}
\newcommand{\bcls}{\tilde{\beta}_{CLS}}
\renewcommand{\L}{\mathcal{L}}
\newcommand{\vt}{\tilde{\varepsilon}}
\renewcommand{\Pr}[1]{Pr\left(#1\right)}
\newcommand{\biv}{\bhat^{IV}}
\newcommand{\xbar}{\overline{X}}
\newcommand{\ybar}{\overline{Y}}
\newcommand{\zbar}{\overline{Z}}
\newcommand{\eps}{\varepsilon}
\newcommand{\esti}{\frac{1}{T_i-1}\sum_{t=1}^{T_i}}
\newcommand{\oinv}{\Omega^{-1}}
\newcommand{\olg}{\overline{g}_n}
\newcommand{\e}[1]{\text{exp}\left(#1\right)}
\DeclareRobustCommand{\bbone}{\text{\usefont{U}{bbold}{m}{n}1}}

\newcommand{\E}[1]{\mathbb{E}\left[#1\right]}% expected value
\renewcommand{\exp}[1]{\E\left[#1\right]}

\definecolor{mygreen}{RGB}{28,172,0} % color values Red, Green, Blue
\definecolor{mylilas}{RGB}{170,55,241}


%%% define column vector command (from Michael Nattinger)
\newcount\colveccount
\newcommand*\colvec[1]{
        \global\colveccount#1
        \begin{pmatrix}
        \colvecnext
}
\def\colvecnext#1{
        #1
        \global\advance\colveccount-1
        \ifnum\colveccount>0
                \\
                \expandafter\colvecnext
        \else
                \end{pmatrix}
        \fi
}

\makeatletter
\let\amsmath@bigm\bigm

\renewcommand{\bigm}[1]{%
  \ifcsname fenced@\string#1\endcsname
    \expandafter\@firstoftwo
  \else
    \expandafter\@secondoftwo
  \fi
  {\expandafter\amsmath@bigm\csname fenced@\string#1\endcsname}%
  {\amsmath@bigm#1}%
}


%________________________________________________________________%

\begin{document}

\lstset{language=Matlab,%
    %basicstyle=\color{red},
    breaklines=true,%
    morekeywords={matlab2tikz},
    keywordstyle=\color{blue},%
    morekeywords=[2]{1}, keywordstyle=[2]{\color{black}},
    identifierstyle=\color{black},%
    stringstyle=\color{mylilas},
    commentstyle=\color{mygreen},%
    showstringspaces=false,%without this there will be a symbol in the places where there is a space
    numbers=left,%
    numberstyle={\tiny \color{black}},% size of the numbers
    numbersep=9pt, % this defines how far the numbers are from the text
    emph=[1]{for,end,break},emphstyle=[1]\color{red}, %some words to emphasise
    %emph=[2]{word1,word2}, emphstyle=[2]{style},    
}

\title{	Problem Set \#12 }
\author{ 	Danny Edgel 										\\ 
			Econ 710: Economic Statistics and Econometrics II	\\
			Spring 2021											\\
		}
\maketitle\thispagestyle{empty}

%%%________________________________________________________________%%%

\noindent\textit{Discussed and/or compared answers with Sarah Bass, Emily Case, Katherine Kwok, Michael Nattinger, and Alex Von Hafften}

% Chapter 27: Exercises 27.1, 27.2, 27.4, 27.8, 27.9.
% Chapter 28: Exercise 28.12. You only need to compare AIC and BIC selection.

%%%________________________________________________________________%%%

\section*{Exercise 27.1}
The latent variable model is:
\[
	Y^* = X'\beta + e
\]
We can derive the conditional means of the censored ($m(X)$) and truncated ($m^{\#}(X)$) variables by assuming that ${e|X\sim\N(0,\sigma^2)}$. Then, for the censored conditional mean,

And for the truncated conditional mean,
\begin{align*}
	m^{\#}(X)	&= \E{Y|X} = X'\beta \E{\one{Y^*>0}|X} = X'\beta \E{\one{e>-X'\beta}|X}	\\
				&= X'\beta\left(1-\Phi\left(-\frac{X'\beta}{\sigma^2}\right)\right)
\end{align*}



%%%________________________________________________________________%%%

\section*{Exercise 27.2}
No, the OLS estimate of $\beta$ is biased downward in this model.

%%%________________________________________________________________%%%

\section*{Exercise 27.4}
An NLLS estimator for the conditional mean of the model in (27.2) is:
\[
	\usmin{\beta,\sigma} \left(Y-X'\beta\Phi\left(\frac{X'\beta}{\sigma}\right) - \sigma\phi\left(\frac{X'\beta}{\sigma}\right)\right)^2
\]


%%%________________________________________________________________%%%

\section*{Exercise 27.8}
The latent variable model for (27.7) is:
\begin{align*}
	Y^* &= X'\beta + e			\\
	S^*	&= Z'\gamma + u
	S 	&= \one{S^*>0}	\\
	Y 	&= 	\begin{cases}
				Y^*,			&S=1	\\
				\text{missing},	&S=0
			\end{cases}
\end{align*}
Assume:
\[
	\colvec{2}{e}{u}\sim\N\left(0,\begin{pmatrix} \sigma^2 & \sigma_{21} \\ \sigma{21} & 1 \end{pmatrix}\right)
\]
Then, 
\begin{align*}
	\E{Y|X,Z,S=1}			&= \E{Y^*|X,Z,S=1} = \E{X'\beta + e|X,Z,S=1} 														\\
							&= X'\beta + \E{e|X,Z,S=1}	= X'\beta + \E{e|X\gamma + u > 0}										\\
	\E{e|X\gamma + u > 0}	&= \E{e|u>-X\gamma} = Cov(e,u)\frac{\phi(Z'\gamma)}{\Phi(Z'\gamma)} = \sigma_{21}\lambda(Z'\gamma)	\\
	\E{Y|X,Z,S=1}			&= X'\beta + \sigma_{21}\lambda(Z'\gamma)
\end{align*}


%%%________________________________________________________________%%%

\section*{Exercise 27.9}

The results of the OLS regression (using robust standard errors) are displayed in the table at the end of this exercise. Absent of misspecification, these results would suggest that the replationship between transfers and income is linear and perfectly symmetric about \$1,000. However, the dependent variable is censored at zero, so this model is misspecified. The share of observations that are censored is \input{27_9b.tex}. I would expect censoring bias to be a problem in this example.

\begin{center}
	\input{table27_9.tex}
\end{center}
%%%________________________________________________________________%%%

\section*{Exercise 28.12}


%%%________________________________________________________________%%%





\end{document}








