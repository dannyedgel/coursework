%%% Econ710: Econometrics
%%% Spring 2020
%%% Danny Edgel
%%%
% Due on Canvas Tuesday, April 13th, 11:59pm Central Time
%%%

%%%
%							PREAMBLE
%%%

\documentclass{article}

%%% declare packages
\usepackage{amsmath}
\usepackage{amssymb}
\usepackage{array}
\usepackage{bm}
\usepackage{bbm}
\usepackage{changepage}
\usepackage{centernot}
\usepackage{color}
\usepackage{courier}
\usepackage{graphicx}
\usepackage{listings}
\usepackage[shortlabels]{enumitem}
\usepackage{boondox-cal}
\usepackage{fancyhdr}
	\fancyhf{} % sets both header and footer to nothing
	\renewcommand{\headrulewidth}{0pt}
    \rfoot{Edgel, \thepage}
    \pagestyle{fancy}
	
%%% define shortcuts for set notation
\newcommand{\N}{\mathcal{N}}
\newcommand{\Z}{\mathbb{Z}}
\newcommand{\R}{\mathbb{R}}
\newcommand{\Q}{\mathbb{Q}}
\newcommand{\union}{\bigcup}
\newcommand{\intersect}{\bigcap}
\newcommand{\lmt}{\underset{x\rightarrow\infty}{\text{lim }}}
\newcommand{\neglmt}{\underset{n\rightarrow-\infty}{\text{lim }}}
\newcommand{\zerolmt}{\underset{x\rightarrow 0}{\text{lim }}}
\newcommand{\usmax}{\underset{1\leq k \leq n}{\text{max }}}
\newcommand{\usmin}[1]{\underset{#1}{\text{min }}}
\newcommand{\intinf}{\int_{-\infty}^{\infty}}
\newcommand{\olx}[1]{\overline{X}_{#1}}
\newcommand{\oly}[1]{\overline{Y}_{#1}}
\newcommand{\olz}[1]{\overline{Z}_{#1}}
%\newcommand{\est}[1]{\frac{1}{#1}\sum_{i=1}^{#1}}
\newcommand{\est}[1]{\frac{1}{\lowercase{#1}}\sum_{i=1}^{\lowercase{#1}}}
\newcommand{\sumn}{\sum_{i=1}^{n}}
\newcommand{\loge}[1]{\text{log}\left(#1\right)}
\renewcommand{\tilde}[1]{\widetilde{#1}}
\newcommand{\tb}{\tilde{\beta}}
\renewcommand{\Pr}[1]{\text{Pr}\left(#1\right)}
\newcommand{\bols}{\hat{\beta}^{OLS}}
\newcommand{\bhat}{\hat{\beta}}
\newcommand{\ahat}{\hat{\alpha}}
\newcommand{\ehat}{\hat{\varepsilon}}
\newcommand{\vols}{\hat{\varepsilon}_{OLS}}
\newcommand{\one}[1]{\mathbbm{1}\left\{#1\right\}}
\newcommand{\tr}[1]{\text{tr}\left(#1\right)}
\newcommand{\pfrac}[2]{\left(\frac{#1}{#2}\right)}
\newcommand{\bcls}{\tilde{\beta}_{CLS}}
\renewcommand{\L}{\mathcal{L}}
\newcommand{\vt}{\tilde{\varepsilon}}
\renewcommand{\Pr}[1]{Pr\left(#1\right)}
\newcommand{\biv}{\bhat^{IV}}
\newcommand{\xbar}{\overline{X}}
\newcommand{\ybar}{\overline{Y}}
\newcommand{\zbar}{\overline{Z}}
\newcommand{\eps}{\varepsilon}
\newcommand{\esti}{\frac{1}{T_i-1}\sum_{t=1}^{T_i}}
\newcommand{\oinv}{\Omega^{-1}}
\newcommand{\olg}{\overline{g}_n}
\DeclareRobustCommand{\bbone}{\text{\usefont{U}{bbold}{m}{n}1}}

\newcommand{\E}[1]{\mathbb{E}\left[#1\right]}% expected value
\renewcommand{\exp}[1]{\E\left[#1\right]}

\definecolor{mygreen}{RGB}{28,172,0} % color values Red, Green, Blue
\definecolor{mylilas}{RGB}{170,55,241}


%%% define column vector command (from Michael Nattinger)
\newcount\colveccount
\newcommand*\colvec[1]{
        \global\colveccount#1
        \begin{pmatrix}
        \colvecnext
}
\def\colvecnext#1{
        #1
        \global\advance\colveccount-1
        \ifnum\colveccount>0
                \\
                \expandafter\colvecnext
        \else
                \end{pmatrix}
        \fi
}

\makeatletter
\let\amsmath@bigm\bigm

\renewcommand{\bigm}[1]{%
  \ifcsname fenced@\string#1\endcsname
    \expandafter\@firstoftwo
  \else
    \expandafter\@secondoftwo
  \fi
  {\expandafter\amsmath@bigm\csname fenced@\string#1\endcsname}%
  {\amsmath@bigm#1}%
}


%________________________________________________________________%

\begin{document}

\lstset{language=Matlab,%
    %basicstyle=\color{red},
    breaklines=true,%
    morekeywords={matlab2tikz},
    keywordstyle=\color{blue},%
    morekeywords=[2]{1}, keywordstyle=[2]{\color{black}},
    identifierstyle=\color{black},%
    stringstyle=\color{mylilas},
    commentstyle=\color{mygreen},%
    showstringspaces=false,%without this there will be a symbol in the places where there is a space
    numbers=left,%
    numberstyle={\tiny \color{black}},% size of the numbers
    numbersep=9pt, % this defines how far the numbers are from the text
    emph=[1]{for,end,break},emphstyle=[1]\color{red}, %some words to emphasise
    %emph=[2]{word1,word2}, emphstyle=[2]{style},    
}

\title{	Problem Set \#10 }
\author{ 	Danny Edgel 										\\ 
			Econ 710: Economic Statistics and Econometrics II	\\
			Spring 2021											\\
		}
\maketitle\thispagestyle{empty}

%%%________________________________________________________________%%%

\noindent\textit{Discussed and/or compared answers with Sarah Bass, Emily Case, Katherine Kwok, Michael Nattinger, and Alex Von Hafften}

% Chapter 22: Exercise 22.1
%
% Chapter 23: Exercises 23.1, 23.2, 23.7, 23.8
%
% Chapter 24: Exercises 24.3, 24.4, 24.5, 24.14

%%%________________________________________________________________%%%

\section*{Exercise 22.1}

\begin{enumerate}[(a)]
	\item The conditional CDF of $Y$ is $Pr(Y\leq y|X=x)$. Given our model, we can solve:
		\[
			Pr(Y\leq y|X=x) = Pr(x'\theta + e\leq y|X=x) = Pr(e\leq y - x'\theta|X=x) = F(y-x'\theta)
		\]
	
	\item Since the distribution of $Y$ is known, we can solve for $\theta$ with MLE. Thus,
		\begin{align*}
			\rho(Y,X,\theta)	&= -\loge{f(Y|X,\theta)} = -\loge{f(y-x'\theta)}	\\
			\phi(Y,X\theta)		&= \frac{\partial}{\partial\theta}\rho(Y,X,\theta) = -\left(\frac{f'(y-x'\theta)}{f(y-x'\theta)}\right)x
		\end{align*}
	
	\item The asymptotic distribution of $\hat{\theta}$ is given by:
		\[
			\sqrt{n}(\hat{\theta}-\theta_0) \rightarrow_d \N\left(0,Q^{-1}\Omega Q^{-1}\right)
		\]
		Where:
		\begin{align*}
					Q		&= \E{\frac{\partial^2}{\partial\theta\partial\theta'}\rho_i(\theta)} = \E{\left(\frac{f'(e_i)^2}{f(e_i)^2}\right)x_ix_i'}		\\
					\Omega 	&= \E{\phi_i\phi_i'} = \E{\left(\frac{\partial}{\partial\theta}\rho_i\right)\left(\frac{\partial}{\partial\theta'}\rho_i'\right)}	
								= \E{\frac{\partial^2}{\partial\theta\partial\theta'}\rho_i} = Q																\\
			\Rightarrow V	&= Q^{-1}\Omega Q^{-1} = Q^{-1} = \E{\left(\frac{f'(e_i)^2}{f(e_i)^2}\right)^{-1}(x_ix_i')^{-1}}
		\end{align*}
	
\end{enumerate}

%%%________________________________________________________________%%%

\section*{Exercise 23.1}

\begin{enumerate}[(a)]
	\item The conditional mean is not linear in $\theta$, since $\E{Y} = \text{exp}(\theta)$. Thus, this is a nonlinear regression model.
	
	\item Yes. We can run OLS on the model as-is, treating $\text{exp}(\theta)$ as our parameter of interest, then transform it post-estimation by taking logs.
	
	\item My answer to part (b) is exactly non-linear least squares, with ${m(\theta)=\text{exp}(\theta)}$.
	
\end{enumerate}

%%%________________________________________________________________%%%

\section*{Exercise 23.2}


%%%________________________________________________________________%%%

\section*{Exercise 23.7}


%%%________________________________________________________________%%%

\section*{Exercise 23.8}


%%%________________________________________________________________%%%

\section*{Exercise 24.3}


%%%________________________________________________________________%%%

\section*{Exercise 24.4}


%%%________________________________________________________________%%%

\section*{Exercise 24.5}


%%%________________________________________________________________%%%

\section*{Exercise 24.14}


%%%________________________________________________________________%%%





\end{document}








