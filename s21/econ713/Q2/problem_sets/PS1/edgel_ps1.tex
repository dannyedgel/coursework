%%% Econ713: Microeconomics II
%%% Spring 2021
%%% Danny Edgel
%%%
% Due on Canvas Thursday, April 1st, 11:59pm Central Time
%%%

%%%
%							PREAMBLE
%%%

\documentclass{article}

%%% declare packages
\usepackage{amsmath}
\usepackage{amssymb}
\usepackage{array}
\usepackage{bm}
\usepackage{changepage}
\usepackage{centernot}
\usepackage{graphicx}
\usepackage{multirow}
\usepackage[shortlabels]{enumitem}
\usepackage{fancyhdr}
	\fancyhf{} % sets both header and footer to nothing
	\renewcommand{\headrulewidth}{0pt}
    \rfoot{Edgel, \thepage}
    \pagestyle{fancy}
	
%%% define shortcuts for set notation
\newcommand{\N}{\mathbb{N}}
\newcommand{\Z}{\mathbb{Z}}
\newcommand{\R}{\mathbb{R}}
\newcommand{\Q}{\mathbb{Q}}
\newcommand{\lmt}{\underset{x\rightarrow\infty}{\text{lim }}}
\newcommand{\neglmt}{\underset{x\rightarrow-\infty}{\text{lim }}}
\newcommand{\zerolmt}{\underset{x\rightarrow 0}{\text{lim }}}
\newcommand{\usmax}[1]{\underset{#1}{\text{max }}}
\newcommand{\usmin}[1]{\underset{#1}{\text{min }}}
\newcommand{\intersect}{\bigcap}
\newcommand{\union}{\bigcup}
\newcommand{\olw}{\overline{w}}
\newcommand{\olx}{\overline{x}}
\newcommand{\loge}[1]{\text{log}\left(#1\right)}
\renewcommand{\P}{\mathcal{P}}
\renewcommand{\L}{\mathcal{L}}
\newcommand{\olp}{\overline{p}}
\renewcommand{\exp}[1]{\text{exp}\left\{#1\right\}}
\newcommand{\binv}[1]{b_j^{-1}\left(#1\right)}

\newcommand{\E}[1]{\mathbb{E}\left[#1\right]} % expected value

%%% define column vector command (from Michael Nattinger)
\newcount\colveccount
\newcommand*\colvec[1]{
        \global\colveccount#1
        \begin{pmatrix}
        \colvecnext
}
\def\colvecnext#1{
        #1
        \global\advance\colveccount-1
        \ifnum\colveccount>0
                \\
                \expandafter\colvecnext
        \else
                \end{pmatrix}
        \fi
}

%%% define function for drawing matrix augmentation lines
\newcommand\aug{\fboxsep=-\fboxrule\!\!\!\fbox{\strut}\!\!\!}

\makeatletter
\let\amsmath@bigm\bigm

\renewcommand{\bigm}[1]{%
  \ifcsname fenced@\string#1\endcsname
    \expandafter\@firstoftwo
  \else
    \expandafter\@secondoftwo
  \fi
  {\expandafter\amsmath@bigm\csname fenced@\string#1\endcsname}%
  {\amsmath@bigm#1}%
}


%________________________________________________________________%

\begin{document}

\title{	Homework \#1 }
\author{ 	Danny Edgel 					\\ 
			Econ 713: Microeconomics II		\\
			Spring 2021						\\
		}
\maketitle\thispagestyle{empty}

%\noindent\textit{Collaborated with Sarah Bass, Emily Case, Michael Nattinger, and Alex Von Hafften}

%%%________________________________________________________________%%%

\subsection*{Question 1}

\begin{itemize}
	\item[a)] The auction is a Bayesian game where each player ${i\in\{1,...,N\}}$ observes type ${v_i\sim F(x)}$ and chooses a continuous action ${b_i\geq 0}$ to maximize their expected payoff, $\E{u_i}$, where:
		\[
			u_i(v_i,b_i,b_{-i}) =	\begin{cases}
										v_i - b_i, 				&b_i>b_j \forall j\neq i												\\
										\frac{1}{k}(v_i-b_i), 	&b_i\in A=\text{max}\{b_1,...,b_N\} \text{, where }|A|=k 	\\
										0						&b_i < \text{max}\{b_1,...,b_N\}
									\end{cases}
		\]
		On any continuous distribution, the probability that any two draws are exactly equal is zero. Therefore, the knife's-edge case is excluded from the reminder of the analysis for brevity.

	
	\item[b)] Player $i$'s expected payoff is:
		\[
			\E{u_i} = (v_i - b_i)\left[\prod_{j\neq i}Pr(b_i>b_j)\right] + 0\left[1-\prod_{j\neq i}Pr(b_i>b_j)\right] = (v_i - b_i)\left[\prod_{j\neq i}Pr(b_i>b_j)\right]
		\]
		Assume that all players bet according to a linear function of their valuation: ${b_j = d + cv_j}$. Then, player $i$'s expected payoff becomes:
		\begin{align*}
			\E{u_i} &= (v_i - b_i)\left[\prod_{j\neq i}Pr(b_i>d + cv_j)\right] = (v_i - b_i)\left[\prod_{j\neq i}Pr\left(v_j < \frac{b_i-d}{c}\right)\right]	\\
					&= (v_i - b_i)\left[\prod_{j\neq i}\left(\frac{b_i-d}{c}\right)^a\right] = (v_i - b_i)\left(\frac{b_i-d}{c}\right)^{Na}
		\end{align*}
		Player $i$'s optimal bet, then, can be derived from the first-order condition of their expected payoff:
		$$	\frac{v_iNa}{c}\left(\frac{b_i-d}{c}\right)^{Na-1} - \left(\frac{b_i-d}{c}\right)^{Na} - \frac{b_iNa}{c}\left(\frac{b_i-d}{c}\right)^{Na-1} = 0 $$
		\begin{align*}
			Nav_i - b_i + d - b_iNa  &= 0								\\
			b_i &= \left(\frac{Na}{1 + Na}\right)v_i + \frac{d}{Na + 1}
		\end{align*}
		Assuming every player uses the same formula, then, we can solve:
		\begin{align*}
			c &= \frac{Na}{Na + 1}
			d &= \frac{d}{Na + 1}\Rightarrow d=0
		\end{align*}
		Thus, the Bayesian Nash equilibrium is for all players to bet:
			\[
				b = \left(\frac{Na}{Na+1}\right)v
			\]
			
	\item[c)] 


	\item[d)] The limit of $b$ as ${a\rightarrow\infty}$ is $v$, so the payoff of the winner converges to zero. Therefore, bidding becomes more competitive as $a$ increases.


	\item[e)] 


\end{itemize}


%%%________________________________________________________________%%%

\subsection*{Question 2}

\begin{itemize}
	\item[a)] 

	
	\item[b)] 


	\item[c)] 


	\item[d)] 


\end{itemize}

%%%________________________________________________________________%%%

\subsection*{Question 3}

\begin{itemize}
	\item[a)] The auction is a Bayesian game where each player ${i\in\{1,2,3\}}$ observes type ${v_i\sim U[0,1]}$ and chooses a continuous action ${b_i\geq 0}$ to maximize their expected payoff, $\E{u_i}$, where, for ${i\neq j \neq k}$:
		\[
			u_i(v_i,b_i,b_j,b_k) =	\begin{cases}
										v_i - b_i, 				&b_i>b_j\geq b_k 					\\
										\frac{1}{2}(v_i-b_i), 	&b_i=b_j>b_k						\\
										\frac{1}{3}(v_i-b_i),	&b_i=b_j=b_k						\\
										0						&b_i < b_j\leq b_k
									\end{cases}
		\]
		On any continuous distribution, the probability that any two draws are exactly equal is zero. Therefore, the knife's-edge cases are excluded from the reminder of the analysis for brevity.

	
	\item[b)] 


	\item[c)] To determine expected revenue, we must first determine the expected value of the third-highest valuation. Begin by solving the pdf of the third-highest price, $f^3$, using the order statistic pdf formula:
		\[
			f^3(x) = \left(frac{n!}{(2!(n-3)!}\right)x^2(1-x)^{n-3} = \left(\frac{n(n-1)(n-2)}{2}\right)x^2(1-x)^{n-3}
		\]
		Then, the expected revenue is the expected value of the third-lowest bid:
		\begin{align*}
			R_s	&= \frac{n(n-1)(n-2)}{2}\int_0^1v^2(1-v)^{n-3}dv = 	\frac{n(n-1)(n-2)}{2}\int_1^0(1-2y+y^2)y^{n-3}dy					\\
				&= \frac{n(n-1)(n-2)}{2}\int_1^0y^{n-3}-2y^{n-2}+y^{n-1}dy
		\end{align*}
		Thus, the expected revenue of the seller is:
		\[
			\E{\frac{n-1}{n-2}v} = \frac{n-1}{(n-2)(n+1)}
		\]

	\item[d)] 


\end{itemize}


%%%________________________________________________________________%%%


\end{document}
