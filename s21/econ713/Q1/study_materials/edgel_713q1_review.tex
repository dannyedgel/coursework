%%% Econ713: Microeconomics II
%%% Spring 2021
%%% Danny Edgel
%%%
% Review of all concepts that will be on the midterm for the first quarter of Econ 713
%%%

%%%
%							PREAMBLE
%%%

\documentclass{article}

%%% declare packages
\usepackage{amsmath}
\usepackage{amssymb}
\usepackage{array}
\usepackage{bm}
\usepackage{booktabs}
\usepackage{changepage}
\usepackage{centernot}
\usepackage{graphicx}
\usepackage{multirow}
\usepackage{xcolor}
\usepackage[shortlabels]{enumitem}
\usepackage{fancyhdr}
	\fancyhf{} % sets both header and footer to nothing
	\renewcommand{\headrulewidth}{0pt}
    \rfoot{Edgel, \thepage}
    \pagestyle{fancy}
	
%%% define shortcuts for set notation
\newcommand{\N}{\mathbb{N}}
\newcommand{\Z}{\mathbb{Z}}
\newcommand{\R}{\mathbb{R}}
\newcommand{\Q}{\mathbb{Q}}
\newcommand{\lmt}{\underset{x\rightarrow\infty}{\text{lim }}}
\newcommand{\neglmt}{\underset{x\rightarrow-\infty}{\text{lim }}}
\newcommand{\zerolmt}{\underset{x\rightarrow 0}{\text{lim }}}
\newcommand{\usmax}[1]{\underset{#1}{\text{max }}}
\newcommand{\usmin}[1]{\underset{#1}{\text{min }}}
\newcommand{\intersect}{\bigcap}
\newcommand{\union}{\bigcup}
\newcommand{\olw}{\overline{w}}
\newcommand{\olx}{\overline{x}}
\newcommand{\oly}{\overline{y}}
\newcommand{\loge}[1]{\text{log}\left(#1\right)}
\renewcommand{\P}{\mathcal{P}}
\renewcommand{\L}{\mathcal{L}}
\newcommand{\B}{\mathcal{B}}
\newcommand{\olp}{\overline{p}}
\renewcommand{\exp}[1]{\text{exp}\left\{#1\right\}}
\newcommand{\binv}[1]{b_j^{-1}\left(#1\right)}
\newcommand{\contains}{\supseteq}
\newcommand{\Epsilon}{\mathcal{E}}
\newcommand{\red}[1]{{\color{red}#1}}
\newcommand{\blue}[1]{{\color{blue}#1}}
\newcommand{\sumn}{\sum_{i=1}^n}

%\DeclareMathOperator{\E}{\mathbb{E}}% expected value

\newcommand{\E}[1]{\mathbb{E}\left[#1\right]} % expected value
\newcommand{\Et}[2]{\mathbb{E}_{#1}\left[#2\right]}

%%% define column vector command (from Michael Nattinger)
\newcount\colveccount
\newcommand*\colvec[1]{
        \global\colveccount#1
        \begin{pmatrix}
        \colvecnext
}
\def\colvecnext#1{
        #1
        \global\advance\colveccount-1
        \ifnum\colveccount>0
                \\
                \expandafter\colvecnext
        \else
                \end{pmatrix}
        \fi
}

%%% define function for drawing matrix augmentation lines
\newcommand\aug{\fboxsep=-\fboxrule\!\!\!\fbox{\strut}\!\!\!}

\makeatletter
\let\amsmath@bigm\bigm

\renewcommand{\bigm}[1]{%
  \ifcsname fenced@\string#1\endcsname
    \expandafter\@firstoftwo
  \else
    \expandafter\@secondoftwo
  \fi
  {\expandafter\amsmath@bigm\csname fenced@\string#1\endcsname}%
  {\amsmath@bigm#1}%
}




%________________________________________________________________%

\begin{document}

\title{	Micro Quarter 3 Review }
\author{ 	Danny Edgel 					\\ 
		Econ 713: Microeconomics II		\\
		Spring 2021, Quarter 1			\\
		}
\maketitle\thispagestyle{empty}



%%%________________________________________________________________%%%

%%%----------------------------------------------------------------------
%%%
%%% SECTION: Matching foundations of markets
%%%
%%%----------------------------------------------------------------------
\section{Matching Foundations of Markets}


%%%________________________________________________________________%%%

\subsection{Non-Transferable Utility (NTU) Matching}

\begin{itemize}
	\item Positive-Assortative Matching (PAM): The derivative of each side’s payoff function has the same sign
	
	\item Negative-Assortative Matching (NAM): The derivative of each side’s payoff function has the opposite sign
	
	\item Gale-Shapley Theorem:
	
	\begin{itemize}
		
		\item Male-pessimal outcome is the female-optimal outcome, and vice versa
		
		\item If male-pessimal and male-optimal outcomes are the same, then the stable matching is unique
	
	\end{itemize}
	
	\item The DAA can take no more than $n^2-2n+2$ rounds, where there are n men and n women
	
	\item Solving DAA algorithm in discrete case: Example from first question of 2020 midterm below, showing the DAA matching from women proposing.
		\begin{center}
			\begin{tabular}{r|c|c|c|c|c|c|c|c|c|c}
			\multicolumn{11}{c}{\underline{Round}} \\ 
			 	& 1 		& 2 		& 3 		& 4 		& 5 		& 6		& 7 		& 8 		& 9 		& 10	\\\hline
			  A 	& S 		& R$^*$	& 		& 		&		& P$^*$	&		&		&		& M$^*$ \\\hline
			  B 	& R$^*$ 	& 		& S$^*$	& 		&		&		&		&		& P$^*$	& \\\hline
			  C 	& P$^*$ 	& 		& 		& 		& R$^*$ 	&		&		& S$^*$	&		& \\\hline
			  D	& S$^*$	& 		& 		& P$^*$	&		&		& R$^*$ 	&		&		& \\\hline
			\end{tabular}
		\end{center}
	
	
\end{itemize}


%%%________________________________________________________________%%%

\subsection{Welfare Theorems of Matching}

\begin{enumerate}
	\item A competitive equilibrium yields an efficient matching
	
	\item An efficient matching is a competitive equilibrium for a suitable set of wages
\end{enumerate}




%%%________________________________________________________________%%%

\subsection{Transferable Utility (TU) Matching}


\begin{itemize}
	\item PAM: Supermodular
		\begin{itemize}
			\item If differentiable, cross-derivative is positive
			\item If not differentiable, increasing differences 
		\end{itemize}
	\item NAM: Submodular
		\begin{itemize}
			\item If differentiable, cross-derivative is negative
			\item If not differentiable, decreasing differences 
		\end{itemize}
	
	\item Competitive mechanics of a TU matching market: there exists an industry of ``match-makers" who get the output from the matches they make and pay wages to each side.
		\begin{itemize}
			\item If matchmakers are making a profit, then free entry of new match-makers hold that new match-makers will enter and out-compete incumbent match-makers by either making better matches or paying higher wages. Thus, the profit of match-makers has a maximum of 0
			
			\item If matchmakers make inefficient matches, then they will be unable to pay high enough wages to at least one side of the match to stave off more efficient matchmakers from offering higher wages for a more efficient match
		\end{itemize}
	
	\item Finding wages (differentiable case)
		\begin{enumerate}
			\item Let $\pi=h(x,y)-v(x)-w(y)$ be the profit function for matchmakers in this market, where $h(x,y)$ is the output of a match. Find FOC for one side of the market\footnote{if FOCs aren’t symmetric, repeat steps 1-3 for other side of market}
			\item If PAM, solve FOC for first derivative of wage function using $y=x$. If NAM, solve using $y=1-x$.
			\item Take antiderivative to determine wage function, including some constant, $c$. Let k be the constant for the other side’s wage function
			\item Impose free entry/exit condition to let $\pi=0$ at its maximum; solve $\pi(x,y)=0$ for the relationship between $c$ and $k$. 
			\item Suppose $c+k=S$. Then, $k=S-c$ and the range of market-decentralizing wages is given by the range of $c$ such that the wage of each side is weakly greater than the side’s outside option
				\begin{itemize}
					\item In the typical case where the value of not matching for each side is zero, $c\in[0,S]$
					\item Suppose $D$ is the cost of matching for the $x$ side. Then $c\in[-D,S]$
					\item If there is a short side of the market, pin wages down uniquely by setting $c$ and $k$ such that the short side captures all surplus. For example, if the mass of type $y$ exceeds that of type $x$, then $c=0$, $k=S$.
				\end{itemize}
		\end{enumerate}
		
	\item Finding wages (discrete case)
		\begin{enumerate}
			\item Determine the efficient match (i.e. the one that maximimizes total output)
			\item Set up the inequalities for the wages in this match: 
				\begin{itemize}
					\item The sum of the two sides in each efficient match can be no greater than the output of that match 
					\item The sum of each individual's wage and the wage of their next-best match must be weakly greater than their match output (in order to dissuade that match)
				\end{itemize}
			\item Due to competitive mechanics of TU matching, match condition (1) will hold with equality
			\item Use these two sets of inequalities to derive the intervals for each individual's wage
		\end{enumerate}
\end{itemize}

%%%________________________________________________________________%%%

\subsection{Double Auctions}

\begin{itemize}
	\item In a double auction, there are $n$ buyers with valuations $v_i$ for a homogenous good and $m$ sellers of the good with costs $c_j$
	
	\item NAM theorem: NAM of buyers and sellers in a double auction results in a total surplus that is weakly greater than any other matching method
	
	\item Solving for equilibrium in a double auction:
		\begin{enumerate}
			\item Determine market demand by sorting valuations in descending order and summing demand at each price
			\item Determine market supply by sorting costs in ascending order and summing supply at each price 
			\item Tip for steps 1 and 2: Draw up a table of each side of the market, with columns representing the open-closed interval of demand/supply at each price/cost/valuation\textemdash e.g., for ${v_i=\$10}$, if adding $i$'s demand at a price of \$10 takes demand from 20 units to 25, then have one column that represents pre-\$10 demand and another with demand at \$10. This makes graphing the staircase-looking function much easier.
			\item Either visually identfy on the tables where the two curves intersect, or begin testing plausbile prices, raising or lowering price ``guesses" according to supply and demand at each guessed price 
			\item After determining an equilibrium price/quantity, determine whether it is unique by graphing supply and demand around that point
		\end{enumerate}
	
	\item Collusion outcomes:
		\begin{itemize}
			\item Sellers collude: Total seller profit is maximized. To solve:
				\begin{enumerate}
					\item Begin with the highest price that won't decrease demand 
					\item Calculate the change in total seller profit if you raise to the next buyer's valuation
					\item If (2) is negative, then the price from (1) is optimal. If (2) is positive, repeat step 2 until you find the price that results in negative marginal profit
				\end{enumerate}
			
			\item Buyers collude: Total consumer surplus is maximized. To solve, use same method as solving for the seller collusion outcome, but with progressively lowering the price
		\end{itemize}
\end{itemize}

%%%________________________________________________________________%%%
\pagebreak
%%%----------------------------------------------------------------------
%%%
%%% SECTION: Partial Equilibrium
%%%
%%%----------------------------------------------------------------------
\section{Partial Equilibrium}


%%%________________________________________________________________%%%

\subsection{Supply and Demand}
\subsubsection{Intensive vs. Extensive Margins}
\begin{itemize}
	\item \textbf{Extensive Margins} reflect who is in the market
		\begin{itemize}
			\item e.g. fixed costs decrease $\Rightarrow$ new firms enter the market $\Rightarrow$ supply increases on the extensive margin
		\end{itemize}
	\item \textbf{Intensive Margins} reflect how much each actor in the market buys/sells
		\begin{itemize}
			\item e.g. incomes increase $\Rightarrow$ consumers have more money $\Rightarrow$ demand increases on the intensive margin
		\end{itemize}
	\item Supply:
		\begin{itemize}
			\item Marginal costs determine intensive margin ($MR = MC$ sets optimal quantity)
			\item Average costs determine extensive margin ($P=AVC$ determines shutdown decision in short run, $P=ATC$ determines entry/exit in long run)
		\end{itemize}
	\item Supply and demand curves reflect \blue{both} intensive and extensive margins
\end{itemize}

\subsubsection{Competitive Equilibrium}
\begin{itemize}
	\item \textbf{Walrasian Price Stability}: Use net demand to determine whether to raise/lower price
		\begin{itemize}
			\item During price adjustment, the \blue{short side} of the market determines the quantity
			\item $Q^D>Q^S\Rightarrow$ market quantity is $Q^S$, price increases
			\item $Q^S>Q^D\Rightarrow$ market quantity is $Q^D$, price decreases
			\item $Q^S=Q^D=Q^*\Rightarrow$ market is Walrasian stable
		\end{itemize}
	\item \textbf{Marshallian Quantity Stability}: Use gap between demand price and supply price to raise/lower supply prices
		\begin{itemize}
			\item During price adjustment, the \blue{short side} of the market determines the quantity
			\item $P^D>P^S\Rightarrow$ Suppliers raise prices
			\item $P^S>P^D\Rightarrow$ Suppliers lower prices
			\item $P^S=P^D=P^*\Rightarrow$ market is Marshallian stable
		\end{itemize}
\end{itemize}

\subsubsection{Elasticity}
\begin{itemize}
	\item Demand: $\varepsilon = \frac{dQ^D}{dP}\frac{P}{Q} = \frac{d\loge{Q^S}}{d\loge{P}}\approx \frac{\%\Delta Q^D}{\%\Delta  P} < 0$ (usually)
	\item Supply: $\eta = \frac{dQ^S}{dP}\frac{P}{Q} = \frac{d\loge{Q^S}}{d\loge{P}}\approx \frac{\%\Delta Q^S}{\%\Delta  P} > 0$ (usually)
	\item Le Chatellier's Principle: demand/supply is more elastic (i.e. the absolute value is greater) in the long run than in the short run
	\item Elasticity and price volatility:
		\begin{itemize}
			\item More elastic $\Rightarrow$ greater quantity volatility, lower price volatility
			\item Less elastic $\Rightarrow$ greater price volatility, lower quantity volatility
		\end{itemize}
	\item Elasticity and tax incidence:
		\begin{itemize}
			\item \blue{Incidence Theorem:} The more elastic side of the market bears the lower burden, regardless of who ``pays" the tax
			\begin{itemize}
				\item Impose excise tax $\tau\equiv dP^D-dP^S$. Then:
					\[
						dP^D\approx \left(\frac{\eta}{\eta-\varepsilon}\right)\tau>0\text{ and }dP^S\approx\left(\frac{\varepsilon}{\eta-\varepsilon}\right)\tau<0
					\]
					Where $\frac{\eta}{\eta-\varepsilon}$ is the share of the tax paid by consumers, and so on.
			\end{itemize}
			\item Deadweight loss for small taxes: Using $\varepsilon$ and tax $\tau$,
				\[
					DWL = \frac{1}{2}dQ(dP^D-DP^S)=\left(\frac{1}{1/\varepsilon-1/\eta}\right)\left(\frac{Q}{2P^D}\right)\tau^2
				\]
			\item \blue{Tax Irrelevance Theorem:} Regardless of whether demand or supply ``pays" the tax, the demand and supply prices, market quantity, and efficiency loss are the same.
			\item \blue{Ramsey Inverse Elasticity Rule:} taxes should be proportional to the sum of the reciprocals of its supply and demand elasticities
			\begin{itemize}
				\item In other words, DWL is lower and tax revenue is higher if you tax less elastically-supplied or -demanded goods.
			\end{itemize}
		\end{itemize}
	
\end{itemize}

%%%________________________________________________________________%%%
\subsection{Market Power}

\subsubsection{Barriers to Entry}
\begin{itemize}
	\item Technical Barriers to Entry: Barriers that are due to the specific nature of production, distribution, etc.
		\begin{itemize}
			\item Minimum efficient scale, i.e. fixed costs that are so high, you can only profitably operate if sales and/or prices are sufficiently high
			\item Ownership of unique resources - exhaustible resources or a location (e.g. ski resorts)
		\end{itemize}
	\item Legal Barriers to Entry:
		\begin{itemize}
			\item Natural monopolies chartered by the government (e.g. the post office, public utilities)
			\item Patents, trademarks, copyrights
		\end{itemize}
	\item Cartels (legal and illegal)
	\item Noncompete Agreements
	\item Network Externalities: the value of a good/service increases with scale (e.g. social media)
\end{itemize}

\subsubsection{Monopoly}
\begin{itemize}
	\item 101 stuff: $MC=MR$ comes from maximizing profit where the price, rather than being taken as given, is the inverse demand curve
	\item \blue{Inverse Elasticity Rule:} Rewriting the FOC yields:
		\[
			P(Q)\left(1-\frac{1}{|\varepsilon|}\right) = C'(Q)
		\]
		From which markups are an inversely related to demand elasticity. Which gives us the \blue{Lerner index} of market power, which goes from 0 (perfect competition) to 1 (captive market):
		\[
			L = \frac{P(Q) - C'(Q)}{P(Q)} = \frac{1}{|\varepsilon|} < 1
		\]
\end{itemize}

\subsubsection{Monopsony}
\begin{itemize}
	\item Rather than a single seller having full control over quantity supplied, a single buyer with full control over quantity demanded 
	\item Canonical example: employers with monospony power hiring at $W<VMP$
		\begin{itemize}
			\item Inverse elasticity rule: $VMP(L) = w(L)\left(1 + \frac{1}{\eta}\right)$
		\end{itemize}
\end{itemize}

\subsubsection{Oligopoly}
\begin{itemize}
	\item Cartels
		\begin{itemize}
			\item Successful cartels operate as a multiplant firm, setting market quantity equal to the monopoly's profit-maximizing quantity
			\item Problem: individual firms face $MR>MC$, tempting them to ``chisel" by increasing supply
			\item \blue{Cartel outcome:} Each firm assumes that each other firm will produce the same amount. Then, for $N$ firms, each firm, $i$, facing market demand $P(Q)$ and cost function $C_i(q_i)$, solves:
				\[
					\usmax{q_i}P(Nq_i)q_i - C_i(q_i)
				\]
				Note that this is a stable outcome if all firms have the same marginal cost, but otherwise collapses into the Cournot outcome
		\end{itemize}
	\item Cournot competition: $i\in\{1,...,n\}$ firms solve:
		\[
			\usmax{q_i}P(Q)q_i-C(q_i)\text{, }Q=\sum_{j=1}^nq_j
		\]
		\begin{itemize}
			\item The Cournot equilibrium converges to the competitive equilibrium as $n\rightarrow\infty$
			\item The Cournot game is submodular (``strategic substitutes"), because each firm $i$'s best reply function is monotonically decreasing in other firms' choice variables ($q_j$, $j\neq i$)
		\end{itemize}
	\item Stackelberg competition: a large, dominant firm ($L$) moves first, followed by another firm (or set of firms), $F$. Solved via backward induction by the leading firm:
		\[
			\usmax{q_F}P(q_L+q_F)q_F-C(q_F)
		\]
	\item Stackelberg leads to a higher equilibrium supply than Cournot, which is higher than monopoly, and all of them are lower than perfect competition
\end{itemize}

\subsubsection{Price Discrimination}
\begin{itemize}
	\item \blue{Def:} Charging different prices to different consumers.
	\item First degree: Charging different prices at the individual level (profit-maximizing 1st degree PD: charging each consumer their willingness to pay)
	\item Second degree: Charging different prices based on quantity
		\begin{itemize}
			\item \blue{Two-part tariff:} paying a fixed fee for the right to trade at a linear price (e.g. Costco memberships)
			\item Quantity discounts (e.g. lower per-unit price when you buy in bulk)
		\end{itemize}
	\item Third degree: Charging different prices to different consumer groups (e.g. senior or student discounts)
\end{itemize}


%%%________________________________________________________________%%%

\subsection{Externalities}
Pecuniary vs. technical externalities:
\begin{itemize}
	\item Pecuniary externality: an increase in demand for some good increases the price, harming existing consumers 
		\begin{itemize}
			\item The price system reallocates gains from trade, maximizing welfare
		\end{itemize}
	\item \blue{Technical externality:} A transaction between two agents imposes a cost or benefit on a third party, which is not intenalized in the price of that transaction.
		\begin{itemize}
			\item An uninternalized benefit is a \textit{positive externality} (e.g. beekeepers near a flower garden)
			\item An uninternalized cost is a \textit{negative externality} (e.g. carbon pollution)
		\end{itemize}
	\item The efficient allocation comes from social marginal cost being equal to social marginal benefit: ($SMC=SMB$)
		\begin{itemize}
			\item Without externalities, private marginal costs/benefits are equal to social marginal costs/benefits
		\end{itemize}
\end{itemize}

\subsubsection{Coase Theorem}
\begin{itemize}
	\item \blue{Big idea:} If property rights are well-defined and agents can costlessly bargain, then the efficient allocation will be reached even in the prescence of externalities.
	\item E.g.: If a cattle rancher is given cattle-grazing rights, then the farmer pays the rancher to reduce grazing; If the farmer is given the right not to have cattle graze on their farm, then the rancher pays the farmer for grazing access.
\end{itemize}

\subsubsection{Pigouvian Taxation}
\begin{itemize}
	\item The efficient outcome can be attained in the presence of externalities by taxing activities with negative externalities and subsidizing those with positive externalities such that ${PMC(Q) -\tau(Q)=SMC(Q)}$ or ${PMB(Q) + \tau(Q)=SMB(Q)}$
	\item The efficient tax is set at the marginal damage (or marginal benefit) of the activity
	\item Alternatively, tradable permits for the activity/good with negative externalities can be issued; the permits should be equal to the optimal quantity and will trade at cost of marginal damage
\end{itemize}

%%%________________________________________________________________%%%

\subsection{Public Goods}
\begin{center}
	\begin{tabular}{rp{1cm}p{1cm}}
					& Rival 										& Nonrival 										\\\cline{2-3}
	Excludable		& \multicolumn{1}{|c|}{Private Goods}			& \multicolumn{1}{|c|}{Club Goods}				\\\cline{2-3}
	Non-Excludable	& \multicolumn{1}{|c|}{Congestion Public Goods}	& \multicolumn{1}{|c|}{Pure Public Goods}		\\\cline{2-3}
	\end{tabular}
\end{center}
\begin{itemize}
	\item \blue{Rival:} One agent's consumption of a good diminishes another agent's benefit (e.g. you and I can't both drink the same 1ml of coca-cola)
	\item \blue{Excludable:} Providers of a good/service cannot prevent those who haven't paid for it from consuming it (e.g. a fireworks display)
	\item For nonrival goods, the aggregate demand curve is calculated by adding inverse demand curves: $P^D_{agg}=\sum_iP^D_i$
\end{itemize}

\subsubsection{Efficient Provisioning}
\begin{itemize}
	\item \blue{Tragedy of the Commons:} Because a consumers return on (or cost of) a public good is equal to the average return (or cost) rather than the marginal return (or cost) $PMC\neq SMC$ and/or $PMB\neq SMB$ $\Rightarrow$ the competitive equilibrium will not be efficient
	\item The efficient allocation can be reached via Pigouvian taxes
		\begin{itemize}
			\item Example: fishing between to lakes, $A$ and $B$, with returns $F(X_A)$ and $G(X_B)$, with $X_A+X_B=1$.
			\item SPP: $F'(X_A)=G'(X_B)\Rightarrow X_A^*,X_B^*$
			\item Suppose $X_B>X_B^*$ in the competitive equilibrium. Determine Pigouvian tax, $\tau^*$ with:
				\[
					\frac{G(X^*_B)}{X_B^*}-\tau^* = \frac{F(X^*_A)}{X_A^*}
				\]
		\end{itemize}
	\item Discrete nonrival goods: provision of one more unit of the good at cost $c$ is efficient if $\exists\{t_i\}_1^n$ such that $\sum_i^nt_i=c$ and each agent $i$ is weakly better off paying $t_i$ and having one more unit of the good, and at least one agent is strictly better off.
	\item Continuous nonrival goods: maximize some social welfare function (SWF), $W$ 
		\begin{itemize}
			\item a strictly quasi-concave one $\Rightarrow$ $\exists!$ solution
			\item \blue{Lemma: Samuelson Condition} The opimal consumption of a public good, $G$ is given by $\sum_{i=1}^nMRS^i_{G,w}=MRT_{G,w}$
				\begin{itemize}
					\item Reduces to $\sum_iMB^i(G)=MC(G)$
				\end{itemize}
			\item \blue{Lindahl Equilibrium:} Let $G$ be a public good and each $i$ of $n$ consumers are endowed with $w_i$ of some private good. Then, a LE is an allocation, ${(G^*,x_1^*,...,x_n^*)}$, and set of prices, $\{p_1,...,p_n\}$, paid by each consumer for the public good, such that, for each consumer:
				\[
					(G^*,x_i^*) = \text{arg}\usmax{x_i,G}U^i(G,x_i)\text{ s.t. }x_i + p_iG = w_i
				\]
				The Lindahl equilibrium satisfied the Samuelson condition, because each consumer contributes their marginal benefit.
			\item \blue{Preak Load Pricing}. Solving these problems typically takes the following steps:
				\begin{enumerate}
					\item Set up a Langrangian function that maximizes total suplus (consumer and producer separately, where producer surplus is generally a rectangle, since public goods usually have a constant marginal cost), with multiplier for off-peak and peak supply, which must be weakly less than total capacity (e.g. buses purchased)
					\item Take first-order conditions
					\item Use complementary slackness from the Kuhn-Tucker conditions to solve for optimal peak and off-peak supply under different scenarios (e.g. peak and off-peak are equal, in which case the Lagrangian multiplier on off-peak supply isn't zero, etc.)
				\end{enumerate}
				\red{Example: Public goods exercise 9}
		\end{itemize}
\end{itemize}

%%%________________________________________________________________%%%
\pagebreak
%%%----------------------------------------------------------------------
%%%
%%% SECTION: General Equilibrium
%%%
%%%----------------------------------------------------------------------
\section{General Equilibrium}


%%%________________________________________________________________%%%

\subsection{GE in Exchange Economies}
Notation:
\begin{itemize}
	\item $\Epsilon = (\{u^i\},\overline{x})$
	\item $L\geq 2$ goods, $\ell\in\{1,...,L\}$
	\item $n\geq 2$ traders, $i\in\{1,...,n\}$
	\item Endowments $\overline{x}^i=(\olx_1^i,...,\olx_L^i)'\in\R_+^L$ 
	\item An allocation is a matrix $x=(x^1,...,x^n)\in\R_+^{L\times n}$
	\item Price vector $p=(p_1,...,p_L)\in\R^L$
	\item Trader $i$ has utility $u^i:\R_+^L\rightarrow\R$
	\item Trader $i$'s income is the market value of their endowment: $p\cdot \olx^i$
	\item Trader $i$'s budget set is $\B^i(\olx^i,p) = \left\{x^i\in\R_+^L|p\cdot x^i\leq p\cdot\olx^i\right\}$
\end{itemize}
Then, each consumer solves:
\[
	\usmax{x^i}u^i(x^i)\text{ s.t. }x^i\in\B^i(\olx^i,p)
\]
An allocation is \blue{feasible} for $\Epsilon$ if $\sum_{i=1}^Nx_\ell^i\leq\sumn \olx_\ell^i$ $\forall \ell$. A feasible allocation $x$ is \blue{socially optimal} if $\nexists$ a feasible allocation $z$ such that no trader is worse off but at least one trader is strictly better off.

\pagebreak
\subsubsection{Competitive Equilibrium}
\red{Most questions on exchange economy GE appear to be given as two-person, two-good economies and involve both solving for and plotting the contract curve.}
\begin{itemize}
	\item \blue{Edgeworth box:} a graph of a two-person, two-good economy:
		\begin{center}
			\includegraphics[scale=.75]{edgeworth.png}
		\end{center}
	\item \blue{Contract curve:} The set of all socially efficient allocations
	\item Any individually rational allocation is one where $u^i(x^i)\geq u^i(\olx^i)$ $\forall i$
	\item The \blue{core} is the set of individually rational allocations
	\item A competitive equilirbium (CE) in $\Epsilon$ is a $(x,p)$ s.t. $x$ is feasible and optimal for traders, given $p$
		\begin{itemize}
			\item Optimal in a 2-good, 2-person economy $\Rightarrow$ $MRS^1_{x,y}=MRS^2_{x,y}$
		\end{itemize}
	\item In an exchange economy, gains from trade arise from differences in preferences and/or endowments
	\item \blue{First Welfare Theorem:} If $(x,p)$ is a CE of $\Epsilon$ and preferences are locall non-satiated, then $x$ is socially efficient
	\item \blue{Second Welfare Theorem:} If traders have continuous, monotonic, and quasiconcave utility functions and $x$ is a socially efficient allocation, then ${\exists p}$ such that $(x,p)$ is a CE of $\Epsilon$ 
		\begin{itemize}
			\item If at least one consumer has smooth, convex preferences, $p$ is unique
		\end{itemize}
\end{itemize}

\subsubsection{Excess Demand Functions}
\begin{itemize}
	\item Since preferences are assumed to be strictly convex, each consumer has a unique demand for each good $\ell$ at each price $p$: $x_\ell^i(p)$
	\item Consumer $i$'s \blue{excess demand} for $\ell$ is their demand for $\ell$, net of their endowment of $\ell$: ${ED^i_\ell(p)=x_\ell^i(p)-\olx_\ell^i(p)}$
	\item Market clearing implies $\sumn ED_\ell^i(p)=0$ $\forall \ell$
	\item \blue{Walras's Law:} If traders cosume their entire income at allocation $x(p)$, then the market calue of net excess demand vanishes: ${\sum_\ell^Lp_\ell \sumn ED^i_\ell(p)=0}$
		\begin{itemize}
			\item Implication: If $L-1$ markets clear, then $L$ markets clear
			\item Other implication: Can normalize all prices by a numeraire and solve for the price ratio, e.g. $\frac{p_y}{p_x}$
			\item $\therefore$ CE problem is $L-1$ equations in $L-1$ goods
			\item e.g. $L=2$ with $x$ and $y$: let $p=\frac{p_y}{p_x}$, then set ${ED^1_x(p) + ED^2_x(p) = 0}$. The only unknown is $p$.
		\end{itemize}
	\item \blue{Theorem (existence):} If every trader $i$ has strictly monotone and convex preferences over $x$ and $y$ and owns a positive endowment, $(\olx^i,\oly^i)$, then there exists a Walrasian stable competitive equilibrium, $(x,y,p)$
		\begin{itemize}
			\item Proof: monotonicity and convexity imply that ${ED_x(0)<0<ED_x(\infty)}$. By intermediate value theorem, ${\exists p\text{ s.t. }ED_x(p)=0}$
			\item Even though excess demand functions can look basically any which way, they're almost always locally unique, so you don't need to worry about cases where they're not
		\end{itemize}
\end{itemize}

\subsubsection{Trade Offer Curves}
\red{Reference: 15.B in MWG; Figures 15.B.3 through 15.B.5 illustrate much better than Lones's slides}
\begin{itemize}
	\item Plots, for each trader $i$, the optimal consumption at each possible price, holding endowments constant (think of it as a best response curve)
	\item The TOC can be non-monotone, even with monotone preferences; with two goods, $x$ and $y$:
		\begin{itemize}
			\item If $x$ and $y$ are both normal, TOC strictly falls in the x-y plane
			\item If $x$ is normal and $y$ is inferior, TOC may fall or rose
			\item If $x$ is inferior and $y$ is normal, TOC may fall or rose
			\item If $x$ and $y$ are both inferior, TOC may fall or rise, and may turn backward
		\end{itemize}
	\item \blue{Solving the TOC with $L=2$:} Set the marginal rate of substitution: 
		\[
			MRS = \frac{\partial U/\partial x}{\partial U/\partial y}
		\]
		Equal to the budget constraint in terms of endowments:
		\[
			p = \frac{\olx - x}{y-\oly}
		\]
		And solve for $y$ as a funciton of $x$, with endowments fixed as TOC parameters
	\item The intersection of each trader's TOC yields an equilibrium, which is unique if demand has the \blue{gross substitutes property}: an increase in price $p_k$ raises the demand of every other good $x_\ell$, for $\ell\neq k$
	\item \blue{Proposition (Uniqueness):} If the aggregate excess demand function satisfies gross substitutes, the economy has at most one Walrasian equilibrium
\end{itemize}

\subsubsection{Monopoly in an Exchange Economy}
\red{I found Lones's explanation of this subject incomprehensible; the exchange chapter in Varian's \textit{Intermediate Micro} textbook explains monopoly in the Edgeworth box clearly. PDFs of the book are found easily online (thank you for this heads-up, Anya!)}
\begin{itemize}
	\item Monopoly can occur in a pure exchange economy if there is no auctioneer but instead one agent (say, $A$) knows the other agent $B$'s offer curve and decides to quote prices to $B$, knowing which bundle $B$ will choose 
	\item \blue{Monopoly with Linear Prices:} 
		\begin{itemize}
			\item Since the TOC represents an agent's optimal bundle (i.e. the tangency point between their highest MRS and a given budget constraint), the monopolist, $A$ is able to draw any budget constraint they choose; where this budget constraint intersects $B$'s TOC determines the equilibrium
			\item The monopolist, then, chooses the point on $B$'s TOC that is tangent to $A$'s highest indifference curve:
				\begin{center}
					\includegraphics[scale=.95]{monopolist_ge.png}
				\end{center}
			\item This outcome is not socially efficient because the tangency does not occur on the contract curve
		\end{itemize}
	\item \blue{Monopoly with Nonlinear Prices:}
		\begin{itemize}
			\item Suppose that $A$ decides instead to extract all possible surplus from $B$ by offering $B$ a bundle that is on the same utility curve as $B$'s endowment
			\item $A$ wasn't able to do this via linear pricing, since this optimal point for $A$ does not lie on $B$'s trade offer curve, which depends on $\olx$. However, by charging a fee to $B$ in order to engage in trade, $A$ shifts $B$'s trade offer curve post-fee transfer so that $A$'s best feasible (given the aggregate endowment) point is tangent to $A$'s indifference curve at $A$'s endowment (see graph below)
				\begin{center}
					\includegraphics[scale=.75]{monopolist_ge_nonlinear.png}
				\end{center}
				
			\item Because this allocation maximizes total surplus (but transfers it to $A$), this allocation is socially efficient
		\end{itemize}
\end{itemize}

%%%________________________________________________________________%%%

\subsection{GE with Production}

\subsubsection{Notation}
\begin{itemize}
	\item $m$ firms, indexed $j=1,...,m$
	\item $n$ consumers, indexed $i=1,...,n$ 
	\item A firm is $y^j\subset\R^L$, were positive entries of $y^j$ are outputs and negative entries are inputs; all of the quarter 1 producer theory stuff applies (convexity, free disposal, etc.)
		\begin{itemize}
			\item with price vector $p\in\R^L_+$, then, firm profits are $p\cdot y^j$
		\end{itemize}
	\item Conumers have utility functions $u^i$ and endowments $\olx^i$
	\item Each consumer, $i$, owns a share, $\theta_{ij}\geq 0$ of each firm, $j$, and faces the budget set:
		\[
			B^i(p)=\left\{x^i\in X^i| p\cdot x^i\leq p\cdot\olx^i + \sum_{h=1}^m\theta_{ij}p\cdot y^j \right\}
		\]
\end{itemize}

\subsubsection{Competitive Equilibrium}
\begin{itemize}
	\item Basically a static version of the competitive equilibrium from a standard macro model
	\item \blue{Definition:} A \blue{competitive equilibrium} of a private ownership economy, ${\left(\{Y^j\}_{j=1}^m;\{X^i,u^i,\olx^i,\theta_{i1},...\theta_{im}\}_{i=1}^n\right)}$, is an allocation, ${(x,y)\in\R^{n\times L}\times\R^{m\times L}}$ and a price vector $p\in\R^L$ such that,
		\begin{enumerate}
			\item $y^j$ maximizes profits $\forall j$, given $p$
			\item $x^i$ maximizes utility $u^i$ subject to $B^i(p)$
			\item Markets clear: ${D(p)-\olx-S(p)\leq 0}$, where $D(p)$ is market demand, $S(p)$ is market supply, and ${D(p)-\olx-S(p)}$ is market excess demand in a production economy
				\begin{itemize}
					\item \textbf{Note}: for any good $k$, if excess demand is negative, $p_k=0$
					\item Also note that this condition implies ${D(p)\leq S(p) + \olx}$
				\end{itemize}
		\end{enumerate}
	\item \blue{Theorem (Existence):} If each consumer, $i$ has a continuous, nonsatiated and strictly quasiconcave utility function, positive endowment, and dividend shares and firms have closed and convex production technologies, then a competitive equilibrium exists.
		\begin{itemize}
			\item Corollary: If the consumer conditions are satisfied, then a competitive equilibrium exists in a pure exchange economy
		\end{itemize}
\end{itemize}

\subsubsection{Robinson Crusoe Economies}
\red{Reference: 15.C in MWG}
\medskip \\
\textbf{Def:} An economy with one consumer and one producer (usually with two goods).
\medskip \\
\blue{Setup:} Let $u(y,\ell)$ represent the consumer's utility over production good, $y$, and leisure, $\ell$. Let $f(L)$ represent the production technology with labor as the only input, and let the consumer's endowment be given by ${L + \ell = \overline{L}}$. Then, solving for the CE involves:
\begin{enumerate}
	\item Solving the firm problem for optimal $y$ and $L$ in terms of price vector $(p,w)$:
		\[
			\usmax{y,L}\pi = py-wL
		\]
		The solution to this problem also yields the firm's maximum profit in terms of the price vector, $\pi(p,w)$
	\item Solving the conumser problem for optimal $y$ and $\ell$, subject to the budget constraint:
		\[
			\usmax{y,\ell} u(y,\ell)\text{ s.t. } py \leq wL + \pi(p,w)
		\]
	\item Imposing labor market clearing (this is generally done automatically in step 2):
		\[
			\ell + L=\overline{L}
		\]
	\item Solving for the equilibirum price vector, $(p^*,w^*)$ by imposing the goods market clearing condition:
		\[
			y^d(p,w) = y^s(p,w)
		\]
\end{enumerate}

%%%________________________________________________________________%%%

\subsection{GE under Uncertainty}

\subsubsection{Arrow-Debreu Securities}
\begin{itemize}
	\item Again, let there be $L$ goods and $n$ traders, where goods are indexed by $\ell$ and traders indexed by $i$
	\item Suppose that one of $S$ states of the worls will be realize in time $t=1$
	\item Assume that in time ${t=0}$, the probability, $\pi_s$ of state ${s\in\{1,...,S\}}$ occuring in time ${t=1}$ is known
	\item Then, there is a market in time ${t=0}$ for \blue{state-contingent claims} to consumption of each good in each state: $x_{\ell s}$
	\item $p_\ell s$ is the price to a contingent claim (``Arrow security") to $\ell$ in state $s$\textemdash the market in time ${t=0}$ is for $L\times S$ forward contracts, which are binding agreements to buy/sell goods in the future, contingent on $s$
	\item Consumption vector for trader $i$ is ${x^i\in\R^{L\times S}}$
	\item \blue{Consumer problem}: Suppose $S=2$ and $L=1$. Then consumer $i$'s problem is:
		\[
			\usmax{x^i_1,x^i_2}\pi_1u^i(x^i_1) + \pi_2u^i(x^i_2)
		\]
	\item \blue{Intensive margin risk-smoothing:} Suppose wealth in state 2 (occuring with probability $\pi$) is given by $q$ and costs $p$ at time ${t=0}$. Then, the consumer's problem is:
		\[
			\usmax{q\geq0}\pi u(w-L+q-pq) + (1-\pi)u(w-pq)
		\]
		Where the FOC shows that $q^*=L$ under actuarially fair insurance ($p=\pi$), but $q<L$ (the consumer under-insures) if $p>\pi$ and the consumer is risk-averse
\end{itemize}

\subsubsection{Risk Bearing}
\begin{itemize}
	\item Setup:
		\begin{itemize}
			\item Endowment is state-dependent: $\olx^i\in\R_+^S$
			\item Expected utility is given by $U(x^i_1,...,x^i_S) = \sum_{s=1}^S\pi_su(x^i_s)$
		\end{itemize}
	\item \blue{Fundamental Theorem of Risk Bearing:}
		\[
			\frac{\pi_1u'(x_1)}{p_1} = ... = \frac{\pi_Su'(x_S)}{p_S}
		\]
		\begin{itemize}
			\item Derived from FOC of consumer problem: ${\lambda = \frac{\pi_su'(x_s)}{p_s}\text{ }\forall s}$
			\item Implication: the price of a state-contingent security rises in proportion to the likelihood of the state
		\end{itemize}
\end{itemize}

\subsubsection{Idiosyncratic vs. Aggregate Risk }
\begin{itemize}
	\item Idiosyncratic risk: ${\olx_s = \sum_{i=1}^n\olx_s^i=\olx_r \text{ }\forall r,s\in\{1,...,S\}}$
		\begin{itemize}
			\item Traders fully insure
			\item Fair prices reflect probabilities of states: $\frac{p_1}{p_2}=\frac{\pi_1}{\pi_2}$
			\item Two-agent, two-state equilibrium: simple exchange economy where $x_1$ and $x_2$ are treated as two different goods
		\end{itemize}
	\item Aggregate risk: ${\exists r\neq s: \olx_r<\olx_s}$
		\begin{itemize}
			\item Traders share risk, and disaster state insurance premiums exceed distaster state probability 
			\item Two agent, two-state example: ${\frac{p_2}{p_1}=\left(\frac{\olx_1}{\olx_2}\right)\left(\frac{\pi_2}{\pi_1}\right)>\frac{\pi_2}{\pi_1}}$
		\end{itemize}
\end{itemize}
		
\subsubsection{Imperfect Information}
\red{Reference: MWG Section 19.H}
\begin{itemize}
	\item Suppose exact probabilities are unknown, such that each consumer $i$ believes that state $s$ occurs with probability $\pi_s^i$, where it is not necessarily the case that ${\pi_s^i=\pi_s^j}$ for ${i\neq j}$
		\begin{itemize}
			\item Two-state example: Suppose all consumers have log utility. In order for the market for wealth in state 2, $x$, to clear, market excess demand must equal zero:
				\begin{enumerate}
					\item Consumers solve:
						\[
							\usmax{x_i}\pi_i\loge{w_i + x_i(1-p)} + (1-\pi_i)\loge{w_i-x_ip}\Rightarrow x_i(p) = w_i\left(\frac{\pi_i-p}{p(1-p)}\right)
						\]
					\item Assume $w_i=w\text{ }\forall i$. Then,
						\[
							ED(p) =0\Rightarrow \sum_{\pi_i>p}(\pi_i-p)=\sum_{\pi_i\leq p}(p-\pi_i)\Rightarrow p=\frac{1}{n}\sumn\pi_i
						\]
				\end{enumerate}
		\end{itemize}
	\item \blue{Rational Expectations Equilibrium}
		\begin{itemize}
			\item Suppose that, instead of state $s$ being revealed in time $t=1$, each consumer $i$ receives a signal, $\sigma_i(s)$, where they can distinguish between $s$ and $s'$ if only if ${\sigma_i(s)\neq\sigma_i(s')}$. As a result, it is possible for one consumer to have an informative signal (i.e. know the true state of the world) and another to require an averaging across states
				\begin{itemize}
					\item \red{MWG Example 19.H.2 shows that equilibrium prices can no longer convey information about states of the world; once the information is conveyed, the uninformed consumer infers the state of the world and changes their demand, upsetting the equilibrium}
				\end{itemize}
			\item \blue{Definition} (\red{MWG 19.H.2}): The price function $p(\cdot)$ is a rational expectations equilibrium price function if, for every $s$, $p(s)$ clears the spot market when every consumer $i$ knows that ${s\in E_{p(s),\sigma_i(s)}}$ and, therefore, evaluates commodity bundles ${x^i\in\R^L}$ according to the updated utility function:
				\[
					\sum_{s'=1}^SPr(s'|p(s),\sigma_i(s))u_{s'}^i(x^i)
				\]
				Where:
				\[
					E_{p(s),\sigma_i(s)} = \{s'|p(s')=p(s),\sigma_i(s')=\sigma_i(s)\}
				\]
			\item \red{Lones's explanation of the Rational Expectations Equilibrium (REE) is confusing. His example is explained in MWG 19.H.3. The explanation is here:}
				\begin{itemize}
					\item In an REE, it must be the case that either:
						\begin{enumerate}[(a)]
							\item the price is equal across the two states, such that the information never gets revealed, resulting in a pooled price equilibrium across the two states
							\item the price differs by state such that the information about the state is revealed (in which case, there is a pooled price equilibrium across the two states)
						\end{enumerate}
					\item In the Kreps example from lecture (and MWG 19.H.3), if Joe learns from the price and adjusts his demand, the pooled price that doesn't differ by state is different from the price where he doesn't learn. If he uses a non-state-dependent demand, then there is a state-dependent price that does not clear the market. Thus, a REE doesn't exist in this market
				\end{itemize}
		\end{itemize}
\end{itemize}
		
%%%________________________________________________________________%%%


\end{document}