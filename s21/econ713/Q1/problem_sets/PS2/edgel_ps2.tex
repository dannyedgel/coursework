%%% Econ713: Microeconomics II
%%% Spring 2021
%%% Danny Edgel
%%%
% Due on Canvas Saturday, March 6th, 11:59pm Central Time
%%%

%%%
%							PREAMBLE
%%%

\documentclass{article}

%%% declare packages
\usepackage{amsmath}
\usepackage{amssymb}
\usepackage{array}
\usepackage{bm}
\usepackage{changepage}
\usepackage{centernot}
\usepackage{graphicx}
\usepackage{multirow}
\usepackage[shortlabels]{enumitem}
\usepackage{fancyhdr}
	\fancyhf{} % sets both header and footer to nothing
	\renewcommand{\headrulewidth}{0pt}
    \rfoot{Edgel, \thepage}
    \pagestyle{fancy}
	
%%% define shortcuts for set notation
\newcommand{\N}{\mathbb{N}}
\newcommand{\Z}{\mathbb{Z}}
\newcommand{\R}{\mathbb{R}}
\newcommand{\Q}{\mathbb{Q}}
\newcommand{\lmt}{\underset{x\rightarrow\infty}{\text{lim }}}
\newcommand{\neglmt}{\underset{x\rightarrow-\infty}{\text{lim }}}
\newcommand{\zerolmt}{\underset{x\rightarrow 0}{\text{lim }}}
\newcommand{\usmax}[1]{\underset{#1}{\text{max }}}
\newcommand{\usmin}[1]{\underset{#1}{\text{min }}}
\newcommand{\intersect}{\bigcap}
\newcommand{\union}{\bigcup}
\newcommand{\olw}{\overline{w}}
\newcommand{\olx}{\overline{x}}
\newcommand{\loge}[1]{\text{log}\left(#1\right)}
\renewcommand{\P}{\mathcal{P}}
\renewcommand{\L}{\mathcal{L}}
\newcommand{\olp}{\overline{p}}
\renewcommand{\exp}[1]{\text{exp}\left\{#1\right\}}
\newcommand{\binv}[1]{b_j^{-1}\left(#1\right)}

\DeclareMathOperator{\E}{\mathbb{E}}% expected value

%%% define column vector command (from Michael Nattinger)
\newcount\colveccount
\newcommand*\colvec[1]{
        \global\colveccount#1
        \begin{pmatrix}
        \colvecnext
}
\def\colvecnext#1{
        #1
        \global\advance\colveccount-1
        \ifnum\colveccount>0
                \\
                \expandafter\colvecnext
        \else
                \end{pmatrix}
        \fi
}

%%% define function for drawing matrix augmentation lines
\newcommand\aug{\fboxsep=-\fboxrule\!\!\!\fbox{\strut}\!\!\!}

\makeatletter
\let\amsmath@bigm\bigm

\renewcommand{\bigm}[1]{%
  \ifcsname fenced@\string#1\endcsname
    \expandafter\@firstoftwo
  \else
    \expandafter\@secondoftwo
  \fi
  {\expandafter\amsmath@bigm\csname fenced@\string#1\endcsname}%
  {\amsmath@bigm#1}%
}


%________________________________________________________________%

\begin{document}

\title{	Homework \#2 }
\author{ 	Danny Edgel 					\\ 
			Econ 713: Microeconomics II		\\
			Spring 2021						\\
		}
\maketitle\thispagestyle{empty}

%\noindent\textit{Collaborated with Sarah Bass, Emily Case, Michael Nattinger, and Alex Von Hafften}

%%%________________________________________________________________%%%

\subsection*{Question 1}

In the long run, each firm will only produce if the price is above its long run average cost. All firms have identical marginal cost curves, but variable long run average costs. The firm with the lowest long run average cost is firm ${k=1}$, which will have positive output if the price is strictly greater than 1, assuming the good is divisible. If goods can only be produced in discrete quantities, then there will only be positive output if the price is weakly greater than 3.

%%%________________________________________________________________%%%

\subsection*{Question 2}

The total cost and total revenue of an individual firm in this market, as a function of output $q$, is:
\begin{align*}
	&TC(q) = 1 + \frac{q^2}{4\theta^2} 		&TR(q) = pq = q
\end{align*}

\begin{enumerate}[(a)]
	\item Since there is a continuum of price-taking firms, this market is competitive, with each firm choosing $q$ to maximize total profit:
		\[
			\usmax{q}(1-\tau) pq - 1 - \frac{q^2}{4\theta^2}
		\]
		From the first-order condition of this problem, we can derive the firm supply curve:
		\begin{align*}
			(1-\tau)p - \frac{q}{2\theta^2} &= 0	\\
			q &= 2\theta^2(1-\tau)p
		\end{align*}
		We can determine the mass of developers that supply this market by finding $\theta$ of the marginal developer:
		\begin{align*}
			\frac{q}{2\theta^2} 				&= \frac{1}{q} + \frac{q}{4\theta^2}									\\
			\frac{2\theta^2(1-\tau)}{2\theta^2} &= \frac{1}{2\theta^2(1-\tau)} + \frac{2\theta^2(1-\tau)}{4\theta^2}	\\
										1-\tau	&= \frac{1}{2\theta^2(1-\tau)} + \frac{(1-\tau)}{2}						\\
									\theta^* 	&= \frac{1}{1-\tau}
		\end{align*}
		The aggregate supply curve is derived by summing each firm's suppy curve, where ${m(\theta)=\beta\theta^{-\beta-1}}$ is the probability mass function of $\theta$:
		\begin{align*}
			Q(p) 	&= \int_{\theta^*}^\infty m(\theta) 2\theta^2(1-\tau) pd\theta = 2(1-\tau)\beta p\int_{\theta^*}^\infty \theta^{1-\beta}d\theta	\\
					&= 2(1-\tau)\beta p\left[\frac{1}{2-\beta}\theta^{2-\beta}\right]^\infty_{\theta^*}												\\
			Q(p) 	&= 2\left(\frac{\beta}{\beta-2}\right)(1-\tau)^{\beta-1}p
		\end{align*}
	
	\item When apple raises its tax rate, the $\theta$ of the marginal developer, $\theta^*$, increases, decreasing the number of developers who supply the market. The per-firm supply also decreases, by $2\theta^2\Delta\tau$.
	
	\item Apple's revenue-maximizing tax is found by either maximizing its revenue or finding the value of $\tau$ for which the elasticity of supply with respect to $\tau$ is equal to negative one:
		\begin{align*}
																															\frac{dQ}{d\tau}\frac{\tau}{Q}	&= -1		\\
		\left(\frac{-2\left(\frac{\beta}{\beta-2}\right)(\beta-1)(1-\tau)^{\beta-2}p}{2\left(\frac{\beta}{\beta-2}\right)(1-\tau)^{\beta-1}p}\right)\tau 	&= -1		\\
																																			(\beta-1)\tau	&= 1-\tau	\\
																																			\tau	&= \frac{1}{\beta}
		\end{align*}
	
	\item Holding $\tau$ constant, an increase in $\beta$ does not change $\theta^*$, so the same ``number" of developers supply the market. However, a higher value of $\beta$ is associated with a steeper pmf of $\theta$, with lower mass for higher values of $\theta$. This decreases the heterogeneity of developers, thus decreasing overall surplus in the market.
	
\end{enumerate}

%%%________________________________________________________________%%%

\subsection*{Question 3}

Denote the parameters and price and quantity variables for each flavor with subscript $i\in\{T,F\}$, for triple-chocolate-chunk and fruit-bowl-punch, respectively. Since Ben and Jerry's is a monopoly, it chooses $(Q_i,P_i)$ such that:
\[
	\usmax{Q_i}(a_i-b_iQ_i - c)Q_i
\]
Where $c$ is the constant marginal cost for each flavor. Then, the profit-maximizing price and quantity for each flavor is:
\begin{align*}
	&Q^* = \frac{a_i-c}{2b_i} 	&P^* = \frac{1}{2}(a_i + c)
\end{align*}
Since we know that $F$ has a higher price intercept than $T$, we also know that Ben and Jerry's will charge a higher price for it.


%%%________________________________________________________________%%%

\subsection*{Question 4}

\begin{enumerate}[(a)]
	\item See the graph below.
		\begin{center}
			\includegraphics[scale=.4]{4a.png}
		\end{center}
	
	\item See the graph below.
		\begin{center}
			\includegraphics[scale=.4]{4b.png}
		\end{center}
	
	\item See the graph below. Displayed is the marginal revenue curve for a demand curve that is linear but for quantities between 4 and 5, over which demand is perfectly elastic.
		\begin{center}
			\includegraphics[scale=.4]{4c.png}
		\end{center}
	
\end{enumerate}



%%%________________________________________________________________%%%
\pagebreak
\subsection*{Question 5}
Assuming that the firm has monopoly pricing power and is able to price discriminate between the two groups, the firm should charge each group according to the monopoly pricing formula that I derived in question 3: it should charge group 1 a price of \$2 and group 2 a price of \$3.
\medskip \\
If marginal cost increases proportionally to quantity, then the firm can no longer independently optimize for each group of shoppers. This is because the monopolist pricing rule is to choose $Q$ such that marginal revenue is equal to marginal cost, then set $P$ that clears demand at $Q$. However, if $MC=Q$, then choosing $Q_i$ for group ${i\in\{1,2\}}$ will change the value $Q_j$ for group ${i\neq j}$ that equates marginal revenue to marginal cost.\footnote{In simpler terms, the marginal cost of production for serving group $i$ is not constant in the quantity of production for group ${j\neq i}$.} The firm can still price discriminate between the two groups, but it now must consider a joint marginal revenue curve:
\[
	MR(Q_1 + Q_2) = \text{max}\left\{MR_1(Q_1),MR_2(Q_2)\right\}
\]
Since each group's demand curve has the same slope but different intercepts, this curve is just equal to $MR_2(Q)$ until ${Q=1}$, at which point the firm alternates between supplying each group until its marginal cost is equal to the marginal revenue of either curve. The table below displays the progression of each curve in the optimization problem, assuming the good can only be sold in discrete quantities:
\begin{center}
	\begin{tabular}{c|ccc|ccc|c}
		$Q$ & $Q_1$ & $MR_1$ 	& $P_1$	& $Q_2$ & $MR_2$	& $P_2$	& $MC$	\\\hline
		1	& 0		& -			& 3		& 1		& 4		 	& 4		& 1		\\
		2 	& 0		& -			& 3		& 2		& 2		 	& 3		& 2		\\
		3	& 1		& 2			& 2		& 2		& -			& 3		& 3
	\end{tabular}
\end{center}
So the firm either chooses to supply two units to the second group or one unit to each group, in either case, netting \$3 of profit. In the first case, the firm charges \$3 to the second group. In the second case, the firm charges \$2 to the first group and \$4 to the second.


%%%________________________________________________________________%%%


\end{document}
