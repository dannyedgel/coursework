%%% Econ714: Macroeconomics II
%%% Spring 2021
%%% Danny Edgel
%%%
% Due on Canvas Monday, March 8th, 11:59pm Central Time
%%%

%%%
%							PREAMBLE
%%%

\documentclass{article}

%%% declare packages
\usepackage{amsmath}
\usepackage{amssymb}
\usepackage{array}
\usepackage{bm}
\usepackage{changepage}
\usepackage{centernot}
\usepackage{graphicx}
\usepackage{xcolor}
\usepackage[shortlabels]{enumitem}
\usepackage{fancyhdr}
	\fancyhf{} % sets both header and footer to nothing
	\renewcommand{\headrulewidth}{0pt}
    \rfoot{Edgel, \thepage}
    \pagestyle{fancy}
	
%%% define shortcuts for set notation
\newcommand{\Z}{\mathbb{Z}}
\newcommand{\R}{\mathbb{R}}
\newcommand{\Q}{\mathbb{Q}}
\newcommand{\lmt}{\underset{x\rightarrow\infty}{\text{lim }}}
\newcommand{\neglmt}{\underset{x\rightarrow-\infty}{\text{lim }}}
\newcommand{\zerolmt}{\underset{x\rightarrow 0}{\text{lim }}}
\newcommand{\loge}[1]{\text{log}\left(#1\right)}
\newcommand{\usmax}[1]{\underset{#1}{\text{max }}}
\newcommand{\usmin}[1]{\underset{#1}{\text{min }}}
\newcommand{\Mt}{M_{t+1}^t}
\newcommand{\vhat}{\hat{v}}
\newcommand{\olp}{\overline{p}}
\renewcommand{\L}{\mathcal{L}}
\newcommand{\olq}{\overline{q}}
\newcommand{\zinf}{_{t=0}^\infty}
\newcommand{\aneg}{A^{-1}}
\newcommand{\sneg}{s^{-1}}
\newcommand{\olk}{\overline{k}}
\newcommand{\olc}{\overline{c}}
\newcommand{\olr}{\overline{r}}
\newcommand{\olpi}{\overline{\pi}}
\newcommand{\Aneg}{A^{-1}}
\renewcommand{\sneg}{s^{-1}}
\newcommand{\dc}[1]{\Delta c_{#1}}
\newcommand{\N}{\mathcal{N}}
\newcommand{\suminf}{\sum_{t=0}^\infty}
\newcommand{\sumn}{\sum_{i=1}^{n}}
\newcommand{\sumnk}{\sum_{i=1}^{N_k}}
\newcommand{\red}[1]{{\color{red}#1}}
\newcommand{\Tau}{\mathrm{T}}
\newcommand{\phat}{\hat{p}}

\newcommand{\E}[1]{\mathbb{E}\left[#1\right]} % expected value
\newcommand{\Et}[1]{\mathbb{E}_t\left[#1\right]}

%%% define column vector command (from Michael Nattinger)
\newcount\colveccount
\newcommand*\colvec[1]{
        \global\colveccount#1
        \begin{pmatrix}
        \colvecnext
}
\def\colvecnext#1{
        #1
        \global\advance\colveccount-1
        \ifnum\colveccount>0
                \\
                \expandafter\colvecnext
        \else
                \end{pmatrix}
        \fi
}

%%% define function for drawing matrix augmentation lines
\newcommand\aug{\fboxsep=-\fboxrule\!\!\!\fbox{\strut}\!\!\!}

\makeatletter
\let\amsmath@bigm\bigm

\renewcommand{\bigm}[1]{%
  \ifcsname fenced@\string#1\endcsname
    \expandafter\@firstoftwo
  \else
    \expandafter\@secondoftwo
  \fi
  {\expandafter\amsmath@bigm\csname fenced@\string#1\endcsname}%
  {\amsmath@bigm#1}%
}


%________________________________________________________________%

\begin{document}

\title{	Problem Set \#6 }
\author{ 	Danny Edgel 					\\ 
			Econ 714: Macroeconomics II		\\
			Spring 2021						\\
		}
\maketitle\thispagestyle{empty}

%%%________________________________________________________________%%%

\noindent\textit{Discussed and/or compared answers with Sarah Bass, Emily Case, Katherine Kwok, Michael Nattinger, and Alex Von Hafften}
 \\

%%%________________________________________________________________%%%

\subsection*{Questions 1}

The monetary policy authority faces the following problem:
\begin{align*}
	&\usmin{\{x_t,\pi_t,i_t\}}\frac{1}{2}\E{\sum_{t=0}^\infty\beta^t\left(x_t^2 + \alpha\pi_t^2\right)} \text{, s.t. } 
		&\sigma\Et{\Delta x_{t+1}} = i_t - \Et{\pi_{t+1}} - r^n_t	\\
	&	&\pi_t = \kappa x_t + \beta\Et{\pi_{t+1}} + u_t
\end{align*}
Using the primal approach, we can optimize the Lagrangian, considering only the NKPC constraint:
\[
	\L = -\E{\frac{1}{2}\sum_{t=0}^\infty\beta^t\left(x_t^2 + \alpha\pi_t^2\right) - \lambda_t\left(\pi_t - \kappa x_t - \beta\pi_{t+1} - u_t\right)}
\]
Which has the following first order conditions:
\begin{align*}
	\frac{\partial\L}{\partial x_t} 	&= -\beta^tx_t + \kappa\lambda_t 						= 0	\\
	\frac{\partial\L}{\partial \pi_t} 	&= \begin{cases} 
												-\beta^t\alpha\pi_t - \lambda_t + \beta\lambda_{t-1} 	= 0, &t\geq 1	\\
												-\beta^t\alpha\pi_t - \lambda_t							= 0, &t=    0
											\end{cases}
\end{align*}
Combining these FOCs enables us to derive an optimal policy rule:
\begin{align*}
	&\alpha\kappa\pi_t + \Delta x_t = 0\text{, }t\geq 1		&\alpha\kappa\pi_0 + x_0 = 0
\end{align*}
Let $x_{-1}=p_{-1}=0$; then, we can represent the optimal rule as a single equation:
\[
	\alpha\kappa\pi_t + \Delta x_t = 0
\]
Since this holds for all $t$, we can prove via induction that ${\alpha\kappa p_t + x_t = 0}$:
\begin{align*}
	\alpha\kappa(p_0 - p_{-1}) + (x_0 - x_{-1}) &= \alpha\kappa p_0 + x_0 = 0	\\
	\alpha\kappa(p_t - p_{t-1}) + (x_t - x_{t-1}) &= \alpha\kappa p_t + x_t - (\alpha\kappa p_{t-1} + x_{t-1}) = 0
\end{align*}
We can use this optimal policy rule and the NKPC (adjusted to use $p_t-p_{t-1}$ instead of $\pi$) to contruct a linear system from which to solve for equilibrium dynamics:
\begin{align*}
	-\beta\E{p_{t+1}} + p_t - p_{t-1} -\kappa(-\alpha\kappa p_t) &= u_t 	\\
	-\beta\E{p_{t+1}} &= -(1 + \beta + \alpha\kappa^2)p_t + p_{t-1} + u_t 	
\end{align*}
\begin{align*}
	\Rightarrow \colvec{2}{\E{p_{t+1}}}{p_t} 
		&= 	\begin{pmatrix}
				1+\frac{1}{\beta}+\frac{\alpha\kappa^2}{\beta}	& -\frac{1}{\beta} \\
				1												& 0 
			\end{pmatrix}\colvec{2}{p_t}{p_{t-1}} + \colvec{2}{-\frac{1}{\beta}}{0}u_t
\end{align*}
To determine equilibrium dynamics in this model, we must find the eigenvalues of the matrix in this linear system:
\begin{align*}
	(1+\frac{1}{\beta}+\frac{\alpha\kappa^2}{\beta}-\lambda)(-\lambda) + \frac{1}{\beta}	&= 0	\\
	\lambda^2 - (1+\frac{1}{\beta}+\frac{\alpha\kappa^2}{\beta})\lambda + \frac{1}{\beta}	&= 0
\end{align*}
Because this system has one state and one choice variable, $\lambda_1>1$ and $\lambda_2<1$, where $\lambda_1$ is the eigenvalue associated with $\E{p_{t+1}}$. Without paying too much mind to the exact values of $\lambda_1$ and $\lambda_2$ and omitting intermediate (and tedious) steps, we can find:
\begin{align*}
	\lambda 			&= \frac{1}{2\beta}\left[1+\beta+\alpha\kappa^2 \pm \sqrt{(1+\beta+\alpha\kappa^2}-4\beta\right]	\\
	\lambda_1\lambda_2 	&= \frac{1}{4\beta^2}\left[(1+\beta+\alpha\kappa^2)^2 - (1+\beta+\alpha\kappa^2)^2 + 4\beta\right]	\\
						&= \frac{1}{\beta}
\end{align*}
Furthermore, we can see that ${\beta(\lambda_1 + \lambda_2) = 1+\beta+\alpha\kappa^2}$. This enables us to write the NKPC with just our eigenvalues and lag operators:
\begin{align*}
	-\beta(1-\lambda_1L)(1-\lambda_2L)L^{-1}p_t 				&= u_t																			\\
	\left(\beta\lambda_1-\beta L^{-1}\right)(1-\lambda_2L)p_t 	&= u_t																			\\
										p_t - \lambda_2p_{t-1}	&= \left(\frac{1}{\lambda_2}-\beta L^{-1}\right)^{-1}u_t 						\\
															p_t	&= \left(\frac{\lambda_2}{1-\beta\lambda_2L^{-1}}\right)u_t + \lambda_2p_{t-1}
\end{align*}
Since we are given the distribution of the markup shock $u_t$, we can determine solve for $p_t$ at any given $t$, with past realizations accounted for in $p_{t-1}$ and expected future realizations given by the distribution of $u_t$:
\begin{align*}
	p_t &= \lambda_2p_{t-1} + \lambda_2\Et{\sum_{j=0}^\infty(\lambda_2\beta)^ju_{t+j}}										\\
		&= \lambda_2p_{t-1} + \lambda_2\left(u_t + \sum_{j=1}^\infty(\lambda_2\beta)^j\Et{u_{t+j}}\right)					\\
		&= \lambda_2p_{t-1} + \lambda_2\left(u_t + \left(\frac{\lambda_2\beta}{1-\lambda_2\beta}\right)\overline{u}\right)	\\
	p_t &= \lambda_2(p_{t-1} + u_t) + \left(\frac{\lambda_2}{\lambda_1-1}\right)\overline{u}					
\end{align*}
Recalling our equation for the output gap, this equation can be used to describe the dynamics of $x_t$, as well:
\[
	x_t = \lambda_2x_{t-1} - \lambda_2\alpha\kappa u_t - \left(\frac{\lambda_2\alpha\kappa}{\lambda_1-1}\right)\overline{u}
\]

%%%________________________________________________________________%%%

\subsection*{Question 2}
Under a discretionary policy, the planner can ensure that ${\alpha\kappa\pi_t + x_t = 0}$ in every period. Then, since the NKPC holds each period, we can solve:
\begin{align*}
	\pi_t 	&= \kappa x_t + \beta\Et{\pi_{t+1}} + u_t																						\\
	\pi_t	&= -\alpha\kappa^2\pi_t + \beta\Et{\pi_{t+1}} + u_t																				\\
	\pi_t	&= \frac{1}{1 + \alpha\kappa^2}\left(\beta\Et{\pi_{t+1}} + u_t\right)															\\
	\pi_t	&= \frac{1}{1 + \alpha\kappa^2}\sum_{j=0}^\infty \left(\frac{\beta}{1 + \alpha\kappa^2}\right)^j\E{u_{t+j}}						\\
	\pi_t	&= \left(\frac{1}{1 + \alpha\kappa^2}\right)u_t + \sum_{j=0}^\infty \left(\frac{\beta}{1 + \alpha\kappa^2}\right)^j\E{u_{t+j}}	\\
	\pi_t	&= \left(\frac{1}{1 + \alpha\kappa^2}\right)u_t + \left(\frac{\beta}{(1 + \alpha\kappa^2)(1 -\beta + \alpha\kappa^2)}\right)\overline{u}
\end{align*}
Applying this to the optimal policy rule yields our equation for the output gap:
\[
	x_t = -\left(\frac{\alpha\kappa}{1 + \alpha\kappa^2}\right)u_t - \left(\frac{\beta\alpha\kappa}{1 -\beta + \alpha\kappa^2}\right)\overline{u}
\]



%%%________________________________________________________________%%%

\subsection*{Question 3}
Under the $\pi_t=0$ rule, the NKPC yields the equilibrium allocation:
\[
	\pi_t = \kappa x_t + \beta\Et{\pi_{t+1}} + u_t \Rightarrow x_t = -\frac{u_t}{\kappa}
\]



%%%________________________________________________________________%%%

\subsection*{Question 4}
Similar to in question 3, we can determine the equilibrium allocation by setting ${x_t=0}$ in the NKPC:
\begin{align*}
	\pi_t &= \beta\Et{\pi_{t+1}} + u_t = \sum_{j=0}^\infty\beta^j\E{u_{t+j}}	\\
	\pi_t &= u_t + \left(\frac{\beta}{1-\beta}\right)\overline{u}
\end{align*}



%%%________________________________________________________________%%%

\subsection*{Question 5}
To determine under which circumstances one policy is preferable to the other, we must first determine the expected welfare losses under each policy:, letting $W_\pi$ and $W_d$ denote welfare losses under an inflation-targeting and discretionary policy, respectively:
\[
	W_\pi 	= \frac{1}{2}\E{\sum_{t=0}^\infty\beta^t\left(-\frac{u_t}{\kappa}\right)^2}
			= \frac{\sigma^2+\overline{u}^2}{2\kappa^2(1-\beta)}
\]
{\footnotesize \begin{align*}
	W_d		&= \frac{1}{2}\E{\sum_{t=0}^\infty\beta^t\left((-\alpha\kappa\pi_t)^2 + \alpha\pi_t^2\right)}	\\
			&= \frac{\alpha(1+\alpha\kappa^2)}{2}\E{\sum_{t=0}^\infty\beta^t
					\left(\left(\frac{1}{1 + \alpha\kappa^2}\right)u_t + \left(\frac{\beta}{(1 + \alpha\kappa^2)(1 -\beta + \alpha\kappa^2)}\right)\overline{u}\right)^2
				}		\\
			&= \frac{\alpha(1+\alpha\kappa^2)}{2(1-\beta)}\left[
					\left(\frac{1}{(1 + \alpha\kappa^2)^2}\right)\Et{u_t^2} + 
					2\left(\frac{\beta}{(1 + \alpha\kappa^2)^2(1 -\beta + \alpha\kappa^2)}\right)\Et{u_t}\overline{u} +
					\left(\frac{\beta^2}{(1 + \alpha\kappa^2)^2(1 -\beta + \alpha\kappa^2)^2}\right)\overline{u}^2
				\right]		\\
			&= \frac{\alpha}{2(1-\beta)(1 + \alpha\kappa^2)^2}\left[
					\sigma^2+\overline{u}^2 + 
					\left(\frac{2\beta}{1 -\beta + \alpha\kappa^2} + \frac{\beta^2}{(1 -\beta + \alpha\kappa^2)^2}\right)\overline{u}^2
				\right]		\\
			&= \frac{\alpha}{2(1-\beta)(1+\alpha\kappa^2)}\left[
					\sigma^2 + \left(
							1 + \frac{2\beta}{1 -\beta + \alpha\kappa^2} 
							+ \frac{\beta^2}{(1 -\beta + \alpha\kappa^2)^2}
							\right)\overline{u}^2
				\right]		\\
			&= \frac{\alpha}{2(1-\beta)(1+\alpha\kappa^2)}\left[
					\sigma^2 + \left(
							\frac{1+2\alpha\kappa^2 + \alpha^2\kappa^4}{(1 -\beta + \alpha\kappa^2)^2}
							\right)\overline{u}^2
				\right]	
\end{align*} }
The social planner prefers an inflation-targeted policy if $W_\pi\leq W_d$:
\begin{align*}
	\frac{\sigma^2+\overline{u}^2}{2\kappa^2(1-\beta)}	&\leq 
		\frac{\alpha}{2(1-\beta)(1+\alpha\kappa^2)}\left[
					\sigma^2 + \left(
							\frac{1+2\alpha\kappa^2 + \alpha^2\kappa^4}{(1 -\beta + \alpha\kappa^2)^2}
							\right)\overline{u}^2 \right]	\\
	\left(\frac{1+\alpha\kappa^2}{\kappa^2}\right)\sigma^2-\alpha\sigma^2 &\leq	
		\left[\frac{\alpha(1+2\alpha\kappa^2 + \alpha^2\kappa^4)}{(1 -\beta + \alpha\kappa^2)^2} - \frac{1+\alpha\kappa^2}{\kappa^2}\right]\overline{u}^2	\\
	\sigma^2 &\leq \left[\frac{\alpha\kappa^2(1+2\alpha\kappa^2 + \alpha^2\kappa^4)}{(1 -\beta + \alpha\kappa^2)^2} - 1-\alpha\kappa^2\right]\overline{u}^2
\end{align*}
At the limit as $\beta\rightarrow 1$, this inequality simplifies cleanly:
\begin{align*}
	\sigma^2 	&\leq \left[\frac{\alpha\kappa^2(1+2\alpha\kappa^2 + \alpha^2\kappa^4)}{(\alpha\kappa^2)^2} - 1-\alpha\kappa^2\right]\overline{u}^2	\\
				&\leq \left[\frac{(\alpha\kappa^2)^2((\alpha\kappa^2)^{-2} + 2 + \alpha\kappa^2)}{(\alpha\kappa^2)^2} - 1-\alpha\kappa^2\right]\overline{u}^2				\\
				&\leq \left((\alpha\kappa^2)^{-2} 2 + \alpha\kappa^2 - 1-\alpha\kappa^2\right)\overline{u}^2																\\
	\sigma^2 	&\leq \frac{\overline{u}^2}{1 + \alpha^2\kappa^4}	
\end{align*}


%%%________________________________________________________________%%%

\subsection*{Question 6}
Following the same steps and logic as in question 5, we can determine the necessary relationship between $\sigma^2$ and $\overline{u}^2$ such that an output-targeting montary policy is optimal. First, we must find the welfare loss from an output policy, $W_x$:
\begin{align*}
	W_x	&= \frac{1}{2}\E{\sum_{t=0}^\infty\beta^t\alpha\left(u_t + \left(\frac{\beta}{1-\beta}\right)\overline{u}\right)^2}	\\
		&= \frac{\alpha\beta}{2(1-\beta)}\left[\Et{u_t^2} + 2\left(\frac{\beta}{1-\beta}\right)\overline{u}\Et{u_t} + \left(\frac{\beta}{1-\beta}\right)^2\overline{u}^2\right]	\\
		&= \frac{\alpha\beta}{2(1-\beta)}\left[\sigma^2 + \left(\frac{2\beta}{1-\beta} + \frac{\beta^2}{(1-\beta)^2}\right)\overline{u}^2\right]	\\
	W_x	&= \frac{\alpha\beta}{2(1-\beta)}\left[\sigma^2 + \left(\frac{\beta(2-\beta)}{(1-\beta)^2}\right)\overline{u}^2\right]
\end{align*}
Then,
\begin{align*}
	W_x	&\leq W_\pi \\
	\frac{\alpha\beta}{2(1-\beta)}\left[\sigma^2 + \left(\frac{\beta(2-\beta)}{(1-\beta)^2}\right)\overline{u}^2\right]
		&\leq \frac{\sigma^2+\overline{u}^2}{2\kappa^2(1-\beta)}	\\
	\alpha\beta\left[\sigma^2 + \left(\frac{\beta(2-\beta)}{(1-\beta)^2}\right)\overline{u}^2\right]
		&\leq \frac{\sigma^2+\overline{u}^2}{\kappa^2}	\\
	(\alpha\beta\kappa^2-1)\sigma^2 &\leq \left[1 - \frac{2\kappa^2\beta(2-\beta)}{(1-\beta)^2}\right]\overline{u}^2	\\
	\sigma^2 \leq \left(\frac{1 - \frac{2\kappa^2\beta(2-\beta)}{(1-\beta)^2}}{\alpha\beta\kappa^2-1}\right)\overline{u}^2
\end{align*}
At the limit of $\beta\rightarrow 1$, this converges to:
\[
	\sigma^2 \leq \frac{\overline{u}^2}{\alpha\kappa^2-1}
\]
Thus, if $\alpha\kappa^2<1$, output targeting is never preferred to inflation targeting.

%%%________________________________________________________________%%%

\subsection*{Question 7}
Recall the final two equations from question 1, which decribe the dynamics of inflation and output in this model:
\begin{align*}
	p_t &= \lambda_2(p_{t-1} + u_t) + \left(\frac{\lambda_2}{\lambda_1-1}\right)\overline{u}									\\
	x_t &= \lambda_2x_{t-1} - \lambda_2\alpha\kappa u_t - \left(\frac{\lambda_2\alpha\kappa}{\lambda_1-1}\right)\overline{u}
\end{align*}
Setting ${u_t=\overline{u}=0}$ for all $t$, the value of $p$ and $x$ in any period $t$ are:
\begin{align*}
	&p_t = \lambda_2^tp_{-1}	&x_t = \lambda_2^tx_{-1}
\end{align*}
Then ${p_{-1}=x_{-1}=0}$ would give a first-best allocation. From the NKPC, we know that ${\Delta x_t=-\alpha\kappa\pi_t}$. Then, by the NKIS, we can determine $\pi_t$ under the Taylor rule:
\begin{align*}
	\sigma\Et{\Delta x_{t+1}} &= i_t - \Et{\pi_{t+1}} - r^n_t						\\
	-\alpha\kappa\sigma\Et{\pi_{t+1}} &= \phi\pi_t - \Et{\pi_{t+1}} - r^n_t			\\
	\pi_t &= \frac{1}{\phi}\left[(1-\alpha\kappa\sigma)\Et{\pi_{t+1}}+r^n_t\right]
\end{align*}
Thus, as $\phi\rightarrow\infty$, $\pi_t\rightarrow 0$ for all $t$, indicating that the first-best allocation is achieved.


%%%________________________________________________________________%%%





\end{document}






