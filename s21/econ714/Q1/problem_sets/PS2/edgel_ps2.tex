%%% Econ714: Macroeconomics II
%%% Spring 2021
%%% Danny Edgel
%%%
% Due on Canvas Monday, February 8th, 11:59pm Central Time
%%%

%%%
%							PREAMBLE
%%%

\documentclass{article}

%%% declare packages
\usepackage{amsmath}
\usepackage{amssymb}
\usepackage{array}
\usepackage{bm}
\usepackage{changepage}
\usepackage{centernot}
\usepackage{graphicx}
\usepackage[shortlabels]{enumitem}
\usepackage{fancyhdr}
	\fancyhf{} % sets both header and footer to nothing
	\renewcommand{\headrulewidth}{0pt}
    \rfoot{Edgel, \thepage}
    \pagestyle{fancy}
	
%%% define shortcuts for set notation
\newcommand{\Z}{\mathbb{Z}}
\newcommand{\R}{\mathbb{R}}
\newcommand{\Q}{\mathbb{Q}}
\newcommand{\lmt}{\underset{x\rightarrow\infty}{\text{lim }}}
\newcommand{\neglmt}{\underset{x\rightarrow-\infty}{\text{lim }}}
\newcommand{\zerolmt}{\underset{x\rightarrow 0}{\text{lim }}}
\newcommand{\loge}[1]{\text{log}\left(#1\right)}
\newcommand{\usmax}[1]{\underset{#1}{\text{max }}}
\newcommand{\Mt}{M_{t+1}^t}
\newcommand{\vhat}{\hat{v}}
\newcommand{\olp}{\overline{p}}
\renewcommand{\L}{\mathcal{L}}
\newcommand{\olq}{\overline{q}}
\newcommand{\zinf}{_{t=0}^\infty}
\newcommand{\aneg}{A^{-1}}
\newcommand{\sneg}{s^{-1}}
\newcommand{\olk}{\overline{k}}
\newcommand{\olc}{\overline{c}}
\newcommand{\olr}{\overline{r}}
\newcommand{\olpi}{\overline{\pi}}
\newcommand{\Aneg}{A^{-1}}
\renewcommand{\sneg}{s^{-1}}
\newcommand{\dc}[1]{\Delta c_{#1}}
\newcommand{\N}{\mathcal{N}}
\newcommand{\suminf}{\sum_{t=0}^\infty}

\newcommand{\E}[1]{\mathbb{E}\left[#1\right]} % expected value
\newcommand{\Et}[1]{\mathbb{E}_t\left[#1\right]}

%%% define column vector command (from Michael Nattinger)
\newcount\colveccount
\newcommand*\colvec[1]{
        \global\colveccount#1
        \begin{pmatrix}
        \colvecnext
}
\def\colvecnext#1{
        #1
        \global\advance\colveccount-1
        \ifnum\colveccount>0
                \\
                \expandafter\colvecnext
        \else
                \end{pmatrix}
        \fi
}

%%% define function for drawing matrix augmentation lines
\newcommand\aug{\fboxsep=-\fboxrule\!\!\!\fbox{\strut}\!\!\!}

\makeatletter
\let\amsmath@bigm\bigm

\renewcommand{\bigm}[1]{%
  \ifcsname fenced@\string#1\endcsname
    \expandafter\@firstoftwo
  \else
    \expandafter\@secondoftwo
  \fi
  {\expandafter\amsmath@bigm\csname fenced@\string#1\endcsname}%
  {\amsmath@bigm#1}%
}


%________________________________________________________________%

\begin{document}

\title{	Problem Set \#2 }
\author{ 	Danny Edgel 					\\ 
			Econ 714: Macroeconomics II		\\
			Spring 2021						\\
		}
\maketitle\thispagestyle{empty}

%%%________________________________________________________________%%%

\noindent\textit{Collaborated with Sarah Bass, Emily Case, Katherine Kwok, Michael Nattinger, and Alex Von Hafften}

%%%________________________________________________________________%%%

\subsection*{Question 1}
The social planner in this problem seeks to maximize utility subject to the production function, resource constraint, and law of motion:
\[
	\usmax{\{C_t,K_t,I_t\}_{t=0}^\infty}\suminf\beta^t\loge{C_t}\text{, s.t. }Y_t=AK_t^\alpha\text{, }K_{t+1}=K_t^{1-\delta}I_t^\delta\text{, }Y_t=C_t + I_t
\]
Combining the production function, resource constraint, and law of motion gives the following Lagrangian function:
\[
	\L = \suminf \beta^t\loge{C_t} - \lambda_t\left(K_{t+1} - K_t^{1-\delta}(AK_t^\alpha-C_t)^\delta\right)
\]
Which has the following first-order conditions:
\begin{align*}
	\frac{\partial\L}{\partial C_t} 	&= \frac{\beta^t}{C_t} - \lambda_t\delta K_t^{1-\delta}\left(AK_t^\alpha-C_t\right)^{\delta-1} 						= 0	\\
	\frac{\partial\L}{\partial K_{t+1}} 
		&= -\lambda_t + 
		\lambda_{t+1}\left[(1-\delta)K_{t+1}^{-\delta}\left(AK_{t+1}^\alpha-C_{t+1}\right)^\delta + \delta K_{t+1}^{1-\delta}\alpha AK_{t+1}^{\alpha-1}\left(AK_{t+1}^\alpha - C_{t+1}\right)^{\delta-1}\right] = 0 	\\
	\Rightarrow	\frac{C_{t+1}}{C_t} 
	&= \beta \left(K_t^{1-\delta}I_t^{\delta-1}\right)\left(K_{t+1}^{\delta-1}I_{t+1}^{1-\delta}\right)\left(K_{t+1}^{-\delta}I_{t+1}^\delta\right)\left(1-\delta + \delta\alpha AK_{t+1}^\alpha I_{t+1}^{-1}\right)
\end{align*}
Which simplifies to the following Euler equation:
\[
	\frac{C_{t+1}}{C_t} = \beta \left(1-\delta+\delta\alpha AK_{t+1}^\alpha I_{t+1}^{-1}\right)
\]
\textit{Perturbation:} Suppose that we are on the optimal growth trajectory at time $t$, and let there be a one-time deviation, $\Delta C_t>0$. This leads to an increase in utility of ${\frac{\beta}{C_t}\Delta C}$, but a decrease in capital:
\[
	\Delta K_{t+1} = -\delta K_t^{1-\delta}I_t^{\delta-1}\Delta C_t
\]
This decrease in capital in the second period, in turn, decreases output in the second period:
\[
	\Delta Y_{t+1} = \alpha A K^{\alpha-1}_{t+1}\Delta K_{t+1}
\]
Then, in period $t+2$, there must be an increase in investment in order to return to the equilibrium path level of capital:
\begin{align*}
	(1-\delta)\Delta K_{t+1}(K_{t+1}^{-\delta}I_{t+1}^\delta) &= -\delta\Delta I_{t+1}(K_{t+1}^{1-\delta}I_{t+1}^{\delta-1})	\\
	\Rightarrow \Delta I_{t+1} &= \dfrac{\delta-1}{\delta}K_{t+1}^{-1}I_{t+1}\Delta K_{t+1}
\end{align*}
Thus, we can derive the change in consumption in period $t+1$ from the resource constraint:
\begin{align*}
	\Delta C_{t+1} 	&= \Delta Y_{t+1} - \Delta I_{t+1} 																						\\
					&= \alpha A K^{\alpha-1}_{t+1}\Delta K_{t+1} - \dfrac{\delta-1}{\delta}K_{t+1}^{-1}I_{t+1}\Delta K_{t+1}				\\
					&= -\alpha A K^{\alpha-1}_{t+1}\delta K_t^{1-\delta}I_t^{\delta-1}\Delta C_t 
						+ \dfrac{\delta-1}{\delta}K_{t+1}^{-1}I_{t+1}\delta K_t^{1-\delta}I_t^{\delta-1}\Delta C_t							\\
					&= -\delta K_t^{-\delta}I_t^{\delta-1}\left(\dfrac{1-\delta}{\delta}I_{t+1}+\alpha A K^{\alpha}_{t+1}\right)\Delta C_t
\end{align*}
Taken together, we can solve for the consumption level in period $t+1$ that ensures utility stays maximized:
\begin{align*}
	\Delta U 																																			&= 0	\\
	\frac{\beta^t}{C_t}\Delta C_t - 
		\frac{\beta^{t+1}}{C_{t+1}}\delta K_t^{-\delta}I_t^{\delta-1}\left(\dfrac{1-\delta}{\delta}I_{t+1}+\alpha A K^{\alpha}_{t+1}\right)\Delta C_t 	&= 0	\\
\end{align*}
\[
	C_{t+1} = \beta C_t K_t^{-\delta}I_t^{\delta-1}\left((1-\delta)I_{t+1}+\delta\alpha A K^{\alpha}_{t+1}\right)	
\]
Which is equivalent to our Euler equation.

%%%________________________________________________________________%%%

\subsection*{Question 2}
The two equations that pin down the steady-state of this model are the Euler equation from (1) and the combined production function, law of motion, and resource constraint (which we will simply refer to as the consolidated resource contraint):
\begin{align*}
	C_{t+1} &= \beta C_t\left(1-\delta+\delta\alpha AK^\alpha_{t+1}\left(AK_{t+1}^\alpha - C_{t+1}\right)^{-1}\right)	\\
	K_{t+1} &= K_t^{1-\delta}\left(AK_t^\alpha - C_t\right)^\delta
\end{align*}

%%%________________________________________________________________%%%

\subsection*{Question 3}
Before log-linearizing this system, let us first simplify the steady-state values of the model's variables using the two equations we have. First, take the law of motion:
\begin{align*}
	\overline{K}		&= \overline{K}^{1-\delta}\overline{I}^\delta 	\\
	\overline{K}^\delta &= \overline{I}^\delta	\Rightarrow \overline{K} = \overline{I}
\end{align*}
Then, if we plug this equality into the resource constraint, we get:
\[
	A\overline{K} = \overline{C} + \overline{K} \Rightarrow \overline{C} = \overline{K}\left(A\overline{K}^{\alpha-1} - 1\right)
\]
To simply notation, denote ${\phi = A\overline{K}^{\alpha-1}}$. Finally, using the law of motion equality and the Euler equation, we can determine:
\[
	\overline{C} = \beta\overline{C}(1-\delta+\delta\alpha A\overline{K}^\alpha\overline{I}^{-1}) \Rightarrow 1 = \beta(1-\delta+\delta\alpha\phi)
\]
Or, ${\beta^{-1} = 1-\delta+\delta\alpha\phi}$.
\medskip \\
\textit{The Euler Equation:} We can log-linearize the Euler equation in stages. Let ${I_{t+1} = AK_{t+1}^\alpha - C_{t+1}}$ and ${Z_{t+1} = 1-\delta+\delta\alpha AK^\alpha_{t+1}I_{t+1}^{-1}}$. Then:
	\begin{align*}
		c_{t+1} - c_t &= \left(\alpha k_{t+1} - i_{t+1}\right)\left(\frac{\delta\alpha A\overline{K}^\alpha\overline{I}^{-1}}{\overline{Z}}\right)	\\
		\frac{1}{\overline{Z}} &= \beta																												\\
		i_{t+1}	&= \left(\frac{A\overline{K}^\alpha}{\overline{I}}\right)\alpha k_{t+1} - \left(\frac{\overline{C}}{\overline{I}}\right)c_{t+1}			\\
	\Rightarrow 	
		c_{t+1} - c_t &= \beta\delta\alpha^2\phi(1-\phi)k_{t+1} - \beta\delta\alpha\phi(1-\phi)c_{t+1}	\\
		\left(1 + \beta\delta\alpha\phi(1-\phi)\right)c_{t+1} &= \beta\delta\alpha^2\phi(1-\phi)k_{t+1} + c_t	\\
		c_{t+1} &= \frac{\beta\delta\alpha^2\phi(1-\phi)}{1 + \beta\delta\alpha\phi(1-\phi)}k_{t+1} + \frac{1}{1 + \beta\delta\alpha\phi(1-\phi)}c_t
	\end{align*}
\\
\textit{The Consolidated Resource Constraint:} Once again letting ${I_t = AK_t^\alpha - C_t}$:
	\begin{align*}
		k_{t+1} &= (1-\delta)k_t + \delta i_t										\\
		i_t 	&= \alpha\phi k_t - (\phi-1)c_t										\\
		k_{t+1} &= (1-\delta)k_t + \delta\left(\alpha\phi k_t - (\phi-1)c_t\right)	\\
		k_{t+1} &= (1-\delta + \delta\alpha\phi)k_t - \delta(\phi-1)c_t				\\
		k_{t+1} &= \beta^{-1}k_t - \delta(\phi-1)c_t				
	\end{align*}

%%%________________________________________________________________%%%

\subsection*{Question 4}
The log-linearized consolidated resource constraint from (3) is already given as a function of one state variable ($k_t$) and one choice variable ($c_t$). We can plug this function into $k_{t+1}$ in the log-linearized Euler equation from (3) to transform it into a function with one state variable and one choice variable, as well:
\begin{align*}
	c_{t+1} &= \frac{\beta\delta\alpha^2\phi(1-\phi)}{1 + \beta\delta\alpha\phi(1-\phi)}\left[\beta^{-1}k_t - \delta(\phi-1)c_t\right] + \frac{1}{1 + \beta\delta\alpha\phi(1-\phi)}c_t	\\
	c_{t+1} &= \frac{\delta\alpha^2\phi(1-\phi)}{1 + \beta\delta\alpha\phi(1-\phi)}k_t + \frac{\delta^2\alpha^2\phi(1-\phi)^2}{1 + \beta\delta\alpha\phi(1-\phi)}c_t	\\
\end{align*}
To simplify the notation, denote ${\theta = \delta\alpha\phi(1-\phi)}$. We begin the Blanchard-Kahn method by writing the system in vector-matrix form:
\[
	x_{t+1}	= \colvec{2}{c_{t+1}}{k_{t+1}} = 
			\begin{pmatrix}
				\frac{\delta\alpha(1-\phi)\theta}{1 + \beta\theta}	&	\frac{\alpha\theta}{1 + \beta\theta}	\\
				- \delta(\phi-1)									&	\beta^{-1}
			\end{pmatrix}\colvec{2}{c_t}{k_t} = Bx_t
\]
Next, we can find the eigenvalues of $B$ by solving ${B - \lambda I	= 0}$:
\begin{align*}
	\left(\frac{\delta\alpha(1-\phi)\theta}{1 + \beta\theta} - \lambda\right)(\beta^{-1}-\lambda) - \frac{\delta(\phi-1)\alpha\theta}{1 + \beta\theta} &= 0	\\
	\Rightarrow \lambda^2 - \left(\frac{\beta^{-1}+\theta+\delta\alpha(1-\phi)\theta}{1+\beta\theta}\right)\lambda + \frac{(\beta^{-1}+1)\delta\alpha(1-\phi)\theta}{1+\beta\theta} &= 0
\end{align*}
Using the quadratic formula, we can\footnote{Uh, in theory, I guess.} solve:
\[
	\lambda = \frac{\beta^{-1} + \theta + \delta\alpha(1-\phi)\theta}{2(1+\beta\theta)} \pm \sqrt{ \frac{(\beta^{-1} + \theta + \delta\alpha(1-\phi)\theta)^2}{4(1+\beta\theta)^2} - \frac{(\beta^{-1} + 1)\delta\alpha(1-\phi)\theta}{1+\beta\theta} }
\]
Whether I made a mistake earlier or am unable to simplify the above equality, I don't know. The two eigenvalues are:
\[
	\lambda_1 = \frac{1}{\beta(1-\delta+\delta\alpha)}\text{, }\lambda_2 = 1-\delta+\delta\alpha
\]
Note that $\lambda_2$ can be rewritten as ${1-\delta(1-\alpha)}$, where ${\alpha,\delta\in(0,1)}$. Thus, ${\lambda_2\in(0,1)}$. By the same logic, since ${\beta<1}$, ${\lambda_1>1}$. 
\medskip \\ 
Now that we have the eigenvalues of $B$, we can represent the system as ${B=Q\Lambda Q^{-1}}$, where the columns of $Q$ are the eigenvectors associated with $\lambda_1$ and $\lambda_2$, and:
\begin{align*}
	Bx_t =	x_{t+1}						&=  Q\Lambda Q^{-1}x_t							\\
			Q^{-1}x_{t+1}				&= \Lambda Q^{-1}x_{t}							\\
			y_{t+1}						&= \Lambda y_t									\\
	\colvec{2}{y_{1,t+1}}{y_{2,t+1}}	&= \colvec{2}{\lambda_1y_{1t}}{\lambda_2y_{2t}}
\end{align*}
Where ${y_t = Q^{-1}x_t}$. Since ${\lambda_1>1}$, we can conclude that the solution to the system includes ${y_t=0}$ ${\forall t}$, since:
\[
	y_{1t} = \underset{j\rightarrow\infty}{\text{lim }}\lambda_1^{-j}y_{1,t+j}=0
\]
We have not solved for $Q$ yet, which would involve deriving the eigenvectors associated with $\lambda_1$ and $\lambda_2$. This may be possible analytically, but instead, let us begin with an arbitrary matrix:
\[
	Q = \begin{pmatrix} a & b \\ c & d \end{pmatrix} \Rightarrow Q^{-1} = \frac{1}{ad-bc}\begin{pmatrix} d & -b \\ -c & a \end{pmatrix}
\]
Then, our value of $y_t$ is:
\[
	y_t = Q^{-1}x_t = \frac{1}{ad-bc}\begin{pmatrix} d & -b \\ -c & a \end{pmatrix}\colvec{2}{c_t}{k_t}
\]
Recall that ${y_{1t}=0}$ for all $t$. Then, the solution to our system can be written as ${c_t = \frac{b}{d}k_t}$. In other words, to a first-order approximation, deviations from steady-state capital correlate with capital's deviations from the steady-state by some constant proportion. Moreover, this proportion comes from the eigenvector of the non-explosive eigenvalue, $\lambda_2$.


%%%________________________________________________________________%%%
\pagebreak
\subsection*{Question 5}
If the true saddle path of the system is of the form ${C_t = ZK_t^a}$, where ${Z,a\in\R}$, then the log-linearized approximation is also the generalized solution to the social planner's problem. 
\medskip \\
One solution is that the consumer never maintains a capital stock and instead consumes all of the production each period. This is consistent with the generalized solution: ${C_t = AK_t^\alpha}$. We can solve for an interior solution by substituting $C_t$ for $ZK_t^a$ in our Euler equation and solving for $Z$, letting ${a=\alpha}$.\footnote{Full disclosure: I realized while solving this question that I made an algebraic error while reducing the Euler equation in question 1. I use the correct Euler equation below, but I simply don't have the time to go back and re-do everything up to this point.}
{\small
\begin{align*}
	C_{t+1} 		&= \beta C_tK_t^{1-\delta}I_t^{\delta-1}K_{t+1}^{-1}\left((1-\delta)I_{t+1} + \delta\alpha AK_{t+1}^\alpha\right)	\\
	ZK_{t+1}^\alpha &= \beta ZK_t^\alpha K_t^{1-\delta}\left(AK_t^\alpha - ZK_t^\alpha\right)^{\delta-1}K_{t+1}^{-1}
							\left[(1-\delta)\left(AK_{t+1}^\alpha - ZK_{t+1}^\alpha\right) + \delta\alpha AK_{t+1}^\alpha\right]		\\
	K_{t+1}^\alpha 	&= \beta K_t^\alpha K_t^{1-\delta}(A-Z)^{\delta-1}\left(K_t^\alpha\right)^{\delta-1}K_{t+1}^{-1}
							\left[(1-\delta)(A-Z)+ \delta\alpha\right]K_{t+1}^\alpha 													\\
	K_{t+1}			&= \beta K_t^\alpha K_t^{1-\delta +\delta\alpha-\alpha}(A-Z)^{\delta-1}\left[(1-\delta)(A-Z)+ \delta\alpha\right]	\\
	K_t^{1-\delta}\left((A-Z)K_t^\alpha\right)^\delta 
					&= \beta K_t^{1-\delta +\delta\alpha}(A-Z)^{\delta-1}\left[(1-\delta)(A-Z)+ \delta\alpha\right]						\\
	(A-Z)^\delta	&= \beta K_t^{1-\delta +\delta\alpha}K_t^{\delta-1-\alpha\delta}(A-Z)^{\delta-1}
							\left[(1-\delta)(A-Z)+ \delta\alpha\right]																	\\
	A-Z				&= \beta(1-\delta)(A-Z) + \beta\delta\alpha A																		\\
	Z\left[\beta(1-\delta)-1\right]
					&= A\beta(1-\delta+\delta\alpha)-A\beta																				\\
				Z 	&= \frac{A\beta(1-\beta)}{1-\beta(1-\delta)}
\end{align*}
}
Thus, the saddle path of the system is:
\[
	C_t = \left(\frac{A\beta(1-\beta)}{1-\beta(1-\delta)}\right)K_t^\alpha
\]

%%%________________________________________________________________%%%

\subsection*{Question 6}
[Question omitted becase we only learned stochastic shocks in today's lecture]

%%%________________________________________________________________%%%

\subsection*{Question 7}

The most notable functional form assumption in this model is the law of motion of capital, which is both strange and leads to a perfect correlation between capital and investment. As a result, any shock to capital also shocks investment which, due to the resource constraint, shocks consumption. This leads to a perfect correlation between capital and consumption. The economic intuition behind this correlation is that capital holdings influence lifetime wealth, leading consumption (which is based on discounted lifetime earnings) to swing with it.

%%%________________________________________________________________%%%



\end{document}






