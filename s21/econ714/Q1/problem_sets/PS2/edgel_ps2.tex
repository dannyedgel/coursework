%%% Econ714: Macroeconomics II
%%% Spring 2021
%%% Danny Edgel
%%%
% Due on Canvas Monday, February 8th, 11:59pm Central Time
%%%

%%%
%							PREAMBLE
%%%

\documentclass{article}

%%% declare packages
\usepackage{amsmath}
\usepackage{amssymb}
\usepackage{array}
\usepackage{bm}
\usepackage{changepage}
\usepackage{centernot}
\usepackage{graphicx}
\usepackage[shortlabels]{enumitem}
\usepackage{fancyhdr}
	\fancyhf{} % sets both header and footer to nothing
	\renewcommand{\headrulewidth}{0pt}
    \rfoot{Edgel, \thepage}
    \pagestyle{fancy}
	
%%% define shortcuts for set notation
\newcommand{\Z}{\mathbb{Z}}
\newcommand{\R}{\mathbb{R}}
\newcommand{\Q}{\mathbb{Q}}
\newcommand{\lmt}{\underset{x\rightarrow\infty}{\text{lim }}}
\newcommand{\neglmt}{\underset{x\rightarrow-\infty}{\text{lim }}}
\newcommand{\zerolmt}{\underset{x\rightarrow 0}{\text{lim }}}
\newcommand{\loge}[1]{\text{log}\left(#1\right)}
\newcommand{\usmax}[1]{\underset{#1}{\text{max }}}
\newcommand{\Mt}{M_{t+1}^t}
\newcommand{\vhat}{\hat{v}}
\newcommand{\olp}{\overline{p}}
\renewcommand{\L}{\mathcal{L}}
\newcommand{\olq}{\overline{q}}
\newcommand{\zinf}{_{t=0}^\infty}
\newcommand{\aneg}{A^{-1}}
\newcommand{\sneg}{s^{-1}}
\newcommand{\olk}{\overline{k}}
\newcommand{\olc}{\overline{c}}
\newcommand{\olr}{\overline{r}}
\newcommand{\olpi}{\overline{\pi}}
\newcommand{\Aneg}{A^{-1}}
\renewcommand{\sneg}{s^{-1}}
\newcommand{\dc}[1]{\Delta c_{#1}}
\newcommand{\N}{\mathcal{N}}
\newcommand{\suminf}{\sum_{t=0}^\infty}

\newcommand{\E}[1]{\mathbb{E}\left[#1\right]} % expected value
\newcommand{\Et}[1]{\mathbb{E}_t\left[#1\right]}

%%% define column vector command (from Michael Nattinger)
\newcount\colveccount
\newcommand*\colvec[1]{
        \global\colveccount#1
        \begin{pmatrix}
        \colvecnext
}
\def\colvecnext#1{
        #1
        \global\advance\colveccount-1
        \ifnum\colveccount>0
                \\
                \expandafter\colvecnext
        \else
                \end{pmatrix}
        \fi
}

%%% define function for drawing matrix augmentation lines
\newcommand\aug{\fboxsep=-\fboxrule\!\!\!\fbox{\strut}\!\!\!}

\makeatletter
\let\amsmath@bigm\bigm

\renewcommand{\bigm}[1]{%
  \ifcsname fenced@\string#1\endcsname
    \expandafter\@firstoftwo
  \else
    \expandafter\@secondoftwo
  \fi
  {\expandafter\amsmath@bigm\csname fenced@\string#1\endcsname}%
  {\amsmath@bigm#1}%
}


%________________________________________________________________%

\begin{document}

\title{	Problem Set \#2 }
\author{ 	Danny Edgel 					\\ 
			Econ 714: Macroeconomics II		\\
			Spring 2021						\\
		}
\maketitle\thispagestyle{empty}

%%%________________________________________________________________%%%

\noindent\textit{Collaborated with Sarah Bass, Emily Case, Michael Nattinger, and Alex Von Hafften}

%%%________________________________________________________________%%%

\subsection*{Question 1}
The social planner in this problem seeks to maximize utility subject to the production function, resource constraint, and law of motion:
\[
	\usmax{\{C_t,K_t,I_t\}_{t=0}^\infty}\suminf\beta^t\loge{C_t}\text{, s.t. }Y_t=AK_t^\alpha\text{, }K_{t+1}=K_t^{1-\delta}I_t^\delta\text{, }Y_t=C_t + I_t
\]
Combining the production function, resource constraint, and law of motion gives the following Lagrangian function:
\[
	\L = \suminf \beta^t\loge{C_t} - \lambda_t\left(K_{t+1} - K_t^{1-\delta}(AK_t^\alpha-C_t)^\delta\right)
\]
Which has the following first-order conditions:
\begin{align*}
	\frac{\partial\L}{\partial C_t} 	&= \frac{\beta^t}{C_t} - \lambda_t\delta\left(AK_t^\alpha-C_t\right)^{\delta-1} 						= 0	\\
	\frac{\partial\L}{\partial K_{t+1}} 
		&= -\lambda_t + \lambda_{t+1}K_t^{-\delta}(AK_t^\alpha-C_t)^\delta\left[1-\delta + \alpha\delta AK_t^\alpha(AK_t^\alpha-C_t)^{-1}\right] = 0 	\\
	\Rightarrow	\frac{C_{t+1}}{C_t} &= \beta K_t^\delta\left[\frac{(AK_{t+1}^\alpha-C_{t+1})^{1-\delta}}{AK_t^\alpha-C_t}\right]\left(1-\delta+\frac{\alpha\delta AK_t^\alpha}{AK_t^\alpha-C_t}\right)^{-1}
\end{align*}
Which simplifies to the following Euler equation:
\[
	\frac{C_{t+1}}{C_t} = \beta K_t^\delta\left[\frac{(AK_{t+1}^\alpha-C_{t+1})^{1-\delta}}{(1-\delta)(AK_t^\alpha-C_t) + \alpha\delta AK_t^\alpha}\right] 
									=  \beta K_t^\delta\left[\frac{(Y_{t+1}-C_{t+1})^{1-\delta}}{(1-\delta)(Y_t-C_t) + \alpha\delta Y_t}\right]
\]


%%%________________________________________________________________%%%

\subsection*{Question 2}


%%%________________________________________________________________%%%

\subsection*{Question 3}

%%%________________________________________________________________%%%

\subsection*{Question 4}


%%%________________________________________________________________%%%

\subsection*{Question 5}


%%%________________________________________________________________%%%

\subsection*{Question 6}


%%%________________________________________________________________%%%

\subsection*{Question 7}


%%%________________________________________________________________%%%

\subsection*{Question 8}


%%%________________________________________________________________%%%


\end{document}







