%%% Econ714: Macroeconomics II
%%% Spring 2021
%%% Danny Edgel
%%%
% Due on Canvas Monday, February 8th, 11:59pm Central Time
%%%

%%%
%							PREAMBLE
%%%

\documentclass{article}

%%% declare packages
\usepackage{amsmath}
\usepackage{amssymb}
\usepackage{array}
\usepackage{bm}
\usepackage{changepage}
\usepackage{centernot}
\usepackage{graphicx}
\usepackage[shortlabels]{enumitem}
\usepackage{fancyhdr}
	\fancyhf{} % sets both header and footer to nothing
	\renewcommand{\headrulewidth}{0pt}
    \rfoot{Edgel, \thepage}
    \pagestyle{fancy}
	
%%% define shortcuts for set notation
\newcommand{\Z}{\mathbb{Z}}
\newcommand{\R}{\mathbb{R}}
\newcommand{\Q}{\mathbb{Q}}
\newcommand{\lmt}{\underset{x\rightarrow\infty}{\text{lim }}}
\newcommand{\neglmt}{\underset{x\rightarrow-\infty}{\text{lim }}}
\newcommand{\zerolmt}{\underset{x\rightarrow 0}{\text{lim }}}
\newcommand{\loge}[1]{\text{log}\left(#1\right)}
\newcommand{\usmax}[1]{\underset{#1}{\text{max }}}
\newcommand{\Mt}{M_{t+1}^t}
\newcommand{\vhat}{\hat{v}}
\newcommand{\olp}{\overline{p}}
\renewcommand{\L}{\mathcal{L}}
\newcommand{\olq}{\overline{q}}
\newcommand{\zinf}{_{t=0}^\infty}
\newcommand{\aneg}{A^{-1}}
\newcommand{\sneg}{s^{-1}}
\newcommand{\olk}{\overline{k}}
\newcommand{\olc}{\overline{c}}
\newcommand{\olr}{\overline{r}}
\newcommand{\olpi}{\overline{\pi}}
\newcommand{\Aneg}{A^{-1}}
\renewcommand{\sneg}{s^{-1}}
\newcommand{\dc}[1]{\Delta c_{#1}}
\newcommand{\N}{\mathcal{N}}
\newcommand{\suminf}{\sum_{t=0}^\infty}

\newcommand{\E}[1]{\mathbb{E}\left[#1\right]} % expected value
\newcommand{\Et}[1]{\mathbb{E}_t\left[#1\right]}

%%% define column vector command (from Michael Nattinger)
\newcount\colveccount
\newcommand*\colvec[1]{
        \global\colveccount#1
        \begin{pmatrix}
        \colvecnext
}
\def\colvecnext#1{
        #1
        \global\advance\colveccount-1
        \ifnum\colveccount>0
                \\
                \expandafter\colvecnext
        \else
                \end{pmatrix}
        \fi
}

%%% define function for drawing matrix augmentation lines
\newcommand\aug{\fboxsep=-\fboxrule\!\!\!\fbox{\strut}\!\!\!}

\makeatletter
\let\amsmath@bigm\bigm

\renewcommand{\bigm}[1]{%
  \ifcsname fenced@\string#1\endcsname
    \expandafter\@firstoftwo
  \else
    \expandafter\@secondoftwo
  \fi
  {\expandafter\amsmath@bigm\csname fenced@\string#1\endcsname}%
  {\amsmath@bigm#1}%
}


%________________________________________________________________%

\begin{document}

\title{	Problem Set \#2 }
\author{ 	Danny Edgel 					\\ 
			Econ 714: Macroeconomics II		\\
			Spring 2021						\\
		}
\maketitle\thispagestyle{empty}

%%%________________________________________________________________%%%

\noindent\textit{Collaborated with Sarah Bass, Emily Case, Katherine Kwok, Michael Nattinger, and Alex Von Hafften}

%%%________________________________________________________________%%%

\subsection*{Question 1}
The social planner in this problem seeks to maximize utility subject to the production function, resource constraint, and law of motion:
\[
	\usmax{\{C_t,K_t,I_t\}_{t=0}^\infty}\suminf\beta^t\loge{C_t}\text{, s.t. }Y_t=AK_t^\alpha\text{, }K_{t+1}=K_t^{1-\delta}I_t^\delta\text{, }Y_t=C_t + I_t
\]
Combining the production function, resource constraint, and law of motion gives the following Lagrangian function:
\[
	\L = \suminf \beta^t\loge{C_t} - \lambda_t\left(K_{t+1} - K_t^{1-\delta}(AK_t^\alpha-C_t)^\delta\right)
\]
Which has the following first-order conditions:
\begin{align*}
	\frac{\partial\L}{\partial C_t} 	&= \frac{\beta^t}{C_t} - \lambda_t\delta K_t^{1-\delta}\left(AK_t^\alpha-C_t\right)^{\delta-1} 						= 0	\\
	\frac{\partial\L}{\partial K_{t+1}} 
		&= -\lambda_t + 
		\lambda_{t+1}\left[(1-\delta)K_{t+1}^{-\delta}\left(AK_{t+1}^\alpha-C_{t+1}\right)^\delta + \delta K_{t+1}^{1-\delta}\alpha AK_{t+1}^{\alpha-1}\left(AK_{t+1}^\alpha - C_{t+1}\right)^{\delta-1}\right] = 0 	\\
	\Rightarrow	\frac{C_{t+1}}{C_t} 
	&= \beta \left(K_t^{1-\delta}I_t^{\delta-1}\right)\left(K_{t+1}^{\delta-1}I_{t+1}^{1-\delta}\right)\left(K_{t+1}^{-\delta}I_{t+1}^\delta\right)\left(1-\delta + \delta\alpha AK_{t+1}^\alpha I_{t+1}^{-1}\right)
\end{align*}
Which simplifies to the following Euler equation:
\[
	\frac{C_{t+1}}{C_t} = \beta \left(1-\delta+\delta\alpha AK_{t+1}^\alpha I_{t+1}^{-1}\right)
\]


%%%________________________________________________________________%%%

\subsection*{Question 2}
The two equations that pin down the steady-state of this model are the Euler equation from (1) and the combined production function, law of motion, and resource constraint (which we will simply refer to as the consolidated resource contraint):
\begin{align*}
	C_{t+1} &= \beta C_t\left(1-\delta+\delta\alpha AK^\alpha_{t+1}\left(AK_{t+1}^\alpha - C_{t+1}\right)^{-1}\right)	\\
	K_{t+1} &= K_t^{1-\delta}\left(AK_t^\alpha - C_t\right)^\delta
\end{align*}

%%%________________________________________________________________%%%

\subsection*{Question 3}
Before log-linearizing this system, let us first simplify the steady-state values of the model's variables using the two equations we have. First, take the law of motion:
\begin{align*}
	\overline{K}		&= \overline{K}^{1-\delta}\overline{I}^\delta 	\\
	\overline{K}^\delta &= \overline{I}^\delta	\Rightarrow \overline{K} = \overline{I}
\end{align*}
Then, if we plug this equality into the resource constraint, we get:
\[
	A\overline{K} = \overline{C} + \overline{K} \Rightarrow \overline{C} = \overline{K}\left(A\overline{K}^{\alpha-1} - 1\right)
\]
To simply notation, denote ${\phi = A\overline{K}^{\alpha-1}}$. Finally, using the law of motion equality and the Euler equation, we can determine:
\[
	\overline{C} = \beta\overline{C}(1-\delta+\delta\alpha A\overline{K}^\alpha\overline{I}^{-1}) \Rightarrow 1 = \beta(1-\delta+\delta\alpha\phi)
\]
Or, ${\beta^{-1} = 1-\delta+\delta\alpha\phi}$.
\medskip \\
\textit{The Euler Equation:} We can log-linearize the Euler equation in stages. Let ${I_{t+1} = AK_{t+1}^\alpha - C_{t+1}}$ and ${Z_{t+1} = 1-\delta+\delta\alpha AK^\alpha_{t+1}I_{t+1}^{-1}}$. Then:
	\begin{align*}
		c_{t+1} - c_t &= \left(\alpha k_{t+1} - i_{t+1}\right)\left(\frac{\delta\alpha A\overline{K}^\alpha\overline{I}^{-1}}{\overline{Z}}\right)	\\
		\frac{1}{\overline{Z}} &= \beta																												\\
		i_{t+1}	&= \left(\frac{A\overline{K}^\alpha}{\overline{I}}\right)\alpha k_{t+1} - \left(\frac{\overline{C}}{\overline{I}}\right)c_{t+1}			\\
	\Rightarrow 	
		c_{t+1} - c_t &= \beta\delta\alpha^2\phi(1-\phi)k_{t+1} - \beta\delta\alpha\phi(1-\phi)c_{t+1}	\\
		\left(1 + \beta\delta\alpha\phi(1-\phi)\right)c_{t+1} &= \beta\delta\alpha^2\phi(1-\phi)k_{t+1} + c_t	\\
		c_{t+1} &= \frac{\beta\delta\alpha^2\phi(1-\phi)}{1 + \beta\delta\alpha\phi(1-\phi)}k_{t+1} + \frac{1}{1 + \beta\delta\alpha\phi(1-\phi)}c_t
	\end{align*}
\\
\textit{The Consolidated Resource Constraint:} Once again letting ${I_t = AK_t^\alpha - C_t}$:
	\begin{align*}
		k_{t+1} &= (1-\delta)k_t + \delta i_t										\\
		i_t 	&= \alpha\phi k_t - (\phi-1)c_t										\\
		k_{t+1} &= (1-\delta)k_t + \delta\left(\alpha\phi k_t - (\phi-1)c_t\right)	\\
		k_{t+1} &= (1-\delta + \delta\alpha\phi)k_t - \delta(\phi-1)c_t				\\
		k_{t+1} &= \beta^{-1}k_t - \delta(\phi-1)c_t				
	\end{align*}

%%%________________________________________________________________%%%

\subsection*{Question 4}
The log-linearized consolidated resource constraint from (3) is already given as a function of one state variable ($k_t$) and one choice variable ($c_t$). We can plug this function into $k_{t+1}$ in the log-linearized Euler equation from (3) to transform it into a function with one state variable and one choice variable, as well:
\begin{align*}
	c_{t+1} &= \frac{\beta\delta\alpha^2\phi(1-\phi)}{1 + \beta\delta\alpha\phi(1-\phi)}\left[\beta^{-1}k_t - \delta(\phi-1)c_t\right] + \frac{1}{1 + \beta\delta\alpha\phi(1-\phi)}c_t	\\
	c_{t+1} &= \frac{\delta\alpha^2\phi(1-\phi)}{1 + \beta\delta\alpha\phi(1-\phi)}k_t + \frac{\delta^2\alpha^2\phi(1-\phi)^2}{1 + \beta\delta\alpha\phi(1-\phi)}c_t	\\
\end{align*}
To simplify the notation, denote ${\theta = \delta\alpha\phi(1-\phi)}$. We begin the Blanchard-Kahn method by writing the system in vector-matrix form:
\[
	x_{t+1}	= \colvec{2}{c_{t+1}}{k_{t+1}} = 
			\begin{pmatrix}
				\frac{\delta\alpha(1-\phi)\theta}{1 + \beta\theta}	&	\frac{\alpha\theta}{1 + \beta\theta}	\\
				- \delta(\phi-1)									&	\beta^{-1}
			\end{pmatrix}\colvec{2}{c_t}{k_t} = Bx_t
\]
Next, we can find the eigenvalues of $B$ by solving ${B - \lambda I	= 0}$:
\begin{align*}
	\left(\frac{\delta\alpha(1-\phi)\theta}{1 + \beta\theta} - \lambda\right)(\beta^{-1}-\lambda) - \frac{\delta(\phi-1)\alpha\theta}{1 + \beta\theta} &= 0	\\
	\Rightarrow \lambda^2 - \left(\frac{\beta^{-1}+\theta+\delta\alpha(1-\phi)\theta}{1+\beta\theta}\right)\lambda + \frac{(\beta^{-1}+1)\delta\alpha(1-\phi)\theta}{1+\beta\theta} &= 0
\end{align*}
Using the quadratic formula, we can solve:
\[
	\lambda = \frac{\beta^{-1} + \theta + \delta\alpha(1-\phi)\theta}{2(1+\beta\theta)} \pm \sqrt{ \frac{(\beta^{-1} + \theta + \delta\alpha(1-\phi)\theta)^2}{4(1+\beta\theta)^2} - \frac{(\beta^{-1} + 1)\delta\alpha(1-\phi)\theta}{1+\beta\theta} }
\]
Note that ${\phi>1}$, since ${\overline{C}/\overline{K}=\phi-1}$, and ${\overline{C},\overline{K}\geq 0}$. 
\medskip \\ 
I'm going to be frank here. I can't solve this. I must have gotten something wrong in the algebra along the way. I have re-calculated the log-linearization multiple times over several hours, with a different approach each time, and I keep getting the same result that I know is incorrect. I've run out of time to get this right, and I hope I can get partial credit. Here's how I would solve it if I found $\lambda_1$ and $\lambda_2$:
\medskip \\
First, I would note that $B=Q\Lambda Q^{-1}$, where $Q=[v_1\text{ }v_2]$, where $v_i$ is the eigenvector that corresponds to $\lambda_i$ and ${y_{t+1} = Q^{-1}x_{t+1}}$. Then, I would identify $i$ such that $|\lambda_i|>1$, then note that this implies that ${y_{it} = \underset{j\rightarrow\infty}{\text{lim }}\lambda_iy_{i,t+j}=0}$. Thus, the saddle path of the system is given by setting $y_{it} = 0$ for all $t$ and solving ${x_{t+1} = Q\Lambda y_t}$.
%Now, let $B=Q\Lambda Q^{-1}$

%%%________________________________________________________________%%%

\subsection*{Question 5}


%%%________________________________________________________________%%%

\subsection*{Question 6}


%%%________________________________________________________________%%%

\subsection*{Question 7}


%%%________________________________________________________________%%%

\subsection*{Question 8}


%%%________________________________________________________________%%%


\end{document}


% Typed into Wolfram Alpha:
% solve l^2 - (((1/b)(2-p)-1-ad(1-p)^2)/(2-p))l + (1/(b(1-p))) = 0 for l
%




