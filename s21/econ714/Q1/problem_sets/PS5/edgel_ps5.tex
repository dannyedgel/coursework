%%% Econ714: Macroeconomics II
%%% Spring 2021
%%% Danny Edgel
%%%
% Due on Canvas Monday, March 1st, 11:59pm Central Time
%%%

%%%
%							PREAMBLE
%%%

\documentclass{article}

%%% declare packages
\usepackage{amsmath}
\usepackage{amssymb}
\usepackage{array}
\usepackage{bm}
\usepackage{changepage}
\usepackage{centernot}
\usepackage{graphicx}
\usepackage{xcolor}
\usepackage[shortlabels]{enumitem}
\usepackage{fancyhdr}
	\fancyhf{} % sets both header and footer to nothing
	\renewcommand{\headrulewidth}{0pt}
    \rfoot{Edgel, \thepage}
    \pagestyle{fancy}
	
%%% define shortcuts for set notation
\newcommand{\Z}{\mathbb{Z}}
\newcommand{\R}{\mathbb{R}}
\newcommand{\Q}{\mathbb{Q}}
\newcommand{\lmt}{\underset{x\rightarrow\infty}{\text{lim }}}
\newcommand{\neglmt}{\underset{x\rightarrow-\infty}{\text{lim }}}
\newcommand{\zerolmt}{\underset{x\rightarrow 0}{\text{lim }}}
\newcommand{\loge}[1]{\text{log}\left(#1\right)}
\newcommand{\usmax}[1]{\underset{#1}{\text{max }}}
\newcommand{\usmin}[1]{\underset{#1}{\text{min }}}
\newcommand{\Mt}{M_{t+1}^t}
\newcommand{\vhat}{\hat{v}}
\newcommand{\olp}{\overline{p}}
\renewcommand{\L}{\mathcal{L}}
\newcommand{\olq}{\overline{q}}
\newcommand{\zinf}{_{t=0}^\infty}
\newcommand{\aneg}{A^{-1}}
\newcommand{\sneg}{s^{-1}}
\newcommand{\olk}{\overline{k}}
\newcommand{\olc}{\overline{c}}
\newcommand{\olr}{\overline{r}}
\newcommand{\olpi}{\overline{\pi}}
\newcommand{\Aneg}{A^{-1}}
\renewcommand{\sneg}{s^{-1}}
\newcommand{\dc}[1]{\Delta c_{#1}}
\newcommand{\N}{\mathcal{N}}
\newcommand{\suminf}{\sum_{t=0}^\infty}
\newcommand{\sumn}{\sum_{i=1}^{n}}
\newcommand{\sumnk}{\sum_{i=1}^{N_k}}
\newcommand{\red}[1]{{\color{red}#1}}
\newcommand{\Tau}{\mathrm{T}}

\newcommand{\E}[1]{\mathbb{E}\left[#1\right]} % expected value
\newcommand{\Et}[1]{\mathbb{E}_t\left[#1\right]}

%%% define column vector command (from Michael Nattinger)
\newcount\colveccount
\newcommand*\colvec[1]{
        \global\colveccount#1
        \begin{pmatrix}
        \colvecnext
}
\def\colvecnext#1{
        #1
        \global\advance\colveccount-1
        \ifnum\colveccount>0
                \\
                \expandafter\colvecnext
        \else
                \end{pmatrix}
        \fi
}

%%% define function for drawing matrix augmentation lines
\newcommand\aug{\fboxsep=-\fboxrule\!\!\!\fbox{\strut}\!\!\!}

\makeatletter
\let\amsmath@bigm\bigm

\renewcommand{\bigm}[1]{%
  \ifcsname fenced@\string#1\endcsname
    \expandafter\@firstoftwo
  \else
    \expandafter\@secondoftwo
  \fi
  {\expandafter\amsmath@bigm\csname fenced@\string#1\endcsname}%
  {\amsmath@bigm#1}%
}


%________________________________________________________________%

\begin{document}

\title{	Problem Set \#5 }
\author{ 	Danny Edgel 					\\ 
			Econ 714: Macroeconomics II		\\
			Spring 2021						\\
		}
\maketitle\thispagestyle{empty}

%%%________________________________________________________________%%%

\noindent\textit{Discussed and/or compared answers with Sarah Bass, Emily Case, Katherine Kwok, Michael Nattinger, and Alex Von Hafften}
 \\

%%%________________________________________________________________%%%

\subsection*{Questions 1}
If we begin with a flexible-price version of this model, we being by solving the household problem with the consumption bundle as a choice variable  alongside labor and investment. Households, then, maximize the Lagrangian:
\[
	\L = \E{\suminf\beta^t\left(\loge{C_t}-L_t\right)-\lambda_t\left(P_tC_t + B_t - W_tL_t + \Pi_t - (1+i_{t-1})B_{t-1} + \Tau_t\right)}
\]
This problem has three first-order conditions, for each choice variable:
\begin{align*}
	\frac{\partial\L}{\partial C_t}	&= \frac{\beta^t}{C_t} - \lambda_tP_t	= 0 	\\
	\frac{\partial\L}{\partial L_t}	&= -\beta^t + \lambda_tw_t				= 0		\\
	\frac{\partial\L}{\partial B_t}	&= -\lambda_t + \lambda_{t+1}(1+i_t)	= 0
\end{align*}
Combining the FOCs for consumption and labor yield a static labor supply curve, while combining the FOCs for consumption yield the model's Euler equation:
\begin{align*}
	&C_t 		= \frac{W_t}{P_t}													
	&C_t^{-1} 	= \beta\Et{\left(\frac{P_t}{P_{t+1}}\right)\frac{1+i_t}{C_{t+1}}}
\end{align*}
Since $C_t$ is a standard CES aggregator and $Y_{it}$ is a standard production function with only labor as an input, we know that the firm problem is solved by the Dixit-Stiglitz price equations:\footnote{An unstated assumption is that ${L_t=1}$. This comes from the fact that ${\varphi=0}$ and thus firms are able to adjust wages and prices instantaneously to ensure a perfectly inelastic labor supply.}
\begin{align*}
	&P_{it} = \left(\frac{\theta}{\theta-1}\right)\frac{W_t}{A_t} 
	&P_t 	= \left(\int P_{it}^{1-\theta}di\right)^{\frac{1}{1-\theta}}
\end{align*}
The steady-state of this model is determined by simply making each variable static across periods and combining the above equations, which yields:
\begin{align*}
	&\overline{C} 	= \frac{\overline{W}}{\overline{P}}															
	&\overline{i}	= \frac{1}{\beta} - 1													\\
	&\overline{P_i}	= \left(\frac{\theta}{\theta-1}\right)\frac{\overline{W}}{\overline{A}}						
	&\overline{P}	= \overline{P_i}N^{\frac{1}{1-\theta}}
\end{align*}
Log-linearizing our equations for this model provides the following, which describe the dynamics of the model:
\begin{align*}
	&p_{it} = w_t-a_t 
	&c_t	= w_t - p_t	\\
	&p_t 	= p_{it}	
	&c_t 	= \Et{c_{t+1} + p_{t+1} - p_t - i_t}
\end{align*}
Finally, since labor is inelastically supplied at ${L_t=1}$, ${\ell_t=0}$. Note that we can combine the equations for $p_t$, $p_{it}$, and $c_t$ to determine that ${c_t=a_t}$, which shows that all fluctuations in consumption come exclusively from productivity shocks.

%%%________________________________________________________________%%%
\pagebreak
\subsection*{Question 2}

\begin{enumerate}[(a)]

	\item The problem for a firm in this market is:
		{\small \begin{align*}
			\usmax{\{P_{it}\}}&\Et{\sum_{j=0}^\infty\Theta_{t,t+j}\left(P_{it+j}C_{it+j}-W_{t+j}L_{it+j} - \frac{\varphi W_{t+j}}{2}\left(\frac{P_{it+j}-P_{it+j-1}}{P_{it+j-1}}\right)^2\right)}	\\
				&\text{ s.t. } C_{it} = \left(\frac{P_{it}}{P_t}\right)^{-\theta}C_t\text{, }C_{it}=A_tL_{it}
		\end{align*} }
		Where $\Theta_{t,t+j}$ is taken from the household's Euler equation as the stochastic discount factor,
		\[
			\Theta_{t,t+j} = \beta^j\left(\frac{P_t}{P_{t+j}}\right)\frac{C_t}{C_{t+j}}
		\]
		and $P_{it}$ is the price charged by the firm $i$ in period $t$. Then, the constraints can be consolidated into the objective function to yield the problem:
		{\tiny \[
			\usmax{\{P_{it}\}}\Et{\sum_{j=0}^\infty\beta^j\left(\frac{P_t}{P_{t+j}}\right)\frac{C_t}{C_{t+j}}
				\left(P_{it+j}^{1-\theta}P_{t+j}^\theta C_{t+j}-\frac{W_{t+j}}{A_{t+j}}\left(\frac{P_{it+j}}{P_{t+j}}\right)^{-\theta}C_{t+j}-\frac{\varphi W_{t+j}}{2}\left(\frac{P_{it+j}-P_{it+j-1}}{P_{it+j-1}}\right)^2\right)}
		\] }
		The firm has a single first-order condition, for $P_{it}$:
		{\footnotesize \[
			(\theta-1)\left(\frac{P_{it}}{P_t}\right)^{-\theta}C_t + \theta\frac{W_t}{A_t}P_{it}^{-\theta-1}P_t^\theta - 
				\frac{\varphi W_t}{P_{it-1}}\left(\frac{P_{it}}{P_{it-1}} - 1\right)
			= \beta\Et{\Theta_{t,t+1}\frac{\varphi W_{t+1}P_{it+1}}{P_{it}^2}\left(\frac{P_{it+1}}{P_{it}}-1\right)}
		\] }
		
	\item Symmetry across producers implies ${P_{it}=P_t}$ and ${C_{it}=C_t}$. Imposing this condition and letting ${\pi_t=\frac{P_t}{P_{t-1}}-1}$, the firm FOC becomes:
		\begin{align*}
			(\theta-1)C_t + \theta\frac{W_t}{P_tA_t} - \frac{\varphi W_t}{P_{t-1}}\pi_t
				&= \beta\Et{\Theta_{t,t+1}\frac{\varphi W_{t+1}P_{t+1}}{P_{t}^2}\pi_{t+1}}	\\
			(\theta-1)P_tC_t + \theta\frac{W_t}{A_t} - \varphi W_t\pi_t(\pi_t+1)
				&= \beta\Et{\Theta_{t,t+1}\varphi W_{t+1}\pi_{t+1}(\pi_{t+1}+1)}	\\
		\end{align*}
	
	\item 
	
	\item 
	
	
\end{enumerate}



%%%________________________________________________________________%%%

\subsection*{Question 3}




%%%________________________________________________________________%%%

\subsection*{Question 4}





%%%________________________________________________________________%%%





\end{document}






