%%% Econ714: Macroeconomics II
%%% Spring 2021
%%% Danny Edgel
%%%
% Due on Canvas Monday, March 1st, 11:59pm Central Time
%%%

%%%
%							PREAMBLE
%%%

\documentclass{article}

%%% declare packages
\usepackage{amsmath}
\usepackage{amssymb}
\usepackage{array}
\usepackage{bm}
\usepackage{changepage}
\usepackage{centernot}
\usepackage{graphicx}
\usepackage{xcolor}
\usepackage[shortlabels]{enumitem}
\usepackage{fancyhdr}
	\fancyhf{} % sets both header and footer to nothing
	\renewcommand{\headrulewidth}{0pt}
    \rfoot{Edgel, \thepage}
    \pagestyle{fancy}
	
%%% define shortcuts for set notation
\newcommand{\Z}{\mathbb{Z}}
\newcommand{\R}{\mathbb{R}}
\newcommand{\Q}{\mathbb{Q}}
\newcommand{\lmt}{\underset{x\rightarrow\infty}{\text{lim }}}
\newcommand{\neglmt}{\underset{x\rightarrow-\infty}{\text{lim }}}
\newcommand{\zerolmt}{\underset{x\rightarrow 0}{\text{lim }}}
\newcommand{\loge}[1]{\text{log}\left(#1\right)}
\newcommand{\usmax}[1]{\underset{#1}{\text{max }}}
\newcommand{\usmin}[1]{\underset{#1}{\text{min }}}
\newcommand{\Mt}{M_{t+1}^t}
\newcommand{\vhat}{\hat{v}}
\newcommand{\olp}{\overline{p}}
\renewcommand{\L}{\mathcal{L}}
\newcommand{\olq}{\overline{q}}
\newcommand{\zinf}{_{t=0}^\infty}
\newcommand{\aneg}{A^{-1}}
\newcommand{\sneg}{s^{-1}}
\newcommand{\olk}{\overline{k}}
\newcommand{\olc}{\overline{c}}
\newcommand{\olr}{\overline{r}}
\newcommand{\olpi}{\overline{\pi}}
\newcommand{\Aneg}{A^{-1}}
\renewcommand{\sneg}{s^{-1}}
\newcommand{\dc}[1]{\Delta c_{#1}}
\newcommand{\N}{\mathcal{N}}
\newcommand{\suminf}{\sum_{t=0}^\infty}
\newcommand{\sumn}{\sum_{i=1}^{n}}
\newcommand{\sumnk}{\sum_{i=1}^{N_k}}
\newcommand{\red}[1]{{\color{red}#1}}
\newcommand{\Tau}{\mathrm{T}}

\newcommand{\E}[1]{\mathbb{E}\left[#1\right]} % expected value
\newcommand{\Et}[1]{\mathbb{E}_t\left[#1\right]}

%%% define column vector command (from Michael Nattinger)
\newcount\colveccount
\newcommand*\colvec[1]{
        \global\colveccount#1
        \begin{pmatrix}
        \colvecnext
}
\def\colvecnext#1{
        #1
        \global\advance\colveccount-1
        \ifnum\colveccount>0
                \\
                \expandafter\colvecnext
        \else
                \end{pmatrix}
        \fi
}

%%% define function for drawing matrix augmentation lines
\newcommand\aug{\fboxsep=-\fboxrule\!\!\!\fbox{\strut}\!\!\!}

\makeatletter
\let\amsmath@bigm\bigm

\renewcommand{\bigm}[1]{%
  \ifcsname fenced@\string#1\endcsname
    \expandafter\@firstoftwo
  \else
    \expandafter\@secondoftwo
  \fi
  {\expandafter\amsmath@bigm\csname fenced@\string#1\endcsname}%
  {\amsmath@bigm#1}%
}


%________________________________________________________________%

\begin{document}

\title{	Problem Set \#5 }
\author{ 	Danny Edgel 					\\ 
			Econ 714: Macroeconomics II		\\
			Spring 2021						\\
		}
\maketitle\thispagestyle{empty}

%%%________________________________________________________________%%%

\noindent\textit{Discussed and/or compared answers with Sarah Bass, Emily Case, Katherine Kwok, Michael Nattinger, and Alex Von Hafften}
 \\

%%%________________________________________________________________%%%

\subsection*{Questions 1}
If we begin with a flexible-price version of this model, we being by solving the household problem with the consumption bundle as a choice variable  alongside labor and investment. Households, then, maximize the Lagrangian:
\[
	\L = \E{\suminf\beta^t\left(\loge{C_t}-L_t\right)-\lambda_t\left(P_tC_t + B_t - W_tL_t + \Pi_t - (1+i_{t-1})B_{t-1} + \Tau_t\right)}
\]
This problem has three first-order conditions, for each choice variable:
\begin{align*}
	\frac{\partial\L}{\partial C_t}	&= \frac{\beta^t}{C_t} - \lambda_tP_t	= 0 	\\
	\frac{\partial\L}{\partial L_t}	&= -\beta^t + \lambda_tw_t				= 0		\\
	\frac{\partial\L}{\partial B_t}	&= -\lambda_t + \lambda_{t+1}(1+i_t)	= 0
\end{align*}
Combining the FOCs for consumption and labor yield a static labor supply curve, while combining the FOCs for consumption yield the model's Euler equation:
\begin{align*}
	&C_t 		= \frac{W_t}{P_t}													
	&C_t^{-1} 	= \beta\Et{\left(\frac{P_t}{P_{t+1}}\right)\frac{1+i_t}{C_{t+1}}}
\end{align*}
Since $C_t$ is a standard CES aggregator and $Y_{it}$ is a standard production function with only labor as an input, we know that the firm problem is solved by the Dixit-Stiglitz price equations:\footnote{An unstated assumption is that ${L_t=1}$. This comes from the fact that ${\varphi=0}$ and thus firms are able to adjust wages and prices instantaneously to ensure a perfectly inelastic labor supply.}
\begin{align*}
	&P_{it} = \left(\frac{\theta}{\theta-1}\right)\frac{W_t}{A_t} 
	&P_t 	= \left(\int P_{it}^{1-\theta}di\right)^{\frac{1}{1-\theta}}
\end{align*}
The steady-state of this model is determined by simply making each variable static across periods and combining the above equations, which yields:
\begin{align*}
	&\overline{C} 	= \frac{\overline{W}}{\overline{P}}															
	&\overline{i}	= \frac{1}{\beta} - 1													\\
	&\overline{P_i}	= \left(\frac{\theta}{\theta-1}\right)\frac{\overline{W}}{\overline{A}}						
	&\overline{P}	= \overline{P_i}N^{\frac{1}{1-\theta}}
\end{align*}
Log-linearizing our equations for this model provides the following, which describe the dynamics of the model:
\begin{align*}
	&p_{it} = w_t-a_t 
	&c_t	= w_t - p_t	\\
	&p_t 	= p_{it}	
	&c_t 	= \Et{c_{t+1} + p_{t+1} - p_t - i_t}
\end{align*}
Finally, since labor is inelastically supplied at ${L_t=1}$, ${\ell_t=0}$. Note that we can combine the equations for $p_t$, $p_{it}$, and $c_t$ to determine that ${c_t=a_t}$, which shows that all fluctuations in consumption come exclusively from productivity shocks.

%%%________________________________________________________________%%%
\pagebreak
\subsection*{Question 2}

\begin{enumerate}[(a)]

	\item The problem for a firm in this market is:
		{\small \begin{align*}
			\usmax{\{P_{it}\}}&\Et{\sum_{j=0}^\infty\Theta_{t,t+j}\left(P_{it+j}C_{it+j}-W_{t+j}L_{it+j} - \frac{\varphi W_{t+j}}{2}\left(\frac{P_{it+j}-P_{it+j-1}}{P_{it+j-1}}\right)^2\right)}	\\
				&\text{ s.t. } C_{it} = \left(\frac{P_{it}}{P_t}\right)^{-\theta}C_t\text{, }C_{it}=A_tL_{it}
		\end{align*} }
		Where $\Theta_{t,t+j}$ is taken from the household's Euler equation as the stochastic discount factor,
		\[
			\Theta_{t,t+j} = \beta^j\left(\frac{P_t}{P_{t+j}}\right)\frac{C_t}{C_{t+j}}
		\]
		and $P_{it}$ is the price charged by the firm $i$ in period $t$. Then, the constraints can be consolidated into the objective function to yield the problem:
		{\tiny \[
			\usmax{\{P_{it}\}}\Et{\sum_{j=0}^\infty\beta^j\left(\frac{P_t}{P_{t+j}}\right)\frac{C_t}{C_{t+j}}
				\left(P_{it+j}^{1-\theta}P_{t+j}^\theta C_{t+j}-\frac{W_{t+j}}{A_{t+j}}\left(\frac{P_{it+j}}{P_{t+j}}\right)^{-\theta}C_{t+j}-\frac{\varphi W_{t+j}}{2}\left(\frac{P_{it+j}-P_{it+j-1}}{P_{it+j-1}}\right)^2\right)}
		\] }
		The firm has a single first-order condition, for $P_{it}$:
		{\footnotesize \[
			(1-\theta)\left(\frac{P_{it}}{P_t}\right)^{-\theta}C_t + \theta\frac{W_t}{A_t}P_{it}^{-\theta-1}P_t^\theta C_t - 
				\frac{\varphi W_t}{P_{it-1}}\left(\frac{P_{it}}{P_{it-1}} - 1\right)
			= -\Et{\Theta_{t,t+1}\frac{\varphi W_{t+1}P_{it+1}}{P_{it}^2}\left(\frac{P_{it+1}}{P_{it}}-1\right)}
		\] }
		
	\item Symmetry across producers implies ${P_{it}=P_t}$ and ${C_{it}=C_t}$. Imposing this condition and letting ${\pi_t=\frac{P_t}{P_{t-1}}-1}$, the firm FOC becomes:
		{\small \begin{align*}
			(1-\theta)C_t + \theta\frac{W_t}{P_tA_t}C_t - \frac{\varphi W_t}{P_{t-1}}\pi_t
				&= -\Et{\Theta_{t,t+1}\frac{\varphi W_{t+1}P_{t+1}}{P_{t}^2}\pi_{t+1}}	\\
			(1-\theta)P_tC_t + \theta\frac{W_t}{A_t}C_t - \varphi W_t\pi_t(\pi_t+1)
				&= -\Et{\Theta_{t,t+1}\varphi W_{t+1}\pi_{t+1}(\pi_{t+1}+1)}	\\
			(1-\theta)P_t + \theta\frac{W_t}{A_t} 
				&= \varphi\Et{\frac{W_t}{C_t}\pi_t(\pi_t+1)-\frac{\Theta_{t,t+1}}{C_t} W_{t+1}\pi_{t+1}(\pi_{t+1}+1)}	
		\end{align*} }
		Where:
		\[
			\frac{\Theta_{t,t+1}}{C_t} = \beta\left(\frac{P_tC_t}{C_tP_{t+1}C_{t+1}}\right) = \beta C_{t+1}^{-1}(\pi_{t+1} + 1)^{-1}
		\]
		Thus, the simplified FOC is:
		\[
			(1-\theta)P_t + \theta\frac{W_t}{A_t} = \varphi\Et{\frac{W_t}{C_t}\pi_t(\pi_t+1)-\beta\frac{W_{t+1}}{C_{t+1}}\pi_{t+1}}
		\]
	
	\item With this simplified FOC, we can log-linearize to get a first-order approximation of deviations from the steady-state, nesting each side of the equation with:
		\begin{align*}
			&X_t = (1-\theta)P_t + \theta\frac{W_t}{A_t}	&Z_t = \frac{W_t}{C_t}\pi_t(\pi_t+1)-\beta\frac{W_{t+1}}{C_{t+1}}\pi_{t+1}
		\end{align*}
		Recall that ${\pi_t=\frac{P_t}{P_{t-1}}-1}$, so $\pi$ is already given as percent deviations from a steady-state of zero. Then, the first-order approximation of the sytstem is:
		\begin{align*}
			x_t	&= \varphi\Et{z_t}																								\\
			x_t &= \left(\frac{(1-\theta)\overline{P}}{(1-\theta)\overline{P}+\theta(\overline{W}/\overline{A})}\right)p_t +
					\left(\frac{\theta(\overline{W}/\overline{A})}{(1-\theta)\overline{P}+\theta(\overline{W}/\overline{A})}\right)(w_t-a_t)	\\
			z_t &= \frac{\overline{W}}{\overline{C}}\pi_t- \beta\frac{\overline{W}}{\overline{C}}\pi_{t+1}					
		\end{align*}
		Since ${\overline{\pi}=0}$ in the steady state, ${\overline{X}=0}$, so the denominators on the left-hand side are irrelevant, enabling the following simplification:
		\[
			(1-\theta)\overline{P}p_t + \theta\frac{\overline{W}}{\overline{A}}(w_t-a_t) 
					= \varphi\frac{\overline{W}}{\overline{C}}\pi_t- \varphi\beta\frac{\overline{W}}{\overline{C}}\Et{\pi_{t+1}}
		\]
		
	\item To determine the output gap in terms of the FOC variables, consider the labor supply curve and market-clearing conditions:
		\begin{align*}
			&Y_t = C_t 	&C_t = \frac{W_t}{P_t}	\\
			&y_t = c_t 	&c_t = w_t - p_t		\\
		\end{align*}
		Also note the steady-state equivalencies from question 1, which enable many cancellations, namely from ${\overline{W}=\overline{P}\overline{C}}$ and ${\frac{\overline{C}}{\overline{A}} = \frac{\theta-1}{\theta}}$:
		\begin{align*}
			\Rightarrow (1-\theta)\overline{P}p_t + \theta\frac{\overline{W}}{\overline{A}}(y_t+p_t-a_t) 
					&= \varphi\frac{\overline{W}}{\overline{C}}\pi_t- \varphi\beta\frac{\overline{W}}{\overline{C}}\Et{\pi_{t+1}}	\\
			\left[(1-\theta)\overline{P} + \theta\frac{\overline{P}\overline{C}}{\overline{A}}\right]p_t + 
				\theta\frac{\overline{P}\overline{C}}{\overline{A}}(y_t-a_t) 
				&= \varphi\frac{\overline{P}\overline{C}}{\overline{C}}\pi_t- \varphi\beta\frac{\overline{P}\overline{C}}{\overline{C}}\Et{\pi_{t+1}}	\\
			\left[(1-\theta) + \theta\frac{\theta-1}{\theta}\right]p_t + 
				\theta\frac{\theta-1}{\theta}(y_t-a_t)  &= \varphi\pi_t- \varphi\beta\Et{\pi_{t+1}}	\\
			\Rightarrow \pi_t = \frac{\theta-1}{\varphi}(y_t-a_t) + \beta\Et{\pi_{t+1}}
		\end{align*}
	
	\item The NKPC under Calvo pricing is:
		\[
			\pi_t = \frac{1}{\lambda}(1-\lambda)(1-\beta\lambda)(\sigma + \phi)\left(y_t + \frac{1+\phi}{\sigma+\phi}a_t\right) + \beta\Et{\pi_{t+1}}
		\]
		Where $\lambda$ is a price friction parameter between 0 and 1, and $\phi$ is the reciprocal of the IES for labor. Note that setting ${\phi=0}$ and ${\sigma=1}$ enables easier comparability between the utility function in our model and the one used in Calvo pricing. In this case, the output gaps of the two Phillips curves are equal, and the coefficient on the output gaps, while different, are more comparable. In each case, this coefficient is positive and increases with greater price stickiness.
	
\end{enumerate}



%%%________________________________________________________________%%%

\subsection*{Question 3}

The source of inflation costs in the Calvo model is the misallocation of labor that comes from some firms being unable to adjust their prices while others are able to: The firms that raise their prices are able to hire more workers than non-price-flexible firms, even though they have the same marginal costs. In this model, firms are symmetric, so there can be no such misallocation. Instead, there is an aggregate misallocation of labor, where workers are paid simply to change prices when, in a flexible-price model, nominal wages would adjust downward.


%%%________________________________________________________________%%%
\pagebreak
\subsection*{Question 4}

To find an equilibrium under this model with the Taylor rule ${i_t=\phi x_t + u_t}$, let us rewrite the NKPC and Euler equation as a linear system. First, we must write the Euler equation as a function of inflation and the output gap, $x_t$:
\begin{align*}
	c_t 			&= \Et{c_{t+1} + p_{t+1} - p_t - i_t}				\\
	x_t + a_t 		&= \Et{x_{t+1} + a_{t+1} + \pi_{t+1} - i_t}			\\
	\Et{x_{t+1}} 	&= -\Et{a_{t+1}-a_t} + x_t - \Et{\pi_{t+1}} + i_t
\end{align*}
As under Calvo pricing, we define the ``natural rate" as the one that prevails under flexible prices. Accounting for this and substituting the Taylor Rule and NKPC,
\begin{align*}
	r^n_t			&= \Et{a_{t+1}-a_t}																			\\
	\Et{x_{t+1}} 	&= -r^n_t + x_t + \frac{\theta-1}{\beta\varphi}x_t - \frac{1}{\beta}\pi_t + \phi x_t + u_t	\\
	\Et{x_{t+1}} 	&= \left(1+\phi + \frac{\theta-1}{\beta\varphi}\right)x_t - \frac{1}{\beta}\pi_t+u_t-r^n_t 	
\end{align*}
Taken as a linear system, we have:
\[
	\Et{\colvec{2}{\pi_{t+1}}{x_{t+1}}} = 
		\begin{pmatrix} \frac{1}{\beta}		& -\frac{\theta-1}{\beta\varphi} 			\\
						-\frac{1}{\beta}	& 1+\phi + \frac{\theta-1}{\beta\varphi}  								
		\end{pmatrix} \colvec{2}{\pi_t}{x_t} + \colvec{2}{0}{1}(u_t-r^n_t) 	= Av + Bc
\]
To determine the restriction on $\phi$ that will give this model a unique equilibrium, we can find the eigenvalues of the $A$ matrix:
\begin{align*}
	f(\lambda) &= \left(\frac{1}{\beta}-\lambda\right)\left(1+\phi+\frac{\theta-1}{\beta\varphi}-\lambda\right) + \frac{\theta-1}{\beta^2\varphi}			\\
	f(\lambda) &= \lambda^2 - \left(1+\phi+\frac{1}{\beta}+\frac{\theta-1}{\beta\varphi}\right)\lambda + \frac{1}{\beta}\left(1+\phi+\frac{\theta-1}{\beta\varphi}\right)
\end{align*}
In order for this system to have a unique equilibrium, each eigenvalue must be greater than one. This is true if ${f(0)>f(1)>0}$, or:
\begin{align*}
	f(0)-f(1) 																&> 0														\\
	-1+\left(1+\phi+\frac{1}{\beta}+\frac{\theta-1}{\beta\varphi}\right)	&> 0														\\
																	\phi	&< \frac{1}{\beta}\left(\frac{\theta-1}{\varphi}+1\right)
\end{align*}



%%%________________________________________________________________%%%





\end{document}






