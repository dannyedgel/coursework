%%% Econ714: Macroeconomics II
%%% Spring 2021
%%% Danny Edgel
%%%
% Due on Canvas Monday, February 15th, 11:59pm Central Time
%%%

%%%
%							PREAMBLE
%%%

\documentclass{article}

%%% declare packages
\usepackage{amsmath}
\usepackage{amssymb}
\usepackage{array}
\usepackage{bm}
\usepackage{changepage}
\usepackage{centernot}
\usepackage{graphicx}
\usepackage{xcolor}
\usepackage[shortlabels]{enumitem}
\usepackage{fancyhdr}
	\fancyhf{} % sets both header and footer to nothing
	\renewcommand{\headrulewidth}{0pt}
    \rfoot{Edgel, \thepage}
    \pagestyle{fancy}
	
%%% define shortcuts for set notation
\newcommand{\Z}{\mathbb{Z}}
\newcommand{\R}{\mathbb{R}}
\newcommand{\Q}{\mathbb{Q}}
\newcommand{\lmt}{\underset{x\rightarrow\infty}{\text{lim }}}
\newcommand{\neglmt}{\underset{x\rightarrow-\infty}{\text{lim }}}
\newcommand{\zerolmt}{\underset{x\rightarrow 0}{\text{lim }}}
\newcommand{\loge}[1]{\text{log}\left(#1\right)}
\newcommand{\usmax}[1]{\underset{#1}{\text{max }}}
\newcommand{\Mt}{M_{t+1}^t}
\newcommand{\vhat}{\hat{v}}
\newcommand{\olp}{\overline{p}}
\renewcommand{\L}{\mathcal{L}}
\newcommand{\olq}{\overline{q}}
\newcommand{\zinf}{_{t=0}^\infty}
\newcommand{\aneg}{A^{-1}}
\newcommand{\sneg}{s^{-1}}
\newcommand{\olk}{\overline{k}}
\newcommand{\olc}{\overline{c}}
\newcommand{\olr}{\overline{r}}
\newcommand{\olpi}{\overline{\pi}}
\newcommand{\Aneg}{A^{-1}}
\renewcommand{\sneg}{s^{-1}}
\newcommand{\dc}[1]{\Delta c_{#1}}
\newcommand{\N}{\mathcal{N}}
\newcommand{\suminf}{\sum_{t=0}^\infty}
\newcommand{\red}[1]{{\color{red}#1}}

\newcommand{\E}[1]{\mathbb{E}\left[#1\right]} % expected value
\newcommand{\Et}[1]{\mathbb{E}_t\left[#1\right]}

%%% define column vector command (from Michael Nattinger)
\newcount\colveccount
\newcommand*\colvec[1]{
        \global\colveccount#1
        \begin{pmatrix}
        \colvecnext
}
\def\colvecnext#1{
        #1
        \global\advance\colveccount-1
        \ifnum\colveccount>0
                \\
                \expandafter\colvecnext
        \else
                \end{pmatrix}
        \fi
}

%%% define function for drawing matrix augmentation lines
\newcommand\aug{\fboxsep=-\fboxrule\!\!\!\fbox{\strut}\!\!\!}

\makeatletter
\let\amsmath@bigm\bigm

\renewcommand{\bigm}[1]{%
  \ifcsname fenced@\string#1\endcsname
    \expandafter\@firstoftwo
  \else
    \expandafter\@secondoftwo
  \fi
  {\expandafter\amsmath@bigm\csname fenced@\string#1\endcsname}%
  {\amsmath@bigm#1}%
}


%________________________________________________________________%

\begin{document}

\title{	Problem Set \#3 }
\author{ 	Danny Edgel 					\\ 
			Econ 714: Macroeconomics II		\\
			Spring 2021						\\
		}
\maketitle\thispagestyle{empty}

%%%________________________________________________________________%%%

\noindent\textit{Discussed and/or compared answers with Sarah Bass, Emily Case, Katherine Kwok, Michael Nattinger, and Alex Von Hafften}
\bigskip \\
\noindent For all of the problems below, all computational work is performed in $\texttt{edgel\_ps3.m}$, which is attached. This code is heavily commented so as to mostly stand on its own. As a result, I will provide little explicit information about the code in this document, leaving it only for answering justifications, deriving relationships, etc.

%%%________________________________________________________________%%%

\subsection*{Questions 1 and 2}
See first two sections of the attached code.


%%%________________________________________________________________%%%

\subsection*{Question 3}
Since both capital and investment have moving steady states, it is reasonable to pick an arbitrary date to assume a steady state at some period prior to the sample period, then carrying the variables forward. With enough periods, the effect of assuming a steady state in the prior period is minimal-to-nonexistent. By the time the sample period begins, the steady state that $k_t$ is relative to actually comes from the data rather than the earlier assumption.


%%%________________________________________________________________%%%

\subsection*{Question 4}
The table below displays the persistence parameters from the three wedges, $a_t$, $g_t$, and $\tau_{Lt}$.
\begin{center} \begin{tabular}{r|cccccc}
 & Mean & $\Sigma$ & Income & Income$^2$ & Age & $\one{\text{Child}}$ \\\hline &&&&&&\\ 
Constant   & -2.314 & 0.377  & 3.089  & 0.000 & 1.186 & 0.000\\ 
 & (0.246) & (0.122)  & (1.007)  & (0.000) & (1.010) &(0.000) \\ 
Price & -32.430 & 1.848  & 16.598  & -0.659 & 0.000 & 11.624 \\ 
 & (4.826) & (0.978)  & (104.545)  & (5.457) & (0.000) &(5.137) \\ 
Sugar & 0.117 & -0.003  & -0.193  & 0.000 & 0.030 & 0.000 \\ 
 & (0.246) & (0.012)  & (0.040)  & (0.000) & (0.034) &(0.000) \\ 
Mushy & 0.271 & 0.081  & 1.468  & 0.000 & -1.514 & 0.000 \\ 
 & (0.012) & (0.198)  & (0.676)  & (0.000) & (1.095) &(0.000) \\ 
 &&&&&& \\ 
\multicolumn{7}{l}{GMM objective: 14.998} \\ 
\multicolumn{7}{l}{MD $R^2$: 0.395} \\ 
\multicolumn{7}{l}{MD weighted $R^2$: 0.077} \\\hline 
\end{tabular} \end{center}



%%%________________________________________________________________%%%

\subsection*{Question 5}
Implementing Blanchard-Kahn to solve this model results in a linear relationship between $c_t$ and $k_t$ in the saddle path of the model. Log-linearizing the Euler equation and resource constraint, results in (after some serious algebra) a system of the following form:
\[
	x_{t+1} = \colvec{2}{k_{t+1}}{c_{t+1}} = A\colvec{2}{k_t}{c_t} + B\colvec{4}{a_t}{g_t}{\hat{\tau_{It}}}{\hat{\tau_{Lt}}} = Ax_t + Bz_t
\]
Where $B$ does not need to be directly parametrically solved, and:
\[
	A = \begin{pmatrix}
			\frac{\alpha\delta\overline{Y}}{\overline{I}} + 1-\delta + \frac{\delta(1-\alpha)\alpha\overline{Y}}{\overline{I}(\phi+\alpha)}	
			& \frac{\delta(\alpha-1)\sigma\overline{Y}}{\overline{I}(\phi+\alpha)}-\frac{\delta\overline{C}}{\overline{I}}	\\
			\left(\frac{\phi\Gamma}{1-\Gamma}\right)
			\left(\frac{\alpha\delta\overline{Y}}{\overline{I}} + 1 - \delta + \frac{\delta(1-\alpha)\alpha\overline{Y}}{\overline{I}(\phi + \alpha)}\right) &
			\frac{1}{1-\Gamma}\left(\sigma + \phi\Gamma
			\left(\frac{\delta(\alpha-1)\sigma\overline{Y}}{\overline{I}(\phi+\alpha)}-\frac{\delta\overline{C}}{\overline{I}}\right)\right)
		\end{pmatrix}
\]
Where $\overline{X}$ is the steady state of everything in the parentheses of the RHS of the Euler equation and: $$ \theta = \alpha\overline{A}\overline{K}^{\alpha-1}\overline{L}^{1-\alpha}\text{, }\Gamma = \frac{\theta(1-\alpha)}{\overline{X}(\phi+\alpha)}$$
Note that, since the steady state of each steady state in the model is not constant, the mean of each steady state is used. This does little to influence the entries of $A$, however, as they use ratios of one steady state to another, and the rates of, say, consumption to output, move little in the sample. \input{q5_a21.tex} In the case of the second row, the steady states cancel out, obviating the need for taking means.

The saddle path, holding all wedges constant, is displayed in the chart below.
\begin{center}\includegraphics[scale=.75]{figure5.png}\end{center}


%%%________________________________________________________________%%%

\subsection*{Question 6}
The Euler equation can be re-arranged to isolate $\hat{\tau}_{It}$:
\[
	\hat{\tau}_{It} = \frac{\overline{X}\overline{\tau}_I\sigma(c_t - \E{c_{t+1}}) + \alpha\overline{\tau}_I\overline{A}\overline{K}^{\alpha-1}\overline{L}^{1-\alpha}\left(\rho_aa_t + (\alpha-1)k_{t+1} + (1-\alpha)\E{\ell_{t+1}}\right)}{\overline{X} - \overline{\tau}_I\rho_{\tau_I}(1-\delta)}
\]
Where:
\[
	\overline{\tau}_I = \frac{\alpha\beta\overline{A}\overline{K}^{\alpha-1}\overline{L}^{1-\alpha}}{1-\beta(1-\delta)} - 1
\]
Solving for the fixed-point estimate of $\hat{\tau}_{It}$ results in a persistence parameter of \input{q6.tex}.

%%%________________________________________________________________%%%

\subsection*{Question 7}
\begin{center}\includegraphics[scale=1]{figure7.png}\end{center}


%%%________________________________________________________________%%%

\subsection*{Question 8}
In order to graph the effect of each wage separately, we must explicitly solve for the parameterized $B$ matrix:
\[
	Bz = \begin{pmatrix}
			\kappa & \frac{\theta(1-\alpha)}{\sigma\overline{X}(\phi+\alpha)}\left(\rho_A - \frac{\delta(1-\alpha)\overline{Y}}{\overline{I}}\right) \\ 
			- \frac{\delta\overline{Y}}{3\overline{I}}   & -\frac{\theta(\alpha-1)\delta\overline{Y}}{3\sigma\overline{X}\overline{Y}} \\
			0 & \left(\frac{(1-\delta)\rho_I}{\overline{X}}-\frac{1}{\overline{\tau}_I}\right)\frac{1+\overline{\tau}_I}{\sigma} \\
			 - \left(\frac{\overline{\tau}_L}{1-\overline{\tau}_L}\right)\kappa & \frac{\theta(1-\alpha)\overline{\tau}_L}{\sigma\overline{X}(\phi+\alpha)(1-\overline{\tau}_L)}\left(\rho_L + \frac{\delta(\alpha-1)\overline{Y}}{\overline{I}}\right)
		 \end{pmatrix}'
		 \colvec{4}{a_t}{g_t}{\hat{\tau_{It}}}{\hat{\tau}_{Lt}}
\]
Where $\theta$ and $\overline{X}$ are defined in question 5 and: $$ \kappa = \frac{\delta(1-\alpha)\overline{Y}}{\overline{I}(\phi + \alpha)} $$
With this matrix defined, we can extract counterfactuals of GDP's deviations from its steady state with all wedges except for one set to zero. for example, we can define a counterfactual deviation from consumption's steady state in period $t$ with \textit{only} a productivity wedge as:
\[
	c_t = \frac{q^{-1}_{11}}{q^{-1}_{12}}k_t - \frac{q^{-1}_{11}b_{11} + q^{-1}_{12}b_{21}}{\lambda_1q^{-1}_{12}}
\]
Assuming $\lambda_1$ is the explosive eigenvalue. We can then use the optimal labor condition to calculate labor, and, given labor and consumption, iterate through the rest of the time series using the law of motion of capital. The result is displayed below, against observed GDP deviations.
\begin{center}\includegraphics[scale=1]{figure8.png}\end{center}

%%%________________________________________________________________%%%



\end{document}






