%%% Econ714: Macroeconomics II
%%% Spring 2021
%%% Danny Edgel
%%%
% Due on Canvas Monday, February 22nd, 11:59pm Central Time
%%%

%%%
%							PREAMBLE
%%%

\documentclass{article}

%%% declare packages
\usepackage{amsmath}
\usepackage{amssymb}
\usepackage{array}
\usepackage{bm}
\usepackage{changepage}
\usepackage{centernot}
\usepackage{graphicx}
\usepackage{xcolor}
\usepackage[shortlabels]{enumitem}
\usepackage{fancyhdr}
	\fancyhf{} % sets both header and footer to nothing
	\renewcommand{\headrulewidth}{0pt}
    \rfoot{Edgel, \thepage}
    \pagestyle{fancy}
	
%%% define shortcuts for set notation
\newcommand{\Z}{\mathbb{Z}}
\newcommand{\R}{\mathbb{R}}
\newcommand{\Q}{\mathbb{Q}}
\newcommand{\lmt}{\underset{x\rightarrow\infty}{\text{lim }}}
\newcommand{\neglmt}{\underset{x\rightarrow-\infty}{\text{lim }}}
\newcommand{\zerolmt}{\underset{x\rightarrow 0}{\text{lim }}}
\newcommand{\loge}[1]{\text{log}\left(#1\right)}
\newcommand{\usmax}[1]{\underset{#1}{\text{max }}}
\newcommand{\usmin}[1]{\underset{#1}{\text{min }}}
\newcommand{\Mt}{M_{t+1}^t}
\newcommand{\vhat}{\hat{v}}
\newcommand{\olp}{\overline{p}}
\renewcommand{\L}{\mathcal{L}}
\newcommand{\olq}{\overline{q}}
\newcommand{\zinf}{_{t=0}^\infty}
\newcommand{\aneg}{A^{-1}}
\newcommand{\sneg}{s^{-1}}
\newcommand{\olk}{\overline{k}}
\newcommand{\olc}{\overline{c}}
\newcommand{\olr}{\overline{r}}
\newcommand{\olpi}{\overline{\pi}}
\newcommand{\Aneg}{A^{-1}}
\renewcommand{\sneg}{s^{-1}}
\newcommand{\dc}[1]{\Delta c_{#1}}
\newcommand{\N}{\mathcal{N}}
\newcommand{\suminf}{\sum_{t=0}^\infty}
\newcommand{\sumn}{\sum_{i=1}^{n}}
\newcommand{\sumnk}{\sum_{i=1}^{N_k}}
\newcommand{\red}[1]{{\color{red}#1}}

\newcommand{\E}[1]{\mathbb{E}\left[#1\right]} % expected value
\newcommand{\Et}[1]{\mathbb{E}_t\left[#1\right]}

%%% define column vector command (from Michael Nattinger)
\newcount\colveccount
\newcommand*\colvec[1]{
        \global\colveccount#1
        \begin{pmatrix}
        \colvecnext
}
\def\colvecnext#1{
        #1
        \global\advance\colveccount-1
        \ifnum\colveccount>0
                \\
                \expandafter\colvecnext
        \else
                \end{pmatrix}
        \fi
}

%%% define function for drawing matrix augmentation lines
\newcommand\aug{\fboxsep=-\fboxrule\!\!\!\fbox{\strut}\!\!\!}

\makeatletter
\let\amsmath@bigm\bigm

\renewcommand{\bigm}[1]{%
  \ifcsname fenced@\string#1\endcsname
    \expandafter\@firstoftwo
  \else
    \expandafter\@secondoftwo
  \fi
  {\expandafter\amsmath@bigm\csname fenced@\string#1\endcsname}%
  {\amsmath@bigm#1}%
}


%________________________________________________________________%

\begin{document}

\title{	Problem Set \#4 }
\author{ 	Danny Edgel 					\\ 
			Econ 714: Macroeconomics II		\\
			Spring 2021						\\
		}
\maketitle\thispagestyle{empty}

%%%________________________________________________________________%%%

\noindent\textit{Discussed and/or compared answers with Sarah Bass, Emily Case, Katherine Kwok, Michael Nattinger, and Alex Von Hafften}
\medskip \\

%%%________________________________________________________________%%%

\subsection*{Questions 1}

The static household cost-minimization problem is given by:
\[
	\usmin{\{C_{ik}\}} \int\left(\sumnk P_{ik}C_{ik}\right) dk 
		\text{ s.t. } 	
		\left(\int C_k^{\frac{\rho-1}{\rho}}\right)^{\frac{\rho}{\rho-1}} = C \text{, }
		\left(\sumnk C_{ik}^{\frac{\theta-1}{\theta}}\right)^{\frac{\theta}{\theta-1}} = C_k
\]
This problem is represented by the Lagrangian:
{\small 
	\[
		\L = -\int\left(\sumnk P_{ik}C_{ik}\right) dk 
				+ P\left[\left(\int C_k^{\frac{\rho-1}{\rho}}\right)^{\frac{\rho}{\rho-1}}-C\right] + \int P_k\left[\left(\sumnk C_{ik}^{\frac{\theta-1}{\theta}}\right)^{\frac{\theta}{\theta-1}} - C_k\right]dk
	\]
}
An implicit assumption with this setup is that households minimize their expenditures while setting an overall utility threshold across industries, then optimize consumption utility within each industry (i.e. across firms). This problem has two first order conditions, with respect to $C_{ik}$ and $C_k$:
\begin{align*}
	\frac{\partial\L}{\partial C_{ik}} 	&= -P_{ik} + P_k\left(\frac{\theta}{\theta-1}\right) 
		\left(\sumnk C_{ik}^{\frac{\theta-1}{\theta}}\right)^{\frac{1}{\theta-1}}\left(\frac{\theta-1}{\theta}\right)C_{ik}^{\frac{-1}{\theta}} = 0	\\
	\frac{\partial\L}{\partial C_k} 	&= P\left(\frac{\rho-1}{\rho}\right)
		\left(\int C_k^{\frac{\rho}{\rho-1}}\right)^{\frac{1}{\rho-1}}\left(\frac{\rho-1}{\rho}\right)C_k^{\frac{-1}{\rho}} - P_k = 0
\end{align*}
We can then solve each equation to isolate our choice variables:
\begin{align*}
	&C_{ik} 	= \left(\frac{P_{ik}}{P_k}\right)^{-\theta}C_k
	&C_k		= \left(\frac{P_k}{P}\right)^{-\rho}C
\end{align*}
Substituting these definitions back into the constraints enable us to define the price indexes:
\begin{align*}
	&C_k	= \left[\sumnk \left(\left(\frac{P_{ik}}{P_k}\right)^{-\theta}C_k\right)^{\frac{\theta-1}{\theta}}\right]^{\frac{\theta}{\theta-1}}	
		&C	= \left[\int \left(\left(\frac{P_k}{P}\right)^{-\rho}C\right)^{\frac{\rho-1}{\rho}}dk\right]^{\frac{\rho}{\rho-1}}	\\
	&1		= P_k^\theta\left(\sumnk P_{ik}^{1-\theta}\right)^{\frac{\theta}{\theta-1}}	
		&1	= P^\rho\left(\int P_k^{1-\rho}dk\right)^{\frac{\rho}{\rho-1}}	\\
	&P_k 	= \left(\sumnk P_{ik}^{1-\theta}\right)^{\frac{1}{1-\theta}}
		&P	= \left(\int P_k^{1-\rho}dk\right)^{\frac{1}{1-\rho}}			\\
\end{align*}
Finally, we can substitute these price indices back into the consumption equation to delineate optimal demand for each good, as a function only of overall consumption utility and relative prices, $P_{ik}$:
\[
	C_{ik} 	= P_{ik}^{-\theta}
				\left(\sumnk P_{ik}^{1-\theta}\right)^{\frac{\theta-\rho}{1-\theta}}
				\left[\int \left(\sumnk P_{ik}^{1-\theta}\right)^{\frac{1-\rho}{1-\theta}}dk\right]^{\frac{\rho}{1-\rho}}C
\]

%%%________________________________________________________________%%%

\subsection*{Question 2}

Assuming that firms engage in Bertrand competition, then firm $i$ in industry $k$ maximizes its profit by only choosing its own price, $P_{ik}$, taking the prices of all other firms as a given, rather than treating them as a function of $P_{ik}$. Firms also take the going wage, $W$, as given and seek to match output to demand, ${Y_{ik}=C_{ik}}$. Then, the firm's problem is:
\[
	\usmax{P_{ik},L_{ik},Y_{ik}}P_{ik}Y_{ik} - WL_{ik} \text{ s.t. } Y_{ik} = A_{ik}L_{ik}\text{, }Y_{ik} = C_{ik}
\]
Substituting the demand for the firm's good for $Y_{ik}$ and rearranging the production function yields an objective function that depends only on $P_{ik}$:
{\small \begin{align*}
	P_{ik}Y_{ik} - WL_{ik} 	&= C_{ik}\left(P_{ik}-\frac{W}{A_{ik}}\right)																		\\
							&=  P_{ik}^{-\theta} \left(\sumnk P_{ik}^{1-\theta}\right)^{\frac{\theta-\rho}{1-\theta}}P^\rho C
								\left(P_{ik}-\frac{W}{A_{ik}}\right)																			\\
							&=  P^\rho C \left[P_{ik}^{1-\theta} \left(\sumnk P_{ik}^{1-\theta}\right)^{\frac{\theta-\rho}{1-\theta}}
								- \frac{W}{A_{ik}}P_{ik}^{-\theta} \left(\sumnk P_{ik}^{1-\theta}\right)^{\frac{\theta-\rho}{1-\theta}}\right]
\end{align*} }
Recall our equations for $P$, which is technically a function of $P_{ik}$, but the firm occupies zero mass with respect to $P$. Therefore, $P_{ik}$ does not influence its value. The firm also does not influence $C$. Then, for the purposes of optimation, our objective function is:
\[
	P_{ik}^{1-\theta} \left(\sumnk P_{ik}^{1-\theta}\right)^{\frac{\theta-\rho}{1-\theta}}
								- \frac{W}{A_{ik}}P_{ik}^{-\theta} \left(\sumnk P_{ik}^{1-\theta}\right)^{\frac{\theta-\rho}{1-\theta}}
\]
Recall also our equation for $P_k$, which was not used in defining the objective function but can be used to simplify the final result. The first-order condition for the firm, then, is:
{\small \begin{align*}
	(1-\theta)P_{ik}^{-\theta}P_k^{\theta-\rho} + P_{ik}^{1-\theta}(\theta-\rho)P_k^{2\theta-\rho-1}P_{ik}^{-\theta}
		&= \frac{W}{A_{ik}}\left[-\theta P_{ik}^{-\theta-1}P_k^{\theta-\rho} + P_{ik}^{-\theta}(\theta-\rho)P_k^{2\theta-\rho-1}P_{ik}^{-\theta}\right]			\\
	P_{ik}^{-\theta}P_k^{\theta-\rho}\left[1-\theta + (\theta-\rho)P_{ik}^{1-\theta}P_k^{\theta-1}\right] 
		&= \frac{W}{A_{ik}}P_{ik}^{-\theta}P_k^{\theta-\rho}\left[-\theta P_{ik}^{-1} + (\theta-\rho)P_{ik}^{-\theta}P_k^{\theta-1}\right] 						\\
	1-\theta + (\theta-\rho)P_{ik}^{1-\theta}P_k^{\theta-1} &= \frac{W}{A_{ik}}\left(-\theta P_{ik}^{-1} + (\theta-\rho)P_{ik}^{-\theta}P_k^{\theta-1}\right)	\\
	1-\theta + (\theta-\rho)\left(\frac{P_{ik}}{P_k}\right)^{1-\theta} &= \frac{W}{A_{ik}}P_{ik}^{-1}\left[-\theta + (\theta-\rho)\left(\frac{P_{ik}}{P_k}\right)^{1-\theta}\right]
\end{align*} }
Let $s_{ik}$ represent firm $i$'s share of the price of one unit of utility in its industry\textemdash i.e., ${s_{ik} = \frac{P_{ik}}{P_k}}$. Then,
\[
	P_{ik} = \frac{W}{A_{ik}}\left(\frac{(\theta-\rho)s_{ik}^{1-\theta} - \theta}{(\theta-\rho)s_{ik}^{1-\theta} + 1-\theta}\right)
\]
Finally, the elasticity of demand for the firm, $\varepsilon(P_{ik})$ reuses much of the alegebra from above:
\begin{align*}
	\frac{dC_{ik}}{dP_{ik}}			&= CP^\rho P_{ik}^{-\theta-1}P_k^{\theta-\rho}\left[-\theta + (\theta-\rho)s_{ik}^{1-\theta}\right]		\\
	\frac{P_{ik}}{C_{ik}}			&= \left[CP^\rho P_{ik}^{-\theta-1}P_k^{\theta-\rho}\right]^{-1}										\\
	\Rightarrow \varepsilon(P_{ik})	&= \frac{dC_{ik}}{dP_{ik}}\frac{P_{ik}}{C_{ik}} = (\theta-\rho)s_{ik}^{1-\theta} - \theta
\end{align*}

%%%________________________________________________________________%%%
\pagebreak
\subsection*{Question 3}
The total costs of firm $i$ in industry $k$ are given by $WL_{ik}$, where ${L_{ik} = A_{ik}^{-1}Y_{ik}}$. The marginal cost of production for such a firm, then, is represented by ${WA_{ik}^{-1}}$. Recall that we solved the price charged by firm $i$ in industry $k$ as:
\[
	P_{ik} = \frac{W}{A_{ik}}\left(\frac{(\theta-\rho)s_{ik}^{1-\theta} - \theta}{(\theta-\rho)s_{ik}^{1-\theta} + 1-\theta}\right)
\]
Then, the markup of this firm is
\begin{align*}
	\eta_{ik} 	&= \frac{(\theta-\rho)s_{ik}^{1-\theta} - \theta}{(\theta-\rho)s_{ik}^{1-\theta} + 1-\theta}					\\
				&= 1 - \frac{1}{(\theta-\rho)\left(\frac{P_{ik}}{P_k}\right)^{1-\theta} + 1-\theta}								\\
				&= 1- \left[(\theta-\rho)\left(WA_{ik}^{-1}\eta_{ik}\right)^{1-\theta}P_k^{\theta-1} + 1-\theta\right]^{-1}		\\
				&= 1- \left[(\theta-\rho)W^{1-\theta}A_{ik}^{\theta-1}\eta_{ik}^{1-\theta}P_k^{\theta-1} + 1-\theta\right]^{-1}	
\end{align*}
This is markup function is recursive, so its derivative with respect to $A_{ik}$ cannot be exactly determined, but we can determine the effect of $A_{ik}$ on $\eta_{ik}$ to a first approximation:
\[
	\frac{\partial \eta_{ik}}{\partial A_{ik}} \approx \frac{(\theta-1)(\theta-\rho)s_{ik}^{1-\theta}}{A_{ik}\left[(\theta-\rho)s_{ik}^{1-\theta}+1-\theta\right]^2}
\]
This value is positive, since ${\theta>\rho\geq 1}$ by assumption. Thus, firms with higher $A_{ik}$ charge higher markups, \textit{ceteris paribus}.

%%%________________________________________________________________%%%

\subsection*{Question 4}

See the attached code, which is also pasted at the end of this document, for the solution to this model. The solution simply requires assuming a value of $s_{ik}$ for each firm, calculating $P_{ik}$ and $P_k$, then recalculating $s_{ik}$ and repeating until the absolute difference between iterations of $s_{ik}$ fall below some threshold. The solution yields the following real wage, rounded to the nearest thousandth:


%%%________________________________________________________________%%%
\pagebreak
\subsection*{Question 5}

The real wage in the model, $\frac{W}{P}$, is equal to $C$. The first-best value for $C$ (i.e. the one that a social planner would choose) would come from all firms charging their marginal cost: ${P_{ik} = \frac{W}{A_{ik}}}$. Each of these values from my solution to the model are given below, rounded to the neareast thousandth.
\begin{align*}
	\input{q5.tex}
\end{align*}
It is immediately apparent that household wages are meaningfully higher in the social planner's allocation than the competitive equilibirum. Then is to be expected, as the assumptions of the first fundamental theorem of welfare are not satisfied. Namely, the market is not perfectly competitive.



%%%________________________________________________________________%%%




\end{document}






