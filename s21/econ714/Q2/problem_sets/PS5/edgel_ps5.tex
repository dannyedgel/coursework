%%% Econ714: Macroeconomics II
%%% Spring 2021
%%% Danny Edgel
%%%
% Due on Canvas Friday, April 23rd, 11:59pm Central Time
%%%

%%%
%							PREAMBLE
%%%

\documentclass{article}

%%% declare packages
\usepackage{amsmath}
\usepackage{amssymb}
\usepackage{array}
\usepackage{bm}
\usepackage{changepage}
\usepackage{centernot}
\usepackage{graphicx}
\usepackage{xcolor}
\usepackage[shortlabels]{enumitem}
\usepackage{fancyhdr}
	\fancyhf{} % sets both header and footer to nothing
	\renewcommand{\headrulewidth}{0pt}
    \rfoot{Edgel, \thepage}
    \pagestyle{fancy}
	
%%% define shortcuts for set notation
\newcommand{\Z}{\mathbb{Z}}
\newcommand{\R}{\mathbb{R}}
\newcommand{\Q}{\mathbb{Q}}
\newcommand{\lmt}{\underset{x\rightarrow\infty}{\text{lim }}}
\newcommand{\neglmt}{\underset{x\rightarrow-\infty}{\text{lim }}}
\newcommand{\zerolmt}{\underset{x\rightarrow 0}{\text{lim }}}
\newcommand{\loge}[1]{\text{log}\left(#1\right)}
\newcommand{\usmax}[1]{\underset{#1}{\text{max }}}
\newcommand{\usmin}[1]{\underset{#1}{\text{min }}}
\newcommand{\Mt}{M_{t+1}^t}
\newcommand{\vhat}{\hat{v}}
\newcommand{\olp}{\overline{p}}
\renewcommand{\L}{\mathcal{L}}
\newcommand{\olq}{\overline{q}}
\newcommand{\zinf}{_{t=0}^\infty}
\newcommand{\aneg}{A^{-1}}
\newcommand{\sneg}{s^{-1}}
\newcommand{\olk}{\overline{k}}
\newcommand{\olc}{\overline{c}}
\newcommand{\olr}{\overline{r}}
\newcommand{\olpi}{\overline{\pi}}
\newcommand{\Aneg}{A^{-1}}
\renewcommand{\sneg}{s^{-1}}
\newcommand{\dc}[1]{\Delta c_{#1}}
\newcommand{\N}{\mathcal{N}}
\newcommand{\suminf}{\sum_{t=0}^\infty}
\newcommand{\sumn}{\sum_{i=1}^{n}}
\newcommand{\sumnk}{\sum_{i=1}^{N_k}}
\newcommand{\red}[1]{{\color{red}#1}}
\newcommand{\Tau}{\mathrm{T}}
\newcommand{\phat}{\hat{p}}

\newcommand{\E}[1]{\mathbb{E}\left[#1\right]} % expected value
\newcommand{\Et}[1]{\mathbb{E}_t\left[#1\right]}

%%% define column vector command (from Michael Nattinger)
\newcount\colveccount
\newcommand*\colvec[1]{
        \global\colveccount#1
        \begin{pmatrix}
        \colvecnext
}
\def\colvecnext#1{
        #1
        \global\advance\colveccount-1
        \ifnum\colveccount>0
                \\
                \expandafter\colvecnext
        \else
                \end{pmatrix}
        \fi
}

%%% define function for drawing matrix augmentation lines
\newcommand\aug{\fboxsep=-\fboxrule\!\!\!\fbox{\strut}\!\!\!}

\makeatletter
\let\amsmath@bigm\bigm

\renewcommand{\bigm}[1]{%
  \ifcsname fenced@\string#1\endcsname
    \expandafter\@firstoftwo
  \else
    \expandafter\@secondoftwo
  \fi
  {\expandafter\amsmath@bigm\csname fenced@\string#1\endcsname}%
  {\amsmath@bigm#1}%
}


%________________________________________________________________%

\begin{document}

\title{	Problem Set \#5 }
\author{ 	Danny Edgel 					\\ 
			Econ 714: Macroeconomics II		\\
			Spring 2021						\\
		}
\maketitle\thispagestyle{empty}

%%%________________________________________________________________%%%

\noindent\textit{Discussed and/or compared answers with Sarah Bass, Emily Case, Katherine Kwok, Michael Nattinger, and Alex Von Hafften}

%%%________________________________________________________________%%%

\subsection*{Question 1}

\begin{enumerate}
	\item A utilitarian planner chooses an allocation, ${\left\{c_1^l,c_1^h,c_2^l,c_2^h,y^l,y^h\right\}}$ that maximizes total utility:
		\[
			\frac{1}{2}\left[u(c_1^h) + \beta u(c_2^h) - \nu(y^h)\right] + \frac{1}{2}\left[u(c_1^l) + \beta u(c_2^l)\right]
		\]
		Subject to the (simplified\footnote{By recognizing that each period's constraint binds at the optimal allocation, be can eliminate the savings variable and consolidate the period-specific constraints into a single resource constraint.}) resource constraint:
		\[
			c_1^l + c_1^h + \frac{c_2^l + c_2^h}{R} = y^h
		\]
		And each type of agent's incentive compatability constraint:
		\begin{align*}
			u(c_1^h) + \beta u(c_2^h) - \nu(y^h)	&\geq u(c_1^l) + \beta u(c_2^l) - \nu(y^l)	&\text{(high-productivity IC)}	\\
						u(c_1^l) + \beta u(c_2^l)	&\geq u(c_1^h) + \beta u(c_2^h)				&\text{(low-productivity IC)}
		\end{align*}
		Note, however, that ${y^l=0}$ regardless of hours worked. Then the low type cannot report as the high type, but the high type can report as the low type, since the planner can observe output. Thus, the Lagrangian of the planner's problem is:
		{\small \begin{align*}
			\L = \frac{1}{2}\left[u(c_1^h) + \beta u(c_2^h) - \nu(y^h)\right] &+ \frac{1}{2}\left[u(c_1^l) + \beta u(c_2^l)\right] 	
					- \lambda\left[c_1^l + c_1^h + \frac{c_2^l + c_2^h}{R} - y^h\right]											\\
					&+ \mu\left[u(c_1^h) + \beta u(c_2^h) - \nu(y^h)	- u(c_1^l) - \beta u(c_2^l)\right]
		\end{align*} }
		Then, the planner's FOC are:
		\begin{align*}
			\frac{\partial\L}{\partial c_1^l} 	&=  .5u'(c_1^l) - \lambda - \mu u'(c_1^l)	= 0							\\
			\frac{\partial\L}{\partial c_1^h} 	&= 	.5u'(c_1^h) - \lambda + \mu u'(c_1^h)	= 0							\\
			\frac{\partial\L}{\partial c_2^l} 	&= 	.5\beta u'(c_2^l) - \frac{\lambda}{R} - \beta\mu u'(c_2^l) = 0	\\
			\frac{\partial\L}{\partial c_2^h} 	&= 	.5\beta u'(c_2^h) - \frac{\lambda}{R} + \beta\mu u'(c_2^h) = 0	\\
			\frac{\partial\L}{\partial y^h} 	&= -.5\nu'(y^h) + \lambda -\mu\nu'(y^h)		= 0	
		\end{align*}
		Let us begin with the intertemporal optimization condition. This yields the same result for either type, but for illustrative purposes, consider each period's consumption FOC for the low type (recall that ${\beta=1/R}$):
		\begin{align*}
			\lambda 	&= u'(c_1^l)\left(\frac{1}{2} - \mu\right) = R\beta u'(c_2^l)\left(\frac{1}{2} - \mu\right)	\\
			\Rightarrow \frac{u'(c_1^l)}{u'(c_2^l)} &= 1
		\end{align*}
		Thus, ${c_1^l=c_2^l}$. It can be trivially shown that the same is true for the high type. Thus, there is perfect consumption smoothly across periods. Then, for comparing consumption across agents, we need only consider one period's consumption. Consider the FOC for each agent's consumption in the first period:
		\begin{align*}
			\lambda 	&= u'(c_1^l)\left(\frac{1}{2} - \mu\right) = u'(c_1^h)\left(\frac{1}{2} + \mu\right)	\\
			\Rightarrow \frac{u'(c_1^h)}{u'(c_1^l)} &= \frac{1 - 2\mu}{1 + 2\mu}
		\end{align*}
		Since the IC constraint binds, ${\mu>0}$ by complementary slackness. Since $u$ is concave, this ratio being less than one implies that ${c^h_t>c^l_t}$. This is clear from the IC constraint, since the high type needs to be compensated with additional consumption in order to have any incentive to report as a high type and be forced to work. Finally, we can use the FOC for $y^h$ solve for the optimal level of output:
		\[
			\nu'(y^h) = \frac{\lambda}{1/2 + \mu}
		\]
		Note that this can be combined with the first-period consumption FOC for the high type to derive:
		\[
			\nu'(y^h) = u'(c_1^h)
		\]
		In other words, there is no distortion for the productive agent.
		
	\item Let ${\left\{c_1^{l*},c_1^{h*},c_2^{l*},c_2^{h*},y^*\right\}}$ be the optimal allocation from part 1. Further, let $b_t$ be the transfer paid by the government in period $t$ to all houses. Since only the productive agents work, we do not need to define type-specific transfers and taxes. Instead, let ${b_t=c_t^{l*}}$ If the tax, $\tau$, can both fund the transfers and satisfy the implementability constraint, then the government will be able to implement the optimal allocation.
	\medskip \\
	The government budget constraint pins down the tax rate:
	\begin{align*}
				\tau y^* &= b_1 = c^{l*}		\\
		\Rightarrow \tau &= \frac{c^{l*}}{y^*}
	\end{align*}
	In the second period, the government uses its ``transfer" function to tax productive households and transfer all revenue to unproductive households:
	\[
		b_2 =	\begin{cases}
					c^{l*},		& y=0					\\
					-c^{l*}, 	& y= y^*				\\
					S, 			& \text{otherwise}		\\
				\end{cases}
	\]
	Where $S$ is the savings of the household (this ensures that the household will not choose a positive output that is not equal to the optimal output). We could check, now, to see whether this satisfies incentive compatability by setting up the household's problem, but it is trivially satisfied by the fact that $c^{1*}$ was chosen to ensure incentive compatability. In the planner's problem, the productive agents' output was taken if they reported as productive. In this case, the productive agent's output is taken (i.e. taxed) if they produce, and they are not given a reasonable intensive margin choice about how much to produce. For all intents and purposes, they are making the same choice of reporting or not reporting as they were in the planner's problem.
	
	\end{enumerate}


%%%________________________________________________________________%%%
\pagebreak
\subsection*{Question 2}

\begin{enumerate}
	\item Let $c_i^j$ and $y_i^j$ denote the consumption and output, respectively, of an agent of type $i$ that sends a signal that they are of type $j$. The planner has to solve the following problem:
		\begin{align*}
			\usmax{}	&pq\left[u(c^H_H) - v\left(\frac{y^H_H}{\theta_H}\right)\right] 
							+ p(1-q)\left[u(c_H^L) - v\left(\frac{y_H^L}{\theta_H}\right)\right] + 												\\
						&(1-p)q\left[u(c^L_L) - v\left(\frac{y^L_L}{\theta_L}\right)\right] 
							+ (1-p)(1-q)\left[u(c^H_L) - v\left(\frac{y^H_L}{\theta_L}\right)\right]											\\
						\text{s.t. }	& (c_H-y_H)[pq + (1-p)(1-q)] + (c_L-y_L)[p(1-q) + q(1-p)] \leq 0										\\
										& u(c^H_H) - v\left(\frac{y^H_H}{\theta_H}\right) \geq u(c^H_L) - v\left(\frac{y^H_L}{\theta_H}\right)	\\
										& u(c^L_H) - v\left(\frac{y^L_H}{\theta_H}\right) \geq u(c^L_L) - v\left(\frac{y^L_L}{\theta_H}\right)	\\
										& u(c^L_L) - v\left(\frac{y^H_L}{\theta_L}\right) \geq u(c^L_H) - v\left(\frac{y^H_H}{\theta_L}\right)	\\
										& u(c^L_L) - v\left(\frac{y^H_L}{\theta_L}\right) \geq u(c^L_H) - v\left(\frac{y^H_H}{\theta_L}\right)	
		\end{align*}
	
	\item A full-information efficient allocation would set ${c^j_H=c^j_L}$ for all $j$ but have the high types work more. Thus, the incentive compatability constraint for low types will not bind, so the allocations for the low types (regardless of which signal they send) will be distorted to ensure that the incentive compatability constraints for the high type are satisfied without any distortions. Thus, the allocations for high types, regardless of signal, will be ex post efficient. 
	
\end{enumerate}


%%%________________________________________________________________%%%




\end{document}






