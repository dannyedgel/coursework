%%% Econ714: Macroeconomics II
%%% Spring 2021
%%% Danny Edgel
%%%
% Due on Canvas Friday, April 16th, 11:59pm Central Time
%%%

%%%
%							PREAMBLE
%%%

\documentclass{article}

%%% declare packages
\usepackage{amsmath}
\usepackage{amssymb}
\usepackage{array}
\usepackage{bm}
\usepackage{changepage}
\usepackage{centernot}
\usepackage{graphicx}
\usepackage{xcolor}
\usepackage[shortlabels]{enumitem}
\usepackage{fancyhdr}
	\fancyhf{} % sets both header and footer to nothing
	\renewcommand{\headrulewidth}{0pt}
    \rfoot{Edgel, \thepage}
    \pagestyle{fancy}
	
%%% define shortcuts for set notation
\newcommand{\Z}{\mathbb{Z}}
\newcommand{\R}{\mathbb{R}}
\newcommand{\Q}{\mathbb{Q}}
\newcommand{\lmt}{\underset{x\rightarrow\infty}{\text{lim }}}
\newcommand{\neglmt}{\underset{x\rightarrow-\infty}{\text{lim }}}
\newcommand{\zerolmt}{\underset{x\rightarrow 0}{\text{lim }}}
\newcommand{\loge}[1]{\text{log}\left(#1\right)}
\newcommand{\usmax}[1]{\underset{#1}{\text{max }}}
\newcommand{\usmin}[1]{\underset{#1}{\text{min }}}
\newcommand{\Mt}{M_{t+1}^t}
\newcommand{\vhat}{\hat{v}}
\newcommand{\olp}{\overline{p}}
\renewcommand{\L}{\mathcal{L}}
\newcommand{\olq}{\overline{q}}
\newcommand{\zinf}{_{t=0}^\infty}
\newcommand{\aneg}{A^{-1}}
\newcommand{\sneg}{s^{-1}}
\newcommand{\olk}{\overline{k}}
\newcommand{\olc}{\overline{c}}
\newcommand{\olr}{\overline{r}}
\newcommand{\olpi}{\overline{\pi}}
\newcommand{\Aneg}{A^{-1}}
\renewcommand{\sneg}{s^{-1}}
\newcommand{\dc}[1]{\Delta c_{#1}}
\newcommand{\N}{\mathcal{N}}
\newcommand{\suminf}{\sum_{t=0}^\infty}
\newcommand{\sumn}{\sum_{i=1}^{n}}
\newcommand{\sumnk}{\sum_{i=1}^{N_k}}
\newcommand{\red}[1]{{\color{red}#1}}
\newcommand{\Tau}{\mathrm{T}}
\newcommand{\phat}{\hat{p}}

\newcommand{\E}[1]{\mathbb{E}\left[#1\right]} % expected value
\newcommand{\Et}[1]{\mathbb{E}_t\left[#1\right]}

%%% define column vector command (from Michael Nattinger)
\newcount\colveccount
\newcommand*\colvec[1]{
        \global\colveccount#1
        \begin{pmatrix}
        \colvecnext
}
\def\colvecnext#1{
        #1
        \global\advance\colveccount-1
        \ifnum\colveccount>0
                \\
                \expandafter\colvecnext
        \else
                \end{pmatrix}
        \fi
}

%%% define function for drawing matrix augmentation lines
\newcommand\aug{\fboxsep=-\fboxrule\!\!\!\fbox{\strut}\!\!\!}

\makeatletter
\let\amsmath@bigm\bigm

\renewcommand{\bigm}[1]{%
  \ifcsname fenced@\string#1\endcsname
    \expandafter\@firstoftwo
  \else
    \expandafter\@secondoftwo
  \fi
  {\expandafter\amsmath@bigm\csname fenced@\string#1\endcsname}%
  {\amsmath@bigm#1}%
}


%________________________________________________________________%

\begin{document}

\title{	Problem Set \#4 }
\author{ 	Danny Edgel 					\\ 
			Econ 714: Macroeconomics II		\\
			Spring 2021						\\
		}
\maketitle\thispagestyle{empty}

%%%________________________________________________________________%%%

\noindent\textit{Discussed and/or compared answers with Sarah Bass, Emily Case, Katherine Kwok, Michael Nattinger, and Alex Von Hafften}

%%%________________________________________________________________%%%

\subsection*{Question 1}

To set up the Ramsey problem, we must first solve for the resource constraint and implementability contraint of this economy. The resource constraint is simply:
\[
	c_t + g_t + k_{t+1} = F(k_t,1-l_t) +(1-\delta)k_t
\]
We can derive the implementability constraint by solving the household problem:
\[
	\usmax{c_t,l_t,k_{t+1},b_{t+1}}\sum_{t=1}^\infty \beta^t\left[\frac{c_t^{1-\sigma}}{1-\sigma}+\nu(l_t)\right]\text{ s.t. } 
		(1+\tau_t)c_t + k_{t+1} + b_{t+1} = w_t(1-l_t) +(1-\delta + r_t)k_t + R_{bt}b_t
\]
Let $p_t$ be the Langrangian multiplier for the household budget contraint. Then, the FOC of the household problem are:
\begin{align*}
	\beta^tc_t^{-\sigma} - p_t(1+\tau_t) 	&= 0	&(c_t)		\\
	\beta^tv'(l_t)-p_tw_t 					&= 0	&(l_t)		\\
	[(1-\delta+r_{t+1})p_{t+1} - p_t]k_t	&= 0	&(k_{t+1})	\\
	(R_{bt+1}p_{t+1}-p_t)b_t				&= 0	&(b_{t+1})	
\end{align*}
Multiplying each size of the budget constraint by the lagrangian multiplier and rearranging, we get:
\begin{align*}
	p_t(1+\tau_t)c_t - p_tw_t(1-l_t) 					&= p_t(1-\delta + r_t)k_t - p_tk_{t+1} + p_tR_{bt}b_t - p_tb_{t+1}	\\
	\beta^t\left(c_t^{1-\sigma}-v'(l_t)(1-l_t)\right) 	&= p_t(1-\delta + r_t)k_t - p_tk_{t+1} + p_tR_{bt}b_t - p_tb_{t+1}
\end{align*}
Then, summing each side across all $t$ yields the implementability constraint:
\[
	\sum_{t=1}^\infty\beta^t\left(c_t^{1-\sigma}-v'(l_t)(1-l_t)\right)  = \frac{c_0^{-\sigma}}{1+\tau_0}\left[(1-\delta + r_0)k_{-1} + R_{b0}b{-1}\right]
\]
Then, the Ramsey problem is to maximize household utility subject to the resource and implementability constraints:
\begin{align*}
	\usmax{c_t,l_t,k_{t+1},b_{t+1}}&\sum_{t=1}^\infty \beta^t\left[\frac{c_t^{1-\sigma}}{1-\sigma}+\nu(l_t)\right]	\\
		\text{ s.t. } 	&c_t + g_t + k_{t+1} = F(k_t,1-l_t) +(1-\delta)k_t	\\
						&\sum_{t=1}^\infty\beta^t\left(c_t^{1-\sigma}-v'(l_t)(1-l_t)\right)  = \frac{c_0^{-\sigma}}{1+\tau_0}\left[(1-\delta + r_0)k_{-1} + R_{b0}b{-1}\right]
\end{align*}
This problem can be written by augmenting the objective function with the implementability constraint, as follows:
{\footnotesize \[
	\usmax{c_t,l_t,k_{t+1},b_{t+1}}\sum_{t=1}^\infty \beta^t\left[\frac{c_t^{1-\sigma}}{1-\sigma}-\nu(l_t) + \lambda\left(c_t^{1-\sigma}-v'(l_t)(1-l_t)\right)\right] 
		- \lambda\frac{c_0^{-\sigma}}{1+\tau_0}\left[(1-\delta + r_0)k_{-1} + R_{b0}b{-1}\right]
\] }
Denote the first two terms of the objective function (excluding the discount factor) as $w(c_t,l_t,\lambda)$. Then, the Ramsey problem is represented by the following Lagrangian function:
{ \begin{align*}
	\L = \sum_{t=0}^\infty &\beta^tw(c_t,l_t,\lambda)- \lambda\frac{c_0^{-\sigma}}{1+\tau_0}\left[(1-\delta + r_0)k_{-1} + R_{b0}b{-1}\right] \\
		&- \mu_t\left(c_t + g_t + k_{t+1} - F(k_t,1-l_t) - (1-\delta)k_t\right)
\end{align*} }
Which has the following first-order conditions:
\begin{align*}
	\beta^tw_1(c_t,l_t,\lambda) - \mu_t					&= 0	&(c_t)		\\
	\beta^tw_2(c_t,l_t,\lambda) - \mu_tF_2(k_t,1-l_t)	&= 0	&(l_t)		\\
	-\mu_{t-1} + \mu_t[F_1(k_t,1-l_t)+1-\delta]			&= 0	&(k_t)	
\end{align*}
Combining the first two FOCs, we have the intratemporal optimization condition:
\[
	\frac{w_2(c_t,l_t,\lambda)}{w_1(c_t,l_t,\lambda)} = F_2(k_t,1-l_t)
\]
Combining the first and third FOCs give the intertemporal optimization condition:
\[
	\frac{w_1(c_t,l_t,\lambda)}{w_1(c_{t+1},l_{t+1},\lambda)} = \beta\left[F_1(k_{t+1},1-l_{t+1})+1-\delta\right]
\]
Where:
\[
	w_1(c_t,l_t,\lambda) = c_t^{-\sigma}\left(1 + \lambda(1-\sigma)\right)
\]
Thus, the intertemporal condition becomes:
\[
	\left(\frac{c_t}{c_{t+1}}\right)^{-\sigma} = \beta\left[F_1(k_{t+1},1-l_{t+1})+1-\delta\right]
\]
In a competitive equilibrium, households have the following intertemporal optimization condition:
\[
	\left(\frac{c_t}{c_{t+1}}\right)^{-\sigma} = \beta\frac{1 + \tau_t}{1+\tau_{t+1}}\left(\frac{p_t}{p_{t+1}}\right) 
		= \beta\frac{1 + \tau_t}{1+\tau_{t+1}}[1-\delta+r_{t+1}]
\]
Where, since the production function satisfies the usual conditions, $${r_{t+1} = F_1(k_{t+1},1-l_{t+1})}$$ In a competitive equilibrium. Thus, the optimal policy is to set ${\tau_t = \tau_{t+1}}$ for all ${t\geq 1}$. 



%%%________________________________________________________________%%%
\pagebreak
\subsection*{Question 2}

\begin{enumerate}
	\item A competitive equilibrium is a policy, $(M_t,B_t)$; allocation, ${(c_{1t},c_{2t},n_t)}$; and price system, ${(p_t,w_t,R_t)}$, such that:
		\begin{enumerate}
			\item Given the policy and price system, the allocation solves the household problem
			\item The allocation satisfies the government budget constraint
		\end{enumerate}
	
	\item The first order conditions of the household problem are:
		\begin{align*}
			&\frac{\beta^t}{c_{1t}} - \lambda_{t+1}p_t - \mu_tp_t = 0	&(c_{1t})	\\
			&\frac{\alpha\beta^t}{c_{2t}} - \lambda_{t+1}p_t = 0		&(c_{2t})	\\
			&-\frac{\gamma\beta^t}{1-n_t} + \lambda_{t+1}w_t = 0		&(n_{t})	\\
			&-\lambda_t + \lambda_{t+1}R = 0							&(B_t)		\\
			&-\lambda_t + \lambda_{t+1} + \mu_t = 0						&(M_t)		
		\end{align*}
		From the FOCs for $M_t$, $B_t$, and $c_{1t}$, we can solve:
		\begin{align*}
			&\frac{\lambda_t}{\lambda_{t+1}}	= 1 + \frac{\mu_t}{\lambda_{t+1}}									\\
			\Rightarrow &R_t = \frac{\lambda_t}{\lambda_{t+1}} = 1 + \frac{\mu_t}{\lambda_{t+1}}					\\
			\Rightarrow &\frac{\beta^t}{\lambda_{t+1}c_{1t}} =  p_t\left(1 + \frac{\mu_t}{\lambda_{t+1}}\right) 	\\
			\Rightarrow &\frac{\beta^t}{c_{1t}} = \lambda_{t+1}p_tR
		\end{align*}
		Combining this with the FOC for $c_{2t}$ gives us: $$ \frac{c_{2t}}{\alpha c_{1t}} = R $$ Combining the FOCs for $c_{2t}$ and $n_t$ yields: $$ \frac{\gamma}{\alpha}\left(\frac{c_{2t}}{1-n_t}\right) = \frac{w_t}{p_t} $$ Since production is linear in labor, ${w_t=p_t}$, so the righthand side of the equation becomes 1. To observe the relationship between $n_t$ and $R$, combine our two optimization conditions with the resource constraint to solve for $n_t$ as a function of $R$:
		\begin{align*}
			c_{1t}			&= \frac{c_{2t}}{\alpha R}													\\
			c_{1t} + c_{2t} &= n_t																		\\
			c_{2t}			&= \frac{\alpha Rn_t}{1 + \alpha R}											\\
			\frac{\gamma}{\alpha}\left[\frac{\frac{n_t}{1 + \frac{1}{\alpha R}}}{1-n_t}\right] &= 1		\\
			\gamma n_t &= \left(\alpha + \frac{1}{R}\right)(1-n_t)										\\
			n_t &= \frac{1 + \alpha R}{1 + (\gamma + \alpha)R}
		\end{align*}
		Taking the derivative yields:
		\[
			\frac{dn_t}{dR} = -\frac{\gamma}{\left[1 + (\gamma + \alpha)R\right]^2} < 0
		\]
		Thus, labor decreases when $R$ increases.
	
\end{enumerate}


%%%________________________________________________________________%%%




\end{document}






