%%% Econ714: Macroeconomics II
%%% Spring 2021
%%% Danny Edgel
%%%
% Due on Canvas Friday, March 26th, 11:59pm Central Time
%%%

%%%
%							PREAMBLE
%%%

\documentclass{article}

%%% declare packages
\usepackage{amsmath}
\usepackage{amssymb}
\usepackage{array}
\usepackage{bm}
\usepackage{changepage}
\usepackage{centernot}
\usepackage{graphicx}
\usepackage{xcolor}
\usepackage[shortlabels]{enumitem}
\usepackage{fancyhdr}
	\fancyhf{} % sets both header and footer to nothing
	\renewcommand{\headrulewidth}{0pt}
    \rfoot{Edgel, \thepage}
    \pagestyle{fancy}
	
%%% define shortcuts for set notation
\newcommand{\Z}{\mathbb{Z}}
\newcommand{\R}{\mathbb{R}}
\newcommand{\Q}{\mathbb{Q}}
\newcommand{\lmt}{\underset{x\rightarrow\infty}{\text{lim }}}
\newcommand{\neglmt}{\underset{x\rightarrow-\infty}{\text{lim }}}
\newcommand{\zerolmt}{\underset{x\rightarrow 0}{\text{lim }}}
\newcommand{\loge}[1]{\text{log}\left(#1\right)}
\newcommand{\usmax}[1]{\underset{#1}{\text{max }}}
\newcommand{\usmin}[1]{\underset{#1}{\text{min }}}
\newcommand{\Mt}{M_{t+1}^t}
\newcommand{\vhat}{\hat{v}}
\newcommand{\olp}{\overline{p}}
\renewcommand{\L}{\mathcal{L}}
\newcommand{\olq}{\overline{q}}
\newcommand{\zinf}{_{t=0}^\infty}
\newcommand{\aneg}{A^{-1}}
\newcommand{\sneg}{s^{-1}}
\newcommand{\olk}{\overline{k}}
\newcommand{\olc}{\overline{c}}
\newcommand{\olr}{\overline{r}}
\newcommand{\olpi}{\overline{\pi}}
\newcommand{\Aneg}{A^{-1}}
\renewcommand{\sneg}{s^{-1}}
\newcommand{\dc}[1]{\Delta c_{#1}}
\newcommand{\N}{\mathcal{N}}
\newcommand{\suminf}{\sum_{t=0}^\infty}
\newcommand{\sumn}{\sum_{i=1}^{n}}
\newcommand{\sumnk}{\sum_{i=1}^{N_k}}
\newcommand{\red}[1]{{\color{red}#1}}
\newcommand{\Tau}{\mathrm{T}}
\newcommand{\phat}{\hat{p}}

\newcommand{\E}[1]{\mathbb{E}\left[#1\right]} % expected value
\newcommand{\Et}[1]{\mathbb{E}_t\left[#1\right]}

%%% define column vector command (from Michael Nattinger)
\newcount\colveccount
\newcommand*\colvec[1]{
        \global\colveccount#1
        \begin{pmatrix}
        \colvecnext
}
\def\colvecnext#1{
        #1
        \global\advance\colveccount-1
        \ifnum\colveccount>0
                \\
                \expandafter\colvecnext
        \else
                \end{pmatrix}
        \fi
}

%%% define function for drawing matrix augmentation lines
\newcommand\aug{\fboxsep=-\fboxrule\!\!\!\fbox{\strut}\!\!\!}

\makeatletter
\let\amsmath@bigm\bigm

\renewcommand{\bigm}[1]{%
  \ifcsname fenced@\string#1\endcsname
    \expandafter\@firstoftwo
  \else
    \expandafter\@secondoftwo
  \fi
  {\expandafter\amsmath@bigm\csname fenced@\string#1\endcsname}%
  {\amsmath@bigm#1}%
}


%________________________________________________________________%

\begin{document}

\title{	Problem Set \#1 }
\author{ 	Danny Edgel 					\\ 
			Econ 714: Macroeconomics II		\\
			Spring 2021						\\
		}
\maketitle\thispagestyle{empty}

%%%________________________________________________________________%%%

\noindent\textit{Discussed and/or compared answers with Sarah Bass, Emily Case, Katherine Kwok, Michael Nattinger, and Alex Von Hafften}

%%%________________________________________________________________%%%

\subsection*{Question 1}
Since both $u$ and $w$ are increasing, strinctly concave, and twice differentiable, the solution to the consumer problem comes from the first-order conditions of the Lagrangian function:
\begin{align*}
	\L &= \theta u(c^1) + (1-\theta) w(c^2) - \lambda( c^1 + c^2 - c )		\\
	\frac{\partial\L}{\partial c^1} &= \theta u'(c^1) 		- \lambda = 0	\\
	\frac{\partial\L}{\partial c^2} &= (1-\theta) w'(c^2) 	- \lambda = 0	\\
	\Rightarrow \theta u'(c^1) &= (1-\theta) w'(c^2)
\end{align*}
Using the envelope condition, we can show the change in total utility given a change in $c$:
\begin{align*}
	v_\theta'(c)	&= \theta u'(c^1)\frac{\partial c^1}{\partial c} + (1-\theta) w'(c^2)\frac{\partial c^2}{\partial c}	\\
			&= \theta u'(c^1)\frac{\partial c^1}{\partial c} + \theta u'(c^1)\frac{\partial c^2}{\partial c}		\\
			&= \theta u'(c^1)\frac{\partial (c^1 + c^2)}{\partial c} = \theta u'(c^1) =  (1-\theta) w'(c^2)
\end{align*}
To show that $v_\theta(c)$ is concave, we must prove that, for any $c<c'$ and $\lambda\in(0,1)$,
\[
	v_\theta(\lambda c + (1-\lambda)c') \geq \lambda v_\theta(c) + (1-\lambda)v_\theta(c')
\] 
Let $f(c)=c^1$ and $g(c)=c^2$ determine the value of $c^1$ and $c^2$ that maximize $v_\theta$ such that ${f(c) + g(c) = c}$. Then,
{\small \begin{align*}
	v_\theta(\lambda c + (1-\lambda)c') 			&= \theta u(f(\lambda c + (1-\lambda)c')) + (1-\theta)w(g(\lambda c + (1-\lambda)c'))	\\
	\lambda v_\theta(c) + (1-\lambda)v_\theta(c')	&= \theta\left[\lambda u(f(c))+ (1-\lambda)u(f(c'))\right] + (1-\theta)\left[\lambda w(g(c))  + (1-\lambda)w(g(c'))\right]		
\end{align*} }
Where, since $u$ and $w$ are concave,
\begin{align*}
	u(f(\lambda c + (1-\lambda)c')) &\geq \lambda u(f(c))+ (1-\lambda)u(f(c')) 	\\
	w(g(\lambda c + (1-\lambda)c')) &\geq \lambda w(g(c))  + (1-\lambda)w(g(c'))	
\end{align*}
Thus, ${v_\theta(\lambda c + (1-\lambda)c')  - \left(\lambda v_\theta(c) + (1-\lambda)v_\theta(c')\right)\geq 0}$, so $v_\theta$ is concave.


%%%________________________________________________________________%%%

\subsection*{Question 2}
\begin{itemize}
	\item[a.] A competitive equilibrium in this economy is a price system, ${\left\{q_t\right\}_{t=0}^\infty}$, and an allocation, ${\left\{c^1_t, c^2_t\right\}_{t=0}^\infty}$, that solves each consumer's problem and clears both the goods and claims markets in every period, $t$:
		\begin{align*} &c_t^1 + c_t^2 = y_t^1 + y_t^2	&q_t^1 + q_t^2 = 0  \end{align*}
	
	\item[b.] Each agent solves their utility maximization problem in period zero, which is represented by the following Lagrangian function and features a single budget constraint:
		\[
			\L = \sum_{t=0}^\infty \beta^tu(c_t^i) - \lambda^i\left(\sum_{t=0}^\infty q_tc_t^i - q_ty^i_t\right)
		\]
		Then optimal consumption in each period, for each agent, has the following first order condition:
		\[
			\frac{\partial\L}{\partial c^i_t} = \beta^tu'(c_t^i) - \lambda^i\left(q_t\right) = 0
		\]
		Thus, we can determine that relative consumption across consumers is constant in every state and time period:
		\[
			\frac{u'(c_t^1)}{u'(c_t^2)} = \frac{\lambda^1}{\lambda^2}
		\]
		Furthermore, since markets are complete with full commitment, we know that consumers with perfectly insure, such that ${c^i_t(s^t)=c^i}$ for all $t$ and $s^t$. Thus, using each consumer's first order condition, we can solve:
		\[
			\frac{\beta^tu'(c_t^i)}{u'(c_0^i)} = \frac{q_t}{q_0}\Rightarrow q_t = \beta^tq_0
		\]
		Then each consumer's budget constraint implies:
		\begin{align*}
			\sum_{t=0}^\infty q_t(s^t)c_t^i 	&= \sum_{t=0}^\infty q_ty^i_t		\\
			c^iq_0\sum_{t=0}^\infty \beta^t 	&= q_0\sum_{t=0}^\infty\beta^ty^i_t	\\
		\end{align*}
		So the allocation is independent of date 0 prices and the right-hand side of this equation depends on each consumer's endowment series:
		\begin{align*}
			\sum_{t=0}^\infty\beta^ty^1_t	&= 1 + \beta^3 + \beta^6 + ... = \sum_{t=0}^\infty(\beta^3)^t = \frac{1}{1-\beta^3}										\\
			\sum_{t=0}^\infty\beta^ty^2_t	&= \beta + \beta^2 + \beta^4 + \beta^5 + ... = \beta\sum_{t=0}^\infty(\beta^3)^t + \beta^2\sum_{t=0}^\infty(\beta^3)^t	
											= \frac{\beta + \beta^2}{1-\beta^3}
		\end{align*}
		Thus, the competitive equilibrium allocation is:
		\begin{align*} &c_t^1 = \frac{1-\beta}{1-\beta^3}	&c_t^2 = \frac{\beta-\beta^3}{1-\beta^3}  \end{align*}
	
	\item[c.] The present discounted value of this asset is $$ p = \sum_{t=0}^\infty 0.05\beta^t = \frac{1}{20(1-\beta)} $$ So this would also be the price of the asset.
	
\end{itemize}


%%%________________________________________________________________%%%

\subsection*{Question 3}
Exercise 8.4 from Ljungqvist and Sargent

\subsubsection*{Part I}
\begin{itemize}
	\item[a.] A competitive equilibrium in this economy is a price system, ${\left\{\{q_t(s^t)\}_{s^t\in S}\right\}_{t=0}^\infty}$, and an allocation, ${\left\{\{c_t(s^t)\}_{s^t\in S}\right\}_{t=0}^\infty}$, that solves the consumer's problem and clears the goods in every period, $t$:
		\begin{align*} &c_t(s^t) = d_t(s^t)  \end{align*}
		If there were a market for claims in period 0, then the claims market would also have to clear. However, since there is only one consumer in this economy, there will be no claims market. The consumer's problem is solved by the following first order condition:
		\[
			\beta^t\pi_t(s^t)u'(c_t(s^t)) = \mu q_t(s^t)
		\]
		Where $\pi_t(s^t)$ is the probability of $s^t$ occuring in time $t$ and $\mu$ is the Lagrangian multiplier for the consumer problem. This condition will be used to price claims to consumption in each period, but the goods market clearing condition ensures that the consumer consumes only their endowment in each period. Let ${q_0(s_0)=1}$ to normalize all prices according to date 0 consumption. Then, we can solve for the equilibrium price of consumption at any period, $t$:
		\[
			q_t(s^t) = \beta^t\frac{\pi_t(s^t)u'(c_t(s^t))}{\pi_0(s_0)u'(c_0(s_0))} = (0.95)^t\frac{\pi_t(s^t)d_t(s^t)^{-2}}{1d_0^{-2}} = (0.95)^t\pi_t(s^t)d_t(s^t)^{-2}
		\]
	
	\item[b.] Using the pricing equation derived in part (a), we can calculate:
		\begin{align*}
						q_5(s^5)	&= (0.95)^5\pi_5(s^5)d_5(s^5)^{-2}						\\
						d_5(s^5)	&= (0.97)(0.97)(1.03)(0.97)(1.03) \approx 0.968			\\
						\pi_5(s^5)	&= (0.8)(0.8)(0.2)(0.1)(0.2) = 0.00256					\\
			\Rightarrow q_5(s^5)	&\approx (0.774)(0.00256)(0.968)^{-2} \approx 0.00211
		\end{align*}
	
	\item[c.] Using the same pricing system, we can calculate:
		\[
			q_5(s^5) = (0.95)^5\left(0.2*0.9*0.9*0.9*0.1\right)\left(1.03*1.03*1.03*1.03*0.97\right)^{-2} \approx 0.00946
		\]
	
	\item[d.] The price of a claim to the entire endowment sequence, $Q$, is simply the sum of the price of a claim to each possible state in each period:
		\[
			Q = \sum_{t=1}^\infty \sum_{s^t\in S} d_t(s^t)q_t(s^t) = \sum_{t=1}^\infty \sum_{s^t\in S} (0.95)^t\pi_t(s^t)d_t(s^t)^{-1}
		\]
	
	\item[e.] The price of a claim to consumption in period 5 if $\lambda$ is 0.97 regardless of the transtition sequence is the sum of the price of all claims in period 5 with sequences ending in 0.97:
		\[
			Q_5^{\lambda = .97} = \sum_{s^5|s_5=0.97} d_5(s^5)q_t(s^5) = \sum_{s^5|s_5=0.97}(0.95)^5\pi_5(s^5)d_5(s^5)^{-1}
		\]
	
\end{itemize}

\subsubsection*{Part II}
\begin{itemize}
	\item[f.] A recursive competitive equilibrium in this economy is a price system, ${\left\{\{q_t(s^t,s_{t+1})\}_{s^t\in S}\right\}_{t=0}^\infty}$, and an allocation of consumption and one-period Arrow securities, ${\left\{\{c_t(s^t),a_t(s^t,s_{t+1})\}_{s^t\in S}\right\}_{t=0}^\infty}$, that solves the consumer's problem and clears the goods in every period, $t$:
		\begin{align*} &c_t(s^t) = d_t(s^t)  \end{align*}
		
		
	\item[g.] The natural debt limit in any period, $\overline{a}_{t+1}$, is the present value of the  total endowment stream, priced according to the equilibrium price system:
		\[
			\overline{a}_{t+1} = d_t(s^t) + \sum_{\tau=t}^\infty\sum_{s_{\tau+1}}\beta^\tau d_\tau(s^\tau)q(s^\tau,s_{\tau+1})
		\]
	
	\item[h.] Since there is still only one consumer, the equilibrium allocation is ${c_t(s^t)=d_t(s^t)}$ for all states and periods. However, the price of an asset can be determined from the conumer's problem, which is represented by the following Lagrangian:
	$$ \L = \E{\sum_{t=0}^\infty\sum_{s^t}\beta^tu(s^t) - \lambda_t\left[c_t(s^t) + \sum_{s_{t+1}}q_t(s^t,s_{t+1})a_{t+1}(s^t,s_{t+1})-d_t(s^t)-a(s^{t+1})\right] }$$
	The first-order condition of the consumer's utility maximization problem is now:
	\[
		q_t(s^t,s_{t+1}) = \beta\left(\frac{u'(c_{t+1}(s^{t+1}))}{u'(c_t(s^t))}\right)\pi(s_{t+1}|s^t)
	\]
	Then, given the model's primitives and recognizing the equilibrium allocation, this condition becomes:
	\[
		q_t(s^t,s_{t+1}) = 0.95\left(\frac{d_t(s^t))}{d_{t+1}(s^{t+1})}\right)^2\pi(s_{t+1}|s^t)
	\]
	Furthermore, the endowment series is governed by ${d_{t+1} = \lambda_{t+1}d_t}$, so:
	\[
		q_t(s^t,s_{t+1}) = 0.95\lambda_{t+1}(s^{t+1})^{-2}\pi(s_{t+1}|s^t)
	\]
	
	\item[i.] The price of this security, $p_t$, is equal to the expected value of a risk-free one-period bond in period $t+1$:
		{\footnotesize \begin{align*}
			p_t(s^t) 	&= \sum_{s_{t+1}}\sum_{s_{t+2}}q_t(s^{t+1},s_{t+2})\pi(s_{t+1}|s^t)	\\
						&= \begin{cases}
								(0.95^2)\left[(0.97)^{-2}(0.8)^2 + (1.03)^{-2}(0.2)(0.8) + 
									(0.97)^{-2}(0.1)(0.2) + (1.03)^{-2}(0.9)(0.2)\right], \lambda_t = 0.97	\\
								(0.95^2)\left[(0.97)^{-2}(0.1)(0.9) + (1.03)^{-2}(0.9)^2 + 
									(0.97)^{-2}(0.8)(0.1) + (1.03)^{-2}(0.2)(0.1)\right], \lambda_t = 1.03
							\end{cases}  
		\end{align*} }																				\\
		\[
			p_t(s^t) 	\approx \begin{cases}
									0.922, \lambda_t = 0.97	\\
									0.869, \lambda_t = 1.03
								\end{cases} 
		\]
	
\end{itemize}


%%%________________________________________________________________%%%




\end{document}






