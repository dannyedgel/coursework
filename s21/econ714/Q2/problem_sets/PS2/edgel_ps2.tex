%%% Econ714: Macroeconomics II
%%% Spring 2021
%%% Danny Edgel
%%%
% Due on Canvas Friday, April 2nd, 11:59pm Central Time
%%%

%%%
%							PREAMBLE
%%%

\documentclass{article}

%%% declare packages
\usepackage{amsmath}
\usepackage{amssymb}
\usepackage{array}
\usepackage{bm}
\usepackage{changepage}
\usepackage{centernot}
\usepackage{graphicx}
\usepackage{xcolor}
\usepackage[shortlabels]{enumitem}
\usepackage{fancyhdr}
	\fancyhf{} % sets both header and footer to nothing
	\renewcommand{\headrulewidth}{0pt}
    \rfoot{Edgel, \thepage}
    \pagestyle{fancy}
	
%%% define shortcuts for set notation
\newcommand{\Z}{\mathbb{Z}}
\newcommand{\R}{\mathbb{R}}
\newcommand{\Q}{\mathbb{Q}}
\newcommand{\lmt}{\underset{x\rightarrow\infty}{\text{lim }}}
\newcommand{\neglmt}{\underset{x\rightarrow-\infty}{\text{lim }}}
\newcommand{\zerolmt}{\underset{x\rightarrow 0}{\text{lim }}}
\newcommand{\loge}[1]{\text{log}\left(#1\right)}
\newcommand{\usmax}[1]{\underset{#1}{\text{max }}}
\newcommand{\usmin}[1]{\underset{#1}{\text{min }}}
\newcommand{\Mt}{M_{t+1}^t}
\newcommand{\vhat}{\hat{v}}
\newcommand{\olp}{\overline{p}}
\renewcommand{\L}{\mathcal{L}}
\newcommand{\olq}{\overline{q}}
\newcommand{\zinf}{_{t=0}^\infty}
\newcommand{\aneg}{A^{-1}}
\newcommand{\sneg}{s^{-1}}
\newcommand{\olk}{\overline{k}}
\newcommand{\olc}{\overline{c}}
\newcommand{\olr}{\overline{r}}
\newcommand{\olpi}{\overline{\pi}}
\newcommand{\Aneg}{A^{-1}}
\renewcommand{\sneg}{s^{-1}}
\newcommand{\dc}[1]{\Delta c_{#1}}
\newcommand{\N}{\mathcal{N}}
\newcommand{\suminf}{\sum_{t=0}^\infty}
\newcommand{\sumn}{\sum_{i=1}^{n}}
\newcommand{\sumnk}{\sum_{i=1}^{N_k}}
\newcommand{\red}[1]{{\color{red}#1}}
\newcommand{\Tau}{\mathrm{T}}
\newcommand{\phat}{\hat{p}}

\newcommand{\E}[1]{\mathbb{E}\left[#1\right]} % expected value
\newcommand{\Et}[1]{\mathbb{E}_t\left[#1\right]}

%%% define column vector command (from Michael Nattinger)
\newcount\colveccount
\newcommand*\colvec[1]{
        \global\colveccount#1
        \begin{pmatrix}
        \colvecnext
}
\def\colvecnext#1{
        #1
        \global\advance\colveccount-1
        \ifnum\colveccount>0
                \\
                \expandafter\colvecnext
        \else
                \end{pmatrix}
        \fi
}

%%% define function for drawing matrix augmentation lines
\newcommand\aug{\fboxsep=-\fboxrule\!\!\!\fbox{\strut}\!\!\!}

\makeatletter
\let\amsmath@bigm\bigm

\renewcommand{\bigm}[1]{%
  \ifcsname fenced@\string#1\endcsname
    \expandafter\@firstoftwo
  \else
    \expandafter\@secondoftwo
  \fi
  {\expandafter\amsmath@bigm\csname fenced@\string#1\endcsname}%
  {\amsmath@bigm#1}%
}


%________________________________________________________________%

\begin{document}

\title{	Problem Set \#2 }
\author{ 	Danny Edgel 					\\ 
			Econ 714: Macroeconomics II		\\
			Spring 2021						\\
		}
\maketitle\thispagestyle{empty}

%%%________________________________________________________________%%%

\noindent\textit{Discussed and/or compared answers with Sarah Bass, Emily Case, Katherine Kwok, Michael Nattinger, and Alex Von Hafften}

%%%________________________________________________________________%%%

\subsection*{Question 1}

\begin{enumerate}
	\item In a decentralized environment, each agent faces the following utility maximization problem:
		\[
			\usmax{\{c_{it},b_{i,t+1}\}}\sum_{t=0}^\infty\beta^tu(c_{it})\text{ s.t. }c_{it}+b_{i,t+1}\leq y_{it} + R_tb_{it}\text{, }b_{i,t+1}\geq\phi_t
		\]
		Where, in our example from class, endowments alternate between a high and low endowment for each type of borrower, $i$. Let ${i=l}$ index the low-endowment borrower and ${i=h}$ index the high-endowment borrower, with the following primitives:
		\begin{align*}
			&u(c) = \loge{c} &\beta = 0.5 &(y_h,y_l) = (15,4)
		\end{align*}
		Then, the Euler equation yields:
		\begin{align*}
			\frac{1}{c_{lt}}	&\geq \beta R\frac{1}{c_{h,t+1}}	\\
			\frac{1}{c_{ht}}	&\geq \beta R\frac{1}{c_{l,t+1}}
		\end{align*}
		Assume that the borrowing constraint binds. Since utility is monotonically increasing in consumption, the budget constraint will also bind. Then, in each period,
		\begin{align*}
			c_{ht} + \phi_{t+1} &= 15 - R_t\phi_t	\\
			c_{lt} - \phi_{t+1} &= 4 + R_t\phi_t	\\
		\end{align*}
		In equilibrium, ${\phi_t=\phi{t+1}}$, ${c_{it}=c_{i,t+1}}$, and ${R_t = R_{t+1}}$ for all $t$. Then,
		\begin{align*}
			c_{h}	&= 15 + \phi(1+R)	\\
			c_{l} 	&= 4 - \phi(1+R)		
		\end{align*}
		The constrained efficient allocation in this problem is ${(c_h,c_l)=(10,9)}$. Then, the above equations give us ${\phi=-\frac{5}{1+R}}$. Combining with the Euler equations from above enable us to obtain the following ranges of $R$ that would decentralize the constrained efficient allocation:
		\[
			R\leq \frac{1}{\beta}\frac{9}{10}
		\]
		So each consumer's Euler equation is satisfied. Since the constrained efficient allocation by definition satisfies the voluntarity participation constraint, we need not show that it is satisfied. Now, we just need to show that markets clear. Let ${R=\frac{1}{\beta}\frac{9}{10}}$:
		\begin{align*}
			c_{h} + c_l	&= 15 + \phi(1+R) + 4 - \phi(1+R) = y_l + y_h	\\
			b_l + b_h	&= -\phi + \phi = 0
		\end{align*}
		
	\item The other equilibrium in this market is ${\phi_t=0}$ for all $t$. This satisfies the voluntary participation constraint:
		\begin{align*}
			\frac{\loge{15-\phi(1+R)}}{1-\beta^2} + \beta\frac{\loge{4+\phi(1+R)}}{1-\beta^2} &= 
				\frac{\loge{15}}{1-\beta^2} + \beta\frac{\loge{4}}{1-\beta^2} = V^d_h				\\
			\frac{\loge{4-\phi(1+R)}}{1-\beta^2} + \beta\frac{\loge{15+\phi(1+R)}}{1-\beta^2} &= 
				\frac{\loge{4}}{1-\beta^2} + \beta\frac{\loge{5}}{1-\beta^2} = V^d_l				
		\end{align*}
		The low endowment type cannot borrow and would not choose to save, so market clearing requires that the high endowment type doesn't save. Then each consumer's budget constraint implies that ${c_{it}=y_{it}}$ for all $i$ and $t$. Then the goods market clears and from the Euler equation, we can obtain $R$:
		\begin{align*}
			\frac{15}{4}	&\geq \beta R	\Rightarrow R \leq \frac{1}{\beta}\frac{15}{4}\\
			\frac{4}{15}	&\geq \beta R	\Rightarrow R \leq \frac{1}{\beta}\frac{4}{15}
		\end{align*}
		So the other equilibrium in this market is ${\phi=0}$ and ${R = \frac{1}{\beta}\frac{4}{15}}$
	
\end{enumerate}


%%%________________________________________________________________%%%
\pagebreak
\subsection*{Question 2}

\begin{enumerate}
	\item 
	
	\item 
	
	\item 
	
	\item 
	
	\item 
	
	\item 
	
\end{enumerate}

%%%________________________________________________________________%%%




\end{document}






