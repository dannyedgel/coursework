%%% Econ711: Microeconomics I
%%% Fall 2020
%%% Danny Edgel
%%%
% Due on Canvas Monday November 2, 11:59pm Central Time
%%%

%%%
%							PREAMBLE
%%%

\documentclass{article}

%%% declare packages
\usepackage{amsmath}
\usepackage{amssymb}
\usepackage{array}
\usepackage{bm}
\usepackage{changepage}
\usepackage{centernot}
\usepackage{graphicx}
\usepackage[shortlabels]{enumitem}
\usepackage{fancyhdr}
	\fancyhf{} % sets both header and footer to nothing
	\renewcommand{\headrulewidth}{0pt}
    \rfoot{Edgel, \thepage}
    \pagestyle{fancy}
	
%%% define shortcuts for set notation
\newcommand{\N}{\mathbb{N}}
\newcommand{\Z}{\mathbb{Z}}
\newcommand{\R}{\mathbb{R}}
\newcommand{\Q}{\mathbb{Q}}
\newcommand{\lmt}{\underset{x\rightarrow\infty}{\text{lim }}}
\newcommand{\neglmt}{\underset{x\rightarrow-\infty}{\text{lim }}}
\newcommand{\zerolmt}{\underset{x\rightarrow 0}{\text{lim }}}
\newcommand{\usmax}[1]{\underset{#1}{\text{max }}}
\newcommand{\usmin}[1]{\underset{#1}{\text{min }}}
\newcommand{\intersect}{\bigcap}
\newcommand{\union}{\bigcup}
\newcommand{\loge}[1]{\text{log}\left(#1\right)}
\renewcommand{\P}{\mathcal{P}}
\renewcommand{\L}{\mathcal{L}}
\newcommand{\olp}{\overline{p}}
\renewcommand{\exp}[1]{\text{exp}\left\{#1\right\}}

%%% define column vector command (from Michael Nattinger)
\newcount\colveccount
\newcommand*\colvec[1]{
        \global\colveccount#1
        \begin{pmatrix}
        \colvecnext
}
\def\colvecnext#1{
        #1
        \global\advance\colveccount-1
        \ifnum\colveccount>0
                \\
                \expandafter\colvecnext
        \else
                \end{pmatrix}
        \fi
}

%%% define function for drawing matrix augmentation lines
\newcommand\aug{\fboxsep=-\fboxrule\!\!\!\fbox{\strut}\!\!\!}

\makeatletter
\let\amsmath@bigm\bigm

\renewcommand{\bigm}[1]{%
  \ifcsname fenced@\string#1\endcsname
    \expandafter\@firstoftwo
  \else
    \expandafter\@secondoftwo
  \fi
  {\expandafter\amsmath@bigm\csname fenced@\string#1\endcsname}%
  {\amsmath@bigm#1}%
}


%________________________________________________________________%

\begin{document}

\title{	Homework \#1 }
\author{ 	Danny Edgel 					\\ 
			Econ 711: Microeconomics I		\\
			Fall 2020						\\
		}
\maketitle\thispagestyle{empty}

%\noindent\textit{Collaborated with Sarah Bass, Emily Case, Michael Nattinger, and Alex Von Hafften}

%%%________________________________________________________________%%%

\subsection*{Question 1}

Suppose $t_i\in S_i$ is strictly dominated by $s_i\in S_i$, but that $\sigma_i\in\Delta S_i$, which is supported by $t_i$ is not strictly dominated. Let $\sigma_i'\in\Delta S_i$ be a mixed strategy that has the same support as $\sigma_i$, but with $s_i$ played played with the same frequency as $t_i$ instead of $t_i$. Since $s_i$ strictly dominates $t_i$, this strategy results in a strictly higher payoff than $\sigma_i$. Therefore, $\sigma_i'$ strictly dominates $\sigma_i$.
\medskip \\
$\therefore$ by contradiction, any mixed strategy that contains a strictly-dominated pure strategy in its support is strictly dominated $\blacksquare$

%%%________________________________________________________________%%%

\subsection*{Question 2}

\begin{enumerate}[(a)]
	\item This scenario is a game with two players ($N=\{1,2\}$) with identical strategy sets ${S_i=\{2,3,...,499,500\}}$, $i=1,2$, and payoff functions:
		\[
			u_i(s_i,s_j) = \begin{cases} s_i + 2, & s_i<s_j \\ s_i, & s_i = s_j \\ s_j - 2, & s_i>s_j \end{cases}\text{ , }i\in\{1,2\}\text{, }j\neq i
		\]
		
	\item Player 1's payoff maximization problem is 
		\[
			\usmax{s_1}u_1(s_1,s_2)
		\]
		Where player 1's payoff matrix is:
		\begin{center}
			\begin{tabular}{rcc}
									& $s_2 < \overline{s}_2$						& $s_2 = \overline{s}_2$			\\ \cline{2-3} 
			$s_1 < \overline{s}_2$	& \multicolumn{1}{|c|}{$[s_2 -2, s_1 + 2]$}		& \multicolumn{1}{|c|}{$s_1 + 2$}	\\ \cline{2-3}
			$s_1 = \overline{s}_2$	& \multicolumn{1}{|c|}{$s_2 - 2$}				& \multicolumn{1}{|c|}{$s_1$}		\\ \cline{2-3}
			$s_1 > \overline{s}_2$	& \multicolumn{1}{|c|}{$s_2 - 2$}				& \multicolumn{1}{|c|}{$s_2 - 2$}	\\ \cline{2-3}
			\end{tabular}
		\end{center}
		Thus, if $s_2=\overline{s}_2$, then player 1's best response is clearly less than $\overline{s}$, and if $s_2<\overline{s}_2$, then choosing $s_1<\overline{s}_2$ will, at worst, make player 1 as poor-off as if they chose $s_1\geq\overline{s}_2$ but will possibly make them better-off. Thus, player 1's best response is $s_1<\overline{s}_2$.
	
	\item Player 2 faces the same best response function that player 1 does. If you begin with the presumption that player 1 believes $\overline{s}_2\in[2,500]$ and iteratively remove strictly dominated strategies for each player, then you arrive at $s_1=s_2=2$ regardless of your initial choice of $\overline{s}_2$.
	
\end{enumerate}


%%%________________________________________________________________%%%

\end{document}




































