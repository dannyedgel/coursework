%%% Econ711: Microeconomics I
%%% Fall 2020
%%% Danny Edgel
%%%
% Due on Canvas Monday October 25, 11:59pm Central Time
%%%

%%%
%							PREAMBLE
%%%

\documentclass{article}

%%% declare packages
\usepackage{amsmath}
\usepackage{amssymb}
\usepackage{array}
\usepackage{bm}
\usepackage{changepage}
\usepackage{centernot}
\usepackage{graphicx}
\usepackage[shortlabels]{enumitem}
\usepackage{fancyhdr}
	\fancyhf{} % sets both header and footer to nothing
	\renewcommand{\headrulewidth}{0pt}
    \rfoot{Edgel, \thepage}
    \pagestyle{fancy}
	
%%% define shortcuts for set notation
\newcommand{\N}{\mathbb{N}}
\newcommand{\Z}{\mathbb{Z}}
\newcommand{\R}{\mathbb{R}}
\newcommand{\Q}{\mathbb{Q}}
\newcommand{\lmt}{\underset{x\rightarrow\infty}{\text{lim }}}
\newcommand{\neglmt}{\underset{x\rightarrow-\infty}{\text{lim }}}
\newcommand{\zerolmt}{\underset{x\rightarrow 0}{\text{lim }}}
\newcommand{\usmax}[1]{\underset{#1}{\text{max }}}
\newcommand{\usmin}[1]{\underset{#1}{\text{min }}}
\newcommand{\intersect}{\bigcap}
\newcommand{\union}{\bigcup}
\newcommand{\loge}[1]{\text{log}\left(#1\right)}
\renewcommand{\P}{\mathcal{P}}
\renewcommand{\L}{\mathcal{L}}
\newcommand{\olp}{\overline{p}}
\renewcommand{\exp}[1]{\text{exp}\left\{#1\right\}}

%%% define column vector command (from Michael Nattinger)
\newcount\colveccount
\newcommand*\colvec[1]{
        \global\colveccount#1
        \begin{pmatrix}
        \colvecnext
}
\def\colvecnext#1{
        #1
        \global\advance\colveccount-1
        \ifnum\colveccount>0
                \\
                \expandafter\colvecnext
        \else
                \end{pmatrix}
        \fi
}

%%% define function for drawing matrix augmentation lines
\newcommand\aug{\fboxsep=-\fboxrule\!\!\!\fbox{\strut}\!\!\!}

\makeatletter
\let\amsmath@bigm\bigm

\renewcommand{\bigm}[1]{%
  \ifcsname fenced@\string#1\endcsname
    \expandafter\@firstoftwo
  \else
    \expandafter\@secondoftwo
  \fi
  {\expandafter\amsmath@bigm\csname fenced@\string#1\endcsname}%
  {\amsmath@bigm#1}%
}


%________________________________________________________________%

\begin{document}

\title{	Problem Set \#7 }
\author{ 	Danny Edgel 					\\ 
			Econ 711: Microeconomics I		\\
			Fall 2020						\\
		}
\maketitle\thispagestyle{empty}

%\noindent\textit{Collaborated with Sarah Bass, Emily Case, Michael Nattinger, and Alex Von Hafften}

%%%________________________________________________________________%%%

\begin{enumerate}[(a)]
	\item If $u$ is linear, then then marginal utility is constant, so the agent maximizes utility by maximizing expected wealth. The agent's optimization problem, then, is
		\[
			\usmax{a} p(w+2a) + (1-p)(w-a) = \usmax{a} a(1-3p) + w
		\]
		Thus, expected wealth is maximized by choosing the maximum value of $a$ if ${p>\frac{1}{3}}$ and by choosing the maximum value of $w$ (i.e. $a=0$) if ${p>\frac{1}{3}}$
		
	\item If the marginal utility of investing is strictly positive at $a=0$, then the optimal level of investment is strictly positive. Then,
		\begin{align*}
			\frac{\partial U(a)}{\partial a} &= 2pu'(w+2a) - (1-p)u'(w-a)	\\
			\frac{\partial U(a)}{\partial a}\bigm|_{a=0} &= 2pu'(w) - (1-p)u'(w) = 3pu'(w)-u'(w) = u'(w)(3p-1)
		\end{align*}
		Where $u'>0$ since $u$ is strictly increasing, and $p>\frac{1}{3}$, so $U'(0)=u'(w)(3p-1)>0$.
		
	\item Continuing from the last problem's calculation, the second derivative of the utility function is
		\[
			\frac{\partial^2 U(a)}{\partial a^2} = 5pu''(w+2a) + (1-p)u''(w-a)
		\]
		Where, by assumption, $u''<0$. Thus, $U''<0$, so $U(a)$ is strictly concave in $a$, and the FOC is necessary and sufficient for finding $a^*$.
		
	\item When all wealth is invested, $u'(w-a)=u'(0)$ and $u'(w+2a)=u'(3w)$. It would be optimal to invest all wealth if $U'(w)>0$:
		\begin{align*}
			\frac{\partial U(a)}{\partial a} = 2pu'(3w) - (1-p)u'(0) &> 0	\\
				2pu'(3w)  &> (1-p)u'(0)	\\
				\frac{u'(3w)}{u'(0)} &> \frac{1-p}{2p}
		\end{align*}
		If $u'(0)\rightarrow\infty$, then the left side of the inequality is zero. Since $p\leq 1$, it is not possible for the right side of the inequality to be negative. Thus, it cannot be optimal to invest all wealth. If $u'(0)$ is finite, then we can solve for $\olp$, the probability level above which the agent will invest all of their wealth:
		\begin{align*}
			\frac{u'(3w)}{u'(0)} &> \frac{1-p}{2p}	\\
			2p\left(\frac{u'(3w)}{u'(0)}\right) &> 1-p \\
			p\left(1+ 2\left(\frac{u'(3w)}{u'(0)}\right)\right) &> 1 \\
			\olp &= \frac{1}{1+ 2\left(\frac{u'(3w)}{u'(0)}\right)}
		\end{align*}
		If $p\geq\olp$, then $U'(w)>0$, so the agent will invest all of their wealth.
		
	\item Given CARA utility, $U$ becomes ${U(a) = p\left(1-e^{-c(w+2a)}\right) + (1-p)\left(1-e^{-c(w-a)}\right)}$, where solving the FOC yields:
		\begin{align*}
			\frac{\partial U(a)}{\partial a} = p\left(e^{-c(w+2a)}\right)(-2c) + (1-p)\left(-e^{-c(w-a)}\right)(c) &= 0	\\
			2pce^{-c(w+2a)} &= (1-p)ce^{-c(w-a)}	\\
			e^{-c(w+2a) + c(w-a)} &= \frac{1-p}{2p}	\\
			c(w-a-w-2a) &= \loge{\frac{1-p}{2p}}	\\
			a^*	&= \frac{-1}{3c}\loge{\frac{1-p}{2p}}
		\end{align*}
		Thus, optimal investment, $a^*$, does not depend on $w$.
		
	\item Assume $A(x)= -\frac{u''(x)}{u'(x)}$ is decreasing in $x$ and recall that 
		\[
			\frac{\partial U(a)}{\partial a} = 2pu'(w+2a) - (1-p)u'(w-a)
		\]
		Now, let $a^*=\text{argmax }U(a)$ be a function of $w$ such that $a^*=a(w)$ (for syntactical simplicity, assume that $a(w)$ always refers to the optimal value of $a$ and that $a$, inside of any utility function or derivative thereof, is a function of $w$). Then, since $U'(a^*)=0$, we can solve:
		\begin{align*}
			\frac{\partial}{\partial w}\left[ 2pu'(w+2a(w)) - (1-p)u'(w-a(w)) \right] &= \frac{\partial}{\partial w}(0)	\\
			2pu''(w+2a)(1+2a'(w)) - (1-p)u''(w-a)(1-a'(w)) &= 0	
		\end{align*}
		\begin{align*}
			2pu''(w+2a)) + a'(w)4pu''(w+2a) - (1-p)u''(w-a) + a'(w)(1-p)u''(w-a)) = 0
		\end{align*}
		\begin{align*}
			a'(w)(4pu''(w+2a)+(1-p)u''(w-a))) &=  (1-p)u''(w-a) - 2pu''(w+2a)) \\
			a'(w) &= \frac{(1-p)u''(w-a) - 2pu''(w+2a)}{4pu''(w+2a)+(1-p)u''(w-a)}
		\end{align*}
		$a'(w)>0$ if the right side of the equality is greater than zero. Since $u''<0$, this is only true if:
		\begin{align*}
			2pu''(w+2a) &> (1-p)u''(w-a)	\\
			\frac{u''(w+2a)}{u''(w-a)} &< \frac{1-p}{2p}
		\end{align*}
		Recall that, at $a^*$, ${\frac{1-p}{2p} = \frac{u'(w+2a)}{u'(w-a)}}$.\footnote{This comes from solving the first order condition: \[ 2pu'(w+2a) - (1-p)u'(w-a)=0 \]} Thus, the condition for $a'(w)>0$ is
		\begin{align*}
			\frac{u''(w+2a)}{u''(w-a)} &< \frac{u'(w+2a)}{u'(w-a)}	\\
			\frac{u''(w+2a)}{u'(w+2a)} &> \frac{u''(w-a)}{u'(w-a)}
		\end{align*}
		Where, by assumption, $-\frac{u''(x)}{u'(x)}$ is decreasing in $x$. Since ${w+2a>w-a}$, this inequality holds. Therefore, ${\frac{\partial a^*}{\partial w}>0}$. Thus, if the agent is wealthier, they will invest more in the start-up regardless of ${p\in(\frac{1}{3},\olp)}$.
		
\end{enumerate}


%%%________________________________________________________________%%%



\end{document}



%{ Typed up for no reson:

(f) The agent's utility function at $a^*$ is given by 
		\begin{align*}
			U(a^*) &= p\left(1-\exp{-c\left(w-\frac{2}{3c}\loge{\frac{1-p}{p}}\right)}\right) + 
				(1-p)\left(1-\exp{-c\left(w+\frac{1}{3c}\loge{\frac{1-p}{p}}\right)}\right) \\
				&=  p\left(1-\exp{-cw}\exp{\frac{2}{3}\loge{\frac{1-p}{p}}}\right) + 
				(1-p)\left(1-\exp{-cw}\exp{-\frac{1}{3}\loge{\frac{1-p}{p}}}\right) \\
				&= p\left(1-e^{-cw}\left(\frac{1-p}{p}\right)^\frac{2}{3}\right) + (1-p)\left(1-e^{-cw}\left(\frac{1-p}{p}\right)^{-\frac{1}{3}}\right)	\\
			U(a^*) &= 1-e^{-cw}\left(p\left(\frac{1-p}{p}\right)^\frac{2}{3}+(1-p)\left(\frac{1-p}{p}\right)^{-\frac{1}{3}}\right)
		\end{align*}
		Taking the derivative of $U(a^*)$ with respect to $w$ shows that it varies positively with wealth:
		\[
			\frac{\partial U(a^*)}{\partial w} = ce^{-cw}\left(p\left(\frac{1-p}{p}\right)^\frac{2}{3}+(1-p)\left(\frac{1-p}{p}\right)^{-\frac{1}{3}}\right)
		\]
		Since $c>0$, $e^{-cw}$, and $p\in(\frac{1}{3},\olp)$, where $\olp<1$. $a^*$ is defined as:
			\[
				a^* = \underset{a}{\text{argmax }}U(a)
			\]
		and, as we determined in earlier conditions, is the value of $a$ when ${\frac{\partial U(a)}{\partial a}=0}$. 

%}







