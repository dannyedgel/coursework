%%% Econ711: Microeconomics I
%%% Fall 2020
%%% Danny Edgel
%%%
% Due on Canvas Monday October 19, 11:59pm Central Time
%%%

%%%
%							PREAMBLE
%%%

\documentclass{article}

%%% declare packages
\usepackage{amsmath}
\usepackage{amssymb}
\usepackage{array}
\usepackage{bm}
\usepackage{changepage}
\usepackage{centernot}
\usepackage{graphicx}
\usepackage[shortlabels]{enumitem}
\usepackage{fancyhdr}
	\fancyhf{} % sets both header and footer to nothing
	\renewcommand{\headrulewidth}{0pt}
    \rfoot{Edgel, \thepage}
    \pagestyle{fancy}
	
%%% define shortcuts for set notation
\newcommand{\N}{\mathbb{N}}
\newcommand{\Z}{\mathbb{Z}}
\newcommand{\R}{\mathbb{R}}
\newcommand{\Q}{\mathbb{Q}}
\newcommand{\lmt}{\underset{x\rightarrow\infty}{\text{lim }}}
\newcommand{\neglmt}{\underset{x\rightarrow-\infty}{\text{lim }}}
\newcommand{\zerolmt}{\underset{x\rightarrow 0}{\text{lim }}}
\newcommand{\usmax}[1]{\underset{#1}{\text{max }}}
\newcommand{\usmin}[1]{\underset{#1}{\text{min }}}
\newcommand{\intersect}{\bigcap}
\newcommand{\union}{\bigcup}
\newcommand{\loge}[1]{\text{log}\left(#1\right)}
\renewcommand{\P}{\mathcal{P}}
\renewcommand{\L}{\mathcal{L}}

%%% define column vector command (from Michael Nattinger)
\newcount\colveccount
\newcommand*\colvec[1]{
        \global\colveccount#1
        \begin{pmatrix}
        \colvecnext
}
\def\colvecnext#1{
        #1
        \global\advance\colveccount-1
        \ifnum\colveccount>0
                \\
                \expandafter\colvecnext
        \else
                \end{pmatrix}
        \fi
}

%%% define function for drawing matrix augmentation lines
\newcommand\aug{\fboxsep=-\fboxrule\!\!\!\fbox{\strut}\!\!\!}

\makeatletter
\let\amsmath@bigm\bigm

\renewcommand{\bigm}[1]{%
  \ifcsname fenced@\string#1\endcsname
    \expandafter\@firstoftwo
  \else
    \expandafter\@secondoftwo
  \fi
  {\expandafter\amsmath@bigm\csname fenced@\string#1\endcsname}%
  {\amsmath@bigm#1}%
}


%________________________________________________________________%

\begin{document}

\title{	Problem Set \#6 }
\author{ 	Danny Edgel 					\\ 
			Econ 711: Microeconomics I		\\
			Fall 2020						\\
		}
\maketitle\thispagestyle{empty}

%\noindent\textit{Collaborated with Sarah Bass, Emily Case, Michael Nattinger, and Alex Von Hafften}

%%%________________________________________________________________%%%

\section*{Question 1: Rationalizing Demand}
Suppose we observe the following ``data":
\begin{center}
	\begin{tabular}{c c c}
		w 		& p 		& x 			\\ \hline 
		100		& (5, 5, 5)	& (12, 4, 4) 	\\
		100		& (7, 4, 5)	& (9, 3, 5) 	\\
		100		& (2, 4, 1)	& (27, 9, 10)	\\
		150		& (7, 4, 5)	& (15, 5, 5)
	\end{tabular} 
\end{center}

\begin{enumerate}[(a)]
	\item Under Walras's law, $p_i\cdot x_i = w$ $\forall i$. Then, we can calulate:
		\begin{align*}
			5*12+5*4+4*5	&= 100 = w	\\
			7*9+4*3+5*5		&= 100 = w	\\
			2*27+4*9+10		&= 100 = w	\\
			7*15+4*5+5*5	&= 150 = w
		\end{align*}
		Thus, the data are consistent with Walras's Law.
		
	\item Given that Walras's Law is satisfied for each observation, $x^i>x^j\Rightarrow p\cdot x^i>p\cdot x^j$ for any $p>>0$, and all price vectors in our data are strictly positive, we can conclude the following:
		\begin{enumerate}[i.]
			\item $x^3>x^i$ $\forall i\neq 3$ implies that: 1) all other goods bundles were affordable at $p^3$, and 2) $x^3$ was unaffordable at all $p^i\neq p^3$. Thus, ${x^3\succ^D x^i\text{ }\forall i\neq 3}$.
			\item $x^4>x^1$ implies that 1) $x^1$ was affordable at $p^4$, and 2) $x^4$ was not affordable at $p^1$. Thus, ${x^4\succ^D x^1}$.
			\item Since $p^4=p^2$ and $w^4>w^2$, we know 1) $x^2$ was affordable when $x^4$ was chosen, and 2) $x^4$ was not affordable when $x^2$ was chosen. Thus, ${x^4\succ^D x^2}$.
			\item $p^1\cdot x^2 = 85<100$, so $p^1\cdot x^2 < p^2\cdot x^2$, and $x^2$ was chosen at $p^2$. Therefore, ${x^2\succ^D x^1}$
			\item $p^2\cdot x^1=120>w^2$, so $x^1$ was not affordable when $x^2$ was chosen. Thus, $\neg(x^2\succsim^D x^1)$
		\end{enumerate}
		Taken together, these preference relations indicate:
		\[
			x^3\succ^D x^4 \succ^D x^1 \succ^D x^2
		\]
		Where it is not possible to have any preference relation ``loops". Therefore, these data satisfy GARP. By Afrias's theorem, satisfying GARP is a sufficient condition for conluding that these data can be rationalized by a continuous, monotonic, and concave utility function.
\end{enumerate}

%%%________________________________________________________________%%%

\section*{Question 2: Aggregating Demand}
Suppose there are $n$ consumers, where consumer $i\in\{1,2,...,n\}$ has the indirect utility function 
\[
	v^i(p,w_i) = a_i(p) + b(p)w_i 
\]
where $\{a_i\}_{i=1}^n$ and $b$ are differentiable functions from $\R_+^k$ to $\R$.
\begin{enumerate}[(a)]
	\item Assuming that $b(p)>0$ and $(p,w_i)>>0$ $\forall i$, then by Roy's identity,
		\begin{align*}
			x^i(p,w_i) 	&= \left(-\frac{\partial v^i(p,w_i)/\partial p_1}{\partial v^i(p,w_i)/\partial w_i},...,
							-\frac{\partial v^i(p,w_i)/\partial p_k}{\partial v^i(p,w_i)/\partial w_i} \right)	\\
						&= \left(-\frac{\frac{\partial a_i(p)}{\partial p_1} + \frac{\partial b(p)}{\partial p_1}w_i}{b(p)},...,
							-\frac{\frac{\partial a_i(p)}{\partial p_k} + \frac{\partial b(p)}{\partial p_k}w_i}{b(p)} \right)
		\end{align*}
		
	\item Using Roy's identity on the representative consumer, we get
		\[
			X(p,W) = \left(-\frac{\left(\sum_{i=1}^n\frac{\partial a_i(p)}{\partial p_1}\right) + \frac{\partial b(p)}{\partial p_1}W}{b(p)},...,
							-\frac{\left(\sum_{i=1}^n\frac{\partial a_i(p)}{\partial p_k}\right) + \frac{\partial b(p)}{\partial p_k}w_i}{b(p)} \right)
		\]
		Where, if $W=\sum_{i=1}^n w_i$, we can solve, for each $j=1,...,k$:
		\begin{align*}
			X_j(p,W)	&= -\frac{\left(\sum_{i=1}^n\frac{\partial a_i(p)}{\partial p_j}\right) + \frac{\partial b(p)}{\partial p_j}W}{b(p)}	\\
						&= -\frac{\sum_{i=1}^n\left(\frac{\partial a_i(p)}{\partial p_j}+ \frac{\partial b(p)}{\partial p_j}w_i\right)}{b(p)}	\\
						&= \sum_{i=1}^n\left(-\frac{\frac{\partial a_i(p)}{\partial p_j}+ \frac{\partial b(p)}{\partial p_j}w_i}{b(p)}\right)	\\
			X_j(p,W)	&= \sum_{i=1}^n x^i_j(p,w_i)
		\end{align*}
		
\end{enumerate}

%%%________________________________________________________________%%%

\section*{Question 3: Homothetic Proferences}
Complete, transitive preferences, $\succsim$, are homothetic if, $\forall x,y\in\R_+^k,t>0$,
\[
	x\succsim y\iff tx\succsim ty
\]
\begin{enumerate}[(a)]
	\item Let $x^*\in x(p,w)$ and define $Y=\{y\in\R^k_+|y\notin x(p,w)\}$.
		\begin{enumerate}[1.]
			\item Suppose, for some $t>0$, that ${tx^*(p,w)\notin x(p,tw)}$
				\begin{enumerate}[a.]
					\item Since preferences are complete, there must exist some ${y^*\in Y}$ such that ${ty^*\in x(p,tw)}$
					\item ${x^*\in x(p,w)\land y^*\notin x(p,w)\Rightarrow x^*\succsim y^*}$. By homothetic preferences, this implies that ${tx^*\succsim ty^*\text{ }\forall t>0}$
					\item $tx^*\succsim ty^*\Rightarrow u(tx)\geq u(ty)$. Since $x^*\in x(p,w)$, then ${p\cdot x^*\leq w}$. This implies also that ${p\cdot (tx^*)\leq tw}$
					\item Since ${ty^*\in x(p,tw)}$, 
						\[
							ty^*=\underset{x}{\text{argmax}}u(x)\text{ s.t. }p\cdot x\leq tw
						\]
						And by c., $u(tx^*)\geq u(ty^*)$, where ${p\cdot (tx^*)\leq tw}$. Thus, $tx^*\in x(p,tw)$ 
						\smallskip \\
						$\therefore$ by contradiction, ${x^*\in x(p,w)\Rightarrow tx^*\in x(p,tw)}$ $\blacksquare$
				\end{enumerate}
			\item Suppose $\exists y^*\in Y$ such that $ty^*\in x(p,tw)$
				\begin{enumerate}[a.]
					\item By definition, $ty\succsim z$ $\forall z\in\R^k_+$ such that ${p\cdot z\leq tw}$
					\item $p\cdot(tx^*)=t(p\cdot x^*)$ where, by definition, $p\cdot x^*\leq w$. Then ${p\cdot(tx^*)\leq tw}$. Thus, ${ty^*\succsim tx^*}$
					\item Since preferences are homothetic, ${ty^*\succsim tx^*\Rightarrow y^*\succsim x^*}$. Thus, $y^*\in x(p,w)$ 
					\smallskip \\
					$\therefore$ by contradiction, ${tx^*\in x(p,tw)\Rightarrow x^*\in x(p,w)}\text{ }\blacksquare$
				\end{enumerate}
			\[
				\therefore\text{for any }t>0\text{, }x(p,tw)=tx(p,w)\text{ }\blacksquare
			\]
		\end{enumerate}
		
\end{enumerate}

%%%________________________________________________________________%%%



\end{document}








