%%% Econ711: Microeconomics I
%%% Fall 2020
%%% Danny Edgel
%%%
% Due on Canvas Sunday November 15, 11:59pm Central Time
%%%

%%%
%							PREAMBLE
%%%

\documentclass{article}

%%% declare packages
\usepackage{amsmath}
\usepackage{amssymb}
\usepackage{array}
\usepackage{bm}
\usepackage{changepage}
\usepackage{centernot}
\usepackage{graphicx}
\usepackage{multirow}
\usepackage[shortlabels]{enumitem}
\usepackage{fancyhdr}
	\fancyhf{} % sets both header and footer to nothing
	\renewcommand{\headrulewidth}{0pt}
    \rfoot{Edgel, \thepage}
    \pagestyle{fancy}
	
%%% define shortcuts for set notation
\newcommand{\N}{\mathbb{N}}
\newcommand{\Z}{\mathbb{Z}}
\newcommand{\R}{\mathbb{R}}
\newcommand{\Q}{\mathbb{Q}}
\newcommand{\lmt}{\underset{x\rightarrow\infty}{\text{lim }}}
\newcommand{\neglmt}{\underset{x\rightarrow-\infty}{\text{lim }}}
\newcommand{\zerolmt}{\underset{x\rightarrow 0}{\text{lim }}}
\newcommand{\usmax}[1]{\underset{#1}{\text{max }}}
\newcommand{\usmin}[1]{\underset{#1}{\text{min }}}
\newcommand{\intersect}{\bigcap}
\newcommand{\union}{\bigcup}
\newcommand{\loge}[1]{\text{log}\left(#1\right)}
\renewcommand{\P}{\mathcal{P}}
\renewcommand{\L}{\mathcal{L}}
\newcommand{\olp}{\overline{p}}
\renewcommand{\exp}[1]{\text{exp}\left\{#1\right\}}

%%% define column vector command (from Michael Nattinger)
\newcount\colveccount
\newcommand*\colvec[1]{
        \global\colveccount#1
        \begin{pmatrix}
        \colvecnext
}
\def\colvecnext#1{
        #1
        \global\advance\colveccount-1
        \ifnum\colveccount>0
                \\
                \expandafter\colvecnext
        \else
                \end{pmatrix}
        \fi
}

%%% define function for drawing matrix augmentation lines
\newcommand\aug{\fboxsep=-\fboxrule\!\!\!\fbox{\strut}\!\!\!}

\makeatletter
\let\amsmath@bigm\bigm

\renewcommand{\bigm}[1]{%
  \ifcsname fenced@\string#1\endcsname
    \expandafter\@firstoftwo
  \else
    \expandafter\@secondoftwo
  \fi
  {\expandafter\amsmath@bigm\csname fenced@\string#1\endcsname}%
  {\amsmath@bigm#1}%
}


%________________________________________________________________%

\begin{document}

\title{	Homework \#3 }
\author{ 	Danny Edgel 					\\ 
			Econ 711: Microeconomics I		\\
			Fall 2020						\\
		}
\maketitle\thispagestyle{empty}

%\noindent\textit{Collaborated with Sarah Bass, Emily Case, Michael Nattinger, and Alex Von Hafften}

%%%________________________________________________________________%%%

\subsection*{Question 1}

\begin{enumerate}[(a)]
	\item $\gamma$ is an additive constant to the marginal payoff of purchasing weapons, whereas $\rho$ subtracts from the marginal payoff. Intuitively, depending on the rest of the marginal payoff function, ${\gamma<\rho}$ would lead to a boundary solution at $w_i=0$. Plugging in the price of weapon's and recognizing that ${\overline{w}=\frac{1}{2}(w_i + w_j)}$, agent $i$'s payoff function can be simplified as follows:
		\begin{align*}
			u_i(w_i,w_j) 	&= \gamma w_i - \beta(w_i-w_j)^2-P(\overline{w})w_i	\\
							&= \gamma w_i - \beta(w_i-w_j)^2-\left[\rho+\frac{\alpha}{2}(w_i + w_j)\right]w_i	\\
							&= (\gamma-\rho)w_i-\beta(w_i-w_j)^2 - \frac{\alpha}{2}(w_i+w_j) w_i	\\
		\end{align*}
		Where all parameters are strictly positive, so ${\beta(w_i-w_j)^2\geq0}$ and ${\frac{\alpha}{2}(w_i+w_j) w_i\geq0}$Thus, $\gamma>\rho$ guarantees an interior solution to the agent's problem.
	
	\item This game is supermodular if each payoff function, $u_i(w_i,w_{-i})$ had increasing differences for all $i$. Since there are only two agents with symmetric and twice-differentiable payoff functions, this is true if
		\[
			\frac{\partial^2 u_i}{\partial w_i \partial w_j}(w_i,w_j) \geq 0
		\]
		So, to determine whether this game is supermodular,
		\begin{align*}
			\frac{partial u_i}{\partial w_i}(w_i,w_j) &= \gamma-\rho-2\beta(w_i-w_j)-\alpha w_i - \frac{\alpha}{2}w_j	\\
			\frac{\partial^2 u_i}{\partial w_i \partial w_j}(w_i,w_j)  &= 2\beta - \frac{\alpha}{2}
		\end{align*}
		Where, by assumption, $\alpha,\beta >0$, so this is a supermodular game if ${2\beta \geq \frac{\alpha}{2}}$
	
	\item 
	
	\item 
	
	\item 
	
	\item 
	
	\item 
	
\end{enumerate}



%%%________________________________________________________________%%%

\subsection*{Question 2}

%%%________________________________________________________________%%%

\subsection*{Question 3}

\begin{enumerate}[(a)]
	\item 
	
	\item 
	
	\item 
	
\end{enumerate}

%%%________________________________________________________________%%%

\subsection*{Question 4}


%%%________________________________________________________________%%%

\subsection*{Question 5}



%%%________________________________________________________________%%%

\end{document}




































