%%% Econ711: Microeconomics I
%%% Fall 2020
%%% Danny Edgel
%%%
% Due on Canvas Monday October 5, 11:59pm Central Time
%%%

%%%
%							PREAMBLE
%%%

\documentclass{article}

%%% declare packages
\usepackage{amsmath}
\usepackage{amssymb}
\usepackage{array}
\usepackage{bm}
\usepackage{changepage}
\usepackage{centernot}
\usepackage{graphicx}
\usepackage{fancyhdr}
	\fancyhf{} % sets both header and footer to nothing
	\renewcommand{\headrulewidth}{0pt}
    \rfoot{Edgel, \thepage}
    \pagestyle{fancy}
	
%%% define shortcuts for set notation
\newcommand{\N}{\mathbb{N}}
\newcommand{\Z}{\mathbb{Z}}
\newcommand{\R}{\mathbb{R}}
\newcommand{\Q}{\mathbb{Q}}
\newcommand{\lmt}{\underset{x\rightarrow\infty}{\text{lim }}}
\newcommand{\neglmt}{\underset{x\rightarrow-\infty}{\text{lim }}}
\newcommand{\zerolmt}{\underset{x\rightarrow 0}{\text{lim }}}
\newcommand{\usmax}[1]{\underset{#1}{\text{max }}}
\newcommand{\usmin}[1]{\underset{#1}{\text{min }}}
\newcommand{\intersect}{\bigcap}
\newcommand{\union}{\bigcup}
\renewcommand{\P}{\mathcal{P}}

%%% define column vector command (from Michael Nattinger)
\newcount\colveccount
\newcommand*\colvec[1]{
        \global\colveccount#1
        \begin{pmatrix}
        \colvecnext
}
\def\colvecnext#1{
        #1
        \global\advance\colveccount-1
        \ifnum\colveccount>0
                \\
                \expandafter\colvecnext
        \else
                \end{pmatrix}
        \fi
}

%%% define function for drawing matrix augmentation lines
\newcommand\aug{\fboxsep=-\fboxrule\!\!\!\fbox{\strut}\!\!\!}

\makeatletter
\let\amsmath@bigm\bigm

\renewcommand{\bigm}[1]{%
  \ifcsname fenced@\string#1\endcsname
    \expandafter\@firstoftwo
  \else
    \expandafter\@secondoftwo
  \fi
  {\expandafter\amsmath@bigm\csname fenced@\string#1\endcsname}%
  {\amsmath@bigm#1}%
}


%________________________________________________________________%

\begin{document}

\title{	Problem Set \#4 }
\author{ 	Danny Edgel 					\\ 
			Econ 711: Microeconomics I		\\
			Fall 2020						\\
		}
\maketitle\thispagestyle{empty}

%\noindent\textit{Collaborated with Sarah Bass, Emily Case, Michael Nattinger, and Alex Von Hafften}

%%%________________________________________________________________%%%

\section*{Question 1}
Let $X$ be a choice set and $\succsim$ be a complete and transitive preference relation on $X$. Let 
\[
	C(A,\succsim) = \{x\in A | x\succsim y \text{ } \forall y\in A\}
\]
Be the choice rule induced by $\succsim$.
\begin{enumerate}
	\item Assume $A\subseteq X$ and $B \subseteq X$, where $x,y\in A\intersect B$, $x\in C(A)$, and $y\in C(B)$
	
	\item $y\in A\intersect B\Rightarrow y\in A$
	
	\item $x\in C(A) \rightarrow x\succsim z \forall z\in A$. Thus, $x\succsim y$ 
	
	\item $x \in A\intersect B\Rightarrow x\in B$ 
	
	\item $y\in C(B)\Rightarrow y\succsim z \forall z\in B$
	
	\item Since $\succsim$ is transitive, $x\succsim y$ and $y\succsim z\forall z \in B$ implies that $x\succsim z\forall z \in B$. Thus, $x\in C(B)$
	
	\item Since $y\in C(B)$ and $x\in B$, $y\succsim x$. $y\in A$ and, since $x\in C(A)$, $x\succsim z\forall z\in A$. Thus, by the transitivity of $\succsim$, $y\succsim z\forall z\in A$. Therefore, $y\in C(A)$
	
\end{enumerate}
$\therefore$ If $A,B\subseteq X$ where $x,y\in A\intersect B$, then $x\in C(A)\land y\in C(B)\Rightarrow x\in C(B)\land y\in C(A)$ $\blacksquare$

%%%________________________________________________________________%%%
\pagebreak
\section*{Question 2}
Let $X$ be a choice set and $C:\P(X)\rightarrow\P(X)$ be a nonempty choice rule that satisfies WARP. Define the preference relation defined on $X$, $\succsim_C$, as
\[
	x\succsim_C y \iff \exists A\subseteq X\text{ s.t. } x,y\in A\land x\in C(A)
\]
\textbf{Completeness.}
\begin{enumerate}
	\item Let $x,z\in A\subseteq X$, where $x\in C(A)$
	
	\item Suppose $\neg(x\succsim_C z)\land\neg(z\succsim_C z)$
	
	\item By the definition of $\succsim_C$, $A\subseteq X\land x,z\in A\land x\in C(A)\rightarrow x\succsim_C z$ 
	
	\item By 2 and 3, $\neg(x\succsim_C z)\land (x\succsim_C z)$
	\medskip \\
	$\therefore$ by contradiction, $\succsim_C$ has complete preferences on $X$
	
\end{enumerate}
\textbf{Transitivity.}
\begin{enumerate}
	\item Suppose $x\succsim_C y$ and $y\succsim_C z$.
	
	\item By the definition of $\succsim_C$, $\exists A\subseteq X\text{ s.t. } x,y\in A\land x\in C(A)$ and ${\exists B\subseteq X\text{ s.t. } x,z\in B\land x\in C(B)}$
	
	\item Clearly, $y\in A\intersect B$. By WARP, if $x\in B$, then $x\in C(B)$. By the definition of $\succsim_C$, since $B\subseteq X\land x,z,\in B\land x\in C(B)$, it must be the case that $x\succsim_C Z$
	
	\item Thus, $x\succsim_C y \land y\succsim_C z \Rightarrow x\succsim_C z$
	\medskip \\
	$\therefore$ $\succsim_C$ has transitive preferences on $X$
	
\end{enumerate}
$\mathbf{C\left(\cdot,\succsim_C\right)=C}$
\begin{enumerate}
	\item Suppose $A\subseteq X$ is nonempty, where $x,y\in A$, $x\in C(A)$, and ${y\in C(A,\succsim_C)}$.
	
	\item By the definition of $\succsim_C$, $x\succsim_C y$ $\forall y\in A$. Thus, $x\in C\left(A,\succsim_C\right)$.  Thus, 
		\[
			x\in C(A)\Rightarrow x\in C\left(A,\succsim_C\right)
		\]
	
	\item $y\in C\left(A,\succsim_C\right)\Rightarrow y\succsim_C x \forall x\in A$.  By the definition of $\succsim_C$, $y\in C(A)$. Thus, 
		\[
			y\in C\left(A,\succsim_C\right)\Rightarrow y\in C(A)
		\]
	
	\item By (2) and (3), $x\in C(A)\iff x\in C\left(A,\succsim_C\right)$
	\medskip \\
	$\therefore$  $C\left(\cdot,\succsim_C\right) = C$
	
\end{enumerate}
$\therefore$ $\succsim_C$ is complete and transitive, and $C\left(\cdot,\succsim_C\right) = C$ $\blacksquare$

%%%________________________________________________________________%%%
\pagebreak 
\section*{Question 3}
Let $X$ be finite and $\succsim$ be a complete and transitive relation on $X$.

\subsection*{(a)}
Suppose $A\neq\emptyset$ and $A\subseteq X$.
\begin{enumerate}
	\item \textit{Base step.} Let $A=\{x\}$. $x\sim x$, so $x\succsim x$. Thus, $C(A,\succsim)=\{x\}\neq\emptyset$
	
	\item \textit{Induction step.} Let $A=\{x_1,...,x_n\}$ and assume $C(A,\succsim)\neq\emptyset$. Then, $\exists x^*\in A$ s.t. $x^*\succsim y$ $\forall y\in A$. Say $\exists x_{n+1}\in X$. Then, since $\succsim$ is complete, either $x_{n+1}\succsim x^*$ or $x^*\succsim x_{n+1}$ (or both).
		\begin{enumerate}
			\item If $x_{n+1}\succsim x^*$, then, since $\succsim$ is transitive and $x^*\succsim y$ $\forall y\in A$, ${x_{n+1}\in C(A\union\{x_{n+1}\},\succsim)}$
			\item If $x^*\succsim x_{n+1}$, then $x^*\succsim y\forall y\in A\union\{x_{n+1}\}$. Thus, ${x^*\in C(A\union\{x_{n+1}\},\succsim)}$
		\end{enumerate}
\end{enumerate}
$\therefore$ by induction, $A\neq\emptyset\Rightarrow C(A,\succsim)\neq\emptyset$ $\blacksquare$

\subsection*{(b)}
\begin{enumerate}
	\item \textit{Base step.} Let $A=\{x\}$. Let $u(x) = 1$. Then, $\forall x,y\in A$, ${x\succsim y\Rightarrow u(x)\geq u(y)}$
	
	\item \textit{Induction step.}  Now let $|A|=n$. Since $\succsim$ is complete and transitive, $A$ can be sorted such that
		\[
			A=\{x_1,...,x_n\}\text{, where } x_n\succsim x_{n-1} \succsim ... \succsim x_2\succsim x_1
		\]
		Define $u(x_i)=i$ for all $i\in\{1,...,n\}$. 
		\medskip \\
		Suppose $x_{n+1}\in X$. Since $\succsim$ is complete, then $x_{n+1}\succsim x_n$ or $x_n\succsim x_{n+1}$, or both.
		\begin{enumerate}
			\item If $x_{n+1}\succsim x_n$, define $u(x_{n+1})=n+1$. Then, $\forall x,y\in A\union\{x_{n+1}\}$, $x\succsim y\Rightarrow u(x)\geq u(y)$.
			\item If ${\neg(x_{n+1}\succsim x_n)}$ and ${x_n\succsim x_{n+1}}$, then set $u(x_n)=n+1$. If ${x_{n+1}\succsim x_{n-1}}$, then set ${u(x_{n+1})=n}$ and leave the utility mappings for $i<n$ unchanged. If ${\neg(x_{n+1}\succsim x_{n-1})}$, then continue this reassigment process until, for some ${i}$, ${x_{n+1}\succsim x_i}$. Then, set ${u(x_{n+1})=i+1}$ and ${u(x_j)=j+1}$ ${\forall j>i}$ and leave the utility mappings unchanged for all ${x_k}$, where ${k<=i}$. If ${x_1\succsim x_{n+1}}$ and ${\nexists i\in\{1,...,n\}}$ such that ${x_{n+1}\succsim x_i}$, then set ${u(x_{n+1}) = 1}$ and set ${u(x_i)=i+1}$ for all $x_i$, where ${i\in\{1,...,n\}}$. Then, ${\forall x,y\in A\union\{x_{n+1}\}}$, ${x\succsim y\Rightarrow u(x)\geq u(y)}$
		\end{enumerate}
	
\end{enumerate}
$\therefore$ When $X$ is finite, $\exists u:X\rightarrow\R$ such that $\forall x,y\in X$, $x\succsim y\Rightarrow u(x)\geq u(y)$ $\blacksquare$


%%%________________________________________________________________%%%



\end{document}








