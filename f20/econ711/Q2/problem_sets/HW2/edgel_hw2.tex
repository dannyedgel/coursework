%%% Econ711: Microeconomics I
%%% Fall 2020
%%% Danny Edgel
%%%
% Due on Canvas Monday November 9, 11:59pm Central Time
%%%

%%%
%							PREAMBLE
%%%

\documentclass{article}

%%% declare packages
\usepackage{amsmath}
\usepackage{amssymb}
\usepackage{array}
\usepackage{bm}
\usepackage{changepage}
\usepackage{centernot}
\usepackage{graphicx}
\usepackage{multirow}
\usepackage[shortlabels]{enumitem}
\usepackage{fancyhdr}
	\fancyhf{} % sets both header and footer to nothing
	\renewcommand{\headrulewidth}{0pt}
    \rfoot{Edgel, \thepage}
    \pagestyle{fancy}
	
%%% define shortcuts for set notation
\newcommand{\N}{\mathbb{N}}
\newcommand{\Z}{\mathbb{Z}}
\newcommand{\R}{\mathbb{R}}
\newcommand{\Q}{\mathbb{Q}}
\newcommand{\lmt}{\underset{x\rightarrow\infty}{\text{lim }}}
\newcommand{\neglmt}{\underset{x\rightarrow-\infty}{\text{lim }}}
\newcommand{\zerolmt}{\underset{x\rightarrow 0}{\text{lim }}}
\newcommand{\usmax}[1]{\underset{#1}{\text{max }}}
\newcommand{\usmin}[1]{\underset{#1}{\text{min }}}
\newcommand{\intersect}{\bigcap}
\newcommand{\union}{\bigcup}
\newcommand{\loge}[1]{\text{log}\left(#1\right)}
\renewcommand{\P}{\mathcal{P}}
\renewcommand{\L}{\mathcal{L}}
\newcommand{\olp}{\overline{p}}
\renewcommand{\exp}[1]{\text{exp}\left\{#1\right\}}

%%% define column vector command (from Michael Nattinger)
\newcount\colveccount
\newcommand*\colvec[1]{
        \global\colveccount#1
        \begin{pmatrix}
        \colvecnext
}
\def\colvecnext#1{
        #1
        \global\advance\colveccount-1
        \ifnum\colveccount>0
                \\
                \expandafter\colvecnext
        \else
                \end{pmatrix}
        \fi
}

%%% define function for drawing matrix augmentation lines
\newcommand\aug{\fboxsep=-\fboxrule\!\!\!\fbox{\strut}\!\!\!}

\makeatletter
\let\amsmath@bigm\bigm

\renewcommand{\bigm}[1]{%
  \ifcsname fenced@\string#1\endcsname
    \expandafter\@firstoftwo
  \else
    \expandafter\@secondoftwo
  \fi
  {\expandafter\amsmath@bigm\csname fenced@\string#1\endcsname}%
  {\amsmath@bigm#1}%
}


%________________________________________________________________%

\begin{document}

\title{	Homework \#2 }
\author{ 	Danny Edgel 					\\ 
			Econ 711: Microeconomics I		\\
			Fall 2020						\\
		}
\maketitle\thispagestyle{empty}

%\noindent\textit{Collaborated with Sarah Bass, Emily Case, Michael Nattinger, and Alex Von Hafften}

%%%________________________________________________________________%%%

\subsection*{Question 1}
For each player, playing ${s_i=\text{min}\{s_i,s_{-i}\}}$ strictly dominates ${s_i>\text{min}\{s_i,s_{-i}\}}$, since private costs increase but private benefits from the public good don't change from contributing above the minimum. ${s_i=\text{min}\{s_i,s_{-i}\}}$ strictly dominates ${s_i<\text{min}\{s_i,s_{-i}\}}$, since the decrease in private costs from contributing below the minimum is outweighed by an even greater decrease in private benefits from the public good. Thus, ${s_1=s_2=...=s_n=k}$ is a pure strategy Nash equilibrium for all $k\in[0,w]$


%%%________________________________________________________________%%%

\subsection*{Question 2}

\begin{enumerate}[(a)]
	\item This strategic situation is a zero-sum game, where all losses have equal payoffs and all wins have equal payoffs (which are greater than the loss payoff). The game's formulation, then, is:
		\begin{align*}
			&N = \{\text{Joe},\text{Donald}\}	\\
			&S_D = \{a_1,a_2,a_3\}\text{, where }\sum_{i=1}^3a_i=1\text{, }a_i\geq0\text{ }\forall i	\\
			&S_J = \{b_1,b_2,b_3\}\text{, where }\sum_{i=1}^3b_i=1\text{, }b_i\geq0\text{ }\forall i	\\
			&u_D(a_i,b_i) = \begin{cases} 1, &a_i > b_i\text{, }a_j > b_j\text{, }i\neq j \\ -1, &\text{otherwise} \end{cases} 	\\
			&u_J(a_i,b_i) = \begin{cases} 1, &b_i > a_i\text{, }b_j > a_j\text{, }i\neq j \\ -1, &\text{otherwise} \end{cases} 
		\end{align*}
	
	\item Suppose that $(s_D,s_J)$ is a pure strategy Nash equilibrium for this game. Let $x_i$ denote the spending of the winning player in area $i$ and $y_i$ denote the spending of the losing player in area $i$. Then, $s_D$ is Donald's best response to $s_J$ and vice versa. Thus, at ${(s_D,s_J)}$, neither player could increase his payoff by choosing another strategy. However, at this equilibrium, one of the players has won the election, so either the winning player won two votes or three. If the winning player won three votes, then he spent ${y_i\in(0\frac{1}{3}]}$ in two areas. Thus, the losing player could increase his payoff by splitting his budget equally between these two areas, winning each vote. If the winning player only won two votes, then in one area, ${y_i > x_i}$ and in the other two areas, ${x_j\in(y_j,1-x_i)}$ such that ${x_j+x_k = 1-x_i}$. Thus, it must be the case that ${x_j<1-y_i}$ or ${x_k<1-y_i}$, so the losing player can increase his payoff by investing all of his money not spent on area $i$ on the area where the winning player spent less than ${1-y_i}$. Therefore, $(s_D,s_J)$ is not a Nash equilibrium. By contradiction, this game has no pure strategy Nash equilibria.
\end{enumerate}

%%%________________________________________________________________%%%

\subsection*{Question 3}

\begin{enumerate}[(a)]
	\item Letting $j$ denote either Alice (A) or Bob (B), this normal form game has formulation ${N=\{Alice,Bob\}}$, ${S_j=\{T^j_1,T^j_2,T^j_3\}}$, and 
		\[
			u_j(T_i^A,T_i^B) = \begin{cases} i, &T_i^A = T_i^B	\\ 0, &T^A_i\neq T^B_k\end{cases}\text{, where }i,k\in\{1,2,3\}
		\]
	
	\item The payoff matrix for this game is:
		\begin{center}
			\begin{tabular}{crccc}
			&					& \multicolumn{3}{c}{Bob}													\\
			&					& $T^B_1$						& $T^B_2$						& $T^B_3$				\\ \cline{3-5} 
			\multirow{2}{*}{Alice}						
			& $T^A_1$			& \multicolumn{1}{|c|}{$(1,1)$}	& \multicolumn{1}{|c|}{$(0,0)$}	& \multicolumn{1}{|c|}{$(0,0)$}	\\ \cline{3-5}
			& $T^A_2$			& \multicolumn{1}{|c|}{$(0,0)$}	& \multicolumn{1}{|c|}{$(2,2)$}	& \multicolumn{1}{|c|}{$(0,0)$}	\\ \cline{3-5}
			& $T^A_3$			& \multicolumn{1}{|c|}{$(0,0)$}	& \multicolumn{1}{|c|}{$(0,0)$}	& \multicolumn{1}{|c|}{$(3,3)$}	\\ \cline{3-5}
			\end{tabular}
		\end{center}
		Clearly, all strategies are rationalizable and the game has pure strategy Nash Equilibria at ${(T^A_1,T^B_1)}$, ${(T^A_2,T^B_2)}$, and ${(T^A_3,T^B_3)}$. Any mixed strategy Nash equilibrium requires each player to be indifferent to each pure strategy. Since payoffs and strategy sets are symmetric, we need only calculate the probabilities with which one player plays each move that makes the other player indifferent to each move. Let $a$, $b$, and $c$ represent Alice's beliefs about the probability with with Bob plays  $T^B_1$, $T^B_2$, and $T^B_3$, respectively. Then, for  ${T^A_1\sim T^A_2\sim T^A_3}$:
		\begin{align*}
			&a=2b\text{ and }2b=3c	\\
			&2b + b + \frac{2}{3}b = 1	\\
			&\frac{11}{3}b = 1			\\
			&a=\frac{6}{11}\text{, }b = \frac{3}{11}\text{, }c = \frac{2}{11}
		\end{align*}
		Thus, ${\left(\frac{6}{11}T^A_1 + \frac{3}{11}T^A_2 + \frac{2}{11}T^A_3,\frac{6}{11}T^B_1 + \frac{3}{11}T^B_2 + \frac{2}{11}T^B_3\right)}$ is the only mixed-strategy Nash equilibrium of this game.
\end{enumerate}

%%%________________________________________________________________%%%
\pagebreak
\subsection*{Question 4}

\begin{enumerate}[(a)]
	\item If each firm maximizes profit and this is common knowledge, then the problem for firm $i$ is:
		\[
			\usmax{q_i}\left\{q_i\left(2-q_i-q_j\right) - q_i\right\}\text{, where }q_j = \text{arg}\usmax{q_j}\left\{q_j\left(2-q_i-q_j\right) - q_j\right\}
		\]
		The first-order condition of this problem yields the equilibrium quantity for each firm:
		\begin{align*}
			2-2q_i - q_j -1 &= 0	\\
			2q_i &= 1-q_j 	\\
			q_i &= \frac{1-q_j}{2}\text{, where } q_j = \frac{1-q_i}{2}	\\
			q_i &= \frac{1-\frac{1-q_i}{2}}{2} \\
			2q_i &= 1-\frac{1}{2}-\frac{q_i}{2}	\\
			\frac{3}{2}q_i &= \frac{1}{2}	\\
			q_i = q_j &= \frac{1}{3}
		\end{align*}
		Thus, in equilibrium, ${\left\{q_1,q_2,Q,P,\pi_1,\pi_2\}=\{\frac{1}{3},\frac{1}{3},\frac{2}{3},\frac{4}{3},\frac{1}{9},\frac{1}{9}\right\}}$.
	
	\item Now, firm 2's best response function remains unchanged, but firm 1's new problem is:
		\[
			\usmax{q_1}\left\{\frac{3}{4}q_1\left(2-q_1-q_2\right) - \frac{3}{4}q_1 + \frac{1}{4}q_1\left(2-q_1-q_2\right)\right\} =
			\usmax{q_1}\left\{q_1\left(2-q_1-q_2\right) - \frac{3}{4}q_1\right\}
		\]
		This yields a new best response function for firm 1 of ${q_1 = \frac{5}{8}-\frac{1}{2}q_2}$. Thus, the new equilibrium allocation is:
		\[
			\{q_1,q_2,Q,P,\pi_1,\pi_2\}=\left\{\frac{3}{10},\frac{7}{20},\frac{13}{20},\frac{27}{20},\frac{21}{200},\frac{49}{400}\right\}
		\]
	
	\item As is shown in the equilibrium allocations, firm 1's profit decreases in the second scenario, while firm 2's profit increases. This suggests that a firm in this duopoly can increase its profits if it can convince the other firm that it does not seek to maximize its profit.
\end{enumerate}

%%%________________________________________________________________%%%
\pagebreak
\subsection*{Question 5}

\begin{enumerate}[(a)]
	\item $u(t)$ is maximized at $t=\frac{1}{2}$\footnote{This can be trivially determined using the first-order condition of $u(t)$.}, and, since $v(q)$ is strictly increasing in quantile $q$, there is no incentive to enter the Zoom seminar prior to ${t=\frac{1}{2}}$. Thus, the quantile function, $Q(t)$, can be solved by setting the payoff, $U(t,Q(t))$, equal to the payoff of entering at precisely ${t=\frac{1}{2}}$ to solve at which quantiles, as a function of $t$, each participant is indifferent between entering at or after the optimal time, before anybody else:
		\begin{align*}
			u(t)v(Q(t)) &= u\left(\frac{1}{2}\right)v(0)	\\
			t(1-t)\left(1 + Q(t) + \frac{1}{4}Q(t)^2\right) &= \frac{1}{4} \\
			\frac{1}{4}Q(t)^2 + Q(t)   + 1 &= \frac{1}{4t(1-t)} \\
			Q(t)^2 + 4Q(t) + 4 &= \frac{1}{t(1-t)} \\
			\left(Q(t) + 2\right)^2 &= \frac{1}{t(1-t)} \\
			Q(t) &= \frac{1}{\sqrt{t(1-t)}}	- 2
		\end{align*}
		We already determined that the lower bound of $t$ is $\frac{1}{2}$. Since $Q(t)$ is a CDF, we can solve for the upper bound of $t$ by setting ${Q(t)=1}$ and solving:
		\begin{align*}
			1 &= \frac{1}{\sqrt{t(1-t)}}- 2	\\
			3\sqrt{t(1-t)} &= 1					\\
			t(1-t) &= \frac{1}{9}	\\
			t^2 - t + \frac{1}{9} &= 0	\\
			t &= \frac{3 \pm \sqrt{5}}{6}
		\end{align*}
		$\frac{3 - \sqrt{5}}{6}<\frac{1}{2}$, so the upper bound of $t$ is $\frac{3 + \sqrt{5}}{6}$.
		\smallskip \\
		To determine whether it is possible to have a terminal rush, we must find the quantile $\tilde{q}$ such that the payoff of going just before the rush is equal to the average payoff of going during the rush:
		\begin{align*}
			\frac{1}{1-\tilde{q}}\int_{\tilde{q}} 1v(x)dx &= v(\tilde{q})	\\
			\frac{1}{1-\tilde{q}}\int_{\tilde{q}} ^1 1 + x + \frac{1}{4}x^2dx &= 1 + \tilde{q} + \frac{1}{4}\tilde{q} \\
			\left[x + \frac{1}{2}x^2 + \frac{1}{12}x^3 \right]^1_{\tilde{q}} &= (1-\tilde{q})(1 + \tilde{q} + \frac{1}{4}\tilde{q}) \\
			1+\frac{1}{2} + \frac{1}{12} - \tilde{q}-\frac{1}{2}\tilde{q}^2-\frac{1}{12}\tilde{q}^3 &= 1-\frac{3}{4}\tilde{q}^2-\frac{1}{4}\tilde{q}^3 \\
			7-12\tilde{q}+3\tilde{q}^2+2\tilde{q}^2 &= 0	\\
			\tilde{q} &= -\frac{7}{2}\text{, }1
		\end{align*}
		$-\frac{7}{2}$ is outside of the range of $Q$ and can be ignored. $\tilde{q}=1$ implies that the terminal ``rush" is equal to zero. Thus, there is no terminal rush in this game. 
		
	\item In the last problem, we solved for the upper bound of $Q(t)$'s domain. The support of $Q$ is ${t\in\left[\frac{1}{2},\frac{3 + \sqrt{5}}{6}\right]}$.
\end{enumerate}


%%%________________________________________________________________%%%

\end{document}




































