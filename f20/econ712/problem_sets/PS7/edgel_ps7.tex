%%% Econ712: Macroeconomics I
%%% Fall 2020
%%% Danny Edgel
%%%
% Due on Canvas Thursday October 22, 11:59pm Central Time
%%%

%%%
%							PREAMBLE
%%%

\documentclass{article}

%%% declare packages
\usepackage{amsmath}
\usepackage{amssymb}
\usepackage{array}
\usepackage{bm}
\usepackage{changepage}
\usepackage{centernot}
\usepackage{graphicx}
\usepackage[shortlabels]{enumitem}
\usepackage{fancyhdr}
	\fancyhf{} % sets both header and footer to nothing
	\renewcommand{\headrulewidth}{0pt}
    \rfoot{Edgel, \thepage}
    \pagestyle{fancy}
	
%%% define shortcuts for set notation
\newcommand{\N}{\mathbb{N}}
\newcommand{\Z}{\mathbb{Z}}
\newcommand{\R}{\mathbb{R}}
\newcommand{\Q}{\mathbb{Q}}
\newcommand{\lmt}{\underset{x\rightarrow\infty}{\text{lim }}}
\newcommand{\neglmt}{\underset{x\rightarrow-\infty}{\text{lim }}}
\newcommand{\zerolmt}{\underset{x\rightarrow 0}{\text{lim }}}
\newcommand{\loge}[1]{\text{ln}\left(#1\right)}
\newcommand{\usmax}[1]{\underset{#1}{\text{max }}}
\newcommand{\Mt}{M_{t+1}^t}
\newcommand{\olp}{\overline{p}}
\renewcommand{\L}{\mathcal{L}}
\newcommand{\olq}{\overline{q}}

%%% define column vector command (from Michael Nattinger)
\newcount\colveccount
\newcommand*\colvec[1]{
        \global\colveccount#1
        \begin{pmatrix}
        \colvecnext
}
\def\colvecnext#1{
        #1
        \global\advance\colveccount-1
        \ifnum\colveccount>0
                \\
                \expandafter\colvecnext
        \else
                \end{pmatrix}
        \fi
}

%%% define function for drawing matrix augmentation lines
\newcommand\aug{\fboxsep=-\fboxrule\!\!\!\fbox{\strut}\!\!\!}

\makeatletter
\let\amsmath@bigm\bigm

\renewcommand{\bigm}[1]{%
  \ifcsname fenced@\string#1\endcsname
    \expandafter\@firstoftwo
  \else
    \expandafter\@secondoftwo
  \fi
  {\expandafter\amsmath@bigm\csname fenced@\string#1\endcsname}%
  {\amsmath@bigm#1}%
}


%________________________________________________________________%

\begin{document}

\title{	Problem Set \#7 }
\author{ 	Danny Edgel 					\\ 
			Econ 712: Macroeconomics I		\\
			Fall 2020						\\
		}
\maketitle\thispagestyle{empty}

%%%________________________________________________________________%%%

\noindent\textit{Collaborated with Sarah Bass, Emily Case, Michael Nattinger, and Alex Von Hafften}
\medskip \\

%%%________________________________________________________________%%%

\noindent Consider a two-period overlapping generations model where agents earn $y$ when young and $0$ when old. Housing supply is fixed at $H^s=1$ and preferences are given by 
\[
	U(c_t^t,h_t,c_{t+1}^t) = \loge{c_t^t} + \alpha h_t + \beta c_{t+1}^t 
\]
Assume that the initial old hold the housing stock and $1+\alpha > \beta y$.
\begin{enumerate}
	\item The social planner's problem (SPP) is:
		\[
			\usmax{\{c_t^t,h_t,c^{-t}_t\}^\infty_{t=1}}\loge{c_t^t} + \alpha h_t + \beta c^{t-1}_t \text{ s.t. } h_t=1\text{, }c_t^t+c_t^{t-1}=y
		\]
		Note that, as the choice variables begin at $t=1$, this problem includes the initial old, and the social planner can reallocate the initial old's housing at will. The using the budget constraint to solve for the consumption of the old generation as a function of the consumption of the young generation and setting $h_t=1$ in each period, the SPP can be re-written as:
		\[
			\usmax{\{c_t^t\}^\infty_{t=1}}\loge{c_t^t} + \alpha + \beta (y-c_t^t)
		\]
		Where the FOC for $c_t^t$ can be used to solve for the social planner's allocation in each period:
		\begin{align*}
			\frac{1}{c_t^t}-\beta &= 0	\\
			c_t^t &= \frac{1}{\beta}	\\
			c_t^{t-1} &= y-\frac{1}{\beta}	\\
			h_t &= 1
		\end{align*}
		
	\pagebreak	
	\item Let $p_t$ be the price of a house in period $t$.
		\begin{enumerate}[(a)]
			\item The young agent's problem is 
				\[
					\usmax{\{c_t^t,h_t,c_{t+1}^t\}^\infty_{t=1}}\loge{c_t^t} + \alpha h_t + \beta c_{t+1}^t \text{ s.t. } c_t^t + p_th_t = y\text{, }c_{t+1}^t = p_{t+1}h_t
				\]
				
			\item The market clearing conditions are:
				\begin{align*}
					c_t^t + c_t^{t-1} 	&= y &\text{(Goods market)} 	\\
								h_t 	&= 1 &\text{(Housing market)} 
				\end{align*}
			
			\item A competitive general equilibrium is an allocation, $\{c_t^t,h_t,c_t^{t-1}\}_{t=1}^\infty$ and set of prices, $\{p_t\}_{t=1}^\infty$ that solve every agent's problem in each period and allow markets to clear.
			
			\item Using each period's budget constraint, the young agent's problem can be rewritten as a choice of only housing:
				\[
					\usmax{\{h_t\}^\infty_{t=1}}\loge{y-p_th_t} + \alpha h_t + \beta p_{t+1}h_t
				\]
				Using the FOC for housing, we can solve for the agent's optimal housing and consumption rules:
				\begin{align*}
					\frac{p_t}{y-p_th_t} + \alpha + \beta p_{t+1} &= 0				\\
					(y-p_th_t)(\alpha + \beta p_{t+1}) &= p_t 						\\
					p_th_t &= y-\frac{p_t}{\alpha + \beta p_{t+1}}					\\
					h_t		&= \frac{y}{p_t} - \frac{1}{\alpha + \beta p_{t+1}} 	\\
					c_t^t &= y-p_t\left(\frac{y}{p_t} - \frac{1}{\alpha + \beta p_{t+1}}\right) \\
						&= y - y + \frac{p_t}{\alpha + \beta p_{t+1}} = \frac{p_t}{\alpha + \beta p_{t+1}} \\
					c_{t+1}^t &= p_{t+1}\left(\frac{y}{p_t} - \frac{1}{\alpha + \beta p_{t+1}}\right) \\
						&= \frac{p_{t+1}}{p_t} y - \frac{p_{t+1}}{\alpha + \beta p_{t+1}}	\\
						&=  \frac{p_{t+1}}{p_t} \left( y - \frac{p_t}{\alpha + \beta p_{t+1}} \right)
				\end{align*}
				Thus, the optimal rules for each variable and their associated non-negativity conditions are (assuming prices are weakly positive):\footnote{As housing cannot go negative and positively relates to utility, this is a reasonable assumption.}
				\begin{align*}
					h_t		&= \frac{y}{p_t} - \frac{1}{\alpha + \beta p_{t+1}}, &y>\frac{p_t}{\alpha + \beta p_{t+1}} 	\\
					c_t^t &= \frac{p_t}{\alpha + \beta p_{t+1}}, &\alpha>-\beta p_{t+1} \\
					c_{t+1}^t &=  \frac{p_{t+1}}{p_t} \left( y - \frac{p_t}{\alpha + \beta p_{t+1}} \right), &y>\frac{p_t}{\alpha + \beta p_{t+1}}
				\end{align*}
				
			\item In equilibrium, $h_t=1$, so we can solve:
				\begin{align*}
					h_t		&= 1	\\
					\frac{y}{p_t} - \frac{1}{\alpha + \beta p_{t+1}} &= 1 	\\
					\frac{1}{\alpha + \beta p_{t+1}} &= \frac{y}{p_t} - 1  	\\
					\alpha + \beta p_{t+1} &= \frac{1}{\frac{y}{p_t} - 1}	\\
					p_{t+1} &= \frac{p_t}{\beta(y-p_t)} - \frac{\alpha}{\beta}
				\end{align*}
				\begin{center}
					\includegraphics{priceplot.png}
				\end{center}
				
			\item In the steady state, $p_t = p_{t+1}=\overline{p}$ $\forall t$. So, using the law of motion from 2(e):
				\begin{align*}
					\olp &= \frac{\olp}{\beta(y-\olp)} - \frac{\alpha}{\beta}	\\
					\beta\olp + \alpha &= \frac{\olp}{y-\olp}	\\
					(y-\olp)(\beta\olp) + \alpha &= \olp	\\
					y\beta\olp + \alpha y - \beta\olp^2-\alpha\olp - \olp &= 0	\\
					\beta\olp^2 + (\alpha-y\beta + 1)\olp - \alpha y &= 0		\\
					\olp &= \frac{-\alpha+y\beta - 1 \pm\sqrt{(\alpha-y\beta + 1)^2-4\beta\alpha y}}{2\beta}
				\end{align*}
				Since $p$ cannot be negative, the unique, steady-state value of $p$ is
				\[
					\olp = \frac{-\alpha+y\beta - 1 +\sqrt{(\alpha-y\beta + 1)^2-4\beta\alpha y}}{2\beta}
				\]
				
			\item Given the steady-state price of housing and optimal choice of consumption and housing, the competitive allocation is
				\begin{align*}
					h_t		&= \frac{y}{\frac{-\alpha+y\beta - 1 +\sqrt{(\alpha-y\beta + 1)^2-4\beta\alpha y}}{2\beta}} - \frac{1}{\alpha + \beta p_{t+1}}	\\
					c_t^t &= \left(\frac{-\alpha+y\beta - 1 +\sqrt{(\alpha-y\beta + 1)^2-4\beta\alpha y}}{2\beta}\right)	\\
									&\left(\alpha + \left(\frac{-\alpha+y\beta - 1 +\sqrt{(\alpha-y\beta + 1)^2-4\beta\alpha y}}{2\beta}\right)\right)^{-1} \\
					c_{t+1}^t &=  y - \left(\frac{-\alpha+y\beta - 1 +\sqrt{(\alpha-y\beta + 1)^2-4\beta\alpha y}}{2\beta}\right)	\\
									&\left(\alpha + \left(\frac{-\alpha+y\beta - 1 +\sqrt{(\alpha-y\beta + 1)^2-4\beta\alpha y}}{2\beta}\right)\right)^{-1}
				\end{align*}
				Which clearly does not simplify to the social planner's allocation.
		\end{enumerate}

\end{enumerate}


%%%________________________________________________________________%%%


\end{document}












