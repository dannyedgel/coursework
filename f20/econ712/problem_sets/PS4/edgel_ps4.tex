%%% Econ712: Macroeconomics I
%%% Fall 2020
%%% Danny Edgel
%%%
% Due on Canvas Thursday October 1, 11:59pm Central Time
%%%

%%%
%							PREAMBLE
%%%

\documentclass{article}

%%% declare packages
\usepackage{amsmath}
\usepackage{amssymb}
\usepackage{array}
\usepackage{bm}
\usepackage{changepage}
\usepackage{centernot}
\usepackage{graphicx}
\usepackage{fancyhdr}
	\fancyhf{} % sets both header and footer to nothing
	\renewcommand{\headrulewidth}{0pt}
    \rfoot{Edgel, \thepage}
    \pagestyle{fancy}
	
%%% define shortcuts for set notation
\newcommand{\N}{\mathbb{N}}
\newcommand{\Z}{\mathbb{Z}}
\newcommand{\R}{\mathbb{R}}
\newcommand{\Q}{\mathbb{Q}}
\newcommand{\lmt}{\underset{x\rightarrow\infty}{\text{lim }}}
\newcommand{\neglmt}{\underset{x\rightarrow-\infty}{\text{lim }}}
\newcommand{\zerolmt}{\underset{x\rightarrow 0}{\text{lim }}}
\newcommand{\loge}[1]{\text{ln}\left(#1\right)}
\newcommand{\usmax}[1]{\underset{\{#1\}}{\text{max }}}
\newcommand{\Mt}{M_{t+1}^t}
\renewcommand{\L}{\mathcal{L}}
\newcommand{\olq}{\overline{q}}

%%% define column vector command (from Michael Nattinger)
\newcount\colveccount
\newcommand*\colvec[1]{
        \global\colveccount#1
        \begin{pmatrix}
        \colvecnext
}
\def\colvecnext#1{
        #1
        \global\advance\colveccount-1
        \ifnum\colveccount>0
                \\
                \expandafter\colvecnext
        \else
                \end{pmatrix}
        \fi
}

%%% define function for drawing matrix augmentation lines
\newcommand\aug{\fboxsep=-\fboxrule\!\!\!\fbox{\strut}\!\!\!}

\makeatletter
\let\amsmath@bigm\bigm

\renewcommand{\bigm}[1]{%
  \ifcsname fenced@\string#1\endcsname
    \expandafter\@firstoftwo
  \else
    \expandafter\@secondoftwo
  \fi
  {\expandafter\amsmath@bigm\csname fenced@\string#1\endcsname}%
  {\amsmath@bigm#1}%
}


%________________________________________________________________%

\begin{document}

\title{	Problem Set \#4 }
\author{ 	Danny Edgel 					\\ 
			Econ 712: Macroeconomics I		\\
			Fall 2020						\\
		}
\maketitle\thispagestyle{empty}

%%%________________________________________________________________%%%

\noindent\textit{Collaborated with Sarah Bass, Emily Case, Michael Nattinger, and Alex Von Hafften}
\bigskip \\
\noindent We have an overlapping generations problem with an infinite number of discrete periods and households that live for two periods. Each generation has a measure of households of population $(1+n)^t$, assuming that the inital old generation has a unit measure population. The inital old are endowed with $\overline{M_1}$ units of fiat currency and $w_2$ units of consumption goods. The money supply increases at a rate of $z\geq 0$, with new fiat money distributed to each period's old generation in proportion to their money holdings such that $M_{t+1}^t$ chosen when young becomes $(1+z)M_{t+1}^t$ when old. \\

Consumption goods are non-storable and each young generation is endowed with $w_1$, where $w_1>w_2$. The utility of households in generation $t\geq 1$ is represented by
\[
	U(c_t^t,c_{t+1}^t) = \loge{c_t^t} + \loge{c_{t+1}^t}
\]
Where $U(c_1^0)=\loge{c_1^0}$ represents the utility of the initial old generation.

%%%________________________________________________________________%%%

\begin{enumerate}

	\item An allocation in this environment is defined as $\{c_0^1,\{c_t^t,c_t^{t-1}\}_{t=1}^\infty\}$, so the social planner's problem, weighting all generations equally, is:
		\begin{align*}
			\usmax{c_0^1,\{c_t^t,c_t^{t-1}\}_{t=1}^\infty} &U(c_1^0) + \sum_{t=1}^\infty U(c_t^t,c_{t+1}^t) 				\\
			&\text{ s.t. }(1+n)^tc_t^t + (1+n)^{t-1}c_t^{t-1}\leq (1+n)^t w_1 + (1+n)^{t-1}w_2\text{ }\forall t\geq 1							\\
			\usmax{c_0^1,\{c_t^t,c_t^{t-1}\}_{t=1}^\infty} &\loge{c_1^0} + \sum_{t=1}^\infty \loge{c_t^t} +\loge{c_t^{t-1}}	\\
			 &\text{ s.t. } (1+n)^t c_t^t + (1+n)^{t-1}c_t^{t-1}\leq (1+n)w_1 + w_2\text{ }\forall t\geq 1
		\end{align*}
		The Lagrangian for this problem is
		\[
			\L= \sum_{t=1}^\infty \loge{c_t^t} + \loge{c_t^{t-1}} -\lambda_t\left(c_t^{t-1} + c_t^{t-1}- (1+n)w_1 - w_2\right)
		\]
		Taking first-order conditions and letting $t$ represent any $t\geq 1$, we get:
		\begin{align*}
			\frac{\partial\L}{\partial c_t^t} 		&= \frac{1}{c_t^t} 		- (1+n)\lambda_t	= 0	\\
			\frac{\partial\L}{\partial c_t^{t-1}} 	&= \frac{1}{c_t^{t-1}} 	- \lambda_t			= 0	\\
			\frac{\partial\L}{\partial \lambda_t} 	&= (1+n)c_t^t + c_t^{t-1} - (1+n)w_1 - w_2	= 0		
		\end{align*}
		Since $c_1^0$ is the consumption of the old generation in time $t$, we need only solve for $c_t^{t-1}$ and $c_t^t$ to derive $c_0^1$:
		\begin{align*}
			& \frac{1}{c_t^t}  = \frac{1+n}{c_t^{t-1}}					& c_t^{t-1} = (1+n)(w_1-c_t^t) + w_2 	\\
			& c_t^{t-1} = (1+n)c_t^t = (1+n)(w_1-c_t^t) + w_2												\\
			& 2c_t^t 	= w_1 + \frac{w_2}{1+n}																\\
			& c_t^t		= \frac{1}{2}\left(w_1 + \frac{w_2}{1+n}\right) & c_t^{t-1}	= \frac{1}{2}\left((1+n)w_1 + w_2\right)
		\end{align*}
		Thus, the social planner's optimal allocation, $\{c_0^1,\{c_t^t,c_t^{t-1}\}_{t=1}^\infty\}$, is 
		\[
			\left\{\frac{1}{2}\left((1+n)w_1 + w_2\right),\left\{\frac{1}{2}\left(w_1 + \frac{w_2}{1+n}\right),\frac{1}{2}\left((1+n)w_1 + w_2\right)\right\}_{t=1}^\infty\right\}
		\]
		
	\item A competitive equilibrium in this model is an allocation of consumption, $\{c_0^1,\{c_t^t,c_t^{t-1}\}_{t=1}^\infty\}$, prices, $\{p_t\}_{t=1}^\infty$, and money $\{M_{t+1}^t\}_{t=2}^\infty$ that solves the initial old generation's representative household problem,
		\[
			\usmax{c_1^0}\loge{c_1^0} \text{ s.t. } c_0^1 = w_2 + \frac{\overline{M_1}}{p_1}
		\]
		the representative household problem of each generation $t\geq 1$,
		\[
			\underset{\{c_t^t,c_{t+1}^t, M_{t+1}^t\}}{\text{max }\loge{c_t^t}+}\loge{c_{t+1}^t}\text{ s.t. } 
				c_t^t + \frac{M^t_{t+1}}{p_t} = w_1\text{, }c_{t+1}^t = w_2 + \frac{(1+z)M^t_{t+1}}{p_{t+1}}
		\]
		and that clears all markets:
		\begin{align*}
			(1+n)^tc_t^t + (1+n)^{t-1}c_{t+1}^t  	&= (1+n)^tw_1 + (1+n)^{t-1}w_2 	&\text{ (Goods market)}	\\
			(1+n)^tM_{t+1}^t 						&= \overline{M}_t				&\text{ (Money market)}
		\end{align*}
		Where $\overline{M}_t$ is the money supply in period $t$. The goods market clearing condition simplifies by dividing each side of the equation by $(1+n)^{t-1}$. The money market clearing condition simplifies by recognizing that inducting backward to period 1, when the money supply equals $\overline{M}_1$:
		\begin{align*}
			(1+n)c_t^t + c_{t+1}^t  &= (1+n)w_1 + w_2 				&\text{ (Goods market)}	\\
			(1+n)M_{t+1}^t 			&= (1+z)^{t-1}\overline{M}_1	&\text{ (Money market)}
		\end{align*}
		
	\item In autarkic equilibrium, each agent consumes their own endowment and no trade occurs. Thus, the autarkic allocation of consumption is:
		\begin{align*}
										c_0^1	&= w_2 	\\
			\{c_t^t,c_{t+1}^t\}_{t=1}^\infty	&= \{w_1,w_2\}_{t=1}^\infty
		\end{align*}
		This clears the goods market. By the Walras Law, the money market also clears. We can derive this equilibrium by solving each of the household optimization problems. Since the initial old generation faces an increasing utility function and has a single budget constraint for a single choice variable, we know that the equality of the budget contraint solves the problem: $c_1^0 = w_2 + \frac{\overline{M}_1}{p_1}$.
		\medskip \\
		The first-order conditions of the Lagrangian for the representative household's problem are:
		\begin{align*}
			\frac{\partial\L}{\partial c_t^t} 			&= \frac{1}{c_t^t} 		- \lambda_t 	= 0 \\
			\frac{\partial\L}{\partial c_{t+1}^t} 		&= \frac{1}{c_{t+1}^t} 	- \lambda_{t+1} = 0 \\
			\frac{\partial\L}{\partial M_{t+1}^t}		&= -\lambda_t\frac{M_{t+1}^t}{p_t} + \lambda_{t+1}\frac{M_{t+1}^t}{p_{t+1}} 	= 0 \\
			\frac{\partial\L}{\partial \lambda_t} 		&= c_t^t + \frac{M_{t+1}^t}{p_t} - w_1	= 0	\\
			\frac{\partial\L}{\partial \lambda_{t+1}} 	&= c_{t+1}^t -w_2 - \frac{(1+z)M_{t+1}^t}{p_{t+1}} = 0	
		\end{align*}
		Combining the consumption first-order conditions with that of money holdings yields:
		\begin{align*}
			\lambda_t\frac{M_{t+1}^t}{p_t} 				&= \lambda_{t+1}\frac{M_{t+1}^t}{p_{t+1}}  \\
			\frac{\lambda_t}{\lambda_{t+1}} 			&= \frac{(1+z)p_t}{p_{t+1}}					\\
			\frac{\frac{1}{c_t^t}}{\frac{1}{c_{t+1}^t}} &= \frac{\lambda_t}{\lambda_{t+1}}			\\
			\therefore\text{ }\frac{c^t_{t+1}}{c_t^t} 	&= \frac{(1+z)p_t}{p_{t+1}}
		\end{align*}
		Knowing that agents will consume their endowments in each period, we can derive:
		\[
			\frac{p_t}{p_{t+1}} = \frac{w_2}{(1+z)w_1}
		\]
		To solve for the steady-state money holdings in the autarkic equilibrium, we use the money market clearing condition:
		\[
			M^t_{t+1} = \frac{(1+z)^{t-1}}{1+n}\overline{M}_1 
		\]
		Plugging this back into the household budget contraints yields:
		\begin{align*}
			\frac{(1+z)^{t-1}}{(1+n)p_t}\overline{M}_1 &= w_1-c_t^t 		= 0 \\
			\frac{(1+z)^t}{(1+n)p_{t+1}}\overline{M}_1 &= c_{t+1}^t - w_2	= 0
		\end{align*}
		So $\frac{1}{p_t}=0$ $\forall t$ in this equilibrium. Thus, our steady-state autarkic equilibrium is:
		\[
			\{c_0^1,\{c_t^t,c^t_{t+1},M_{t+1}^t\}_{t=1}^\infty\} = \left\{w_2,\left\{w_1,w_2,\frac{(1+z)^{t-1}}{1+n}\overline{M}_1 \right\}_{t=1}^\infty\right\}
		\]
		With money having no value.
		
	\item To begin solving for a compeitive monetary equilibrium, we can rewrite the representative household problem for generation $t\geq 1$ in terms of solely $M_{t+1}^t$ using each period's budget constraint (since $U(c_t^t,c_{t+1})$ is strictly increasing in $c_t^t$ and $c_{t+1}$, the budget constraints will hold with equality):
		\[
			\usmax{M_{t+1}^t}\loge{w_1-\frac{M_{t+1}^t}{p_t}} + \loge{w_2 + \frac{(1+z)M_{t+1}^t}{p_{t+1}}}
		\]
		The first-order condition of this unconstrained maximization problem enables us to solve for the steady-state optimum of $\Mt$ with respect to prices in each period:
		\begin{align*}
			-\frac{1}{p_t\left(w_1-\frac{\Mt}{p_t}\right)} + \frac{1+z}{p_{t+1}\left(w_2 + \frac{(1+z)\Mt}{p_{t+1}}\right)} &= 0	\\
			(1+z)p_tw_1 + (1+z)\Mt &= p_{t+1}w_2 + (1+z)\Mt	\\
			2(1+z)\Mt &= (1+z)p_tw_1-p_{t+1}w_2	\\
			\Mt & = \frac{1}{2}\left(p_tw_1 - \frac{p_{t+1}}{1+z}w_2\right)
		\end{align*}
		We can use this relation to solve for the steady-state level of consumption with respect to $q_t=\frac{p_t}{p_{t+1}}$:
		\begin{align*}
			&c_t^t 		= w_1 - \frac{1}{2p_t}\left(p_tw_1 - \frac{p_{t+1}}{1+z}w_2\right)\text{, }
			&c_{t+1}^t 	= w_2 + \frac{1+z}{2p_{t+1}}\left(p_tw_1 - \frac{p_{t+1}}{1+z}w_2\right)											\\
			&c_t^t 		= \frac{1}{2}w_1 + \frac{1}{2q_t(1+z)}\text{, }				& c_{t+1}^t = \frac{1+z}{2}q_tw_1 + \frac{1}{2}w_2		\\
			&c_t^t 		= \frac{1}{2}\left(w_1 + \frac{1}{q_t(1+z)}\right)\text{, }	& c_{t+1}^t = \frac{1}{2}\left((1+z)q_tw_1 + w_2\right)	\\
		\end{align*}
		We can then solve for the steady-state level of $q_t$ using the goods market clearing condition and recognizing that, in the steady state, $q_{t-1}=q_t=\overline{q}$ for all $t$:
		\begin{align*}
			(1+n)c_t^t + c_{t+1}^t &= (1+n)w_1 + w_2	\\
			(1+n)\frac{1}{2}\left(w_1 - \frac{1}{\olq(1+z)}\right) + \frac{1}{2}\left((1+z)\olq w_1 + w_2\right) &= (1+n)w_1 + w_2 \\
			\olq(1+n)w_1 - \frac{(1+n)}{(1+z)} + (1+z)\olq^2 w_1 + \olq w_2 &= 2(1+n)\olq w_1 + 2\olq w_2 
		\end{align*}
		\begin{align*}
			(1+z)\olq^2 w_1 - \left[(1+n)w_1 + 2(1+n) w_1 - w_2 + 2w_2 \right]\olq + \frac{(1+n)}{(1+z)}w_2  &= 0  \\
			(1+z)\olq^2 w_1 - \left((1+n)w_1 + w_2 + \right)\olq + \frac{(1+n)}{(1+z)}w_2  &= 0  
		\end{align*}
		\begin{align*}
			\olq &= \frac{(1+n)w_1+w_2\pm \sqrt{((1+n)w_1 + w_2)^2 - 4(1+z)w_1\left(\frac{1+n}{1+z}\right)w_2}}{2(1+z)w_1}	\\
			\olq &= \frac{(a+n)w_1+w_2\pm\sqrt{(1+n)^2w_1^2-2(1+n)w_1w_2+w_2^2}}{2(1+z)w_1}
		\end{align*}
		Then, there are two possible ``candidates" for the steady-state $\olq$:
		\begin{align*}
			&\olq = \frac{(1+n)w_1+w_2+(1+n)w_1-w_2}{2(1+z)w_1}	&\olq = \frac{(1+n)w_1+w_2-(1+n)w_1+w_2}{2(1+z)w_1}	\\
			&\olq = \frac{1+n}{1+z}								&\olq = \frac{w_2}{(1+z)w_1}
		\end{align*}
		We know from the last question that $\frac{w_2}{(1+z)w_1}$ is the price ratio for the autarkic equilibrium. Therefore, the price ratio for the monetary equilibrium is $\frac{1+n}{1+z}$. This can be considered the inverse of the rate of return on money. This rate intuitively makes sense because $1+z$ is the growth rate of money holdings, and $1+n$ is the growth rate of the population. $1+z$ is the nominal return, which, divided by the growth in the population (which also holds money that will grow by $1+z$), becomes the real return. Then, the steady-state level of consumption in each period is:
		\begin{align*}
			&c_t^t 		= \frac{1}{2}\left(w_1 + \frac{1}{\left(\frac{1+n}{1+z}\right)(1+z)}\right)\text{, }	& c_{t+1}^t = \frac{1}{2}\left((1+z)\left(\frac{1+n}{1+z}\right)w_1 + w_2\right) \\
			&c_t^t 		= \frac{1}{2}\left(w_1 + \frac{w_2}{(1+n)}\right)\text{, }	& c_{t+1}^t = \frac{1}{2}\left((1+n)w_1 + w_2\right) \\
		\end{align*}
		This shows that the non-negativity constraint is not binding. Households value money as a means of smoothing their consumption between young and old age. Finally, we can use the initial old generation's budget contraint to derive $p_1$, which enables us to calculate $p_t$ for each $t\geq 1$:
		\begin{align*}
			&c_0^1 = w_2 + \frac{\overline{M}_1}{p_1}								& p_t = \left(\frac{1+z}{1+n}\right)^{t-1}p_1	   \\
			&p_1 = \frac{\overline{M}_1}{c_0^1-w_2}									& \\
			&p_1 = \frac{\overline{M}_1}{\frac{1}{2}(1+n)w_1 + \frac{1}{2}w_2-w_2}	& \\
			&p_1 = \frac{2\overline{M}_1}{(1+n)w_1-w_2}	&p_t = \left(\frac{1+z}{1+n}\right)^{t-1}\frac{2\overline{M}_1}{(1+n)w_1-w_2}
		\end{align*}
		From the  market clearing conditions solved in the last question, we know that, in any steady state:
		\[
			M_{t+1}^t = \frac{(1+z)^{t-1}}{1+n}\overline{M}_1
		\]
		Thus, the equilibrium allocations of consumption, $\{c_0^1,\{c_t^t,c^t_{t+1},M_{t+1}^t,p_t\}_{t=1}^\infty\}$, money holdings, $\{M_{t+1}^t\}_{t=1}^\infty$, and prices, $\{p_t\}_{t=1}^\infty$, in the monetary steady state are:
		\begin{align*}
			\{c_0^1,\{c_t^t,c^t_{t+1}\}_{t=1}^\infty\} &= \left\{\frac{1}{2}\left((1+n)w_1 + w_2\right), \left\{\frac{1}{2}\left(w_1 + \frac{w_2}{(1+n)}\right),\frac{1}{2}\left((1+n)w_1 + w_2\right) \right\}_{t=1}^\infty\right\} \\
			\{M_{t+1}^t\}_{t=1}^\infty &= \left\{\frac{(1+z)^{t-1}}{1+n}\overline{M}_1 \right\}_{t=1}^\infty \\
			\{p_t\}_{t=1}^\infty &= \left\{\left(\frac{1+z}{1+n}\right)^{t-1}p_1 \right\}_{t=1}^\infty
		\end{align*}
		
		
	\item We derived three allocations for each representative consumption (the intial old generation, generation $t$'s consumption when young, and generation $t$'s consumption when old): the social planner's allocation, the autarky allocation, and the monetary equilibrium:
	\begin{center}
		\begin{tabular}{r|c c c}
						& $c_0^1$ 									& $c_t^t$ 											& $c_{t+1}^t$								\\
	\hline				&											&													&										  	\\
			SPP			& $\frac{1}{2}\left((1+n)w_1 + w_2\right)$ 	& $\frac{1}{2}\left(w_1 + \frac{w_2}{1+n}\right)$ 	& $\frac{1}{2}\left((1+n)w_1 + w_2\right)$	\\
			Autarky 	& $w_2$ 									& $w_1$ 											& $w_2$ 									\\
			Monetary	& $\frac{1}{2}\left((1+n)w_1 + w_2\right)$ 	& $\frac{1}{2}\left(w_1 + \frac{w_2}{(1+n)}\right)$ & $\frac{1}{2}\left((1+n)w_1 + w_2\right)$ 
		\end{tabular}
	\end{center}
	To determine whether the stationary monetary equilibrium dominates autarky, we need to show that at least one group is made better off without any others being made worse-off. In our model, we have two groups to compare: the initial old generation, and each generation in year $t\geq1$. The utility of these generations in each equilibrium is:
	\begin{center}
		\begin{tabular}{r|c c}
							& Autarky  					& Monetary 																									\\
		\hline				&							&																											\\
			Initial Old		& $\loge{w_2}$ 				& $\loge{\frac{1}{2}\left((1+n)w_1 + w_2\right)}$ 															\\
			Generation $t$	& $\loge{w_1} + \loge{w_2}$	& $\loge{\frac{1}{2}\left(w_1 + \frac{w_2}{(1+n)}\right)} + \loge{\frac{1}{2}\left((1+n)w_1 + w_2\right)}$ 
		\end{tabular}
	\end{center}
	Since $1+n>1$ and $w_1>w_2$, it is trivially clear that $\frac{1}{2}\left((1+n)w_1 + w_2\right)>w_2$, so the initial old generation is made strictly better off in the monetary equilibrium. All other generations must be at least as well-off in the monetary equilibrium because autarky is always feasible, and the monetary equilibrium is reached by maximizing their utility. Thus, if autarky made them strictly better off, then there would be no monetary equilirium with a different allocation than in autarky.
	
	\item Yes, money in this model exhibits super-neutrality. The stationary monetary equilibrium is determined by a fixed ratio, $\overline{q}$ of the price of the consumption good in one period to the price in the previous period, where $\overline{q}=\frac{1+n}{1+z}$. We can see that this does not depend on the level of inflation by observing the representative agent's intertemporal rate of substitution, which was derived in question 3 as $\frac{c_{t+1}}{c_t^t} = \frac{(1+z)p_t}{p_{t+1}}$. Thus, if we let $\pi$ be the rate of inflation from one period to another such that $p_{t+1}=(1+\pi)p_t$, then
		\[
			\frac{c_{t+1}}{c_t^t} = (1+z)\frac{p_t}{p_{t+1}} = (1+z)\frac{(1+\pi)p_{t-1}}{(1+\pi)p_{t}} =  (1+z)\frac{p_{t-1}}{p_{t}} = \frac{1+z}{\overline{q}}
		\]
		Thus, the optimal price ratio is unchanged by the level of inflation.
	
\end{enumerate}


%%%________________________________________________________________%%%


\end{document}












