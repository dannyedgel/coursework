%%% Econ703: Math Camp
%%% Fall 2020
%%% Danny Edgel
%%%
% Due on Canvas Wednesday September 30, 11pm Central Time
%%%

%%%
%							PREAMBLE
%%%

\documentclass{article}

%%% declare packages
\usepackage{amsmath}
\usepackage{amssymb}
\usepackage{array}
\usepackage{bm}
\usepackage{changepage}
\usepackage{centernot}
\usepackage{graphicx}
\usepackage{fancyhdr}
	\fancyhf{} % sets both header and footer to nothing
	\renewcommand{\headrulewidth}{0pt}
    \rfoot{Edgel, \thepage}
    \pagestyle{fancy}
	
%%% define shortcuts for set notation
\newcommand{\N}{\mathbb{N}}
\newcommand{\Z}{\mathbb{Z}}
\newcommand{\R}{\mathbb{R}}
\newcommand{\Q}{\mathbb{Q}}
\newcommand{\lmt}{\underset{x\rightarrow\infty}{\text{lim }}}
\newcommand{\neglmt}{\underset{x\rightarrow-\infty}{\text{lim }}}
\newcommand{\zerolmt}{\underset{x\rightarrow 0}{\text{lim }}}
\newcommand{\usmax}[1]{\underset{#1}{\text{max }}}
\newcommand{\usmin}[1]{\underset{#1}{\text{min }}}
\newcommand{\inv}{^{-1}}
\newcommand{\at}[2][]{#1|_{#2}}

%%% define column vector command (from Michael Nattinger)
\newcount\colveccount
\newcommand*\colvec[1]{
        \global\colveccount#1
        \begin{pmatrix}
        \colvecnext
}
\def\colvecnext#1{
        #1
        \global\advance\colveccount-1
        \ifnum\colveccount>0
                \\
                \expandafter\colvecnext
        \else
                \end{pmatrix}
        \fi
}

%%% define function for drawing matrix augmentation lines
\newcommand\aug{\fboxsep=-\fboxrule\!\!\!\fbox{\strut}\!\!\!}

\makeatletter
\let\amsmath@bigm\bigm

\renewcommand{\bigm}[1]{%
  \ifcsname fenced@\string#1\endcsname
    \expandafter\@firstoftwo
  \else
    \expandafter\@secondoftwo
  \fi
  {\expandafter\amsmath@bigm\csname fenced@\string#1\endcsname}%
  {\amsmath@bigm#1}%
}


%________________________________________________________________%

\begin{document}

\title{	Problem Set \#6 }
\author{ 	Danny Edgel 							\\ 
			Econ 703: Mathematical Economics I		\\
			Fall 2020								\\
		}
\maketitle\thispagestyle{empty}

%%%________________________________________________________________%%%

\noindent\textit{Collaborated with Sarah Bass, Emily Case, Michael Nattinger, and Alex Von Hafften}

%%%________________________________________________________________%%%

\section*{Question 1}
The time Bob takes to walk to Happy Cow Farm\footnote{Whether "Happy Cow" is an appropriate name for a place that exists primarily to harvest cows is an ethical one and thus beyond the scope of this question.} is given by:
\[
	T = \frac{1}{5}D_R + \frac{1}{3}D_F
\]
Where $D_R$ is the distance Bob travels on the road, and $D_F$ is the distance he travels through the forest. Letting $x$ represent the difference between $D_R$ and the maximim distance Bob would travel on the road, we can solve the problem as:
\begin{align*}
	\usmin{x}\frac{1}{5}(12-x) + \frac{1}{6}\sqrt{25 + x^2}	\\
	\frac{dT}{dx} 						&= 0			 	\\
	\frac{x}{3\sqrt{25+x^2}}			&= \frac{1}{5}		\\
	25x^2								&= 9(25 + x^2)		\\
	x^2									&= \frac{225}{16}	
\end{align*}
Since $x$ cannot be negative, and $\frac{x}{3\sqrt{25+x^2}}-\frac{1}{5}$ is non-decreasing for $x\geq 0$, we know that only $x=\frac{15}{4}$ minimizes $T$. Thus,
\[
	T = \frac{1}{5}(12-\frac{15}{4}) + \frac{1}{6}\sqrt{25 + \left(\frac{15}{4}\right)^2} = \frac{56}{15}
\]
Thus, the shortest amount of time it will take Bob to walk to Happy Cow Farm is 224 minutes, or 3 hours and 44 minutes.
	
%%%________________________________________________________________%%%

\section*{Question 2}
It is not possible for $x_0$ to be a local optimum of $f$. Suppose $f'(x_0)=0$. Then, $x_0$ is an inflection point of $f$ but not a local optimum.
\medskip \\
\textbf{Proof.}
\begin{enumerate}
	\item Let $x_1<x_0<x_2\in B_\varepsilon(x_0)$.\footnote{Since the domain of $f$ is $\R$, the existence of $x_1$ and $x_2$ are trivially proven by the fact that $B_\varepsilon(x_0)$ contains an infinite number of elements for all $\varepsilon>0$.}  By assumption, $f'(x)<0$ for all $x\in  B_\varepsilon(x_0)$. 
	\item Since $f$ is differentiable for all $x\in  B_\varepsilon(x_0)$, $f$ is also continuous  $\forall x\in  B_\varepsilon(x_0)$. Thus, $f(x_1)>f(x_2)$ and, by the mean value theorem, $f(x_1)>f(x_0)>f(x_2)$
\end{enumerate}
$\therefore$ $x_0$ is not a local optimum of $f$ $\blacksquare$


%%%________________________________________________________________%%%

\section*{Question 3}
By the chain rule, for $\beta\in\{r,s,t\}$, $\frac{\partial w}{\partial\beta}=\sum_{\alpha\in\{x,y,z\}}\frac{\partial w}{\partial\alpha}\cdot\frac{\partial\alpha}{\partial\beta}$. Thus,
(r+2s+t)
(2r+3s+t) 
(3r+s+t)
\begin{align*}
	\frac{\partial w}{\partial r} 	&= y^2z  + 4xyz + 3xy^2 \\
									&= (2r+3s+t)^2(3r+s+t)  + 4(r+2s+t)(2r+3s+t)(3r+s+t) + 3(r+2s+t)(2r+3s+t)^2	\\
	\frac{\partial w}{\partial s} 	&= 2y^2z + 6xyz + xy^2  \\
									&= (2r+3s+t)^2(3r+s+t)  + 4(r+2s+t)(2r+3s+t)(3r+s+t) + 3(r+2s+t)(2r+3s+t)^2	\\
	\frac{\partial w}{\partial t} 	&= y^2z  + 2xyz + xy^2  \\
									&= (2r+3s+t)^2(3r+s+t)  + 4(r+2s+t)(2r+3s+t)(3r+s+t) + 3(r+2s+t)(2r+3s+t)^2
\end{align*}


%%%________________________________________________________________%%%

\section*{Question 4}
Let $f:X\rightarrow\R^n$ be continuously differentiable on $X\subset\R^n$. Then, for any $x,y\in X$ and $i,j\in\{1,...,n\}$,
\[
	\frac{\partial f^i}{\partial x_j}(x) = \underset{y_j\rightarrow x_j}{\text{lim }}\frac{f^i(y_j)-f^i(x_j)}{y_j-x_j}
\]
Removing the limit and taking absolute values enables us to derive, when $y\in B_\varepsilon(x)$ for some $\varepsilon>0$,
\[
	|f^i(y_j)-f^i(x_j)| \leq k|y_j-x_j|
\]
Where $k=|\frac{\partial f^i}{\partial x_j}(x)|$. Now let $k=\usmax{j\in\{1,...,n\}}\left\{\frac{\partial f^i}{\partial x_j}(x)\right\}$. Then, for every $i\in\{1,...,n\}$,
\[
	|f^i(y)-f^i(x)| \leq k|y_i-x_i|
\]
Thus, if we let $k=\usmax{i,j\in\{1,...,n\}}\left\{\frac{\partial f^i}{\partial x_j}(x)\right\}$, we can conclude that:
\begin{align*}
	\sqrt{\sum_{i=1}^n\left(f(y_i)-f(x_i)\right)^2} &\leq \sqrt{\sum_{i=1}^nk^2\left(y_i-x_i\right)^2} \\
	d(f(x),f(y)) &\leq \sqrt{n}kd(x,y)
\end{align*}
$\therefore$ $f$ is locally Lipschitz on $X$ $\blacksquare$

%%%________________________________________________________________%%%

\section*{Question 5}
We know that $f(1,1)=0$. $D_xf(x,y) = 5x-2x+1$, so $D_Xf(1,1) \neq 0$. Then the implicit function theorem applies, and we can calcluate:
\[
	\frac{\partial x(y)}{\partial y}\at[\big]{y=1} = -(D_xf(1,1))^{-1}(D_y(1,1)) = -\frac{-3-2}{5-2+1} = \frac{5}{4}
\]

%%%________________________________________________________________%%%


\section*{Question 6}
First, we must solve for the critical points of $f(x,y)=2x^4 + y^2 - xy + 1$:
\[
	\colvec{2}{\partial f/\partial x}{\partial f/\partial y} = \colvec{2}{8x^3-y}{2y-x} = \colvec{2}{0}{0}
\]
By substituting $y=8x^3$ into $x=2y$, we can solve that this is satisfied when $x(16x^2-1)=0$. Thus, the critical points are $\colvec{2}{0}{0}$, $\colvec{2}{1/4}{1/8}$ and $\colvec{2}{-1/4}{-1/8}$. To determine whether these are maxima, minima, or saddle points, we must calculate the function's Hessian matrix:
\[
	H = \begin{pmatrix} \frac{\partial^2 f}{\partial x^2}  & \frac{\partial f}{\partial y\partial x} \\ \frac{\partial f}{\partial x\partial y} & \frac{\partial^2 f}{\partial y^2}\end{pmatrix}  = \begin{pmatrix} 24x^2 & -1 \\ -1 & 2 \end{pmatrix}
\]
Then the determinant of $H$ is $48x^2 - 1$. For each of our critical points, we can solve:
\begin{align*}
	\colvec{2}{0}{0}:\text{ } 		&	|H|	= -1 <0 					\\
	\colvec{2}{1/4}{1/8}:\text{ } 	&	|H| = \frac{48}{16} - 1 =2 >0	\\
	\colvec{2}{-1/4}{-1/8}:\text{ } &	|H| = \frac{48}{16} - 1 =2 >0
\end{align*}
Thus, $\colvec{2}{0}{0}$ is a saddle point. Since $\frac{\partial^2 f}{\partial x^2}>0$ $\forall x\in\R$, $\colvec{2}{1/4}{1/8}$ and $\colvec{2}{-1/4}{-1/8}$ are local minima. To determine whether either of these points are global minima, we must determine the function's behavior at its limits:
\begin{align*}
	\underset{x\rightarrow\infty,y\rightarrow\infty}{\text{lim }}f(x,y) 	&= \infty \\
	\underset{x\rightarrow-\infty,y\rightarrow\infty}{\text{lim }}f(x,y) 	&= \infty \\
	\underset{x\rightarrow\infty,y\rightarrow-\infty}{\text{lim }}f(x,y) 	&= \infty \\
	\underset{x\rightarrow-\infty,y\rightarrow-\infty}{\text{lim }}f(x,y) 	&= \infty
\end{align*}
Further, $f(1/4,1/8)=f(-1/4,-1/8)$. Thus, both  $\colvec{2}{1/4}{1/8}$ and $\colvec{2}{-1/4}{-1/8}$ are global minima.

%%%________________________________________________________________%%%



\end{document}
















