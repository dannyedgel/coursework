%%% Econ703: Math Camp
%%% Fall 2020
%%% Danny Edgel
%%%
% Due on Canvas Wednesday September 30, 11pm Central Time
%%%

%%%
%							PREAMBLE
%%%

\documentclass{article}

%%% declare packages
\usepackage{amsmath}
\usepackage{amssymb}
\usepackage{array}
\usepackage{bm}
\usepackage{changepage}
\usepackage{centernot}
\usepackage{graphicx}
\usepackage{fancyhdr}
	\fancyhf{} % sets both header and footer to nothing
	\renewcommand{\headrulewidth}{0pt}
    \rfoot{Edgel, \thepage}
    \pagestyle{fancy}
	
%%% define shortcuts for set notation
\newcommand{\N}{\mathbb{N}}
\newcommand{\Z}{\mathbb{Z}}
\newcommand{\R}{\mathbb{R}}
\newcommand{\Q}{\mathbb{Q}}
\newcommand{\lmt}{\underset{x\rightarrow\infty}{\text{lim }}}
\newcommand{\neglmt}{\underset{x\rightarrow-\infty}{\text{lim }}}
\newcommand{\zerolmt}{\underset{x\rightarrow 0}{\text{lim }}}
\newcommand{\usmax}[1]{\underset{#1}{\text{max }}}
\newcommand{\usmin}[1]{\underset{#1}{\text{min }}}
\newcommand{\inv}{^{-1}}

%%% define column vector command (from Michael Nattinger)
\newcount\colveccount
\newcommand*\colvec[1]{
        \global\colveccount#1
        \begin{pmatrix}
        \colvecnext
}
\def\colvecnext#1{
        #1
        \global\advance\colveccount-1
        \ifnum\colveccount>0
                \\
                \expandafter\colvecnext
        \else
                \end{pmatrix}
        \fi
}

%%% define function for drawing matrix augmentation lines
\newcommand\aug{\fboxsep=-\fboxrule\!\!\!\fbox{\strut}\!\!\!}

\makeatletter
\let\amsmath@bigm\bigm

\renewcommand{\bigm}[1]{%
  \ifcsname fenced@\string#1\endcsname
    \expandafter\@firstoftwo
  \else
    \expandafter\@secondoftwo
  \fi
  {\expandafter\amsmath@bigm\csname fenced@\string#1\endcsname}%
  {\amsmath@bigm#1}%
}


%________________________________________________________________%

\begin{document}

\title{	Problem Set \#6 }
\author{ 	Danny Edgel 							\\ 
			Econ 703: Mathematical Economics I		\\
			Fall 2020								\\
		}
\maketitle\thispagestyle{empty}

%%%________________________________________________________________%%%

\textit{Collaborated with Sarah Bass, Emily Case, Michael Nattinger, and Alex Von Hafften}

%%%________________________________________________________________%%%

\section*{Question 1}
The time Bob takes to walk to Happy Cow Farm\footnote{Whether "Happy Cow" is an appropriate name for a place that exists primarily to harvest cows is an ethical one and thus beyond the scope of this question.} is given by:
\[
	T = \frac{1}{5}D_R + \frac{1}{3}D_F
\]
Where $D_R$ is the distance Bob travels on the road, and $D_F$ is the distance he travels through the forest. Letting $x$ represent the difference between $D_R$ and the maximim distance Bob would travel on the road, we can solve the problem as:
\begin{align*}
	\usmin{x}\frac{1}{5}(12-x) + \frac{1}{6}\sqrt{25 + x^2}	\\
	\frac{dT}{dx} 						&= 0			 	\\
	\frac{x}{3\sqrt{25+x^2}}			&= \frac{1}{5}		\\
	25x^2								&= 9(25 + x^2)		\\
	x^2									&= \frac{225}{16}	
\end{align*}
Since $x$ cannot be negative, and $\frac{x}{3\sqrt{25+x^2}}-\frac{1}{5}$ is non-decreasing for $x\geq 0$, we know that only $x=\frac{15}{4}$ minimizes $T$. Thus,
\[
	T = \frac{1}{5}(12-\frac{15}{4}) + \frac{1}{6}\sqrt{25 + \left(\frac{15}{4}\right)^2} = \frac{56}{15}
\]
Thus, the shortest amount of time it will take Bob to walk to Happy Cow Farm is 224 minutes, or 3 hours and 11 minutes.
	
%%%________________________________________________________________%%%

\section*{Question 2}
It is not possible for $x_0$ to be a local optimum of $f$. Suppose $f'(x_0)=0$. Then, $x_0$ is an inflection point of $f$ but not a local optimum.
\medskip \\
\textbf{Proof.}
\begin{enumerate}
	\item Let $x_1<x_0<x_2\in B_\varepsilon(x_0)$.  By assumption, $f'(x_1)<0$ and $f'(x_2)<0$.
	\item Since $f$ is differentiable for all $x\in  B_\varepsilon(x_0)$, $f$ is also continuous  $\forall x\in  B_\varepsilon(x_0)$. Thus, $f(x_1)>f(x_0)>f(x_2)$
\end{enumerate}
$\therefore$ $x_0$ is not a local optimum of $f$ $\blacksquare$


%%%________________________________________________________________%%%

\section*{Question 3}
By the chain rule, for $\beta\in\{r,s,t\}$, $\frac{\partial w}{\partial\beta}=\sum_{\alpha\in\{x,y,z\}}\frac{\partial w}{\partial\alpha}\cdot\frac{\partial\alpha}{\partial\beta}$. Thus,
\begin{align*}
	\frac{\partial w}{\partial r} &= (1)y^2z + (2)(2)xyz + (3)xy^2 = y^2z  + 4xyz + 3xy^2	\\
	\frac{\partial w}{\partial s} &= (2)y^2z + (3)(2)xyz + (1)xy^2 = 2y^2z + 6xyz + xy^2 	\\
	\frac{\partial w}{\partial t} &= (1)y^2z + (1)(2)xyz + (1)xy^2 = y^2z  + 2xyz + xy^2
\end{align*}


%%%________________________________________________________________%%%

\section*{Question 4}


%%%________________________________________________________________%%%

\section*{Question 5}


%%%________________________________________________________________%%%


\section*{Question 6}


%%%________________________________________________________________%%%



\end{document}
















