%%% Econ703: Math Camp
%%% Fall 2020
%%% Danny Edgel
%%%
% Due on Canvas Wednesday October 7th, 11pm Central Time
%%%

%%%
%							PREAMBLE
%%%

\documentclass{article}

%%% declare packages
\usepackage{amsmath}
\usepackage{amssymb}
\usepackage{array}
\usepackage{bm}
\usepackage{changepage}
\usepackage{centernot}
\usepackage{graphicx}
\usepackage{fancyhdr}
	\fancyhf{} % sets both header and footer to nothing
	\renewcommand{\headrulewidth}{0pt}
    \rfoot{Edgel, \thepage}
    \pagestyle{fancy}
	
%%% define shortcuts for set notation
\newcommand{\N}{\mathbb{N}}
\newcommand{\Z}{\mathbb{Z}}
\newcommand{\R}{\mathbb{R}}
\newcommand{\Q}{\mathbb{Q}}
\newcommand{\lmt}{\underset{x\rightarrow\infty}{\text{lim }}}
\newcommand{\neglmt}{\underset{x\rightarrow-\infty}{\text{lim }}}
\newcommand{\zerolmt}{\underset{x\rightarrow 0}{\text{lim }}}
\newcommand{\usmax}[1]{\underset{#1}{\text{max }}}
\newcommand{\usmin}[1]{\underset{#1}{\text{min }}}
\newcommand{\inv}{^{-1}}
\newcommand{\at}[2][]{#1|_{#2}}

%%% define column vector command (from Michael Nattinger)
\newcount\colveccount
\newcommand*\colvec[1]{
        \global\colveccount#1
        \begin{pmatrix}
        \colvecnext
}
\def\colvecnext#1{
        #1
        \global\advance\colveccount-1
        \ifnum\colveccount>0
                \\
                \expandafter\colvecnext
        \else
                \end{pmatrix}
        \fi
}

%%% define function for drawing matrix augmentation lines
\newcommand\aug{\fboxsep=-\fboxrule\!\!\!\fbox{\strut}\!\!\!}

\makeatletter
\let\amsmath@bigm\bigm

\renewcommand{\bigm}[1]{%
  \ifcsname fenced@\string#1\endcsname
    \expandafter\@firstoftwo
  \else
    \expandafter\@secondoftwo
  \fi
  {\expandafter\amsmath@bigm\csname fenced@\string#1\endcsname}%
  {\amsmath@bigm#1}%
}


%________________________________________________________________%

\begin{document}

\title{	Problem Set \#7 }
\author{ 	Danny Edgel 							\\ 
			Econ 703: Mathematical Economics I		\\
			Fall 2020								\\
		}
\maketitle\thispagestyle{empty}

%%%________________________________________________________________%%%

\noindent\textit{Collaborated with Sarah Bass, Emily Case, Michael Nattinger, and Alex Von Hafften}

%%%________________________________________________________________%%%

\section*{Question 1}
Let $X\subset\R^n$ be convex. We can prove that, for any $k\in\N$, $\lambda_1,...,\lambda_k\geq 0$, $\sum_{i=1}^k\lambda_i=1$, if ${x_1,...,x_k\in X}$, then ${\sum_{i=1}^k\lambda_ix_1\in X}$.
\medskip \\
\textbf{Proof.}
\begin{enumerate}
	\item \textit{Base step.} Suppose $x_1,x_2\in X$. Since $X$ is convex, $(1-\lambda)x_1 + \lambda x_2$ is also in $X$ for all $\lambda\in[0,1]$
	\item \textit{Induction Step.} Assume that, for some $k\in\N$, $\sum_{i=1}^k\lambda_ix_i\in X$, where $\sum_{i=1}^k=1$. Let $x_{k+1}\in X$ and $\lambda'\in[0,1]$. Then, since $X$ is convex, 
		\[
			(1-\lambda')x_{k+1} + \lambda'\sum_{i=1}\lambda_ix_i
		\]
		is also in $X$. Now, define
		\[
			{\lambda'}_i = \begin{cases} \lambda'\lambda_i, &i\in\{1,...,k\} \\ 1 - \lambda', &i=k+1 \end{cases}
		\]
		Then, $\sum_{i=1}^{k+1}{\lambda'}_ix_i\in X$ and $\sum_{i=1}^{k+1}{\lambda'}_i=1$
\end{enumerate}
$\therefore$ $\sum_{i=1}^k\lambda_ix_i\in X$ for any $k\in\N$ $\blacksquare$

	
%%%________________________________________________________________%%%

\section*{Question 2}


%%%________________________________________________________________%%%



\end{document}
















