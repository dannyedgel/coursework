%%% Econ703: Math Camp
%%% Fall 2020
%%% Danny Edgel
%%%
% Due on Canvas Wednesday October 7th, 11pm Central Time
%%%

%%%
%							PREAMBLE
%%%

\documentclass{article}

%%% declare packages
\usepackage{amsmath}
\usepackage{amssymb}
\usepackage{array}
\usepackage{bm}
\usepackage{changepage}
\usepackage{centernot}
\usepackage{graphicx}
\usepackage{fancyhdr}
	\fancyhf{} % sets both header and footer to nothing
	\renewcommand{\headrulewidth}{0pt}
    \rfoot{Edgel, \thepage}
    \pagestyle{fancy}
	
%%% define shortcuts for set notation
\newcommand{\N}{\mathbb{N}}
\newcommand{\Z}{\mathbb{Z}}
\newcommand{\R}{\mathbb{R}}
\newcommand{\Q}{\mathbb{Q}}
\newcommand{\lmt}{\underset{x\rightarrow\infty}{\text{lim }}}
\newcommand{\neglmt}{\underset{x\rightarrow-\infty}{\text{lim }}}
\newcommand{\zerolmt}{\underset{x\rightarrow 0}{\text{lim }}}
\newcommand{\usmax}[1]{\underset{#1}{\text{max }}}
\newcommand{\usmin}[1]{\underset{#1}{\text{min }}}
\newcommand{\inv}{^{-1}}
\newcommand{\at}[2][]{#1|_{#2}}
\newcommand{\cl}[1]{\text{cl}#1}
\newcommand{\hyp}[1]{\text{hyp}#1}
\newcommand{\coS}{\text{co}S}
\newcommand{\infsum}[2]{\underset{\pi\in\Pi_{#1}}{\text{inf}}\sum_{i=1}^n\pi_i({#2})}
\newcommand{\intersect}{\bigcap}
\newcommand{\union}{\bigcup}

%%% define column vector command (from Michael Nattinger)
\newcount\colveccount
\newcommand*\colvec[1]{
        \global\colveccount#1
        \begin{pmatrix}
        \colvecnext
}
\def\colvecnext#1{
        #1
        \global\advance\colveccount-1
        \ifnum\colveccount>0
                \\
                \expandafter\colvecnext
        \else
                \end{pmatrix}
        \fi
}

%%% define function for drawing matrix augmentation lines
\newcommand\aug{\fboxsep=-\fboxrule\!\!\!\fbox{\strut}\!\!\!}

\makeatletter
\let\amsmath@bigm\bigm

\renewcommand{\bigm}[1]{%
  \ifcsname fenced@\string#1\endcsname
    \expandafter\@firstoftwo
  \else
    \expandafter\@secondoftwo
  \fi
  {\expandafter\amsmath@bigm\csname fenced@\string#1\endcsname}%
  {\amsmath@bigm#1}%
}


%________________________________________________________________%

\begin{document}

\title{	Problem Set \#7 }
\author{ 	Danny Edgel 							\\ 
			Econ 703: Mathematical Economics I		\\
			Fall 2020								\\
		}
\maketitle\thispagestyle{empty}

%%%________________________________________________________________%%%

\noindent\textit{Collaborated with Sarah Bass, Emily Case, Michael Nattinger, and Alex Von Hafften}

%%%________________________________________________________________%%%

\section*{Question 1}
Let $X\subset\R^n$ be convex. We can prove that, for any $k\in\N$, $\lambda_1,...,\lambda_k\geq 0$, $\sum_{i=1}^k\lambda_i=1$, if ${x_1,...,x_k\in X}$, then ${\sum_{i=1}^k\lambda_ix_1\in X}$.
\medskip \\
\textbf{Proof.}
\begin{enumerate}
	\item \textit{Base step.} Suppose $x_1,x_2\in X$. Since $X$ is convex, $(1-\lambda)x_1 + \lambda x_2$ is also in $X$ for all $\lambda\in[0,1]$
	\item \textit{Induction Step.} Assume that, for some $k\in\N$, $\sum_{i=1}^k\lambda_ix_i\in X$, where $\sum_{i=1}^k=1$. Let $x_{k+1}\in X$ and $\lambda'\in[0,1]$. Then, since $X$ is convex, 
		\[
			(1-\lambda')x_{k+1} + \lambda'\sum_{i=1}\lambda_ix_i
		\]
		is also in $X$. Now, define
		\[
			{\lambda'}_i = \begin{cases} \lambda'\lambda_i, &i\in\{1,...,k\} \\ 1 - \lambda', &i=k+1 \end{cases}
		\]
		Then, $\sum_{i=1}^{k+1}{\lambda'}_ix_i\in X$ and $\sum_{i=1}^{k+1}{\lambda'}_i=1$
\end{enumerate}
$\therefore$ $\sum_{i=1}^k\lambda_ix_i\in X$ for any $k\in\N$ $\blacksquare$

	
%%%________________________________________________________________%%%
\pagebreak
\section*{Question 2}
Define $C$ as the set of all convex combinations of $S$.
\begin{enumerate}
	\item Suppose $x\in C$
		\begin{enumerate}
			\item By definition, $\exists s_1,...,s_n\in S$, $\lambda_1,...,\lambda_n\in[0,1]$, $\sum_{i=1}^n\lambda_i=1$ such that ${\sum_{i=1}^n\lambda_i s_i=x}$
			\item Let $X\supset S$ be a convex set. Since $x\in S$, $x\in X$. Since $\coS=\intersect_{\alpha\in\Omega}X_\alpha$, where $\Omega$ is the set of all convex sets that contain $S$, ${x\in S\land x\in X\Rightarrow x\in\coS}$
			\item Thus, $x\in C\Rightarrow x\in\coS$
		\end{enumerate}
		$\therefore$ $C\subseteq\coS$
	
	\item Suppose $x\in\coS$
		\begin{enumerate}
			\item Any intersection of convex sets is also convex, so $\coS$ is convex, so $\exists y_1,...,y_m\in\coS$, $\lambda_1,...,\lambda_m\in[0,1]$, $\sum_{i=1}^m\lambda_i=1$ such that ${\sum_{i=1}^m\lambda_i y_i = x}$
			\item It is clearly apparent that $\coS\subseteq S$, so $y_1,...,y_m\in S$. Then, $x$ is a convex combination of elements of $S$. so $x\in C$
		\end{enumerate}
		$\therefore$ $\coS\subseteq C$
		
\end{enumerate}
$\therefore$ $C=\coS$ $\blacksquare$


%%%________________________________________________________________%%%

\section*{Question 3}
Suppose $X$ is convex.
\begin{enumerate}
	\item Let $x,y\in\cl{X}$ and suppose $\exists z=(1-\lambda)x + \lambda y$, $z\notin\cl{X}$
	\item If $x,y\in X$, then, since $X$ is convex, $(1-\lambda)x + \lambda y\in X$ $\forall \lambda$. Thus, $x,y\in X\Rightarrow z\in\cl{X}$
	\item If $x\in\cl{X}$, $x\notin X$, and $y\in X$, then $x$ is a limit point of $X$. Then, $\forall x'=(1-\lambda')x + \lambda'y$, $x'\in X$ or $x'=x$. Thus, either $z\in X$ or $z$ is a limit point of $x$. Thus, $z\in\cl{X}$.
	\item If $x,y\in\cl{X}$ and $x,y\notin X$, then both $x$ and $y$ are limit points of $X$. Thus, $\forall\varepsilon>0$, $\exists x'\in B_\varepsilon(x)$, $y'\in B_\varepsilon(y)$ such that $x'$ and $y'$ are both in $X$ and are convex combinations of $x$ and $y$. Then, either $z$ is equal to $x$ or $y$, or $\exists\varepsilon$ such that $x'\in B_\varepsilon(x)$, $y'\in B_\varepsilon(y)$, and $z=(1-\lambda')x'+\lambda'y'$ for some $\lambda'\in[0,1]$. Thus, $z\in\cl{X}$
\end{enumerate}
$\therefore$ by contradiction, $\cl{X}$ is convex $\blacksquare$


%%%________________________________________________________________%%%

\section*{Question 4}
\begin{enumerate}
	\item Let $f:X\rightarrow\R$ be a concave function where $X\subseteq\R^n$.
		\begin{enumerate}
			\item Let $x_1,x_2\in X$ and define $y_1=f(x_1)$ and $y_2=f(x_2)$. Then, ${z_1=(x_1,y_1)}$ and ${z_2=(x_2,y_2)}$ are both in $\hyp{f}$
			\item Since $X$ is convex and $f$ is concave, $\forall\lambda\in[0,1]$, ${f((1-\lambda)x_1+\lambda x_2)\geq (1-\lambda)f(x_1) + \lambda f(x_2)}$
			\item Thus, 
				\[
					(1-\lambda)x_1 + \lambda x_2\in X\text{ and } (1-\lambda)y_1 + \lambda y_2\leq f((1-\lambda)x_1+\lambda x_2)
				\]
				so $(1-\lambda)z_1 + \lambda z_2\in\hyp{f}$
		\end{enumerate}
		$\therefore$ $f$ concave $\Rightarrow$ $\hyp{f}$ convex
	
	\item Let $\hyp{f}$ be a convex hypograph of $f:X\rightarrow\R$, where $X\subseteq\R^n$.
		\begin{enumerate}
			\item Let $z_1,z_2\in\hyp{f}$. Then, $\forall\lambda\in[0,1]$, ${(1-\lambda)z_1 + \lambda z_2\in\hyp{f}}$. Then,
				\[
					(1-\lambda)x_1 + \lambda x_2 \in X\text{ and } (1-\lambda)y_1 + \lambda y_2\leq f((1-\lambda)x_1 + \lambda x_2)
				\]
				Thus, $X$ is a convex set and, $\forall x_1,x_2\in X$, $(1-\lambda)f(x_1) + \lambda f(x_2)$
				\[
					f((1-\lambda)x_1+\lambda x_2)\geq (1-\lambda)f(x_1) + \lambda f(x_2)
				\]
		\end{enumerate}
		$\therefore$ $\hyp{f}$ convex $\Rightarrow f$ concave
		
\end{enumerate}
$\therefore$ $f$ is convave if and only if $\hyp{f}$ is convex $\blacksquare$


%%%________________________________________________________________%%%

\section*{Question 5}
\begin{enumerate}
	\item Let $X$ and $Y$ be closed, convex sets, and let $X$ be compact
	\item Fix $y_0\in Y$ and define $A=\{y-x|x\in X\}$. Let $a_n$ be a sequence in $A$ such that $a_n = y_n-x_n$ for all $n$
	\item Since $X$ is compact, $\exists x_{n_k}\rightarrow x\in X$, and $y_n\rightarrow y\in\R^n$. Thus, $a_{n_k}\rightarrow a\in A$. Therefore, $A$ is closed and, given that $X$ and $Y$ are convex, $A$ is also convex. 
	\item Since $X$ and $Y$ are disjoint, $0\notin A$. Thus $\exists H(p,\beta)$ that strictly separates $A$ and $\{0\}$. Then, we can solve:
		\begin{align*}
			0<&\beta<p\cdot a	\\
			0<&\beta<p\cdot(y-x) \\
			0<&\beta<p\cdot y - p\cdot x \\
			p\cdot x < &\beta + p\cdot x < p\cdot y
		\end{align*}
\end{enumerate}
$\therefore$ $\exists H(p,\alpha)$ that strictly separates $X$ and $Y$ $\blacksquare$


%%%________________________________________________________________%%%

\section*{Question 6}
\begin{enumerate}
	\item Suppose $\exists x\in\R^n$ such that ${\infsum{A}{x_i}>0}$ and ${\infsum{B}{-x_i}>0}$
		\begin{enumerate}
			\item Further suppose $\exists \pi'\in\Pi_A\intersect\Pi_B$
			\item If ${\sum_{i=1}^n\pi_i'x_i>0}$, then ${\sum_{i=1}^n\pi_i'(-x_i)<0}$. If ${\sum_{i=1}^n\pi_i'(-x_i)>0}$, then ${\sum_{i=1}^n\pi_i'x_i<0}$
			\item Thus, $\exists \pi'\in\Pi_A\intersect\Pi_B\Rightarrow$$\nexists x\in\R^n$ such that ${\infsum{A}{x_i}>0}$ and ${\infsum{B}{(-x_i)}>0}$
		\end{enumerate}
		$\therefore$ By contradition, if $\exists x\in\R^n$ such that ${\infsum{A}{x_i}>0}$ and ${\infsum{B}{(-x_i)}>0}$, then ${\Pi_A\intersect\Pi_B=\emptyset}$
	
	\item Suppose ${\Pi_A\intersect\Pi_B=\emptyset}$, Therefore, $\Pi_A$ and $\Pi_B$ are disjoint.
		\begin{enumerate}
			\item Since $\Pi_A$ and $\Pi_B$ are disjoint and both compact, convex sets in $\R^n$, $A=\Pi_A-\Pi_B$ is a compact, convex set where $0\notin A$. Then, by the second theorem for a lecture,
				\[
					\exists H(x^*,\beta)\text{ s.t. } 0<\beta<a^*\cdot x^*
				\]
				Where ${a^*=\underset{\pi_A\in\Pi_A,\pi_B\in\Pi_B}{\text{inf}}\{\pi_A\cdot x^* - \pi_B\cdot x^*\}}$ Then, ${a^*=\pi_A^*\cdot x^* - \pi_B^*\cdot x^*}$, where:
				\[
					\pi_A^*=\underset{\pi_A\in\Pi_A}{\text{inf}}\{\pi_A\cdot x^*\}\text{, and }
					\pi_B^*=\underset{\pi_B\in\Pi_B}{\text{sup}}\{\pi_B\cdot x^*\}
				\]
				Then, we can solve $0<\beta<a^*\cdot x^*$ to derive ${\pi_B^*\cdot x^*<\beta+\pi_B^*\cdot x^*<\pi_A^*\cdot x^*}$. Define ${\alpha=\beta+\pi_B^*\cdot x^*}$ and $\overline{alpha}\in\R^n$ such that ${\overline{alpha}_i=\alpha}$ ${\forall i\in\{1,...,n\}}$. Thus, ${\forall\pi_A\in\Pi_A,\pi_B\in\Pi_B}$,
				\[
					\pi_B\cdot(x^*-\overline{\alpha})\leq\pi_B^*\cdot x^*-\alpha<0<\pi_A^*\cdot x^*-\alpha\leq\pi_A\cdot(x^*-\overline{\alpha})
				\]
				So $(x^*-\overline{\alpha})$ is an agreeable trade.
				
		\end{enumerate}
		$\therefore$ ${\Pi_A\intersect\Pi_B=\emptyset\Rightarrow}$ $\exists x\in\R^n$ such that ${\infsum{A}{x_i}>0}$ and ${\infsum{B}{-x_i}>0}$
		
\end{enumerate}
$\therefore$ There is an agreeable trade if and only if there is no common prior $\blacksquare$




%%%________________________________________________________________%%%



\end{document}
















