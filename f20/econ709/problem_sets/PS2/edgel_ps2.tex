%%% Econ709: Econometrics
%%% Fall 2020
%%% Danny Edgel
%%%
% Due on Canvas Monday September 21, 11:59pm Central Time
%%%

%%%
%							PREAMBLE
%%%

\documentclass{article}

%%% declare packages
\usepackage{amsmath}
\usepackage{amssymb}
\usepackage{array}
\usepackage{bm}
\usepackage{changepage}
\usepackage{centernot}
\usepackage{graphicx}
\usepackage{fancyhdr}
	\fancyhf{} % sets both header and footer to nothing
	\renewcommand{\headrulewidth}{0pt}
    \rfoot{Edgel, \thepage}
    \pagestyle{fancy}
	
%%% define shortcuts for set notation
\newcommand{\N}{\mathbb{N}}
\newcommand{\Z}{\mathbb{Z}}
\newcommand{\R}{\mathbb{R}}
\newcommand{\Q}{\mathbb{Q}}
\newcommand{\union}{\bigcup}
\newcommand{\intersect}{\bigcap}
\newcommand{\lmt}{\underset{x\rightarrow\infty}{\text{lim }}}
\newcommand{\neglmt}{\underset{x\rightarrow-\infty}{\text{lim }}}
\newcommand{\zerolmt}{\underset{x\rightarrow 0}{\text{lim }}}
\newcommand{\usmax}{\underset{1\leq k \leq n}{\text{max }}}
\newcommand{\intinf}{\int_{-\infty}^{\infty}}

%%% define column vector command (from Michael Nattinger)
\newcount\colveccount
\newcommand*\colvec[1]{
        \global\colveccount#1
        \begin{pmatrix}
        \colvecnext
}
\def\colvecnext#1{
        #1
        \global\advance\colveccount-1
        \ifnum\colveccount>0
                \\
                \expandafter\colvecnext
        \else
                \end{pmatrix}
        \fi
}

\makeatletter
\let\amsmath@bigm\bigm

\renewcommand{\bigm}[1]{%
  \ifcsname fenced@\string#1\endcsname
    \expandafter\@firstoftwo
  \else
    \expandafter\@secondoftwo
  \fi
  {\expandafter\amsmath@bigm\csname fenced@\string#1\endcsname}%
  {\amsmath@bigm#1}%
}


%________________________________________________________________%

\begin{document}

\title{	Problem Set \#2 }
\author{ 	Danny Edgel 										\\ 
			Econ 709: Economic Statistics and Econometrics I	\\
			Fall 2020											\\
		}
\maketitle\thispagestyle{empty}

%%%________________________________________________________________%%%

\noindent\textit{Collaborated with Sarah Bass, Emily Case, Michael Nattinger, and Alex Von Hafften}

%%%________________________________________________________________%%%

\section*{Question 1}
\textbf{Suppose that $Y=X^3$ and $f_X(x)=42x^5(1-x)$, $x\in(0,1)$. Find the PDF of $Y$, and show that the PDF integrates to 1.}
\bigskip \\
We know that the CDF of $Y$, $F_Y(y)$ is equal to $F_X(f^{-1}(x))$. So we can solve for the PDF of $Y$ by first finding its CDF:
\begin{align*}
	f^{-1}(y)		&= \sqrt[3]{y}																			\\
	F_X(x)			&=\int_0^x 42t^5(1-t)dt=42\int_0^x t^5-t^6dt=42(\frac{1}{6}x^6-\frac{1}{7}x^7)						\\
	F_X(f^{-1}(y)) 	&= 42(\frac{1}{6}y^{6/3}-\frac{1}{7}y^{7/3})=42y^2(\frac{1}{6}-\sqrt[3]{y}) = F_Y(y)	\\
	f_Y(y)			&= \frac{d}{dy} F_Y(y) = 14y-14y\sqrt[3]{y}												\\
\end{align*}
Since we already know $F_Y(y)$, we can easily show that the PDF of $Y$ integrates to $1$:
\[
	\int_0^1 f_Y(y)dy = F_Y(1)-F_Y(0) = 1^2(7-6\sqrt[3]{1}) - 0 = 7-6 = 1
\]

%%%________________________________________________________________%%%

\section*{Question 2}
\textbf{Consider the CDF $F_X(x)=\begin{cases}1.2x&\text{   if } x\in[0,0.5) \\ 0.2+0.8x&\text{   if } x\in[0.5,1] \end{cases}$, and the function}
\[
	f_X(x)=\begin{cases}1.2&\text{   if } x\in[0,0.5) \\ a&\text{   if } x=0.5 \\ 0.8&\text{   if } x\in(0.5,1]\end{cases}
\]
\smallskip \\
\textbf{Show that $f_X$ is the density function of $F_X$ as long as $a\geq0$. That is, show that for all $x\in[0,1]$, $F_X(x)=\int_0^x f_X(t)dt$.}
\bigskip \\
We can define $F_X(x)=\int_0^x f_X(t)dt$ on a case-by-case basis:
\begin{align*}
	x\in[0,0.5): 	\int_0^x f_X(t)dt	&=\int_0^x 1.2dt = [1.2t]_0^x = 1.2x																\\
	x=0.5:			\int_0^x f_X(t)dt	&=\int0^0.5 1.2dt + \int_0.5^x a dt = [1.2t]^0.5_0 + [at]^x_0.5 									\\
										&= 1.2(0.5) - 0 + ax - 0.5a	=0.6 + 0.5x - 0.5x = 0.6												\\
	x\in(0.5,1):	\int_0^x f_X(t)dt	&=\int0^0.5 1.2dt + \int_0.5^x a dt + \int_0.5^x 0.8dt = [1.2t]^0.5_0 + [at]^0.5_0.5 +[0.8]^x_0.5	\\
										&= 0.6 + 0.8x - 0.4 = 0.8x + 0.2
\end{align*}
Since $0.6=1.2x$ when $x=0.5$, then $\int_0^x 1.2dt=\int0^0.5 1.2dt + \int_0.5^x a dt$ when $x=0.5$. Thus,  $\forall x\in[0,1]$, $F_X(x)=\int_0^x f_X(t)dt$.


%%%________________________________________________________________%%%

\section*{Question 3}
\textbf{Let $X$ have the PDF $F_X(x)=\frac{2}{9}(x+1)$, $x\in[-1,2]$. Find the PDF of $Y=X^2$. Note that this is a bit different from the exercise in the lecture note.}
\bigskip \\
Given $f_X(x)$ and $Y=g(X)=X^3$, we can solve:
\[
	f_Y(y) = \begin{cases} f_X\left( g^{-1}(y) \right)|\frac{d}{dy}g^{-1}(y)|, &y\in Y \\ 0, &y\notin Y \end{cases}
\]
Solving for each element of the first case of $f_Y(y)$ yields:
\begin{align*}
	f_X\left( g^{-1}(y) \right)			&= \frac{2}{9}			\\
	\frac{d}{dy}g^{-1}(y)=1/(2\sqrt{y})	&= \frac{1}{2\sqrt{y}}	
\end{align*}
Since $x\in[-1,2]$ and $\frac{d}{dy}g^{-1}(y)=1/(2\sqrt{y})$, our first case of $f_Y(y)$ is defined on $y\in(0,4]$. Thus, 
\[
	f_Y(y) = \begin{cases} \frac{1}{18\sqrt{y}}, &y\in (0,4] \\ 0, &y\leq 0 \lor y>4 \end{cases}
\]


%%%________________________________________________________________%%%

\section*{Question 4}
\textbf{A median of a distribution is a value $m$ such that $P(X\leq m)\geq\frac{1}{2}$ and $P(X\geq m)\geq\frac{1}{2}$. Find the median of the distribution $f(x)=1/\pi(1+x^2)$, $x\in\R$.}
\bigskip \\
Given that $P(X\leq x)=F_X(x)$, we need to find $m$ such that $\frac{1}{2}\leq F_X(m)\leq\frac{1}{2}$:
\begin{align*}
	\frac{1}{2} 			\leq &\frac{1}{\sqrt{\pi}}\int_{-\infty}^m f_X(x) 	\leq \frac{1}{2} 				\\
	\frac{\sqrt{\pi}}{2} 	\leq &\int_{-\infty}^m \frac{1}{\pi}(1+t^2)dt 		\leq \frac{\sqrt{\pi}}{2} 		\\
	\frac{\sqrt{\pi}}{2} 	\leq &[\text{arctan}(x/\sqrt{\pi})]^m_{-\infty}		\leq \frac{\sqrt{\pi}}{2} 		\\
	\frac{\sqrt{\pi}}{2} 	\leq &(\text{arctan}(m/\sqrt{\pi})-1)				\leq \frac{\sqrt{\pi}}{2}		\\
	\text{tan}(\sqrt{\pi}/2)\leq &(m/\sqrt{\pi})-1								\leq \text{tan}(\sqrt{\pi}/2)	
\end{align*}
$\therefore$ $m = \sqrt{\pi}(1 + \text{tan}(\sqrt{\pi}/2))$	$\blacksquare$


%%%________________________________________________________________%%%

\section*{Question 5}
\textbf{Show that if $X$ is a continuous random variable, then $\underset{a}{\text{min}}E|X-a|=E|X-m|$, where $m$ is the median of $X$.}
\bigskip \\
We simply need to find $a$ that minimizes $E(X-a)$. Then:
\begin{align*}
	E(X-a)				&=  \intinf f_X(t)|t-a|dt = \int_{-\infty}^a f_X(t)(a-t)dt + \int_a^\infty f_X(t)(t-a)dt	\\
	\frac{d}{da}E(X-a)	&= \int_{-\infty}^a f_X(t)dt - \int_a^\infty f_X(t)dt = F_X(a) - (1 - F_X(a)) = 0			\\
	F_X(a)				&= \frac{1}{2}
\end{align*}
$\therefore$ $a=m$	$\blacksquare$


%%%________________________________________________________________%%%

\section*{Question 6}
\textbf{Let $\mu_n$ denote the $n$th central moment of a random variable $X$. Two quantities of interest, in addition to the mean and variance are}
\[
	\alpha_3 = \dfrac{\mu_3}{\mu_2^{3/2}}\text{ and }\alpha_4=\dfrac{\mu_4}{\mu_2^2}
\]
\textbf{The value $\alpha_3$ is called the skewness and $\alpha_4$ is called the kurtosis. The skewness measures the lack of symmetry in the density function. The kurtosis measures the peakedness or flatness of the density function.}
\begin{enumerate}
	\item \textbf{Show that if a density function is symmetric about a point $a$, then $\alpha_3=0$}
		\bigskip \\
		Let $X$ be a continuous random variable that is symmetrically distributed about some point $a$. Thus, $E(X)=a$. Now, define $Y=X-a$ and $g(y)=y^3$. Then, we can derive:
		\begin{align*}
			\mu_3 	&= E(Y^3) 																\\
					&= \int_{-\infty}^a f_X(y)g(y)dy 	+ \int^{\infty}_a f_X(y)g(y)dy		\\
					&= \int^{\infty}_a f_X(-y)g(-y)dy	+ \int^{\infty}_a f_X(y)g(y)dy		\\
					&= -\int^{\infty}_a f_X(y)g(y)dy	+ \int^{\infty}_a f_X(y)g(y)dy		\\
			\mu_3 	&= 0
		\end{align*}
		Therefore, $\alpha_3 = \frac{\mu_3}{\mu_2^{3/2}} = 0$.
		
	\item \textbf{Calculate $\alpha_3$ for $f(x)=e^{-x}$, $x\geq0$, a density function that is skewed to the right.}
		\bigskip \\
		First, we must solve for the mean and second and third central moments:
		\begin{align*}
			E(X)	&= \int^\infty_0 e^{-t}tdt = [-te^{-t}]^\infty_0-\int^\infty_0 e^{-t}dt = -0+0-[e^{-x}]^\infty_0 = 1	\\
			\mu_2	&= E(X^2) - 1^2 = \int^\infty_0 e^{-t}t^2dt-1 = -2\int^\infty_0 e^{-t}tdt-1 = 2-1 = 1					\\
			\mu_3	&= \int_0^\infty e^{-t}(t-1)^3dt = \int_0^\infty e^{-t}(t^3-3t^2+3t-1)dt = E(X^3) - 3E(X^2) + 3E(X) - 1	\\
			E(X^3)	&= \int^\infty_0 e^{-t}t^3dt = 0 - 3\int_0^\infty t^2e^{-t}dt = 3(2)=6\\
			\mu_3	&= 6 - 3(2)+3-1 = 2
		\end{align*}
		Thus,
		\[
			\alpha_3 = \dfrac{\mu_3}{\mu_2^{3/2}} = \dfrac{2}{1}=2
		\]
		
	\item \textbf{Calculate $\alpha_4$ for the following density functions and comment on the peakedness of each:}
		\begin{enumerate}
			\item $\mathbf{f(x)=\frac{1}{\sqrt{2\pi}}e^{-x^2/2}\text{, }x\in\R}$
				\bigskip \\
				$f$ is simply the normal distribution, so we know that it has moment-generating function $M(t)=e^{t^2/2}$. Then, we can calculate the following derivatives of $M$:
				\begin{align*}
					M'(t) 		&= -te^{t^2/2}					\\
					M^{(2)}(t)	&= (t^2 - 1)e^{t^2/2}			\\
					M^{(3)}(t)	&= (3t-t^3)e^{t^2/2} 			\\
					M^{(4)}(t)	&= (2t^4-3t^2-6t+3)e^{t^2/2}
				\end{align*}
				Using $\mu_k=M^{(k)}(0)$, we can solve for $\alpha_4$:
				\[
					\alpha_4 = \dfrac{\mu_4}{\mu_2^2} = \dfrac{M^{(4)}(0)}{(M^{(2)}(0))^2} = \dfrac{3}{(-1)^2} = 3
				\]
				A kurtosis of 3 is very low, indicating that the tails of the distribution are fairly skinny, with most of the data falling near the mean, where there is a ``peak". 
			
			\item $\mathbf{f(x)=1/2\text{, }x\in(-1,1)}$
				\bigskip \\
				We can begin by finding $E(X)$, $\mu_2$, and $\mu_4$. This distribution is symmetric about $0$, so $E(X)=0$. Then,
				\begin{align*}
					\mu_2 &= \int^1_{-1} \frac{1}{2}t^2dt = [\frac{1}{6}t^3]^1_{-1} = \frac{1}{6}(1+1) = \frac{1}{3}	\\
					\mu_4 &= \int^1_{-1} \frac{1}{2}t^4dt = [\frac{1}{10}t^5]^1_{-1} = \frac{1}{10}(1+1) = \frac{1}{5}
				\end{align*}
				Thus,
				\[
					\alpha_4 = \dfrac{\mu_4}{\mu_2^2} = \dfrac{1/5}{(1/3)^2} = \dfrac{9}{5}
				\]
				The kurtosis of this function is even smaller than that of the normal distribution, indicating a high peakedness of the distribution, even though this distribution is flat. This is because the values only fall from $-1$ to $1$, wheras most others fall in an infinite range. The density from $-1$ to $1$, then, can be seen as a sharp, steep peak with a flat top.
				
			\item $\mathbf{f(x)=\frac{1}{2}e^{-|x|}\text{, }x\in\R}$
				\bigskip \\
				This distribution is clearly symmetric about $0$, so $E(X)=0$. Then, 
				\begin{align*}
					\mu_2 	&= \int_{-\infty}^0\frac{1}{2} e^tt^2dt + \int_0^\infty\frac{1}{2} e^{-t}t^2dt 											\\ 
							&= [\frac{1}{2}t^2e^t]^0_{-\infty} + \int_{-\infty}^0te^t + [\frac{1}{2}t^2e^{-t}]_0^\infty + \int_0^\infty te^{-t}		\\
							&= [te^t]^0_{-\infty} + \int_{-\infty}^0e^t + [te^{-t}]_0^\infty + \int_0^\infty e^{-t}									\\
							&= [e^t]^0_{-\infty} - [e^{-t}]_0^\infty = 1 - 0 - 0 + 1																\\
					\mu_2	&= 2
				\end{align*}
				And, simplyfying intermediate steps where the definite integral is known to equal zero by the above steps:
				\begin{align*}
					\mu_4 	&= \int_{-\infty}^0\frac{1}{2} e^tt^4dt + \int_0^\infty\frac{1}{2} e^{-t}t^4dt 	\\
							&= 0 + \int_{-\infty}^02 e^tt^3dt - 0 + \int_0^\infty2 e^{-t}t^3dt				\\
							&= 0 + \int_{-\infty}^06 e^tt^2dt - 0 + \int_0^\infty6 e^{-t}t^2dt				\\
							&= 0 + \int_{-\infty}^012 e^ttdt - 0 + \int_0^\infty12 e^{-t}tdt				\\
							&= 0 + \int_{-\infty}^012 e^tdt - 0 + \int_0^\infty12 e^{-t}dt					\\
							&= [12e^t]^0_{-\infty} - [12e^{-t}]^\infty_0 = 12 - 0 - 0 + 12					\\
					\mu_4 	&= 24
				\end{align*}
				Thus,
				\[
					\alpha_4 = \dfrac{\mu_4}{\mu_2^2} = \dfrac{24}{2^2} = 6
				\]
				Therefore, this distribution is half as peaked as the normal distribution.
				
		\end{enumerate}
		
		
\end{enumerate}

%%%________________________________________________________________%%%


\end{document}












