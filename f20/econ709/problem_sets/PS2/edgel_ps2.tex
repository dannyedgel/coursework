%%% Econ709: Econometrics
%%% Fall 2020
%%% Danny Edgel
%%%
% Due on Canvas Monday September 21, 11:59pm Central Time
%%%

%%%
%							PREAMBLE
%%%

\documentclass{article}

%%% declare packages
\usepackage{amsmath}
\usepackage{amssymb}
\usepackage{array}
\usepackage{bm}
\usepackage{changepage}
\usepackage{centernot}
\usepackage{graphicx}
\usepackage{fancyhdr}
	\fancyhf{} % sets both header and footer to nothing
	\renewcommand{\headrulewidth}{0pt}
    \rfoot{Edgel, \thepage}
    \pagestyle{fancy}
	
%%% define shortcuts for set notation
\newcommand{\N}{\mathbb{N}}
\newcommand{\Z}{\mathbb{Z}}
\newcommand{\R}{\mathbb{R}}
\newcommand{\Q}{\mathbb{Q}}
\newcommand{\union}{\bigcup}
\newcommand{\intersect}{\bigcap}
\newcommand{\lmt}{\underset{x\rightarrow\infty}{\text{lim }}}
\newcommand{\neglmt}{\underset{x\rightarrow-\infty}{\text{lim }}}
\newcommand{\zerolmt}{\underset{x\rightarrow 0}{\text{lim }}}
\newcommand{\usmax}{\underset{1\leq k \leq n}{\text{max }}}

%%% define column vector command (from Michael Nattinger)
\newcount\colveccount
\newcommand*\colvec[1]{
        \global\colveccount#1
        \begin{pmatrix}
        \colvecnext
}
\def\colvecnext#1{
        #1
        \global\advance\colveccount-1
        \ifnum\colveccount>0
                \\
                \expandafter\colvecnext
        \else
                \end{pmatrix}
        \fi
}

\makeatletter
\let\amsmath@bigm\bigm

\renewcommand{\bigm}[1]{%
  \ifcsname fenced@\string#1\endcsname
    \expandafter\@firstoftwo
  \else
    \expandafter\@secondoftwo
  \fi
  {\expandafter\amsmath@bigm\csname fenced@\string#1\endcsname}%
  {\amsmath@bigm#1}%
}


%________________________________________________________________%

\begin{document}

\title{	Problem Set \#1 }
\author{ 	Danny Edgel 										\\ 
			Econ 709: Economic Statistics and Econometrics I	\\
			Fall 2020											\\
		}
\maketitle\thispagestyle{empty}

%%%________________________________________________________________%%%

\section*{Question 1}
\textbf{Suppose that $Y=X^3$ and $f_X(x)=42x^5(1-x)$, $x\in(0,1)$. Find the PDF of $Y$, and show that the PDF integrates to 1.}
\bigskip \\


%%%________________________________________________________________%%%

\section*{Question 2}
\textbf{Consider the CDF $F_X(x)=\begin{cases}1.2x&\text{   if } x\in[0,0.5) \\ 0.2+0.8x&\text{   if } x\in[0.5,1] \end{cases}$, and the function}
\[
	f_X(x)=\begin{cases}1.2&\text{   if } x\in[0,0.5) \\ a&\text{   if } x=0.5 \\ 0.8&\text{   if } x\in[0.5,1]\end{cases}
\]
\smallskip \\
\textbf{Show that $f_X$ is the density function of $F_X$ as long as $a\geq0$. That is, show that for all $x\in[0,1]$, $F_X(x)=\int_0^x f_X(t)dt$.}
\bigskip \\



%%%________________________________________________________________%%%

\section*{Question 3}
\textbf{Let $X$ have the PDF $F_X(x)=\frac{2}{9}(x+1)$, $x\in[-1,2]$. Find the PDF of $Y=X^2$. Note that this is a bit different from the exercise in the lecture note.}
\bigskip \\



%%%________________________________________________________________%%%

\section*{Question 4}
\textbf{A median of a distribution is a value $m$ such that $P(X\leq m)\geq\frac{1}{2}$ and $P(X\geq m)\geq\frac{1}{2}$. Find the median of the distribution $f(x)=\frac{1}{\pi(1+x^2)}$, $x\in\R$.}
\bigskip \\



%%%________________________________________________________________%%%

\section*{Question 5}
\textbf{Show that if $X$ is a continuous random variable, then $\underset{a}{\text{min}}E|X-a|=E|X-m|$, where $m$ is the median of $X$.}
\bigskip \\
(hint: work out the integral expression of $E|X-a|$ and notice that it is differentiable)


%%%________________________________________________________________%%%

\section*{Question 6}
\textbf{Let $\mu_n$ denote the $n$th central moment of a random variable $X$. Two quantities of interest, in addition to the mean and variance are}
\[
	\alpha_3 = \dfrac{\mu_3}{\mu_2^{3/2}}\text{ and }\alpha_4=\dfrac{\mu_4}{\mu_2^2}
\]
\textbf{The value $\alpha_3$ is called the skewness and $\alpha_4$ is called the kurtosis. The skewness measures the lack of symmetry in the density function. The kurtosis measures the peakedness or flatness of the density function.}
\begin{enumerate}
	\item \textbf{Show that if a density function is symmetric about a point $a$, then $\alpha_3=0$}
		\bigskip \\
		
		
	\item \textbf{Calculate $\alpha_3$ for $f(x)=e^{-x}$, $x\geq0$, a density function that is skewed to the right.}
		\bigskip \\
	
	
	\item \textbf{Calculate $\alpha_4$ for the following density functions and commend on the peakedness of each:}
		\begin{enumerate}
			\item $\mathbf{f(x)=\frac{1}{\sqrt{2\pi}}e^{-x^2/2}\text{, }x\in\R}$
				\bigskip \\
			
			\item $\mathbf{f(x)=1/2\text{, }x\in(-1,1)}$
				\bigskip \\
			\item $\mathbf{f(x)=\frac{1}{2}e^{-|x|}\text{, }x\in\R}$
				\bigskip \\
		\end{enumerate}
		
		
\end{enumerate}

%%%________________________________________________________________%%%


\end{document}












