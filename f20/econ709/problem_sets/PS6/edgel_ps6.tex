%%% Econ709: Econometrics
%%% Fall 2020
%%% Danny Edgel
%%%
% Due on Canvas Sunday, October 18th, 11:59pm Central Time
%%%

%%%
%							PREAMBLE
%%%

\documentclass{article}

%%% declare packages
\usepackage{amsmath}
\usepackage{amssymb}
\usepackage{array}
\usepackage{bm}
\usepackage{bbm}
\usepackage{changepage}
\usepackage{centernot}
\usepackage{graphicx}
\usepackage[shortlabels]{enumitem}
\usepackage{fancyhdr}
	\fancyhf{} % sets both header and footer to nothing
	\renewcommand{\headrulewidth}{0pt}
    \rfoot{Edgel, \thepage}
    \pagestyle{fancy}
	
%%% define shortcuts for set notation
\newcommand{\N}{\mathcal{N}}
\newcommand{\Z}{\mathbb{Z}}
\newcommand{\R}{\mathbb{R}}
\newcommand{\Q}{\mathbb{Q}}
\newcommand{\union}{\bigcup}
\newcommand{\intersect}{\bigcap}
\newcommand{\lmt}{\underset{x\rightarrow\infty}{\text{lim }}}
\newcommand{\neglmt}{\underset{x\rightarrow-\infty}{\text{lim }}}
\newcommand{\zerolmt}{\underset{x\rightarrow 0}{\text{lim }}}
\newcommand{\usmax}{\underset{1\leq k \leq n}{\text{max }}}
\newcommand{\intinf}{\int_{-\infty}^{\infty}}
\newcommand{\olx}[1]{\overline{X}_{#1}}
\newcommand{\oly}[1]{\overline{Y}_{#1}}
\newcommand{\est}[1]{\frac{1}{#1}\sum_{i=1}^{#1}}
\newcommand{\sumn}{\sum_{i=1}^{n}}
\newcommand{\loge}[1]{\text{log}\left(#1\right)}

%%% define column vector command (from Michael Nattinger)
\newcount\colveccount
\newcommand*\colvec[1]{
        \global\colveccount#1
        \begin{pmatrix}
        \colvecnext
}
\def\colvecnext#1{
        #1
        \global\advance\colveccount-1
        \ifnum\colveccount>0
                \\
                \expandafter\colvecnext
        \else
                \end{pmatrix}
        \fi
}

\makeatletter
\let\amsmath@bigm\bigm

\renewcommand{\bigm}[1]{%
  \ifcsname fenced@\string#1\endcsname
    \expandafter\@firstoftwo
  \else
    \expandafter\@secondoftwo
  \fi
  {\expandafter\amsmath@bigm\csname fenced@\string#1\endcsname}%
  {\amsmath@bigm#1}%
}


%________________________________________________________________%

\begin{document}

\title{	Problem Set \#6 }
\author{ 	Danny Edgel 										\\ 
			Econ 709: Economic Statistics and Econometrics I	\\
			Fall 2020											\\
		}
\maketitle\thispagestyle{empty}

%%%________________________________________________________________%%%

\noindent\textit{Collaborated with Sarah Bass, Emily Case, Michael Nattinger, and Alex Von Hafften}
%%%________________________________________________________________%%%

\section*{Question 1}
Say $P(X=1)=p$ and $P(X=0)=1-p$, where $0<p<1$.
\begin{itemize}
	\item[(a)] Say $f(x)=p^x(1-p)^{1-x}$. Then,
		\begin{align*}
			f(0) &= p^0(1-p)^{1-0} = 1-p = P(X=0)	\\
			f(1) &= p^1(1-p)^{1-1} = p = P(X=1)
		\end{align*}
	\item[(b)] 
		\[
			\ell_n = \sum_{i=1}^n \loge{f(x_i)} = \sum_{i=1}^nx_i\loge{p} + (1-x_i)\loge{1-p} = n\loge{p} + \loge{1-p}\sumn1-x_i
		\]
	\item[(c)] To find $\hat{p}$, we simply maximize $\ell_n$ with repspect to $p$:
		\begin{align*}
			\frac{\partial\ell_n}{\partial p} &= \frac{1}{p}\sumn x_i - \frac{1}{1-p}\sumn 1-x_i = 0	\\
			\frac{n}{p}\overline{X}_n &= \frac{n}{1-p} - \frac{n}{1-p}\overline{X}_n	\\
			\frac{p-1}{p}\olx{n} &= 1 - \olx{n} \\
			\left(\frac{p-1}{p}+1\right)\olx{n} &= 1 \\
			\frac{1}{p}\olx{n} &= 1 \\
			\hat{p}_n &= \olx{n}
		\end{align*}
\end{itemize}


%%%________________________________________________________________%%%

\section*{Question 2}
$X\sim f(x)=\frac{\alpha}{x^{1+\alpha}}$, $x\geq 1$
\begin{enumerate}[(a)]
	\item The log-likelihood function is:
		\[
			\ell_n = \sum_{i=1}^n\loge{f(x_i)} = \sumn\loge{\alpha}-(1+\alpha)\loge{x_i} = n\loge{\alpha} - (1+\alpha)\sumn\loge{x_i}
		\]
		
	\item To find $\hat{\alpha}$, we simply maximize $\ell_n$ with repspect to $\alpha$:
		\begin{align*}
			\frac{\partial\ell_n}{\partial\alpha} &= \frac{n}{\alpha} - \sumn\loge{x_i} = 0 \\
			\frac{n}{\hat{\alpha}} &= \sumn\loge{x_i}	\\
			\hat{\alpha}_n^{-1} &= \frac{1}{n}\sumn\loge{x_i}
		\end{align*}
\end{enumerate}


%%%________________________________________________________________%%%

\section*{Question 3}
$X\sim f(x)=\left[\pi(1+(x-\theta)^2)\right]^{-1}$, $x\in\R$
\begin{enumerate}[(a)]
	\item The log-likelihood function is:
		\[
			\ell_n = \sum_{i=1}^n\loge{f(x_i)} = \sumn\loge{\pi}+\loge{1+(x_i-\theta)^2} = -n\loge{\pi}-\sumn\loge{1+(x_i-\theta)^2}
		\]
		
	\item The first-order condition for the MLE $\hat{\theta}$ is:
		\[
			\frac{\partial\ell_n}{\partial\theta} = \sumn\frac{2(x_i-\hat{\theta}_n)}{1+(x_i-\hat{\theta}_n)}=0
		\]
\end{enumerate}


%%%________________________________________________________________%%%

\section*{Question 4}
$X\sim f(x)=\frac{1}{2}\text{exp}(-|x-\theta|)$, $x\in\R$
\begin{enumerate}[(a)]
	\item The log-likelihood function is:
		\[
			\ell_n = \sum_{i=1}^n\loge{f(x_i)} = \sumn\loge{\frac{1}{2}}-|x_i-\theta| = n\loge{\frac{1}{2}}-\sumn-|x_i-\theta|
		\]
		
	\item The MLE will be $\hat{\theta}_n$ that minimizes $\sumn|x_i-\hat{\theta}_n|$, so we want to choose theta that will minimize the sum of the absolute deviations from $X_i$. We already know that this value is $\est{n}x_i=\olx{n}$. Thus,
		\[
			\hat{\theta}_n = \est{n}X_i
		\]
\end{enumerate}


%%%________________________________________________________________%%%

\section*{Question 5}


%%%________________________________________________________________%%%

\section*{Question 6}


%%%________________________________________________________________%%%

\section*{Question 7}


%%%________________________________________________________________%%%

\section*{Question 8}



%%%________________________________________________________________%%%




\end{document}












