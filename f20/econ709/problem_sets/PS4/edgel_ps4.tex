%%% Econ709: Econometrics
%%% Fall 2020
%%% Danny Edgel
%%%
% Due on Canvas Sunday, October 4th, 11:59pm Central Time
%%%

%%%
%							PREAMBLE
%%%

\documentclass{article}

%%% declare packages
\usepackage{amsmath}
\usepackage{amssymb}
\usepackage{array}
\usepackage{bm}
\usepackage{changepage}
\usepackage{centernot}
\usepackage{graphicx}
\usepackage{fancyhdr}
	\fancyhf{} % sets both header and footer to nothing
	\renewcommand{\headrulewidth}{0pt}
    \rfoot{Edgel, \thepage}
    \pagestyle{fancy}
	
%%% define shortcuts for set notation
\newcommand{\N}{\mathcal{N}}
\newcommand{\Z}{\mathbb{Z}}
\newcommand{\R}{\mathbb{R}}
\newcommand{\Q}{\mathbb{Q}}
\newcommand{\union}{\bigcup}
\newcommand{\intersect}{\bigcap}
\newcommand{\lmt}{\underset{x\rightarrow\infty}{\text{lim }}}
\newcommand{\neglmt}{\underset{x\rightarrow-\infty}{\text{lim }}}
\newcommand{\zerolmt}{\underset{x\rightarrow 0}{\text{lim }}}
\newcommand{\usmax}{\underset{1\leq k \leq n}{\text{max }}}
\newcommand{\intinf}{\int_{-\infty}^{\infty}}
\newcommand{\olx}[1]{\overline{X}_{#1}}

%%% define column vector command (from Michael Nattinger)
\newcount\colveccount
\newcommand*\colvec[1]{
        \global\colveccount#1
        \begin{pmatrix}
        \colvecnext
}
\def\colvecnext#1{
        #1
        \global\advance\colveccount-1
        \ifnum\colveccount>0
                \\
                \expandafter\colvecnext
        \else
                \end{pmatrix}
        \fi
}

\makeatletter
\let\amsmath@bigm\bigm

\renewcommand{\bigm}[1]{%
  \ifcsname fenced@\string#1\endcsname
    \expandafter\@firstoftwo
  \else
    \expandafter\@secondoftwo
  \fi
  {\expandafter\amsmath@bigm\csname fenced@\string#1\endcsname}%
  {\amsmath@bigm#1}%
}


%________________________________________________________________%

\begin{document}

\title{	Problem Set \#4 }
\author{ 	Danny Edgel 										\\ 
			Econ 709: Economic Statistics and Econometrics I	\\
			Fall 2020											\\
		}
\maketitle\thispagestyle{empty}

%%%________________________________________________________________%%%

\noindent\textit{Collaborated with Sarah Bass, Emily Case, Michael Nattinger, and Alex Von Hafften}
%%%________________________________________________________________%%%

\section*{Question 1}
\textbf{Suppose that another observation $X_{n+1}$ becomes available. Show that:}

\begin{itemize}
	\item[(a)] $\mathbf{\overline{X}_{n+1}=(n\overline{X}_n + X_{n+1})/(n+1)}$ \\
		\begin{align*}
			\overline{X}_{n+1} 	&= \frac{1}{n+1}\sum_{i=1}^{n+1}X_i 	\\
								&= \frac{1}{n+1}\left(\sum_{i=1}^nX_i + X_{n+1}\right)	\\
								&= \frac{1}{n+1}\left(n\overline{X}_n + X_{n+1}\right)
		\end{align*}
	
	\item[(b)] $\mathbf{s^2_{n+1}=\frac{1}{n}((n-1)s_n^2+(n/(n+1))(X_{n+1}-\overline{X}_n)^2)}$ \\
	
		Using the relation from (a), we can derive:
		\begin{align*}
			s^2_{n+1} 	&= \frac{1}{n}\sum_{i=1}^{n+1}(X_i-\overline{X}_{n+1})^2 	\\
						&= \frac{1}{n}\sum_{i=1}^{n+1}\left((X_i-\olx{n})+(\olx{n}-\olx{n+1})\right)^2	\\
						&= \frac{1}{n}\sum_{i=1}^{n+1}\left[(X_i-\olx{n})^2+2(X_i-\olx{n})(\olx{n}-\olx{n+1})+(\olx{n}-\olx{n+1})^2\right]	\\
						&= \frac{1}{n}\left[\sum_{i=1}^{n}(X_i-\olx{n})^2 + (X_{n+1}-\olx{n})^2 +2(\olx{n}-\olx{n+1})\sum_{i=1}^{n+1}(X_i-\olx{n})
						+\sum_{i=1}^{n+1}(\olx{n}-\olx{n+1})^2\right]	\\
						&= \frac{1}{n}\left[(n-1)s^2_n + (X_{n+1}-\olx{n})^2 +2(n+1)(\olx{n}-\olx{n+1})(\olx{n+1}-\olx{n}) +(n+1)(\olx{n}-\olx{n+1})^2\right]	\\
						&= \frac{1}{n}\left[(n-1)s^2_n + (X_{n+1}-\olx{n})^2 -2(n+1)(\olx{n}-\olx{n+1})^2 +(n+1)(\olx{n}-\olx{n+1})^2\right]	\\
						&= \frac{1}{n}\left[(n-1)s^2_n + (X_{n+1}-\olx{n})^2 -(n+1)(\olx{n}-\olx{n+1})^2 \right]	\\
						&= \frac{1}{n}\left[(n-1)s^2_n + (X_{n+1}-\olx{n})^2 -(n+1)\left(\olx{n}-\frac{1}{n+1}(n\olx{n}+X_{n+1})\right)^2 \right]	\\
						&= \frac{1}{n}\left[(n-1)s^2_n + (X_{n+1}-\olx{n})^2 -(n+1)\left(\frac{1}{n+1}\olx{n}-\frac{1}{n+1}X_{n+1})\right)^2 \right]	\\
						&= \frac{1}{n}\left[(n-1)s^2_n + (X_{n+1}-\olx{n})^2 -(n+1)\left(-\frac{1}{n+1}\right)^2\left(X_{n+1}-\olx{n})\right)^2 \right]	\\
						&= \frac{1}{n}\left[(n-1)s^2_n + \left(1-\frac{1}{n+1}\right)(X_{n+1}-\olx{n})^2 \right]	\\
						&= \frac{(n-1)s^2_n + \frac{n}{n+1}(X_{n+1}-\olx{n})^2}{n}
		\end{align*}
	
\end{itemize}	


%%%________________________________________________________________%%%

\section*{Question 2}
\textbf{For some integer $k$, set $\mu_k=E(X^k)$. Construct an unbiased estimator $\hat{\mu}_k$ for $\mu_k$, and show its unbiasedness.}
\bigskip
Define $\hat{\mu}_k = \frac{1}{n}\sum_{i=1}^nX_i^k$. If the bias of this estimator is equal to zero, then it is unbiased:
\begin{align*}
	E(\hat{\mu}_k)-\mu_k &= 0						\\
	E(\frac{1}{n}\sum_{i=1}^nX_i^k) - E(X^k) &= 0	\\
	\frac{1}{n}\sum_{i=1}^nX_i^k &= X^k
\end{align*}
Since $\{X_i\}_{i=1}^n$ is assumed to be a random sample, this is equality holds. Thus, $\hat{\mu}_k$ is an unbiased estimator.

%%%________________________________________________________________%%%

\section*{Question 3}
\textbf{Consider the central moment $m_k=E((X-\mu)^k)$. Construct an estimator $\hat{m}_k$ for $m_k$ without assuming a known $\mu$. In general, do you expect $\hat{m}_k$ to be biased or unbiased?}
\bigskip

%%%________________________________________________________________%%%

\section*{Question 4}
\textbf{Calculate the variance of $\hat{\mu}_k$ that you proposed above, and call it $Var(\hat{\mu}_k)$.}
\bigskip

%%%________________________________________________________________%%%

\section*{Question 5}
\textbf{Show that $E(s_n)\leq\sigma$ using Jensen's inequality (CB Theorem 4.7.7).}
\bigskip

%%%________________________________________________________________%%%

\section*{Question 6}
\textbf{Show algebraically that $\hat{\sigma}^2=n^{-1}\sum_{i-1}^n(X_i-\mu)^2-(\overline{X}_n-\mu)^2$.}
\bigskip

%%%________________________________________________________________%%%

\section*{Question 7}
\textbf{Find the covariance of $\hat{\sigma}^2$ and $\overline{X}_n$. Under what condition is this zero? (See lecture question for hint)}
\bigskip

%%%________________________________________________________________%%%

\section*{Question 8}
\textbf{Suppose that $X_i$ are independent but not necessarily identically distributed (i.n.i.d.) with $E(X_i)=\mu_i$ and $Var(X_i)=\sigma_i^2$.}

\begin{itemize}
	\item[(a)] \textbf{Find $E(\overline{X}_n)$.} \\
	
	\item[(b)] \textbf{Find $Var(\overline{X}_n)$.} \\
	
\end{itemize}	

%%%________________________________________________________________%%%

\section*{Question 9}
\textbf{Show that if $Q\sim \chi^2_r$, then $E(Q)=r$ and $Var(Q)=2r$ (hint: use the representation $Q=\sum_{i=1}^n X_i^2$ with $X_i$ being i.i.d $\N(0,1)$).}
\bigskip

%%%________________________________________________________________%%%

\section*{Question 10}
\textbf{Suppose that $X_i\sim\N(\mu_X,\sigma^2_X):i=1,...,n_1$ and $Y_i\sim\N(\mu_Y,\sigma_Y^2),i=1,...,n_2$ are mutually independent. Set $\overline{X}_n=n_1^{-1}\sum_{i=1}^{n_2}Y_i$.}

\begin{itemize}
	\item[(a)] \textbf{Find $E(\overline{X}_n-\overline{Y}_n)$.} \\
	
	\item[(b)] \textbf{Find $Var(\overline{X}_n-\overline{Y}_n)$.} \\
	
	\item[(c)] \textbf{Find the distribution of $\overline{X}_n-\overline{Y}_n$.} \\
	
\end{itemize}	

%%%________________________________________________________________%%%





\end{document}












